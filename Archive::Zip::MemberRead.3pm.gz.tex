\documentclass[]{article}
\usepackage[T1]{fontenc}
\usepackage{lmodern}
\usepackage{amssymb,amsmath}
\usepackage{ifxetex,ifluatex}
\usepackage{fixltx2e} % provides \textsubscript
% use upquote if available, for straight quotes in verbatim environments
\IfFileExists{upquote.sty}{\usepackage{upquote}}{}
\ifnum 0\ifxetex 1\fi\ifluatex 1\fi=0 % if pdftex
  \usepackage[utf8]{inputenc}
\else % if luatex or xelatex
  \ifxetex
    \usepackage{mathspec}
    \usepackage{xltxtra,xunicode}
  \else
    \usepackage{fontspec}
  \fi
  \defaultfontfeatures{Mapping=tex-text,Scale=MatchLowercase}
  \newcommand{\euro}{€}
\fi
% use microtype if available
\IfFileExists{microtype.sty}{\usepackage{microtype}}{}
\usepackage{longtable,booktabs}
\ifxetex
  \usepackage[setpagesize=false, % page size defined by xetex
              unicode=false, % unicode breaks when used with xetex
              xetex]{hyperref}
\else
  \usepackage[unicode=true]{hyperref}
\fi
\hypersetup{breaklinks=true,
            bookmarks=true,
            pdfauthor={},
            pdftitle={Archive::Zip::MemberRead(3pm)},
            colorlinks=true,
            citecolor=blue,
            urlcolor=blue,
            linkcolor=magenta,
            pdfborder={0 0 0}}
\urlstyle{same}  % don't use monospace font for urls
\setlength{\parindent}{0pt}
\setlength{\parskip}{6pt plus 2pt minus 1pt}
\setlength{\emergencystretch}{3em}  % prevent overfull lines
\setcounter{secnumdepth}{0}
\usepackage{pagecolor}

% Set background colour (of the page)
\definecolor{weirdbgcolor}{HTML}{FCF4F0}
\pagecolor{weirdbgcolor}

% Make bold text appear in a particular colour
\definecolor{boldcolor}{HTML}{6E0002}
\let\realtextbf=\textbf
\renewcommand{\textbf}[1]{\textcolor{boldcolor}{\realtextbf{#1}}}

% Use underlines instead of emphasis (ugh)
\renewcommand{\emph}[1]{\underline{#1}}

% % Use fixed-width font by default
% \renewcommand*\familydefault{\ttdefault}

\title{Archive::Zip::MemberRead(3pm)}
\author{}
\date{}

\begin{document}
\maketitle

\begin{longtable}[c]{@{}lll@{}}
\toprule\addlinespace
Archive::Zip::MemberRead(3pm) & User Contributed Perl Documentation &
Archive::Zip::MemberRead(3pm)
\\\addlinespace
\bottomrule
\end{longtable}

\hyperdef{}{NAME}{\section{\hyperref[NAME]{NAME}}\label{NAME}}

Archive::Zip::MemberRead - A wrapper that lets you read Zip archive
members as if they were files.

\hyperdef{}{SYNOPSIS}{\section{\hyperref[SYNOPSIS]{SYNOPSIS}}\label{SYNOPSIS}}

\begin{verbatim}
  use Archive::Zip;
  use Archive::Zip::MemberRead;
  $zip = Archive::Zip->new("file.zip");
  $fh  = Archive::Zip::MemberRead->new($zip, "subdir/abc.txt");
  while (defined($line = $fh->getline()))
  {
      print $fh->input_line_number . "#: $line\n";
  }
  $read = $fh->read($buffer, 32*1024);
  print "Read $read bytes as :$buffer:\n";
\end{verbatim}

\hyperdef{}{DESCRIPTION}{\section{\hyperref[DESCRIPTION]{DESCRIPTION}}\label{DESCRIPTION}}

The Archive::Zip::MemberRead module lets you read Zip archive member
data just like you read data from files.

\hyperdef{}{METHODS}{\section{\hyperref[METHODS]{METHODS}}\label{METHODS}}

\begin{description}
\item[\emph{Archive::Zip::Member::readFileHandle()}]
You can get a ``Archive::Zip::MemberRead'' from an archive member by
calling ``readFileHandle()'':

~

\begin{verbatim}
  my $member = $zip->memberNamed('abc/def.c');
  my $fh = $member->readFileHandle();
  while (defined($line = $fh->getline()))
  {
      # ...
  }
  $fh->close();
    
\end{verbatim}
\end{description}

\begin{description}
\item[Archive::Zip::MemberRead-\textgreater{}new(\$zip, \$fileName)]
\end{description}

\begin{description}
\item[Archive::Zip::MemberRead-\textgreater{}new(\$zip, \$member)]
\end{description}

\begin{description}
\item[Archive::Zip::MemberRead-\textgreater{}new(\$member)]
Construct a new Archive::Zip::MemberRead on the specified member.

~

\begin{verbatim}
  my $fh = Archive::Zip::MemberRead->new($zip, 'fred.c')
    
\end{verbatim}
\end{description}

\begin{description}
\itemsep1pt\parskip0pt\parsep0pt
\item[setLineEnd(expr)]
Set the line end character to use. This is set to \textbackslash{}n by
default except on Windows systems where it is set to
\textbackslash{}r\textbackslash{}n. You will only need to set this on
systems which are not Windows or Unix based and require a line end
diffrent from \textbackslash{}n. This is a class method so call as
``Archive::Zip::MemberRead''-\textgreater{}``setLineEnd(\$nl)''
\end{description}

\begin{description}
\itemsep1pt\parskip0pt\parsep0pt
\item[\emph{rewind()}]
Rewinds an ``Archive::Zip::MemberRead'' so that you can read from it
again starting at the beginning.
\end{description}

\begin{description}
\itemsep1pt\parskip0pt\parsep0pt
\item[input\_record\_separator(expr)]
If the argumnet is given, input\_record\_separator for this instance is
set to it. The current setting (which may be the global \$/) is always
returned.
\end{description}

\begin{description}
\itemsep1pt\parskip0pt\parsep0pt
\item[\emph{input\_line\_number()}]
Returns the current line number, but only if you're using ``getline()''.
Using ``read()'' will not update the line number.
\end{description}

\begin{description}
\itemsep1pt\parskip0pt\parsep0pt
\item[\emph{close()}]
Closes the given file handle.
\end{description}

\begin{description}
\itemsep1pt\parskip0pt\parsep0pt
\item[buffer\_size({[} \$size {]})]
Gets or sets the buffer size used for reads. Default is the chunk size
used by Archive::Zip.
\end{description}

\begin{description}
\itemsep1pt\parskip0pt\parsep0pt
\item[\emph{getline()}]
Returns the next line from the currently open member. Makes sense only
for text files. A read error is considered fatal enough to die. Returns
undef on eof. All subsequent calls would return undef, unless a
\emph{rewind()} is called. Note: The line returned has the
input\_record\_separator (default: newline) removed.
\end{description}

\begin{description}
\item[read(\$buffer, \$num\_bytes\_to\_read)]
Simulates a normal ``read()'' system call. Returns the no. of bytes
read. ``undef'' on error, 0 on eof, \emph{e.g.}:

~

\begin{verbatim}
  $fh = Archive::Zip::MemberRead->new($zip, "sreeji/secrets.bin");
  while (1)
  {
    $read = $fh->read($buffer, 1024);
    die "FATAL ERROR reading my secrets !\n" if (!defined($read));
    last if (!$read);
    # Do processing.
    ....
   }
    
\end{verbatim}
\end{description}

\hyperdef{}{AUTHOR}{\section{\hyperref[AUTHOR]{AUTHOR}}\label{AUTHOR}}

Sreeji K. Das, \textless{}sreeji\_k@yahoo.com\textgreater{} See
Archive::Zip by Ned Konz without which this module does not make any
sense!

Minor mods by Ned Konz.

\hyperdef{}{COPYRIGHT}{\section{\hyperref[COPYRIGHT]{COPYRIGHT}}\label{COPYRIGHT}}

Copyright 2002 Sreeji K. Das.

This program is free software; you can redistribute it and/or modify it
under the same terms as Perl itself.

\begin{longtable}[c]{@{}ll@{}}
\toprule\addlinespace
2009-06-30 & perl v5.14.2
\\\addlinespace
\bottomrule
\end{longtable}

\end{document}
