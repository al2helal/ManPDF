\documentclass[]{article}
\usepackage[T1]{fontenc}
\usepackage{lmodern}
\usepackage{amssymb,amsmath}
\usepackage{ifxetex,ifluatex}
\usepackage{fixltx2e} % provides \textsubscript
% use upquote if available, for straight quotes in verbatim environments
\IfFileExists{upquote.sty}{\usepackage{upquote}}{}
\ifnum 0\ifxetex 1\fi\ifluatex 1\fi=0 % if pdftex
  \usepackage[utf8]{inputenc}
\else % if luatex or xelatex
  \ifxetex
    \usepackage{mathspec}
    \usepackage{xltxtra,xunicode}
  \else
    \usepackage{fontspec}
  \fi
  \defaultfontfeatures{Mapping=tex-text,Scale=MatchLowercase}
  \newcommand{\euro}{€}
\fi
% use microtype if available
\IfFileExists{microtype.sty}{\usepackage{microtype}}{}
\usepackage{longtable,booktabs}
\ifxetex
  \usepackage[setpagesize=false, % page size defined by xetex
              unicode=false, % unicode breaks when used with xetex
              xetex]{hyperref}
\else
  \usepackage[unicode=true]{hyperref}
\fi
\hypersetup{breaklinks=true,
            bookmarks=true,
            pdfauthor={},
            pdftitle={AIO\_ERROR(3)},
            colorlinks=true,
            citecolor=blue,
            urlcolor=blue,
            linkcolor=magenta,
            pdfborder={0 0 0}}
\urlstyle{same}  % don't use monospace font for urls
\setlength{\parindent}{0pt}
\setlength{\parskip}{6pt plus 2pt minus 1pt}
\setlength{\emergencystretch}{3em}  % prevent overfull lines
\setcounter{secnumdepth}{0}
\usepackage{pagecolor}

% Set background colour (of the page)
\definecolor{weirdbgcolor}{HTML}{FCF4F0}
\pagecolor{weirdbgcolor}

% Make bold text appear in a particular colour
\definecolor{boldcolor}{HTML}{6E0002}
\let\realtextbf=\textbf
\renewcommand{\textbf}[1]{\textcolor{boldcolor}{\realtextbf{#1}}}

% Use underlines instead of emphasis (ugh)
\renewcommand{\emph}[1]{\underline{#1}}

% % Use fixed-width font by default
% \renewcommand*\familydefault{\ttdefault}

\title{AIO\_ERROR(3)}
\author{}
\date{}

\begin{document}
\maketitle

\begin{longtable}[c]{@{}lll@{}}
\toprule\addlinespace
AIO\_ERROR(3) & Linux Programmer's Manual & AIO\_ERROR(3)
\\\addlinespace
\bottomrule
\end{longtable}

\hyperdef{}{NAME}{\section{\hyperref[NAME]{NAME}}\label{NAME}}

aio\_error - get error status of asynchronous I/O operation

\hyperdef{}{SYNOPSIS}{\section{\hyperref[SYNOPSIS]{SYNOPSIS}}\label{SYNOPSIS}}

\textbf{\#include \textless{}aio.h\textgreater{}}

~

\textbf{int aio\_error(const struct aiocb *}\emph{aiocbp}\textbf{);}

~

Link with \emph{-lrt}.

\hyperdef{}{DESCRIPTION}{\section{\hyperref[DESCRIPTION]{DESCRIPTION}}\label{DESCRIPTION}}

The \textbf{aio\_error}() function returns the error status for the
asynchronous I/O request with control block pointed to by \emph{aiocbp}.
(See \textbf{aio}(7) for a description of the \emph{aiocb} structure.)

\hyperdef{}{RETURNux5fVALUE}{\section{\hyperref[RETURNux5fVALUE]{RETURN
VALUE}}\label{RETURNux5fVALUE}}

This function returns one of the following:

\begin{description}
\itemsep1pt\parskip0pt\parsep0pt
\item[*]
\textbf{EINPROGRESS}, if the request has not been completed yet.
\end{description}

\begin{description}
\itemsep1pt\parskip0pt\parsep0pt
\item[*]
\textbf{ECANCELED}, if the request was canceled.
\end{description}

\begin{description}
\itemsep1pt\parskip0pt\parsep0pt
\item[*]
0, if the request completed successfully.
\end{description}

\begin{description}
\itemsep1pt\parskip0pt\parsep0pt
\item[*]
A positive error number, if the asynchronous I/O operation failed. This
is the same value that would have been stored in the \emph{errno}
variable in the case of a synchronous \textbf{read}(2),
\textbf{write}(2), \textbf{fsync}(2), or \textbf{fdatasync}(2) call.
\end{description}

\hyperdef{}{ERRORS}{\section{\hyperref[ERRORS]{ERRORS}}\label{ERRORS}}

\begin{description}
\itemsep1pt\parskip0pt\parsep0pt
\item[\textbf{EINVAL}]
\emph{aiocbp} does not point at a control block for an asynchronous I/O
request of which the return status (see \textbf{aio\_return}(3)) has not
been retrieved yet.
\end{description}

\begin{description}
\itemsep1pt\parskip0pt\parsep0pt
\item[\textbf{ENOSYS}]
\textbf{aio\_error}() is not implemented.
\end{description}

\hyperdef{}{VERSIONS}{\section{\hyperref[VERSIONS]{VERSIONS}}\label{VERSIONS}}

The \textbf{aio\_error}() function is available since glibc 2.1.

\hyperdef{}{ATTRIBUTES}{\section{\hyperref[ATTRIBUTES]{ATTRIBUTES}}\label{ATTRIBUTES}}

\hyperdef{}{Multithreadingux5fux28seeux5fpthreadsux287ux29ux29}{\subsection{\hyperref[Multithreadingux5fux28seeux5fpthreadsux287ux29ux29]{Multithreading
(see
pthreads(7))}}\label{Multithreadingux5fux28seeux5fpthreadsux287ux29ux29}}

The \textbf{aio\_error}() function is thread-safe.

\hyperdef{}{CONFORMINGux5fTO}{\section{\hyperref[CONFORMINGux5fTO]{CONFORMING
TO}}\label{CONFORMINGux5fTO}}

POSIX.1-2001, POSIX.1-2008.

\hyperdef{}{EXAMPLE}{\section{\hyperref[EXAMPLE]{EXAMPLE}}\label{EXAMPLE}}

See \textbf{aio}(7).

\hyperdef{}{SEEux5fALSO}{\section{\hyperref[SEEux5fALSO]{SEE
ALSO}}\label{SEEux5fALSO}}

\textbf{aio\_cancel}(3), \textbf{aio\_fsync}(3), \textbf{aio\_read}(3),
\textbf{aio\_return}(3), \textbf{aio\_suspend}(3),
\textbf{aio\_write}(3), \textbf{lio\_listio}(3), \textbf{aio}(7)

\hyperdef{}{COLOPHON}{\section{\hyperref[COLOPHON]{COLOPHON}}\label{COLOPHON}}

This page is part of release 3.54 of the Linux \emph{man-pages} project.
A description of the project, and information about reporting bugs, can
be found at http://www.kernel.org/doc/man-pages/.

\begin{longtable}[c]{@{}ll@{}}
\toprule\addlinespace
2013-07-04 &
\\\addlinespace
\bottomrule
\end{longtable}

\end{document}
