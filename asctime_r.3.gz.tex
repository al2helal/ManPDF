\documentclass[]{article}
\usepackage[T1]{fontenc}
\usepackage{lmodern}
\usepackage{amssymb,amsmath}
\usepackage{ifxetex,ifluatex}
\usepackage{fixltx2e} % provides \textsubscript
% use upquote if available, for straight quotes in verbatim environments
\IfFileExists{upquote.sty}{\usepackage{upquote}}{}
\ifnum 0\ifxetex 1\fi\ifluatex 1\fi=0 % if pdftex
  \usepackage[utf8]{inputenc}
\else % if luatex or xelatex
  \ifxetex
    \usepackage{mathspec}
    \usepackage{xltxtra,xunicode}
  \else
    \usepackage{fontspec}
  \fi
  \defaultfontfeatures{Mapping=tex-text,Scale=MatchLowercase}
  \newcommand{\euro}{€}
\fi
% use microtype if available
\IfFileExists{microtype.sty}{\usepackage{microtype}}{}
\usepackage{longtable,booktabs}
\ifxetex
  \usepackage[setpagesize=false, % page size defined by xetex
              unicode=false, % unicode breaks when used with xetex
              xetex]{hyperref}
\else
  \usepackage[unicode=true]{hyperref}
\fi
\hypersetup{breaklinks=true,
            bookmarks=true,
            pdfauthor={},
            pdftitle={CTIME(3)},
            colorlinks=true,
            citecolor=blue,
            urlcolor=blue,
            linkcolor=magenta,
            pdfborder={0 0 0}}
\urlstyle{same}  % don't use monospace font for urls
\setlength{\parindent}{0pt}
\setlength{\parskip}{6pt plus 2pt minus 1pt}
\setlength{\emergencystretch}{3em}  % prevent overfull lines
\setcounter{secnumdepth}{0}
\usepackage{pagecolor}

% Set background colour (of the page)
\definecolor{weirdbgcolor}{HTML}{FCF4F0}
\pagecolor{weirdbgcolor}

% Make bold text appear in a particular colour
\definecolor{boldcolor}{HTML}{6E0002}
\let\realtextbf=\textbf
\renewcommand{\textbf}[1]{\textcolor{boldcolor}{\realtextbf{#1}}}

% Use underlines instead of emphasis (ugh)
\renewcommand{\emph}[1]{\underline{#1}}

% % Use fixed-width font by default
% \renewcommand*\familydefault{\ttdefault}

\title{CTIME(3)}
\author{}
\date{}

\begin{document}
\maketitle

\begin{longtable}[c]{@{}lll@{}}
\toprule\addlinespace
CTIME(3) & Linux Programmer's Manual & CTIME(3)
\\\addlinespace
\bottomrule
\end{longtable}

\hyperdef{}{NAME}{\section{\hyperref[NAME]{NAME}}\label{NAME}}

asctime, ctime, gmtime, localtime, mktime, asctime\_r, ctime\_r,
gmtime\_r, localtime\_r - transform date and time to broken-down time or
ASCII

\hyperdef{}{SYNOPSIS}{\section{\hyperref[SYNOPSIS]{SYNOPSIS}}\label{SYNOPSIS}}

\begin{verbatim}
#include <time.h>
 
char *asctime(const struct tm *tm);
 
char *asctime_r(const struct tm *tm, char *buf);
 
char *ctime(const time_t *timep);
 
char *ctime_r(const time_t *timep, char *buf);
 
struct tm *gmtime(const time_t *timep);
 
struct tm *gmtime_r(const time_t *timep, struct tm *result);
 
struct tm *localtime(const time_t *timep);
 
struct tm *localtime_r(const time_t *timep, struct tm *result);
 
time_t mktime(struct tm *tm);
\end{verbatim}

~

Feature Test Macro Requirements for glibc (see
\textbf{feature\_test\_macros}(7)): \\

~

\textbf{asctime\_r}(), \textbf{ctime\_r}(), \textbf{gmtime\_r}(),
\textbf{localtime\_r}():

\_POSIX\_C\_SOURCE~\textgreater{}=~1 \textbar{}\textbar{}
\_XOPEN\_SOURCE \textbar{}\textbar{} \_BSD\_SOURCE \textbar{}\textbar{}
\_SVID\_SOURCE \textbar{}\textbar{} \_POSIX\_SOURCE

\hyperdef{}{DESCRIPTION}{\section{\hyperref[DESCRIPTION]{DESCRIPTION}}\label{DESCRIPTION}}

The \textbf{ctime}(), \textbf{gmtime}() and \textbf{localtime}()
functions all take an argument of data type \emph{time\_t} which
represents calendar time. When interpreted as an absolute time value, it
represents the number of seconds elapsed since the Epoch, 1970-01-01
00:00:00 +0000 (UTC).

The \textbf{asctime}() and \textbf{mktime}() functions both take an
argument representing broken-down time which is a representation
separated into year, month, day, and so on.

Broken-down time is stored in the structure \emph{tm} which is defined
in \emph{\textless{}time.h\textgreater{}} as follows:

~

\begin{verbatim}
struct tm {
    int tm_sec;         /* seconds */
    int tm_min;         /* minutes */
    int tm_hour;        /* hours */
    int tm_mday;        /* day of the month */
    int tm_mon;         /* month */
    int tm_year;        /* year */
    int tm_wday;        /* day of the week */
    int tm_yday;        /* day in the year */
    int tm_isdst;       /* daylight saving time */
};
\end{verbatim}

The members of the \emph{tm} structure are:

\begin{description}
\itemsep1pt\parskip0pt\parsep0pt
\item[\emph{tm\_sec}]
The number of seconds after the minute, normally in the range 0 to 59,
but can be up to 60 to allow for leap seconds.
\end{description}

\begin{description}
\itemsep1pt\parskip0pt\parsep0pt
\item[\emph{tm\_min}]
The number of minutes after the hour, in the range 0 to 59.
\end{description}

\begin{description}
\itemsep1pt\parskip0pt\parsep0pt
\item[\emph{tm\_hour}]
The number of hours past midnight, in the range 0 to 23.
\end{description}

\begin{description}
\itemsep1pt\parskip0pt\parsep0pt
\item[\emph{tm\_mday}]
The day of the month, in the range 1 to 31.
\end{description}

\begin{description}
\itemsep1pt\parskip0pt\parsep0pt
\item[\emph{tm\_mon}]
The number of months since January, in the range 0 to 11.
\end{description}

\begin{description}
\itemsep1pt\parskip0pt\parsep0pt
\item[\emph{tm\_year}]
The number of years since 1900.
\end{description}

\begin{description}
\itemsep1pt\parskip0pt\parsep0pt
\item[\emph{tm\_wday}]
The number of days since Sunday, in the range 0 to 6.
\end{description}

\begin{description}
\itemsep1pt\parskip0pt\parsep0pt
\item[\emph{tm\_yday}]
The number of days since January 1, in the range 0 to 365.
\end{description}

\begin{description}
\itemsep1pt\parskip0pt\parsep0pt
\item[\emph{tm\_isdst}]
A flag that indicates whether daylight saving time is in effect at the
time described. The value is positive if daylight saving time is in
effect, zero if it is not, and negative if the information is not
available.
\end{description}

The call \textbf{ctime(}\emph{t}\textbf{)} is equivalent to
\textbf{asctime(localtime(}\emph{t}\textbf{))}\emph{.} It converts the
calendar time \emph{t} into a null-terminated string of the form

~

``Wed Jun 30 21:49:08 1993\textbackslash{}n''

~

The abbreviations for the days of the week are ``Sun'', ``Mon'',
``Tue'', ``Wed'', ``Thu'', ``Fri'', and ``Sat''. The abbreviations for
the months are ``Jan'', ``Feb'', ``Mar'', ``Apr'', ``May'', ``Jun'',
``Jul'', ``Aug'', ``Sep'', ``Oct'', ``Nov'', and ``Dec''. The return
value points to a statically allocated string which might be overwritten
by subsequent calls to any of the date and time functions. The function
also sets the external variables \emph{tzname}, \emph{timezone}, and
\emph{daylight} (see \textbf{tzset}(3)) with information about the
current timezone. The reentrant version \textbf{ctime\_r}() does the
same, but stores the string in a user-supplied buffer which should have
room for at least 26 bytes. It need not set \emph{tzname},
\emph{timezone}, and \emph{daylight}.

The \textbf{gmtime}() function converts the calendar time \emph{timep}
to broken-down time representation, expressed in Coordinated Universal
Time (UTC). It may return NULL when the year does not fit into an
integer. The return value points to a statically allocated struct which
might be overwritten by subsequent calls to any of the date and time
functions. The \textbf{gmtime\_r}() function does the same, but stores
the data in a user-supplied struct.

The \textbf{localtime}() function converts the calendar time
\emph{timep} to broken-down time representation, expressed relative to
the user's specified timezone. The function acts as if it called
\textbf{tzset}(3) and sets the external variables \emph{tzname} with
information about the current timezone, \emph{timezone} with the
difference between Coordinated Universal Time (UTC) and local standard
time in seconds, and \emph{daylight} to a nonzero value if daylight
savings time rules apply during some part of the year. The return value
points to a statically allocated struct which might be overwritten by
subsequent calls to any of the date and time functions. The
\textbf{localtime\_r}() function does the same, but stores the data in a
user-supplied struct. It need not set \emph{tzname}, \emph{timezone},
and \emph{daylight}.

The \textbf{asctime}() function converts the broken-down time value
\emph{tm} into a null-terminated string with the same format as
\textbf{ctime}(). The return value points to a statically allocated
string which might be overwritten by subsequent calls to any of the date
and time functions. The \textbf{asctime\_r}() function does the same,
but stores the string in a user-supplied buffer which should have room
for at least 26 bytes.

The \textbf{mktime}() function converts a broken-down time structure,
expressed as local time, to calendar time representation. The function
ignores the values supplied by the caller in the \emph{tm\_wday} and
\emph{tm\_yday} fields. The value specified in the \emph{tm\_isdst}
field informs \textbf{mktime}() whether or not daylight saving time
(DST) is in effect for the time supplied in the \emph{tm} structure: a
positive value means DST is in effect; zero means that DST is not in
effect; and a negative value means that \textbf{mktime}() should (use
timezone information and system databases to) attempt to determine
whether DST is in effect at the specified time.

~

The \textbf{mktime}() function modifies the fields of the \emph{tm}
structure as follows: \emph{tm\_wday} and \emph{tm\_yday} are set to
values determined from the contents of the other fields; if structure
members are outside their valid interval, they will be normalized (so
that, for example, 40 October is changed into 9 November);
\emph{tm\_isdst} is set (regardless of its initial value) to a positive
value or to 0, respectively, to indicate whether DST is or is not in
effect at the specified time. Calling \textbf{mktime}() also sets the
external variable \emph{tzname} with information about the current
timezone.

~

If the specified broken-down time cannot be represented as calendar time
(seconds since the Epoch), \textbf{mktime}() returns \emph{(time\_t)~-1}
and does not alter the members of the broken-down time structure.

\hyperdef{}{RETURNux5fVALUE}{\section{\hyperref[RETURNux5fVALUE]{RETURN
VALUE}}\label{RETURNux5fVALUE}}

Each of these functions returns the value described, or NULL (-1 in case
of \textbf{mktime}()) in case an error was detected.

\hyperdef{}{CONFORMINGux5fTO}{\section{\hyperref[CONFORMINGux5fTO]{CONFORMING
TO}}\label{CONFORMINGux5fTO}}

POSIX.1-2001. C89 and C99 specify \textbf{asctime}(), \textbf{ctime}(),
\textbf{gmtime}(), \textbf{localtime}(), and \textbf{mktime}().
POSIX.1-2008 marks \textbf{asctime}(), \textbf{asctime\_r}(),
\textbf{ctime}(), and \textbf{ctime\_r}() as obsolete, recommending the
use of \textbf{strftime}(3) instead.

\hyperdef{}{NOTES}{\section{\hyperref[NOTES]{NOTES}}\label{NOTES}}

The four functions \textbf{asctime}(), \textbf{ctime}(),
\textbf{gmtime}() and \textbf{localtime}() return a pointer to static
data and hence are not thread-safe. Thread-safe versions
\textbf{asctime\_r}(), \textbf{ctime\_r}(), \textbf{gmtime\_r}() and
\textbf{localtime\_r}() are specified by SUSv2, and available since libc
5.2.5.

~

POSIX.1-2001 says: "The \textbf{asctime}(), \textbf{ctime}(),
\textbf{gmtime}(), and \textbf{localtime}() functions shall return
values in one of two static objects: a broken-down time structure and an
array of type \emph{char}. Execution of any of the functions may
overwrite the information returned in either of these objects by any of
the other functions." This can occur in the glibc implementation.

In many implementations, including glibc, a 0 in \emph{tm\_mday} is
interpreted as meaning the last day of the preceding month.

The glibc version of \emph{struct tm} has additional fields

~

\begin{verbatim}
long tm_gmtoff;           /* Seconds east of UTC */
const char *tm_zone;      /* Timezone abbreviation */
\end{verbatim}

~

defined when \textbf{\_BSD\_SOURCE} was set before including
\emph{\textless{}time.h\textgreater{}}. This is a BSD extension, present
in 4.3BSD-Reno.

~

According to POSIX.1-2004, \textbf{localtime}() is required to behave as
though \textbf{tzset}(3) was called, while \textbf{localtime\_r}() does
not have this requirement. For portable code \textbf{tzset}(3) should be
called before \textbf{localtime\_r}().

\hyperdef{}{SEEux5fALSO}{\section{\hyperref[SEEux5fALSO]{SEE
ALSO}}\label{SEEux5fALSO}}

\textbf{date}(1), \textbf{gettimeofday}(2), \textbf{time}(2),
\textbf{utime}(2), \textbf{clock}(3), \textbf{difftime}(3),
\textbf{strftime}(3), \textbf{strptime}(3), \textbf{timegm}(3),
\textbf{tzset}(3), \textbf{time}(7)

\hyperdef{}{COLOPHON}{\section{\hyperref[COLOPHON]{COLOPHON}}\label{COLOPHON}}

This page is part of release 3.54 of the Linux \emph{man-pages} project.
A description of the project, and information about reporting bugs, can
be found at http://www.kernel.org/doc/man-pages/.

\begin{longtable}[c]{@{}ll@{}}
\toprule\addlinespace
2010-02-25 &
\\\addlinespace
\bottomrule
\end{longtable}

\end{document}
