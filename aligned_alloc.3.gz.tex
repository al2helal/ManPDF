\documentclass[]{article}
\usepackage[T1]{fontenc}
\usepackage{lmodern}
\usepackage{amssymb,amsmath}
\usepackage{ifxetex,ifluatex}
\usepackage{fixltx2e} % provides \textsubscript
% use upquote if available, for straight quotes in verbatim environments
\IfFileExists{upquote.sty}{\usepackage{upquote}}{}
\ifnum 0\ifxetex 1\fi\ifluatex 1\fi=0 % if pdftex
  \usepackage[utf8]{inputenc}
\else % if luatex or xelatex
  \ifxetex
    \usepackage{mathspec}
    \usepackage{xltxtra,xunicode}
  \else
    \usepackage{fontspec}
  \fi
  \defaultfontfeatures{Mapping=tex-text,Scale=MatchLowercase}
  \newcommand{\euro}{€}
\fi
% use microtype if available
\IfFileExists{microtype.sty}{\usepackage{microtype}}{}
\usepackage{longtable,booktabs}
\ifxetex
  \usepackage[setpagesize=false, % page size defined by xetex
              unicode=false, % unicode breaks when used with xetex
              xetex]{hyperref}
\else
  \usepackage[unicode=true]{hyperref}
\fi
\hypersetup{breaklinks=true,
            bookmarks=true,
            pdfauthor={},
            pdftitle={POSIX\_MEMALIGN(3)},
            colorlinks=true,
            citecolor=blue,
            urlcolor=blue,
            linkcolor=magenta,
            pdfborder={0 0 0}}
\urlstyle{same}  % don't use monospace font for urls
\setlength{\parindent}{0pt}
\setlength{\parskip}{6pt plus 2pt minus 1pt}
\setlength{\emergencystretch}{3em}  % prevent overfull lines
\setcounter{secnumdepth}{0}
\usepackage{pagecolor}

% Set background colour (of the page)
\definecolor{weirdbgcolor}{HTML}{FCF4F0}
\pagecolor{weirdbgcolor}

% Make bold text appear in a particular colour
\definecolor{boldcolor}{HTML}{6E0002}
\let\realtextbf=\textbf
\renewcommand{\textbf}[1]{\textcolor{boldcolor}{\realtextbf{#1}}}

% Use underlines instead of emphasis (ugh)
\renewcommand{\emph}[1]{\underline{#1}}

% % Use fixed-width font by default
% \renewcommand*\familydefault{\ttdefault}

\title{POSIX\_MEMALIGN(3)}
\author{}
\date{}

\begin{document}
\maketitle

\begin{longtable}[c]{@{}lll@{}}
\toprule\addlinespace
POSIX\_MEMALIGN(3) & Linux Programmer's Manual & POSIX\_MEMALIGN(3)
\\\addlinespace
\bottomrule
\end{longtable}

\hyperdef{}{NAME}{\section{\hyperref[NAME]{NAME}}\label{NAME}}

posix\_memalign, aligned\_alloc, memalign, valloc, pvalloc - allocate
aligned memory

\hyperdef{}{SYNOPSIS}{\section{\hyperref[SYNOPSIS]{SYNOPSIS}}\label{SYNOPSIS}}

\begin{verbatim}
#include <stdlib.h>
 
int posix_memalign(void **memptr, size_t alignment, size_t size);
void *aligned_alloc(size_t alignment, size_t size);
void *valloc(size_t size);
 
#include <malloc.h>
 
void *memalign(size_t alignment, size_t size);
void *pvalloc(size_t size);
\end{verbatim}

~

Feature Test Macro Requirements for glibc (see
\textbf{feature\_test\_macros}(7)): \\

~

\textbf{posix\_memalign}(): \_POSIX\_C\_SOURCE~\textgreater{}=~200112L
\textbar{}\textbar{} \_XOPEN\_SOURCE~\textgreater{}=~600

~

\textbf{aligned\_alloc}(): \_ISOC11\_SOURCE

~

\textbf{valloc}():

~

\begin{description}
\item[Since glibc 2.12:]
\begin{verbatim}
_BSD_SOURCE ||
    (_XOPEN_SOURCE >= 500 ||
        _XOPEN_SOURCE && _XOPEN_SOURCE_EXTENDED) &&
    !(_POSIX_C_SOURCE >= 200112L || _XOPEN_SOURCE >= 600)
 
    
\end{verbatim}
\end{description}

\begin{description}
\itemsep1pt\parskip0pt\parsep0pt
\item[Before glibc 2.12:]
\_BSD\_SOURCE \textbar{}\textbar{} \_XOPEN\_SOURCE~\textgreater{}=~500
\textbar{}\textbar{} \_XOPEN\_SOURCE~\&\&~\_XOPEN\_SOURCE\_EXTENDED

~

(The (nonstandard) header file \emph{\textless{}malloc.h\textgreater{}}
also exposes the declaration of \textbf{valloc}(); no feature test
macros are required.)
\end{description}

\hyperdef{}{DESCRIPTION}{\section{\hyperref[DESCRIPTION]{DESCRIPTION}}\label{DESCRIPTION}}

The function \textbf{posix\_memalign}() allocates \emph{size} bytes and
places the address of the allocated memory in \emph{*memptr}. The
address of the allocated memory will be a multiple of \emph{alignment},
which must be a power of two and a multiple of \emph{sizeof(void~*)}. If
\emph{size} is 0, then the value placed in \emph{*memptr} is either
NULL, or a unique pointer value that can later be successfully passed to
\textbf{free}(3).

~

The obsolete function \textbf{memalign}() allocates \emph{size} bytes
and returns a pointer to the allocated memory. The memory address will
be a multiple of \emph{alignment}, which must be a power of two.

~

The function \textbf{aligned\_alloc}() is the same as
\textbf{memalign}(), except for the added restriction that \emph{size}
should be a multiple of \emph{alignment}.

~

The obsolete function \textbf{valloc}() allocates \emph{size} bytes and
returns a pointer to the allocated memory. The memory address will be a
multiple of the page size. It is equivalent to
\emph{memalign(sysconf(\_SC\_PAGESIZE),size)}.

~

The obsolete function \textbf{pvalloc}() is similar to
\textbf{valloc}(), but rounds the size of the allocation up to the next
multiple of the system page size.

~

For all of these functions, the memory is not zeroed.

\hyperdef{}{RETURNux5fVALUE}{\section{\hyperref[RETURNux5fVALUE]{RETURN
VALUE}}\label{RETURNux5fVALUE}}

\textbf{aligned\_alloc}(), \textbf{memalign}(), \textbf{valloc}(), and
\textbf{pvalloc}() return a pointer to the allocated memory, or NULL if
the request fails.

~

\textbf{posix\_memalign}() returns zero on success, or one of the error
values listed in the next section on failure. The value of \emph{errno}
is indeterminate after a call to \textbf{posix\_memalign}().

\hyperdef{}{ERRORS}{\section{\hyperref[ERRORS]{ERRORS}}\label{ERRORS}}

\begin{description}
\itemsep1pt\parskip0pt\parsep0pt
\item[\textbf{EINVAL}]
The \emph{alignment} argument was not a power of two, or was not a
multiple of \emph{sizeof(void~*)}.
\end{description}

\begin{description}
\itemsep1pt\parskip0pt\parsep0pt
\item[\textbf{ENOMEM}]
There was insufficient memory to fulfill the allocation request.
\end{description}

\hyperdef{}{VERSIONS}{\section{\hyperref[VERSIONS]{VERSIONS}}\label{VERSIONS}}

The functions \textbf{memalign}(), \textbf{valloc}(), and
\textbf{pvalloc}() have been available in all Linux libc libraries.

~

The function \textbf{aligned\_alloc}() was added to glibc in version
2.16.

~

The function \textbf{posix\_memalign}() is available since glibc 2.1.91.

\hyperdef{}{CONFORMINGux5fTO}{\section{\hyperref[CONFORMINGux5fTO]{CONFORMING
TO}}\label{CONFORMINGux5fTO}}

The function \textbf{valloc}() appeared in 3.0BSD. It is documented as
being obsolete in 4.3BSD, and as legacy in SUSv2. It does not appear in
POSIX.1-2001.

~

The function \textbf{pvalloc}() is a GNU extension.

~

The function \textbf{memalign}() appears in SunOS 4.1.3 but not in
4.4BSD.

~

The function \textbf{posix\_memalign}() comes from POSIX.1d.

~

The function \textbf{aligned\_alloc}() is specified in the C11 standard.

\hyperdef{}{Headers}{\subsection{\hyperref[Headers]{Headers}}\label{Headers}}

Everybody agrees that \textbf{posix\_memalign}() is declared in
\emph{\textless{}stdlib.h\textgreater{}}.

~

On some systems \textbf{memalign}() is declared in
\emph{\textless{}stdlib.h\textgreater{}} instead of
\emph{\textless{}malloc.h\textgreater{}}.

~

According to SUSv2, \textbf{valloc}() is declared in
\emph{\textless{}stdlib.h\textgreater{}}. Libc4,5 and glibc declare it
in \emph{\textless{}malloc.h\textgreater{}}, and also in
\emph{\textless{}stdlib.h\textgreater{}} if suitable feature test macros
are defined (see above).

\hyperdef{}{NOTES}{\section{\hyperref[NOTES]{NOTES}}\label{NOTES}}

On many systems there are alignment restrictions, for example, on
buffers used for direct block device I/O. POSIX specifies the
\emph{pathconf(path,\_PC\_REC\_XFER\_ALIGN)} call that tells what
alignment is needed. Now one can use \textbf{posix\_memalign}() to
satisfy this requirement.

~

\textbf{posix\_memalign}() verifies that \emph{alignment} matches the
requirements detailed above. \textbf{memalign}() may not check that the
\emph{alignment} argument is correct.

~

POSIX requires that memory obtained from \textbf{posix\_memalign}() can
be freed using \textbf{free}(3). Some systems provide no way to reclaim
memory allocated with \textbf{memalign}() or \textbf{valloc}() (because
one can pass to \textbf{free}(3) only a pointer obtained from
\textbf{malloc}(3), while, for example, \textbf{memalign}() would call
\textbf{malloc}(3) and then align the obtained value). The glibc
implementation allows memory obtained from any of these functions to be
reclaimed with \textbf{free}(3).

~

The glibc \textbf{malloc}(3) always returns 8-byte aligned memory
addresses, so these functions are needed only if you require larger
alignment values.

\hyperdef{}{SEEux5fALSO}{\section{\hyperref[SEEux5fALSO]{SEE
ALSO}}\label{SEEux5fALSO}}

\textbf{brk}(2), \textbf{getpagesize}(2), \textbf{free}(3),
\textbf{malloc}(3)

\hyperdef{}{COLOPHON}{\section{\hyperref[COLOPHON]{COLOPHON}}\label{COLOPHON}}

This page is part of release 3.54 of the Linux \emph{man-pages} project.
A description of the project, and information about reporting bugs, can
be found at http://www.kernel.org/doc/man-pages/.

\begin{longtable}[c]{@{}ll@{}}
\toprule\addlinespace
2013-09-02 & GNU
\\\addlinespace
\bottomrule
\end{longtable}

\end{document}
