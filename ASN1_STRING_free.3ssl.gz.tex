\documentclass[]{article}
\usepackage[T1]{fontenc}
\usepackage{lmodern}
\usepackage{amssymb,amsmath}
\usepackage{ifxetex,ifluatex}
\usepackage{fixltx2e} % provides \textsubscript
% use upquote if available, for straight quotes in verbatim environments
\IfFileExists{upquote.sty}{\usepackage{upquote}}{}
\ifnum 0\ifxetex 1\fi\ifluatex 1\fi=0 % if pdftex
  \usepackage[utf8]{inputenc}
\else % if luatex or xelatex
  \ifxetex
    \usepackage{mathspec}
    \usepackage{xltxtra,xunicode}
  \else
    \usepackage{fontspec}
  \fi
  \defaultfontfeatures{Mapping=tex-text,Scale=MatchLowercase}
  \newcommand{\euro}{€}
\fi
% use microtype if available
\IfFileExists{microtype.sty}{\usepackage{microtype}}{}
\usepackage{longtable,booktabs}
\ifxetex
  \usepackage[setpagesize=false, % page size defined by xetex
              unicode=false, % unicode breaks when used with xetex
              xetex]{hyperref}
\else
  \usepackage[unicode=true]{hyperref}
\fi
\hypersetup{breaklinks=true,
            bookmarks=true,
            pdfauthor={},
            pdftitle={ASN1\_STRING\_new(3SSL)},
            colorlinks=true,
            citecolor=blue,
            urlcolor=blue,
            linkcolor=magenta,
            pdfborder={0 0 0}}
\urlstyle{same}  % don't use monospace font for urls
\setlength{\parindent}{0pt}
\setlength{\parskip}{6pt plus 2pt minus 1pt}
\setlength{\emergencystretch}{3em}  % prevent overfull lines
\setcounter{secnumdepth}{0}
\usepackage{pagecolor}

% Set background colour (of the page)
\definecolor{weirdbgcolor}{HTML}{FCF4F0}
\pagecolor{weirdbgcolor}

% Make bold text appear in a particular colour
\definecolor{boldcolor}{HTML}{6E0002}
\let\realtextbf=\textbf
\renewcommand{\textbf}[1]{\textcolor{boldcolor}{\realtextbf{#1}}}

% Use underlines instead of emphasis (ugh)
\renewcommand{\emph}[1]{\underline{#1}}

% % Use fixed-width font by default
% \renewcommand*\familydefault{\ttdefault}

\title{ASN1\_STRING\_new(3SSL)}
\author{}
\date{}

\begin{document}
\maketitle

\begin{longtable}[c]{@{}lll@{}}
\toprule\addlinespace
ASN1\_STRING\_new(3SSL) & OpenSSL & ASN1\_STRING\_new(3SSL)
\\\addlinespace
\bottomrule
\end{longtable}

\hyperdef{}{NAME}{\section{\hyperref[NAME]{NAME}}\label{NAME}}

ASN1\_STRING\_new, ASN1\_STRING\_type\_new, ASN1\_STRING\_free -
ASN1\_STRING allocation functions

\hyperdef{}{SYNOPSIS}{\section{\hyperref[SYNOPSIS]{SYNOPSIS}}\label{SYNOPSIS}}

\begin{verbatim}
 #include <openssl/asn1.h>
 ASN1_STRING * ASN1_STRING_new(void);
 ASN1_STRING * ASN1_STRING_type_new(int type);
 void ASN1_STRING_free(ASN1_STRING *a);
\end{verbatim}

\hyperdef{}{DESCRIPTION}{\section{\hyperref[DESCRIPTION]{DESCRIPTION}}\label{DESCRIPTION}}

\emph{ASN1\_STRING\_new()} returns an allocated \textbf{ASN1\_STRING}
structure. Its type is undefined.

\emph{ASN1\_STRING\_type\_new()} returns an allocated
\textbf{ASN1\_STRING} structure of type \textbf{type}.

\emph{ASN1\_STRING\_free()} frees up \textbf{a}.

\hyperdef{}{NOTES}{\section{\hyperref[NOTES]{NOTES}}\label{NOTES}}

Other string types call the \textbf{ASN1\_STRING} functions. For example
\emph{ASN1\_OCTET\_STRING\_new()} calls
ASN1\_STRING\_type(V\_ASN1\_OCTET\_STRING).

\hyperdef{}{RETURNux5fVALUES}{\section{\hyperref[RETURNux5fVALUES]{RETURN
VALUES}}\label{RETURNux5fVALUES}}

\emph{ASN1\_STRING\_new()} and \emph{ASN1\_STRING\_type\_new()} return a
valid ASN1\_STRING structure or \textbf{NULL} if an error occurred.

\emph{ASN1\_STRING\_free()} does not return a value.

\hyperdef{}{SEEux5fALSO}{\section{\hyperref[SEEux5fALSO]{SEE
ALSO}}\label{SEEux5fALSO}}

\emph{ERR\_get\_error}(3)

\hyperdef{}{HISTORY}{\section{\hyperref[HISTORY]{HISTORY}}\label{HISTORY}}

TBA

\begin{longtable}[c]{@{}ll@{}}
\toprule\addlinespace
2014-01-06 & 1.0.1f
\\\addlinespace
\bottomrule
\end{longtable}

\end{document}
