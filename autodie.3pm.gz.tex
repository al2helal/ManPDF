\documentclass[]{article}
\usepackage[T1]{fontenc}
\usepackage{lmodern}
\usepackage{amssymb,amsmath}
\usepackage{ifxetex,ifluatex}
\usepackage{fixltx2e} % provides \textsubscript
% use upquote if available, for straight quotes in verbatim environments
\IfFileExists{upquote.sty}{\usepackage{upquote}}{}
\ifnum 0\ifxetex 1\fi\ifluatex 1\fi=0 % if pdftex
  \usepackage[utf8]{inputenc}
\else % if luatex or xelatex
  \ifxetex
    \usepackage{mathspec}
    \usepackage{xltxtra,xunicode}
  \else
    \usepackage{fontspec}
  \fi
  \defaultfontfeatures{Mapping=tex-text,Scale=MatchLowercase}
  \newcommand{\euro}{€}
\fi
% use microtype if available
\IfFileExists{microtype.sty}{\usepackage{microtype}}{}
\usepackage{longtable,booktabs}
\ifxetex
  \usepackage[setpagesize=false, % page size defined by xetex
              unicode=false, % unicode breaks when used with xetex
              xetex]{hyperref}
\else
  \usepackage[unicode=true]{hyperref}
\fi
\hypersetup{breaklinks=true,
            bookmarks=true,
            pdfauthor={},
            pdftitle={autodie(3pm)},
            colorlinks=true,
            citecolor=blue,
            urlcolor=blue,
            linkcolor=magenta,
            pdfborder={0 0 0}}
\urlstyle{same}  % don't use monospace font for urls
\setlength{\parindent}{0pt}
\setlength{\parskip}{6pt plus 2pt minus 1pt}
\setlength{\emergencystretch}{3em}  % prevent overfull lines
\setcounter{secnumdepth}{0}
\usepackage{pagecolor}

% Set background colour (of the page)
\definecolor{weirdbgcolor}{HTML}{FCF4F0}
\pagecolor{weirdbgcolor}

% Make bold text appear in a particular colour
\definecolor{boldcolor}{HTML}{6E0002}
\let\realtextbf=\textbf
\renewcommand{\textbf}[1]{\textcolor{boldcolor}{\realtextbf{#1}}}

% Use underlines instead of emphasis (ugh)
\renewcommand{\emph}[1]{\underline{#1}}

% % Use fixed-width font by default
% \renewcommand*\familydefault{\ttdefault}

\title{autodie(3pm)}
\author{}
\date{}

\begin{document}
\maketitle

\begin{longtable}[c]{@{}lll@{}}
\toprule\addlinespace
autodie(3pm) & User Contributed Perl Documentation & autodie(3pm)
\\\addlinespace
\bottomrule
\end{longtable}

\hyperdef{}{NAME}{\section{\hyperref[NAME]{NAME}}\label{NAME}}

autodie - Replace functions with ones that succeed or die with lexical
scope

\hyperdef{}{SYNOPSIS}{\section{\hyperref[SYNOPSIS]{SYNOPSIS}}\label{SYNOPSIS}}

\begin{verbatim}
    use autodie;            # Recommended: implies 'use autodie qw(:default)'
    use autodie qw(:all);   # Recommended more: defaults and system/exec.
    use autodie qw(open close);   # open/close succeed or die
    open(my $fh, "<", $filename); # No need to check!
    {
        no autodie qw(open);          # open failures won't die
        open(my $fh, "<", $filename); # Could fail silently!
        no autodie;                   # disable all autodies
    }
\end{verbatim}

\hyperdef{}{DESCRIPTION}{\section{\hyperref[DESCRIPTION]{DESCRIPTION}}\label{DESCRIPTION}}

\begin{verbatim}
        bIlujDI' yIchegh()Qo'; yIHegh()!
        It is better to die() than to return() in failure.
                -- Klingon programming proverb.
\end{verbatim}

The ``autodie'' pragma provides a convenient way to replace functions
that normally return false on failure with equivalents that throw an
exception on failure.

The ``autodie'' pragma has \emph{lexical scope}, meaning that functions
and subroutines altered with ``autodie'' will only change their
behaviour until the end of the enclosing block, file, or ``eval''.

If ``system'' is specified as an argument to ``autodie'', then it uses
IPC::System::Simple to do the heavy lifting. See the description of that
module for more information.

\hyperdef{}{EXCEPTIONS}{\section{\hyperref[EXCEPTIONS]{EXCEPTIONS}}\label{EXCEPTIONS}}

Exceptions produced by the ``autodie'' pragma are members of the
autodie::exception class. The preferred way to work with these
exceptions under Perl 5.10 is as follows:

\begin{verbatim}
    use feature qw(switch);
    eval {
        use autodie;
        open(my $fh, '<', $some_file);
        my @records = <$fh>;
        # Do things with @records...
        close($fh);
    };
    given ($@) {
        when (undef)   { say "No error";                    }
        when ('open')  { say "Error from open";             }
        when (':io')   { say "Non-open, IO error.";         }
        when (':all')  { say "All other autodie errors."    }
        default        { say "Not an autodie error at all." }
    }
\end{verbatim}

Under Perl 5.8, the ``given/when'' structure is not available, so the
following structure may be used:

\begin{verbatim}
    eval {
        use autodie;
        open(my $fh, '<', $some_file);
        my @records = <$fh>;
        # Do things with @records...
        close($fh);
    };
    if ($@ and $@->isa('autodie::exception')) {
        if ($@->matches('open')) { print "Error from open\n";   }
        if ($@->matches(':io' )) { print "Non-open, IO error."; }
    } elsif ($@) {
        # A non-autodie exception.
    }
\end{verbatim}

See autodie::exception for further information on interrogating
exceptions.

\hyperdef{}{CATEGORIES}{\section{\hyperref[CATEGORIES]{CATEGORIES}}\label{CATEGORIES}}

Autodie uses a simple set of categories to group together similar
built-ins. Requesting a category type (starting with a colon) will
enable autodie for all built-ins beneath that category. For example,
requesting ``:file'' will enable autodie for ``close'', ``fcntl'',
``fileno'', ``open'' and ``sysopen''.

The categories are currently:

\begin{verbatim}
    :all
        :default
            :io
                read
                seek
                sysread
                sysseek
                syswrite
                :dbm
                    dbmclose
                    dbmopen
                :file
                    binmode
                    close
                    chmod
                    chown
                    fcntl
                    fileno
                    flock
                    ioctl
                    open
                    sysopen
                    truncate
                :filesys
                    chdir
                    closedir
                    opendir
                    link
                    mkdir
                    readlink
                    rename
                    rmdir
                    symlink
                    unlink
                :ipc
                    pipe
                    :msg
                        msgctl
                        msgget
                        msgrcv
                        msgsnd
                    :semaphore
                        semctl
                        semget
                        semop
                    :shm
                        shmctl
                        shmget
                        shmread
                :socket
                    accept
                    bind
                    connect
                    getsockopt
                    listen
                    recv
                    send
                    setsockopt
                    shutdown
                    socketpair
            :threads
                fork
        :system
            system
            exec
\end{verbatim}

Note that while the above category system is presently a strict
hierarchy, this should not be assumed.

A plain ``use autodie'' implies ``use autodie qw(:default)''. Note that
``system'' and ``exec'' are not enabled by default. ``system'' requires
the optional IPC::System::Simple module to be installed, and enabling
``system'' or ``exec'' will invalidate their exotic forms. See ``BUGS''
below for more details.

The syntax:

\begin{verbatim}
    use autodie qw(:1.994);
\end{verbatim}

allows the ``:default'' list from a particular version to be used. This
provides the convenience of using the default methods, but the surety
that no behavioral changes will occur if the ``autodie'' module is
upgraded.

``autodie'' can be enabled for all of Perl's built-ins, including
``system'' and ``exec'' with:

\begin{verbatim}
    use autodie qw(:all);
\end{verbatim}

\hyperdef{}{FUNCTIONux5fSPECIFICux5fNOTES}{\section{\hyperref[FUNCTIONux5fSPECIFICux5fNOTES]{FUNCTION
SPECIFIC NOTES}}\label{FUNCTIONux5fSPECIFICux5fNOTES}}

\hyperdef{}{print}{\subsection{\hyperref[print]{print}}\label{print}}

The autodie pragma \textbf{\textless{}does not check calls
to}\textbf{``print''}\textbf{}\textgreater{}.

\hyperdef{}{flock}{\subsection{\hyperref[flock]{flock}}\label{flock}}

It is not considered an error for ``flock'' to return false if it fails
due to an ``EWOULDBLOCK'' (or equivalent) condition. This means one can
still use the common convention of testing the return value of ``flock''
when called with the ``LOCK\_NB'' option:

\begin{verbatim}
    use autodie;
    if ( flock($fh, LOCK_EX | LOCK_NB) ) {
        # We have a lock
    }
\end{verbatim}

Autodying ``flock'' will generate an exception if ``flock'' returns
false with any other error.

\hyperdef{}{systemux2fexec}{\subsection{\hyperref[systemux2fexec]{system/exec}}\label{systemux2fexec}}

The ``system'' built-in is considered to have failed in the following
circumstances:

\begin{description}
\itemsep1pt\parskip0pt\parsep0pt
\item[•]
The command does not start.
\end{description}

\begin{description}
\itemsep1pt\parskip0pt\parsep0pt
\item[•]
The command is killed by a signal.
\end{description}

\begin{description}
\itemsep1pt\parskip0pt\parsep0pt
\item[•]
The command returns a non-zero exit value (but see below).
\end{description}

On success, the autodying form of ``system'' returns the \emph{exit
value} rather than the contents of \$?.

Additional allowable exit values can be supplied as an optional first
argument to autodying ``system'':

\begin{verbatim}
    system( [ 0, 1, 2 ], $cmd, @args);  # 0,1,2 are good exit values
\end{verbatim}

``autodie'' uses the IPC::System::Simple module to change ``system''.
See its documentation for further information.

Applying ``autodie'' to ``system'' or ``exec'' causes the exotic forms
``system \{ \$cmd \} @args'' or ``exec \{ \$cmd \} @args'' to be
considered a syntax error until the end of the lexical scope. If you
really need to use the exotic form, you can call ``CORE::system'' or
``CORE::exec'' instead, or use ``no autodie qw(system exec)'' before
calling the exotic form.

\hyperdef{}{GOTCHAS}{\section{\hyperref[GOTCHAS]{GOTCHAS}}\label{GOTCHAS}}

Functions called in list context are assumed to have failed if they
return an empty list, or a list consisting only of a single undef
element.

\hyperdef{}{DIAGNOSTICS}{\section{\hyperref[DIAGNOSTICS]{DIAGNOSTICS}}\label{DIAGNOSTICS}}

\begin{description}
\itemsep1pt\parskip0pt\parsep0pt
\item[:void cannot be used with lexical scope]
The ``:void'' option is supported in Fatal, but not ``autodie''. To
workaround this, ``autodie'' may be explicitly disabled until the end of
the current block with ``no autodie''. To disable autodie for only a
single function (eg, open) use ``no autodie qw(open)''.

~

``autodie'' performs no checking of called context to determine whether
to throw an exception; the explicitness of error handling with
``autodie'' is a deliberate feature.
\end{description}

\begin{description}
\itemsep1pt\parskip0pt\parsep0pt
\item[No user hints defined for \%s]
You've insisted on hints for user-subroutines, either by pre-pending a
``!'' to the subroutine name itself, or earlier in the list of arguments
to ``autodie''. However the subroutine in question does not have any
hints available.
\end{description}

See also ``DIAGNOSTICS'' in Fatal.

\hyperdef{}{BUGS}{\section{\hyperref[BUGS]{BUGS}}\label{BUGS}}

``Used only once'' warnings can be generated when ``autodie'' or
``Fatal'' is used with package filehandles (eg, ``FILE''). Scalar
filehandles are strongly recommended instead.

When using ``autodie'' or ``Fatal'' with user subroutines, the
declaration of those subroutines must appear before the first use of
``Fatal'' or ``autodie'', or have been exported from a module.
Attempting to use ``Fatal'' or ``autodie'' on other user subroutines
will result in a compile-time error.

Due to a bug in Perl, ``autodie'' may ``lose'' any format which has the
same name as an autodying built-in or function.

``autodie'' may not work correctly if used inside a file with a name
that looks like a string eval, such as \emph{eval (3)}.

\hyperdef{}{autodieux5fandux5fstringux5feval}{\subsection{\hyperref[autodieux5fandux5fstringux5feval]{autodie
and string eval}}\label{autodieux5fandux5fstringux5feval}}

Due to the current implementation of ``autodie'', unexpected results may
be seen when used near or with the string version of eval. \emph{None of
these bugs exist when using block eval}.

Under Perl 5.8 only, ``autodie'' \emph{does not} propagate into string
``eval'' statements, although it can be explicitly enabled inside a
string ``eval''.

Under Perl 5.10 only, using a string eval when ``autodie'' is in effect
can cause the autodie behaviour to leak into the surrounding scope. This
can be worked around by using a ``no autodie'' at the end of the scope
to explicitly remove autodie's effects, or by avoiding the use of string
eval.

\emph{None of these bugs exist when using block eval}. The use of
``autodie'' with block eval is considered good practice.

\hyperdef{}{REPORTINGux5fBUGS}{\subsection{\hyperref[REPORTINGux5fBUGS]{REPORTING
BUGS}}\label{REPORTINGux5fBUGS}}

Please report bugs via the CPAN Request Tracker at
\textless{}http://rt.cpan.org/NoAuth/Bugs.html?Dist=autodie\textgreater{}.

\hyperdef{}{FEEDBACK}{\section{\hyperref[FEEDBACK]{FEEDBACK}}\label{FEEDBACK}}

If you find this module useful, please consider rating it on the CPAN
Ratings service at
\textless{}http://cpanratings.perl.org/rate?distribution=autodie\textgreater{}
.

The module author loves to hear how ``autodie'' has made your life
better (or worse). Feedback can be sent to
\textless{}pjf@perltraining.com.au\textgreater{}.

\hyperdef{}{AUTHOR}{\section{\hyperref[AUTHOR]{AUTHOR}}\label{AUTHOR}}

Copyright 2008-2009, Paul Fenwick
\textless{}pjf@perltraining.com.au\textgreater{}

\hyperdef{}{LICENSE}{\section{\hyperref[LICENSE]{LICENSE}}\label{LICENSE}}

This module is free software. You may distribute it under the same terms
as Perl itself.

\hyperdef{}{SEEux5fALSO}{\section{\hyperref[SEEux5fALSO]{SEE
ALSO}}\label{SEEux5fALSO}}

Fatal, autodie::exception, autodie::hints, IPC::System::Simple

\emph{Perl tips, autodie} at
\textless{}http://perltraining.com.au/tips/2008-08-20.html\textgreater{}

\hyperdef{}{ACKNOWLEDGEMENTS}{\section{\hyperref[ACKNOWLEDGEMENTS]{ACKNOWLEDGEMENTS}}\label{ACKNOWLEDGEMENTS}}

Mark Reed and Roland Giersig -- Klingon translators.

See the \emph{AUTHORS} file for full credits. The latest version of this
file can be found at
\textless{}https://github.com/pjf/autodie/tree/master/AUTHORS\textgreater{}
.

\begin{longtable}[c]{@{}ll@{}}
\toprule\addlinespace
2014-01-27 & perl v5.18.2
\\\addlinespace
\bottomrule
\end{longtable}

\end{document}
