\documentclass[]{article}
\usepackage[T1]{fontenc}
\usepackage{lmodern}
\usepackage{amssymb,amsmath}
\usepackage{ifxetex,ifluatex}
\usepackage{fixltx2e} % provides \textsubscript
% use upquote if available, for straight quotes in verbatim environments
\IfFileExists{upquote.sty}{\usepackage{upquote}}{}
\ifnum 0\ifxetex 1\fi\ifluatex 1\fi=0 % if pdftex
  \usepackage[utf8]{inputenc}
\else % if luatex or xelatex
  \ifxetex
    \usepackage{mathspec}
    \usepackage{xltxtra,xunicode}
  \else
    \usepackage{fontspec}
  \fi
  \defaultfontfeatures{Mapping=tex-text,Scale=MatchLowercase}
  \newcommand{\euro}{€}
\fi
% use microtype if available
\IfFileExists{microtype.sty}{\usepackage{microtype}}{}
\usepackage{longtable,booktabs}
\ifxetex
  \usepackage[setpagesize=false, % page size defined by xetex
              unicode=false, % unicode breaks when used with xetex
              xetex]{hyperref}
\else
  \usepackage[unicode=true]{hyperref}
\fi
\hypersetup{breaklinks=true,
            bookmarks=true,
            pdfauthor={},
            pdftitle={AptPkg::Source(3pm)},
            colorlinks=true,
            citecolor=blue,
            urlcolor=blue,
            linkcolor=magenta,
            pdfborder={0 0 0}}
\urlstyle{same}  % don't use monospace font for urls
\setlength{\parindent}{0pt}
\setlength{\parskip}{6pt plus 2pt minus 1pt}
\setlength{\emergencystretch}{3em}  % prevent overfull lines
\setcounter{secnumdepth}{0}
\usepackage{pagecolor}

% Set background colour (of the page)
\definecolor{weirdbgcolor}{HTML}{FCF4F0}
\pagecolor{weirdbgcolor}

% Make bold text appear in a particular colour
\definecolor{boldcolor}{HTML}{6E0002}
\let\realtextbf=\textbf
\renewcommand{\textbf}[1]{\textcolor{boldcolor}{\realtextbf{#1}}}

% Use underlines instead of emphasis (ugh)
\renewcommand{\emph}[1]{\underline{#1}}

% % Use fixed-width font by default
% \renewcommand*\familydefault{\ttdefault}

\title{AptPkg::Source(3pm)}
\author{}
\date{}

\begin{document}
\maketitle

\begin{longtable}[c]{@{}lll@{}}
\toprule\addlinespace
AptPkg::Source(3pm) & User Contributed Perl Documentation &
AptPkg::Source(3pm)
\\\addlinespace
\bottomrule
\end{longtable}

\hyperdef{}{NAME}{\section{\hyperref[NAME]{NAME}}\label{NAME}}

AptPkg::Source - APT source package interface

\hyperdef{}{SYNOPSIS}{\section{\hyperref[SYNOPSIS]{SYNOPSIS}}\label{SYNOPSIS}}

use AptPkg::Source;

\hyperdef{}{DESCRIPTION}{\section{\hyperref[DESCRIPTION]{DESCRIPTION}}\label{DESCRIPTION}}

The AptPkg::Source module provides an interface to \textbf{APT}'s source
package lists.

\hyperdef{}{AptPkg::Source}{\subsection{\hyperref[AptPkg::Source]{AptPkg::Source}}\label{AptPkg::Source}}

The AptPkg::Source package implements the \textbf{APT} pkgSrcRecords
class as a hash reference (inherits from AptPkg::hash). The hash is
keyed on source or binary package name and the value is an array
reference of the details of matching source packages.

Note that there is no iterator class, so it is not possible to get a
list of all keys (with keys or each).

\emph{Constructor}

\begin{description}
\itemsep1pt\parskip0pt\parsep0pt
\item[new({[}\emph{SOURCELIST}{]})]
Instantiation of the object uses configuration from the
\$AptPkg::Config::\_config object (automatically initialised if not done
explicitly).

~

If no \emph{SOURCELIST} is specified, then the value of
Dir::Etc::sourcelist from the configuration object is used (generally
/etc/apt/sources.list).
\end{description}

\emph{Methods}

\begin{description}
\itemsep1pt\parskip0pt\parsep0pt
\item[find(\emph{PACK}, {[}\emph{SRCONLY}{]})]
In a list context, return a list of source package details for the given
\emph{PACK}, which may either be a source package name, or the name of
one of the binaries provided (unless \emph{SRCONLY} is provided and
true).

~

In a scalar context, the source package name of the first entry is
returned.
\end{description}

\begin{description}
\itemsep1pt\parskip0pt\parsep0pt
\item[get, exists]
These methods are used to implement the hashref abstraction:
\$obj-\textgreater{}get(\$pack) and \$obj-\textgreater{}\{\$pack\} are
equivalent.

~

The get method has the same semantics as find, but returns an array
reference in a scalar context.
\end{description}

The list returned by the find (and get) methods consists of hashes which
describe each available source package (in order of discovery from the
deb-src files described in sources.list).

Each hash contains the following entries:

\begin{description}
\item[Package]
\end{description}

\begin{description}
\item[Version]
\end{description}

\begin{description}
\item[Maintainer]
\end{description}

\begin{description}
\itemsep1pt\parskip0pt\parsep0pt
\item[Section]
Strings giving the source package name, version, maintainer and section.
\end{description}

\begin{description}
\itemsep1pt\parskip0pt\parsep0pt
\item[Binaries]
A list of binary package names from the package.
\end{description}

\begin{description}
\itemsep1pt\parskip0pt\parsep0pt
\item[BuildDepends]
A hash describing the build dependencies of the package. Possible keys
are:
\end{description}

~

``Build-Depends'', ``Build-Depends-Indep'', ``Build-Conflicts'',
``Build-Conflicts-Indep''.

~

The values are a list of dependencies/conflicts with each item being a
list containing the package name followed optionally by an operator and
version number.

~

Operator values evaluate to a comparison string* (\textgreater{},
\textgreater{}=, etc) or one of the AptPkg::Dep:: constants in a numeric
context (see ``pkgCache::Dep::DepCompareOp'' in \emph{AptPkg}(3pm)).

~

*Note that this is a normalised, rather than Debian-style
(\textgreater{}\textgreater{} vs \textgreater{}) string.

\begin{description}
\itemsep1pt\parskip0pt\parsep0pt
\item[Files]
A list of files making up the source package, each described by a hash
containing the keys:
\end{description}

~

``MD5Hash'', ``Size'', ``ArchiveURI'', ``Type''.

\hyperdef{}{SEEux5fALSO}{\section{\hyperref[SEEux5fALSO]{SEE
ALSO}}\label{SEEux5fALSO}}

\emph{AptPkg::Config}(3pm), \emph{AptPkg::Cache}(3pm),
\emph{AptPkg}(3pm), \emph{AptPkg::hash}(3pm).

\hyperdef{}{AUTHOR}{\section{\hyperref[AUTHOR]{AUTHOR}}\label{AUTHOR}}

Brendan O'Dea \textless{}bod@debian.org\textgreater{}

\begin{longtable}[c]{@{}ll@{}}
\toprule\addlinespace
2013-05-10 & perl v5.18.1
\\\addlinespace
\bottomrule
\end{longtable}

\end{document}
