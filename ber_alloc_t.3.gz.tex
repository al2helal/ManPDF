\documentclass[]{article}
\usepackage[T1]{fontenc}
\usepackage{lmodern}
\usepackage{amssymb,amsmath}
\usepackage{ifxetex,ifluatex}
\usepackage{fixltx2e} % provides \textsubscript
% use upquote if available, for straight quotes in verbatim environments
\IfFileExists{upquote.sty}{\usepackage{upquote}}{}
\ifnum 0\ifxetex 1\fi\ifluatex 1\fi=0 % if pdftex
  \usepackage[utf8]{inputenc}
\else % if luatex or xelatex
  \ifxetex
    \usepackage{mathspec}
    \usepackage{xltxtra,xunicode}
  \else
    \usepackage{fontspec}
  \fi
  \defaultfontfeatures{Mapping=tex-text,Scale=MatchLowercase}
  \newcommand{\euro}{€}
\fi
% use microtype if available
\IfFileExists{microtype.sty}{\usepackage{microtype}}{}
\usepackage{longtable,booktabs}
\ifxetex
  \usepackage[setpagesize=false, % page size defined by xetex
              unicode=false, % unicode breaks when used with xetex
              xetex]{hyperref}
\else
  \usepackage[unicode=true]{hyperref}
\fi
\hypersetup{breaklinks=true,
            bookmarks=true,
            pdfauthor={},
            pdftitle={LBER\_ENCODE(3)},
            colorlinks=true,
            citecolor=blue,
            urlcolor=blue,
            linkcolor=magenta,
            pdfborder={0 0 0}}
\urlstyle{same}  % don't use monospace font for urls
\setlength{\parindent}{0pt}
\setlength{\parskip}{6pt plus 2pt minus 1pt}
\setlength{\emergencystretch}{3em}  % prevent overfull lines
\setcounter{secnumdepth}{0}
\usepackage{pagecolor}

% Set background colour (of the page)
\definecolor{weirdbgcolor}{HTML}{FCF4F0}
\pagecolor{weirdbgcolor}

% Make bold text appear in a particular colour
\definecolor{boldcolor}{HTML}{6E0002}
\let\realtextbf=\textbf
\renewcommand{\textbf}[1]{\textcolor{boldcolor}{\realtextbf{#1}}}

% Use underlines instead of emphasis (ugh)
\renewcommand{\emph}[1]{\underline{#1}}

% % Use fixed-width font by default
% \renewcommand*\familydefault{\ttdefault}

\title{LBER\_ENCODE(3)}
\author{}
\date{}

\begin{document}
\maketitle

\begin{longtable}[c]{@{}lll@{}}
\toprule\addlinespace
LBER\_ENCODE(3) & Library Functions Manual & LBER\_ENCODE(3)
\\\addlinespace
\bottomrule
\end{longtable}

\hyperdef{}{NAME}{\section{\hyperref[NAME]{NAME}}\label{NAME}}

ber\_alloc\_t, ber\_flush, ber\_flush2, ber\_printf, ber\_put\_int,
ber\_put\_enum, ber\_put\_ostring, ber\_put\_string, ber\_put\_null,
ber\_put\_boolean, ber\_put\_bitstring, ber\_start\_seq,
ber\_start\_set, ber\_put\_seq, ber\_put\_set - OpenLDAP LBER simplified
Basic Encoding Rules library routines for encoding

\hyperdef{}{LIBRARY}{\section{\hyperref[LIBRARY]{LIBRARY}}\label{LIBRARY}}

OpenLDAP LBER (liblber, -llber)

\hyperdef{}{SYNOPSIS}{\section{\hyperref[SYNOPSIS]{SYNOPSIS}}\label{SYNOPSIS}}

\textbf{\#include \textless{}lber.h\textgreater{}}

\textbf{BerElement *ber\_alloc\_t(int}\emph{options}\textbf{);}

\textbf{int ber\_flush(Sockbuf *}\emph{sb}\textbf{, BerElement
*}\emph{ber}\textbf{, int}\emph{freeit}\textbf{);}

\textbf{int ber\_flush2(Sockbuf *}\emph{sb}\textbf{, BerElement
*}\emph{ber}\textbf{, int}\emph{freeit}\textbf{);}

\textbf{int ber\_printf(BerElement *}\emph{ber}\textbf{, const char
*}\emph{fmt}\textbf{, \ldots{});}

\textbf{int ber\_put\_int(BerElement *}\emph{ber}\textbf{,
ber\_int\_t}\emph{num}\textbf{, ber\_tag\_t}\emph{tag}\textbf{);}

\textbf{int ber\_put\_enum(BerElement *}\emph{ber}\textbf{,
ber\_int\_t}\emph{num}\textbf{, ber\_tag\_t}\emph{tag}\textbf{);}

\textbf{int ber\_put\_ostring(BerElement *}\emph{ber}\textbf{, const
char *}\emph{str}\textbf{, ber\_len\_t}\emph{len}\textbf{,
ber\_tag\_t}\emph{tag}\textbf{);}

\textbf{int ber\_put\_string(BerElement *}\emph{ber}\textbf{, const char
*}\emph{str}\textbf{, ber\_tag\_t}\emph{tag}\textbf{);}

\textbf{int ber\_put\_null(BerElement *}\emph{ber}\textbf{,
ber\_tag\_t}\emph{tag}\textbf{);}

\textbf{int ber\_put\_boolean(BerElement *}\emph{ber}\textbf{,
ber\_int\_t}\emph{bool}\textbf{, ber\_tag\_t}\emph{tag}\textbf{);}

\textbf{int ber\_put\_bitstring(BerElement *}\emph{ber}\textbf{, const
char *}\emph{str}\textbf{, ber\_len\_t}\emph{blen}\textbf{,
ber\_tag\_t}\emph{tag}\textbf{);}

\textbf{int ber\_start\_seq(BerElement *}\emph{ber}\textbf{,
ber\_tag\_t}\emph{tag}\textbf{);}

\textbf{int ber\_start\_set(BerElement *}\emph{ber}\textbf{,
ber\_tag\_t}\emph{tag}\textbf{);}

\textbf{int ber\_put\_seq(BerElement *}\emph{ber}\textbf{);}

\textbf{int ber\_put\_set(BerElement *}\emph{ber}\textbf{);}

\hyperdef{}{DESCRIPTION}{\section{\hyperref[DESCRIPTION]{DESCRIPTION}}\label{DESCRIPTION}}

These routines provide a subroutine interface to a simplified
implementation of the Basic Encoding Rules of ASN.1. The version of BER
these routines support is the one defined for the LDAP protocol. The
encoding rules are the same as BER, except that only definite form
lengths are used, and bitstrings and octet strings are always encoded in
primitive form. This man page describes the encoding routines in the
lber library. See \textbf{lber-decode}(3) for details on the
corresponding decoding routines. Consult \textbf{lber-types}(3) for
information about types, allocators, and deallocators.

Normally, the only routines that need to be called by an application are
\textbf{ber\_alloc\_t}() to allocate a BER element for encoding,
\textbf{ber\_printf}() to do the actual encoding, and
\textbf{ber\_flush2}() to actually write the element. The other routines
are provided for those applications that need more control than
\textbf{ber\_printf}() provides. In general, these routines return the
length of the element encoded, or -1 if an error occurred.

The \textbf{ber\_alloc\_t}() routine is used to allocate a new BER
element. It should be called with an argument of LBER\_USE\_DER.

The \textbf{ber\_flush2}() routine is used to actually write the element
to a socket (or file) descriptor, once it has been fully encoded (using
\textbf{ber\_printf}() and friends). See \textbf{lber-sockbuf}(3) for
more details on the Sockbuf implementation of the \emph{sb} parameter.
If the \emph{freeit} parameter is non-zero, the supplied \emph{ber} will
be freed. If \emph{LBER\_FLUSH\_FREE\_ON\_SUCCESS} is used, the
\emph{ber} is only freed when successfully flushed, otherwise it is left
intact; if \emph{LBER\_FLUSH\_FREE\_ON\_ERROR} is used, the \emph{ber}
is only freed when an error occurs, otherwise it is left intact; if
\emph{LBER\_FLUSH\_FREE\_ALWAYS} is used, the \emph{ber} is freed
anyway. This function differs from the original \textbf{ber\_flush}(3)
function, whose behavior corresponds to that indicated for
\emph{LBER\_FLUSH\_FREE\_ON\_SUCCESS}. Note that in the future, the
behavior of \textbf{ber\_flush}(3) with \emph{freeit} non-zero might
change into that of \textbf{ber\_flush2}(3) with \emph{freeit} set to
\emph{LBER\_FLUSH\_FREE\_ALWAYS}.

The \textbf{ber\_printf}() routine is used to encode a BER element in
much the same way that \textbf{sprintf}(3) works. One important
difference, though, is that some state information is kept with the
\emph{ber} parameter so that multiple calls can be made to
\textbf{ber\_printf}() to append things to the end of the BER element.
\textbf{Ber\_printf}() writes to \emph{ber}, a pointer to a BerElement
such as returned by \textbf{ber\_alloc\_t}(). It interprets and formats
its arguments according to the format string \emph{fmt}. The format
string can contain the following characters:

\begin{description}
\itemsep1pt\parskip0pt\parsep0pt
\item[\textbf{b}]
Boolean. An ber\_int\_t parameter should be supplied. A boolean element
is output.
\end{description}

\begin{description}
\itemsep1pt\parskip0pt\parsep0pt
\item[\textbf{e}]
Enumeration. An ber\_int\_t parameter should be supplied. An enumeration
element is output.
\end{description}

\begin{description}
\itemsep1pt\parskip0pt\parsep0pt
\item[\textbf{i}]
Integer. An ber\_int\_t parameter should be supplied. An integer element
is output.
\end{description}

\begin{description}
\itemsep1pt\parskip0pt\parsep0pt
\item[\textbf{B}]
Bitstring. A char * pointer to the start of the bitstring is supplied,
followed by the number of bits in the bitstring. A bitstring element is
output.
\end{description}

\begin{description}
\itemsep1pt\parskip0pt\parsep0pt
\item[\textbf{n}]
Null. No parameter is required. A null element is output.
\end{description}

\begin{description}
\itemsep1pt\parskip0pt\parsep0pt
\item[\textbf{o}]
Octet string. A char * is supplied, followed by the length of the string
pointed to. An octet string element is output.
\end{description}

\begin{description}
\itemsep1pt\parskip0pt\parsep0pt
\item[\textbf{O}]
Octet string. A struct berval * is supplied. An octet string element is
output.
\end{description}

\begin{description}
\itemsep1pt\parskip0pt\parsep0pt
\item[\textbf{s}]
Octet string. A null-terminated string is supplied. An octet string
element is output, not including the trailing NULL octet.
\end{description}

\begin{description}
\itemsep1pt\parskip0pt\parsep0pt
\item[\textbf{t}]
Tag. A ber\_tag\_t specifying the tag to give the next element is
provided. This works across calls.
\end{description}

\begin{description}
\itemsep1pt\parskip0pt\parsep0pt
\item[\textbf{v}]
Several octet strings. A null-terminated array of char *'s is supplied.
Note that a construct like `\{v\}' is required to get an actual SEQUENCE
OF octet strings.
\end{description}

\begin{description}
\itemsep1pt\parskip0pt\parsep0pt
\item[\textbf{V}]
Several octet strings. A null-terminated array of struct berval *'s is
supplied. Note that a construct like `\{V\}' is required to get an
actual SEQUENCE OF octet strings.
\end{description}

\begin{description}
\itemsep1pt\parskip0pt\parsep0pt
\item[\textbf{W}]
Several octet strings. An array of struct berval's is supplied. The
array is terminated by a struct berval with a NULL bv\_val. Note that a
construct like `\{W\}' is required to get an actual SEQUENCE OF octet
strings.
\end{description}

\begin{description}
\itemsep1pt\parskip0pt\parsep0pt
\item[\textbf{\{}]
Begin sequence. No parameter is required.
\end{description}

\begin{description}
\itemsep1pt\parskip0pt\parsep0pt
\item[\textbf{\}}]
End sequence. No parameter is required.
\end{description}

\begin{description}
\itemsep1pt\parskip0pt\parsep0pt
\item[\textbf{{[}}]
Begin set. No parameter is required.
\end{description}

\begin{description}
\itemsep1pt\parskip0pt\parsep0pt
\item[\textbf{{]}}]
End set. No parameter is required.
\end{description}

The \textbf{ber\_put\_int}() routine writes the integer element
\emph{num} to the BER element \emph{ber}.

The \textbf{ber\_put\_enum}() routine writes the enumeration element
\emph{num} to the BER element \emph{ber}.

The \textbf{ber\_put\_boolean}() routine writes the boolean value given
by \emph{bool} to the BER element.

The \textbf{ber\_put\_bitstring}() routine writes \emph{blen} bits
starting at \emph{str} as a bitstring value to the given BER element.
Note that \emph{blen} is the length \emph{in bits} of the bitstring.

The \textbf{ber\_put\_ostring}() routine writes \emph{len} bytes
starting at \emph{str} to the BER element as an octet string.

The \textbf{ber\_put\_string}() routine writes the null-terminated
string (minus the terminating ` ') to the BER element as an octet
string.

The \textbf{ber\_put\_null}() routine writes a NULL element to the BER
element.

The \textbf{ber\_start\_seq}() routine is used to start a sequence in
the BER element. The \textbf{ber\_start\_set}() routine works similarly.
The end of the sequence or set is marked by the nearest matching call to
\textbf{ber\_put\_seq}() or \textbf{ber\_put\_set}(), respectively.

\hyperdef{}{EXAMPLES}{\section{\hyperref[EXAMPLES]{EXAMPLES}}\label{EXAMPLES}}

Assuming the following variable declarations, and that the variables
have been assigned appropriately, an lber encoding of the following
ASN.1 object:

\begin{verbatim}
      AlmostASearchRequest := SEQUENCE {
          baseObject      DistinguishedName,
          scope           ENUMERATED {
              baseObject    (0),
              singleLevel   (1),
              wholeSubtree  (2)
          },
          derefAliases    ENUMERATED {
              neverDerefaliases   (0),
              derefInSearching    (1),
              derefFindingBaseObj (2),
              alwaysDerefAliases  (3)
          },
          sizelimit       INTEGER (0 .. 65535),
          timelimit       INTEGER (0 .. 65535),
          attrsOnly       BOOLEAN,
          attributes      SEQUENCE OF AttributeType
      }
\end{verbatim}

can be achieved like so:

\begin{verbatim}
      int rc;
      ber_int_t    scope, ali, size, time, attrsonly;
      char   *dn, **attrs;
      BerElement *ber;

      /* ... fill in values ... */

      ber = ber_alloc_t( LBER_USE_DER );

      if ( ber == NULL ) {
              /* error */
      }

      rc = ber_printf( ber, "{siiiib{v}}", dn, scope, ali,
          size, time, attrsonly, attrs );

      if( rc == -1 ) {
              /* error */
      } else {
              /* success */
      }
\end{verbatim}

\hyperdef{}{ERRORS}{\section{\hyperref[ERRORS]{ERRORS}}\label{ERRORS}}

If an error occurs during encoding, generally these routines return -1.

\hyperdef{}{NOTES}{\section{\hyperref[NOTES]{NOTES}}\label{NOTES}}

The return values for all of these functions are declared in the
\textless{}lber.h\textgreater{} header file.

\hyperdef{}{SEEux5fALSO}{\section{\hyperref[SEEux5fALSO]{SEE
ALSO}}\label{SEEux5fALSO}}

\textbf{lber-decode}(3), \textbf{lber-memory}(3),
\textbf{lber-sockbuf}(3), \textbf{lber-types}(3)

\hyperdef{}{ACKNOWLEDGEMENTS}{\section{\hyperref[ACKNOWLEDGEMENTS]{ACKNOWLEDGEMENTS}}\label{ACKNOWLEDGEMENTS}}

\textbf{OpenLDAP Software} is developed and maintained by The OpenLDAP
Project \textless{}http://www.openldap.org/\textgreater{}.
\textbf{OpenLDAP Software} is derived from University of Michigan LDAP
3.3 Release.

\begin{longtable}[c]{@{}ll@{}}
\toprule\addlinespace
2012/04/23 & OpenLDAP
\\\addlinespace
\bottomrule
\end{longtable}

\end{document}
