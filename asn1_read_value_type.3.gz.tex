\documentclass[]{article}
\usepackage[T1]{fontenc}
\usepackage{lmodern}
\usepackage{amssymb,amsmath}
\usepackage{ifxetex,ifluatex}
\usepackage{fixltx2e} % provides \textsubscript
% use upquote if available, for straight quotes in verbatim environments
\IfFileExists{upquote.sty}{\usepackage{upquote}}{}
\ifnum 0\ifxetex 1\fi\ifluatex 1\fi=0 % if pdftex
  \usepackage[utf8]{inputenc}
\else % if luatex or xelatex
  \ifxetex
    \usepackage{mathspec}
    \usepackage{xltxtra,xunicode}
  \else
    \usepackage{fontspec}
  \fi
  \defaultfontfeatures{Mapping=tex-text,Scale=MatchLowercase}
  \newcommand{\euro}{€}
\fi
% use microtype if available
\IfFileExists{microtype.sty}{\usepackage{microtype}}{}
\usepackage{longtable,booktabs}
\ifxetex
  \usepackage[setpagesize=false, % page size defined by xetex
              unicode=false, % unicode breaks when used with xetex
              xetex]{hyperref}
\else
  \usepackage[unicode=true]{hyperref}
\fi
\hypersetup{breaklinks=true,
            bookmarks=true,
            pdfauthor={},
            pdftitle={asn1\_read\_value\_type(3)},
            colorlinks=true,
            citecolor=blue,
            urlcolor=blue,
            linkcolor=magenta,
            pdfborder={0 0 0}}
\urlstyle{same}  % don't use monospace font for urls
\setlength{\parindent}{0pt}
\setlength{\parskip}{6pt plus 2pt minus 1pt}
\setlength{\emergencystretch}{3em}  % prevent overfull lines
\setcounter{secnumdepth}{0}
\usepackage{pagecolor}

% Set background colour (of the page)
\definecolor{weirdbgcolor}{HTML}{FCF4F0}
\pagecolor{weirdbgcolor}

% Make bold text appear in a particular colour
\definecolor{boldcolor}{HTML}{6E0002}
\let\realtextbf=\textbf
\renewcommand{\textbf}[1]{\textcolor{boldcolor}{\realtextbf{#1}}}

% Use underlines instead of emphasis (ugh)
\renewcommand{\emph}[1]{\underline{#1}}

% % Use fixed-width font by default
% \renewcommand*\familydefault{\ttdefault}

\title{asn1\_read\_value\_type(3)}
\author{}
\date{}

\begin{document}
\maketitle

\begin{longtable}[c]{@{}lll@{}}
\toprule\addlinespace
asn1\_read\_value\_type(3) & libtasn1 & asn1\_read\_value\_type(3)
\\\addlinespace
\bottomrule
\end{longtable}

\hyperdef{}{NAME}{\section{\hyperref[NAME]{NAME}}\label{NAME}}

asn1\_read\_value\_type - API function

\hyperdef{}{SYNOPSIS}{\section{\hyperref[SYNOPSIS]{SYNOPSIS}}\label{SYNOPSIS}}

\textbf{\#include \textless{}libtasn1.h\textgreater{}}

~

\textbf{int asn1\_read\_value\_type(asn1\_node}\emph{root}\textbf{,
const char *}\emph{name}\textbf{, void *}\emph{ivalue}\textbf{, int
*}\emph{len}\textbf{, unsigned int *}\emph{etype}\textbf{);}

\hyperdef{}{ARGUMENTS}{\section{\hyperref[ARGUMENTS]{ARGUMENTS}}\label{ARGUMENTS}}

\begin{description}
\itemsep1pt\parskip0pt\parsep0pt
\item[asn1\_node root]
pointer to a structure.
\end{description}

\begin{description}
\itemsep1pt\parskip0pt\parsep0pt
\item[const char * name]
the name of the element inside a structure that you want to read.
\end{description}

\begin{description}
\itemsep1pt\parskip0pt\parsep0pt
\item[void * ivalue]
vector that will contain the element's content, must be a pointer to
memory cells already allocated (may be \textbf{NULL}).
\end{description}

\begin{description}
\itemsep1pt\parskip0pt\parsep0pt
\item[int * len]
number of bytes of *value: value{[}0{]}..value{[}len-1{]}. Initialy
holds the sizeof value.
\end{description}

\begin{description}
\itemsep1pt\parskip0pt\parsep0pt
\item[unsigned int * etype]
The type of the value read (ASN1\_ETYPE)
\end{description}

\hyperdef{}{DESCRIPTION}{\section{\hyperref[DESCRIPTION]{DESCRIPTION}}\label{DESCRIPTION}}

Returns the value of one element inside a structure. If an element is
OPTIONAL and this returns \textbf{ASN1\_ELEMENT\_NOT\_FOUND}, it means
that this element wasn't present in the der encoding that created the
structure. The first element of a SEQUENCE\_OF or SET\_OF is named
``?1''. The second one ``?2'' and so on.

~

Note that there can be valid values with length zero. In these case this
function will succeed and \emph{len} will be zero.

\hyperdef{}{INTEGER}{\section{\hyperref[INTEGER]{INTEGER}}\label{INTEGER}}

VALUE will contain a two's complement form integer.

~

integer=-1 -\textgreater{} value{[}0{]}=0xFF , len=1. integer=1
-\textgreater{} value{[}0{]}=0x01 , len=1.

\hyperdef{}{ENUMERATED}{\section{\hyperref[ENUMERATED]{ENUMERATED}}\label{ENUMERATED}}

As INTEGER (but only with not negative numbers).

\hyperdef{}{BOOLEAN}{\section{\hyperref[BOOLEAN]{BOOLEAN}}\label{BOOLEAN}}

VALUE will be the null terminated string ``TRUE'' or ``FALSE'' and LEN=5
or LEN=6.

~

OBJECT IDENTIFIER: VALUE will be a null terminated string with each
number separated by a dot (i.e. ``1.2.3.543.1'').

~

LEN = strlen(VALUE)+1

\hyperdef{}{UTCTIME}{\section{\hyperref[UTCTIME]{UTCTIME}}\label{UTCTIME}}

VALUE will be a null terminated string in one of these formats:
``YYMMDDhhmmss+hh'mm''' or ``YYMMDDhhmmss-hh'mm'''. LEN=strlen(VALUE)+1.

\hyperdef{}{GENERALIZEDTIME}{\section{\hyperref[GENERALIZEDTIME]{GENERALIZEDTIME}}\label{GENERALIZEDTIME}}

VALUE will be a null terminated string in the same format used to set
the value.

~

OCTET STRING: VALUE will contain the octet string and LEN will be the
number of octets.

\hyperdef{}{GENERALSTRING}{\section{\hyperref[GENERALSTRING]{GENERALSTRING}}\label{GENERALSTRING}}

VALUE will contain the generalstring and LEN will be the number of
octets.

~

BIT STRING: VALUE will contain the bit string organized by bytes and LEN
will be the number of bits.

\hyperdef{}{CHOICE}{\section{\hyperref[CHOICE]{CHOICE}}\label{CHOICE}}

If NAME indicates a choice type, VALUE will specify the alternative
selected.

\hyperdef{}{ANY}{\section{\hyperref[ANY]{ANY}}\label{ANY}}

If NAME indicates an any type, VALUE will indicate the DER encoding of
the structure actually used.

\hyperdef{}{RETURNS}{\section{\hyperref[RETURNS]{RETURNS}}\label{RETURNS}}

\textbf{ASN1\_SUCCESS} if value is returned,
\textbf{ASN1\_ELEMENT\_NOT\_FOUND} if \emph{name} is not a valid
element, \textbf{ASN1\_VALUE\_NOT\_FOUND} if there isn't any value for
the element selected, and \textbf{ASN1\_MEM\_ERROR} if The value vector
isn't big enough to store the result, and in this case \emph{len} will
contain the number of bytes needed.

\hyperdef{}{COPYRIGHT}{\section{\hyperref[COPYRIGHT]{COPYRIGHT}}\label{COPYRIGHT}}

Copyright © 2006-2013 Free Software Foundation, Inc..

~

Copying and distribution of this file, with or without modification, are
permitted in any medium without royalty provided the copyright notice
and this notice are preserved.

\hyperdef{}{SEEux5fALSO}{\section{\hyperref[SEEux5fALSO]{SEE
ALSO}}\label{SEEux5fALSO}}

The full documentation for \textbf{libtasn1} is maintained as a Texinfo
manual. If the \textbf{info} and \textbf{libtasn1} programs are properly
installed at your site, the command

\begin{description}
\itemsep1pt\parskip0pt\parsep0pt
\item[]
\textbf{info libtasn1}
\end{description}

should give you access to the complete manual. As an alternative you may
obtain the manual from:

\begin{description}
\itemsep1pt\parskip0pt\parsep0pt
\item[]
\textbf{http://www.gnu.org/software/libtasn1/manual/}
\end{description}

\begin{longtable}[c]{@{}ll@{}}
\toprule\addlinespace
3.4 & libtasn1
\\\addlinespace
\bottomrule
\end{longtable}

\end{document}
