\documentclass[]{article}
\usepackage[T1]{fontenc}
\usepackage{lmodern}
\usepackage{amssymb,amsmath}
\usepackage{ifxetex,ifluatex}
\usepackage{fixltx2e} % provides \textsubscript
% use upquote if available, for straight quotes in verbatim environments
\IfFileExists{upquote.sty}{\usepackage{upquote}}{}
\ifnum 0\ifxetex 1\fi\ifluatex 1\fi=0 % if pdftex
  \usepackage[utf8]{inputenc}
\else % if luatex or xelatex
  \ifxetex
    \usepackage{mathspec}
    \usepackage{xltxtra,xunicode}
  \else
    \usepackage{fontspec}
  \fi
  \defaultfontfeatures{Mapping=tex-text,Scale=MatchLowercase}
  \newcommand{\euro}{€}
\fi
% use microtype if available
\IfFileExists{microtype.sty}{\usepackage{microtype}}{}
\usepackage{longtable,booktabs}
\ifxetex
  \usepackage[setpagesize=false, % page size defined by xetex
              unicode=false, % unicode breaks when used with xetex
              xetex]{hyperref}
\else
  \usepackage[unicode=true]{hyperref}
\fi
\hypersetup{breaklinks=true,
            bookmarks=true,
            pdfauthor={},
            pdftitle={ATOI(3)},
            colorlinks=true,
            citecolor=blue,
            urlcolor=blue,
            linkcolor=magenta,
            pdfborder={0 0 0}}
\urlstyle{same}  % don't use monospace font for urls
\setlength{\parindent}{0pt}
\setlength{\parskip}{6pt plus 2pt minus 1pt}
\setlength{\emergencystretch}{3em}  % prevent overfull lines
\setcounter{secnumdepth}{0}
\usepackage{pagecolor}

% Set background colour (of the page)
\definecolor{weirdbgcolor}{HTML}{FCF4F0}
\pagecolor{weirdbgcolor}

% Make bold text appear in a particular colour
\definecolor{boldcolor}{HTML}{6E0002}
\let\realtextbf=\textbf
\renewcommand{\textbf}[1]{\textcolor{boldcolor}{\realtextbf{#1}}}

% Use underlines instead of emphasis (ugh)
\renewcommand{\emph}[1]{\underline{#1}}

% % Use fixed-width font by default
% \renewcommand*\familydefault{\ttdefault}

\title{ATOI(3)}
\author{}
\date{}

\begin{document}
\maketitle

\begin{longtable}[c]{@{}lll@{}}
\toprule\addlinespace
ATOI(3) & Linux Programmer's Manual & ATOI(3)
\\\addlinespace
\bottomrule
\end{longtable}

\hyperdef{}{NAME}{\section{\hyperref[NAME]{NAME}}\label{NAME}}

atoi, atol, atoll, atoq - convert a string to an integer

\hyperdef{}{SYNOPSIS}{\section{\hyperref[SYNOPSIS]{SYNOPSIS}}\label{SYNOPSIS}}

\begin{verbatim}
#include <stdlib.h>
 
int atoi(const char *nptr);
 
long atol(const char *nptr);
 
long long atoll(const char *nptr);
 
long long atoq(const char *nptr);
\end{verbatim}

~

Feature Test Macro Requirements for glibc (see
\textbf{feature\_test\_macros}(7)): \\

~

\textbf{atoll}():

\_BSD\_SOURCE \textbar{}\textbar{} \_SVID\_SOURCE \textbar{}\textbar{}
\_XOPEN\_SOURCE~\textgreater{}=~600 \textbar{}\textbar{}
\_ISOC99\_SOURCE \textbar{}\textbar{}
\_POSIX\_C\_SOURCE~\textgreater{}=~200112L;

~

or \emph{cc~-std=c99}

\hyperdef{}{DESCRIPTION}{\section{\hyperref[DESCRIPTION]{DESCRIPTION}}\label{DESCRIPTION}}

The \textbf{atoi}() function converts the initial portion of the string
pointed to by \emph{nptr} to \emph{int}. The behavior is the same as

~

strtol(nptr, NULL, 10); \\

~

except that \textbf{atoi}() does not detect errors.

The \textbf{atol}() and \textbf{atoll}() functions behave the same as
\textbf{atoi}(), except that they convert the initial portion of the
string to their return type of \emph{long} or \emph{long long}.
\textbf{atoq}() is an obsolete name for \textbf{atoll}().

\hyperdef{}{RETURNux5fVALUE}{\section{\hyperref[RETURNux5fVALUE]{RETURN
VALUE}}\label{RETURNux5fVALUE}}

The converted value.

\hyperdef{}{CONFORMINGux5fTO}{\section{\hyperref[CONFORMINGux5fTO]{CONFORMING
TO}}\label{CONFORMINGux5fTO}}

SVr4, POSIX.1-2001, 4.3BSD, C99. C89 and POSIX.1-1996 include the
functions \textbf{atoi}() and \textbf{atol}() only. \textbf{atoq}() is a
GNU extension.

\hyperdef{}{NOTES}{\section{\hyperref[NOTES]{NOTES}}\label{NOTES}}

The nonstandard \textbf{atoq}() function is not present in libc 4.6.27
or glibc 2, but is present in libc5 and libc 4.7 (though only as an
inline function in \emph{\textless{}stdlib.h\textgreater{}} until libc
5.4.44). The \textbf{atoll}() function is present in glibc 2 since
version 2.0.2, but not in libc4 or libc5.

\hyperdef{}{SEEux5fALSO}{\section{\hyperref[SEEux5fALSO]{SEE
ALSO}}\label{SEEux5fALSO}}

\textbf{atof}(3), \textbf{strtod}(3), \textbf{strtol}(3),
\textbf{strtoul}(3)

\hyperdef{}{COLOPHON}{\section{\hyperref[COLOPHON]{COLOPHON}}\label{COLOPHON}}

This page is part of release 3.54 of the Linux \emph{man-pages} project.
A description of the project, and information about reporting bugs, can
be found at http://www.kernel.org/doc/man-pages/.

\begin{longtable}[c]{@{}ll@{}}
\toprule\addlinespace
2012-08-03 & GNU
\\\addlinespace
\bottomrule
\end{longtable}

\end{document}
