\documentclass[]{article}
\usepackage[T1]{fontenc}
\usepackage{lmodern}
\usepackage{amssymb,amsmath}
\usepackage{ifxetex,ifluatex}
\usepackage{fixltx2e} % provides \textsubscript
% use upquote if available, for straight quotes in verbatim environments
\IfFileExists{upquote.sty}{\usepackage{upquote}}{}
\ifnum 0\ifxetex 1\fi\ifluatex 1\fi=0 % if pdftex
  \usepackage[utf8]{inputenc}
\else % if luatex or xelatex
  \ifxetex
    \usepackage{mathspec}
    \usepackage{xltxtra,xunicode}
  \else
    \usepackage{fontspec}
  \fi
  \defaultfontfeatures{Mapping=tex-text,Scale=MatchLowercase}
  \newcommand{\euro}{€}
\fi
% use microtype if available
\IfFileExists{microtype.sty}{\usepackage{microtype}}{}
\usepackage{longtable,booktabs}
\ifxetex
  \usepackage[setpagesize=false, % page size defined by xetex
              unicode=false, % unicode breaks when used with xetex
              xetex]{hyperref}
\else
  \usepackage[unicode=true]{hyperref}
\fi
\hypersetup{breaklinks=true,
            bookmarks=true,
            pdfauthor={},
            pdftitle={Archive::Extract(3pm)},
            colorlinks=true,
            citecolor=blue,
            urlcolor=blue,
            linkcolor=magenta,
            pdfborder={0 0 0}}
\urlstyle{same}  % don't use monospace font for urls
\setlength{\parindent}{0pt}
\setlength{\parskip}{6pt plus 2pt minus 1pt}
\setlength{\emergencystretch}{3em}  % prevent overfull lines
\setcounter{secnumdepth}{0}
\usepackage{pagecolor}

% Set background colour (of the page)
\definecolor{weirdbgcolor}{HTML}{FCF4F0}
\pagecolor{weirdbgcolor}

% Make bold text appear in a particular colour
\definecolor{boldcolor}{HTML}{6E0002}
\let\realtextbf=\textbf
\renewcommand{\textbf}[1]{\textcolor{boldcolor}{\realtextbf{#1}}}

% Use underlines instead of emphasis (ugh)
\renewcommand{\emph}[1]{\underline{#1}}

% % Use fixed-width font by default
% \renewcommand*\familydefault{\ttdefault}

\title{Archive::Extract(3pm)}
\author{}
\date{}

\begin{document}
\maketitle

\begin{longtable}[c]{@{}lll@{}}
\toprule\addlinespace
Archive::Extract(3pm) & User Contributed Perl Documentation &
Archive::Extract(3pm)
\\\addlinespace
\bottomrule
\end{longtable}

\hyperdef{}{NAME}{\section{\hyperref[NAME]{NAME}}\label{NAME}}

Archive::Extract - A generic archive extracting mechanism

\hyperdef{}{SYNOPSIS}{\section{\hyperref[SYNOPSIS]{SYNOPSIS}}\label{SYNOPSIS}}

\begin{verbatim}
    use Archive::Extract;
    ### build an Archive::Extract object ###
    my $ae = Archive::Extract->new( archive => 'foo.tgz' );
    ### extract to cwd() ###
    my $ok = $ae->extract;
    ### extract to /tmp ###
    my $ok = $ae->extract( to => '/tmp' );
    ### what if something went wrong?
    my $ok = $ae->extract or die $ae->error;
    ### files from the archive ###
    my $files   = $ae->files;
    ### dir that was extracted to ###
    my $outdir  = $ae->extract_path;
    ### quick check methods ###
    $ae->is_tar     # is it a .tar file?
    $ae->is_tgz     # is it a .tar.gz or .tgz file?
    $ae->is_gz;     # is it a .gz file?
    $ae->is_zip;    # is it a .zip file?
    $ae->is_bz2;    # is it a .bz2 file?
    $ae->is_tbz;    # is it a .tar.bz2 or .tbz file?
    $ae->is_lzma;   # is it a .lzma file?
    $ae->is_xz;     # is it a .xz file?
    $ae->is_txz;    # is it a .tar.xz or .txz file?
    ### absolute path to the archive you provided ###
    $ae->archive;
    ### commandline tools, if found ###
    $ae->bin_tar     # path to /bin/tar, if found
    $ae->bin_gzip    # path to /bin/gzip, if found
    $ae->bin_unzip   # path to /bin/unzip, if found
    $ae->bin_bunzip2 # path to /bin/bunzip2 if found
    $ae->bin_unlzma  # path to /bin/unlzma if found
    $ae->bin_unxz    # path to /bin/unxz if found
\end{verbatim}

\hyperdef{}{DESCRIPTION}{\section{\hyperref[DESCRIPTION]{DESCRIPTION}}\label{DESCRIPTION}}

Archive::Extract is a generic archive extraction mechanism.

It allows you to extract any archive file of the type .tar, .tar.gz,
.gz, .Z, tar.bz2, .tbz, .bz2, .zip, .xz,, .txz, .tar.xz or .lzma without
having to worry how it does so, or use different interfaces for each
type by using either perl modules, or commandline tools on your system.

See the ``HOW IT WORKS'' section further down for details.

\hyperdef{}{METHODS}{\section{\hyperref[METHODS]{METHODS}}\label{METHODS}}

\hyperdef{}{ux24aeux5f=ux5fArchive::Extract-ux3enewux28archiveux5f=ux3eux5fux27ux2fpathux2ftoux2farchiveux27ux2cux5btypeux5f=ux3eux5fTYPEux5dux29}{\subsection{\hyperref[ux5cux24aeux5f=ux5fArchive::Extract-ux5ctextgreaterux7bux7dnewux28archiveux5f=ux5ctextgreaterux7bux7dux5fux27ux2fpathux2ftoux2farchiveux27ux2cux7bux5bux7dtypeux5f=ux5ctextgreaterux7bux7dux5fTYPEux7bux5dux7dux29]{\$ae
= Archive::Extract-\textgreater{}new(archive =\textgreater{}
`/path/to/archive',{[}type =\textgreater{}
TYPE{]})}}\label{ux24aeux5f=ux5fArchive::Extract-ux3enewux28archiveux5f=ux3eux5fux27ux2fpathux2ftoux2farchiveux27ux2cux5btypeux5f=ux3eux5fTYPEux5dux29}}

Creates a new ``Archive::Extract'' object based on the archive file you
passed it. Automatically determines the type of archive based on the
extension, but you can override that by explicitly providing the
``type'' argument.

Valid values for ``type'' are:

\begin{description}
\itemsep1pt\parskip0pt\parsep0pt
\item[tar]
Standard tar files, as produced by, for example, ``/bin/tar''.
Corresponds to a ``.tar'' suffix.
\end{description}

\begin{description}
\itemsep1pt\parskip0pt\parsep0pt
\item[tgz]
Gzip compressed tar files, as produced by, for example ``/bin/tar -z''.
Corresponds to a ``.tgz'' or ``.tar.gz'' suffix.
\end{description}

\begin{description}
\itemsep1pt\parskip0pt\parsep0pt
\item[gz]
Gzip compressed file, as produced by, for example ``/bin/gzip''.
Corresponds to a ``.gz'' suffix.
\end{description}

\begin{description}
\itemsep1pt\parskip0pt\parsep0pt
\item[Z]
Lempel-Ziv compressed file, as produced by, for example
``/bin/compress''. Corresponds to a ``.Z'' suffix.
\end{description}

\begin{description}
\itemsep1pt\parskip0pt\parsep0pt
\item[zip]
Zip compressed file, as produced by, for example ``/bin/zip''.
Corresponds to a ``.zip'', ``.jar'' or ``.par'' suffix.
\end{description}

\begin{description}
\itemsep1pt\parskip0pt\parsep0pt
\item[bz2]
Bzip2 compressed file, as produced by, for example, ``/bin/bzip2''.
Corresponds to a ``.bz2'' suffix.
\end{description}

\begin{description}
\itemsep1pt\parskip0pt\parsep0pt
\item[tbz]
Bzip2 compressed tar file, as produced by, for example ``/bin/tar -j''.
Corresponds to a ``.tbz'' or ``.tar.bz2'' suffix.
\end{description}

\begin{description}
\itemsep1pt\parskip0pt\parsep0pt
\item[lzma]
Lzma compressed file, as produced by ``/bin/lzma''. Corresponds to a
``.lzma'' suffix.
\end{description}

\begin{description}
\itemsep1pt\parskip0pt\parsep0pt
\item[xz]
Xz compressed file, as produced by ``/bin/xz''. Corresponds to a ``.xz''
suffix.
\end{description}

\begin{description}
\itemsep1pt\parskip0pt\parsep0pt
\item[txz]
Xz compressed tar file, as produced by, for example ``/bin/tar -J''.
Corresponds to a ``.txz'' or ``.tar.xz'' suffix.
\end{description}

Returns a ``Archive::Extract'' object on success, or false on failure.

\hyperdef{}{ux24ae-ux3eextractux28ux5fux5btoux5f=ux3eux5fux27ux2foutputux2fpathux27ux5dux5fux29}{\subsection{\hyperref[ux5cux24ae-ux5ctextgreaterux7bux7dextractux28ux5fux7bux5bux7dtoux5f=ux5ctextgreaterux7bux7dux5fux27ux2foutputux2fpathux27ux7bux5dux7dux5fux29]{\$ae-\textgreater{}extract(
{[}to =\textgreater{} `/output/path'{]}
)}}\label{ux24ae-ux3eextractux28ux5fux5btoux5f=ux3eux5fux27ux2foutputux2fpathux27ux5dux5fux29}}

Extracts the archive represented by the ``Archive::Extract'' object to
the path of your choice as specified by the ``to'' argument. Defaults to
``cwd()''.

Since ``.gz'' files never hold a directory, but only a single file; if
the ``to'' argument is an existing directory, the file is extracted
there, with its ``.gz'' suffix stripped. If the ``to'' argument is not
an existing directory, the ``to'' argument is understood to be a
filename, if the archive type is ``gz''. In the case that you did not
specify a ``to'' argument, the output file will be the name of the
archive file, stripped from its ``.gz'' suffix, in the current working
directory.

``extract'' will try a pure perl solution first, and then fall back to
commandline tools if they are available. See the ``GLOBAL VARIABLES''
section below on how to alter this behaviour.

It will return true on success, and false on failure.

On success, it will also set the follow attributes in the object:

\begin{description}
\itemsep1pt\parskip0pt\parsep0pt
\item[\$ae-\textgreater{}extract\_path]
This is the directory that the files where extracted to.
\end{description}

\begin{description}
\item[\$ae-\textgreater{}files]
This is an array ref with the paths of all the files in the archive,
relative to the ``to'' argument you specified. To get the full path to
an extracted file, you would use:

~

\begin{verbatim}
    File::Spec->catfile( $to, $ae->files->[0] );
    
\end{verbatim}

~

Note that all files from a tar archive will be in unix format, as per
the tar specification.
\end{description}

\hyperdef{}{ACCESSORS}{\section{\hyperref[ACCESSORS]{ACCESSORS}}\label{ACCESSORS}}

\hyperdef{}{ux24ae-ux3eerrorux28ux5bBOOLux5dux29}{\subsection{\hyperref[ux5cux24ae-ux5ctextgreaterux7bux7derrorux28ux7bux5bux7dBOOLux7bux5dux7dux29]{\$ae-\textgreater{}error({[}BOOL{]})}}\label{ux24ae-ux3eerrorux28ux5bBOOLux5dux29}}

Returns the last encountered error as string. Pass it a true value to
get the ``Carp::longmess()'' output instead.

\hyperdef{}{ux24ae-ux3eextractux5fpath}{\subsection{\hyperref[ux5cux24ae-ux5ctextgreaterux7bux7dextractux5fpath]{\$ae-\textgreater{}extract\_path}}\label{ux24ae-ux3eextractux5fpath}}

This is the directory the archive got extracted to. See ``extract()''
for details.

\hyperdef{}{ux24ae-ux3efiles}{\subsection{\hyperref[ux5cux24ae-ux5ctextgreaterux7bux7dfiles]{\$ae-\textgreater{}files}}\label{ux24ae-ux3efiles}}

This is an array ref holding all the paths from the archive. See
``extract()'' for details.

\hyperdef{}{ux24ae-ux3earchive}{\subsection{\hyperref[ux5cux24ae-ux5ctextgreaterux7bux7darchive]{\$ae-\textgreater{}archive}}\label{ux24ae-ux3earchive}}

This is the full path to the archive file represented by this
``Archive::Extract'' object.

\hyperdef{}{ux24ae-ux3etype}{\subsection{\hyperref[ux5cux24ae-ux5ctextgreaterux7bux7dtype]{\$ae-\textgreater{}type}}\label{ux24ae-ux3etype}}

This is the type of archive represented by this ``Archive::Extract''
object. See accessors below for an easier way to use this. See the
``new()'' method for details.

\hyperdef{}{ux24ae-ux3etypes}{\subsection{\hyperref[ux5cux24ae-ux5ctextgreaterux7bux7dtypes]{\$ae-\textgreater{}types}}\label{ux24ae-ux3etypes}}

Returns a list of all known ``types'' for ``Archive::Extract'''s ``new''
method.

\hyperdef{}{ux24ae-ux3eisux5ftgz}{\subsection{\hyperref[ux5cux24ae-ux5ctextgreaterux7bux7disux5ftgz]{\$ae-\textgreater{}is\_tgz}}\label{ux24ae-ux3eisux5ftgz}}

Returns true if the file is of type ``.tar.gz''. See the ``new()''
method for details.

\hyperdef{}{ux24ae-ux3eisux5ftar}{\subsection{\hyperref[ux5cux24ae-ux5ctextgreaterux7bux7disux5ftar]{\$ae-\textgreater{}is\_tar}}\label{ux24ae-ux3eisux5ftar}}

Returns true if the file is of type ``.tar''. See the ``new()'' method
for details.

\hyperdef{}{ux24ae-ux3eisux5fgz}{\subsection{\hyperref[ux5cux24ae-ux5ctextgreaterux7bux7disux5fgz]{\$ae-\textgreater{}is\_gz}}\label{ux24ae-ux3eisux5fgz}}

Returns true if the file is of type ``.gz''. See the ``new()'' method
for details.

\hyperdef{}{ux24ae-ux3eisux5fZ}{\subsection{\hyperref[ux5cux24ae-ux5ctextgreaterux7bux7disux5fZ]{\$ae-\textgreater{}is\_Z}}\label{ux24ae-ux3eisux5fZ}}

Returns true if the file is of type ``.Z''. See the ``new()'' method for
details.

\hyperdef{}{ux24ae-ux3eisux5fzip}{\subsection{\hyperref[ux5cux24ae-ux5ctextgreaterux7bux7disux5fzip]{\$ae-\textgreater{}is\_zip}}\label{ux24ae-ux3eisux5fzip}}

Returns true if the file is of type ``.zip''. See the ``new()'' method
for details.

\hyperdef{}{ux24ae-ux3eisux5flzma}{\subsection{\hyperref[ux5cux24ae-ux5ctextgreaterux7bux7disux5flzma]{\$ae-\textgreater{}is\_lzma}}\label{ux24ae-ux3eisux5flzma}}

Returns true if the file is of type ``.lzma''. See the ``new()'' method
for details.

\hyperdef{}{ux24ae-ux3eisux5fxz}{\subsection{\hyperref[ux5cux24ae-ux5ctextgreaterux7bux7disux5fxz]{\$ae-\textgreater{}is\_xz}}\label{ux24ae-ux3eisux5fxz}}

Returns true if the file is of type ``.xz''. See the ``new()'' method
for details.

\hyperdef{}{ux24ae-ux3ebinux5ftar}{\subsection{\hyperref[ux5cux24ae-ux5ctextgreaterux7bux7dbinux5ftar]{\$ae-\textgreater{}bin\_tar}}\label{ux24ae-ux3ebinux5ftar}}

Returns the full path to your tar binary, if found.

\hyperdef{}{ux24ae-ux3ebinux5fgzip}{\subsection{\hyperref[ux5cux24ae-ux5ctextgreaterux7bux7dbinux5fgzip]{\$ae-\textgreater{}bin\_gzip}}\label{ux24ae-ux3ebinux5fgzip}}

Returns the full path to your gzip binary, if found

\hyperdef{}{ux24ae-ux3ebinux5funzip}{\subsection{\hyperref[ux5cux24ae-ux5ctextgreaterux7bux7dbinux5funzip]{\$ae-\textgreater{}bin\_unzip}}\label{ux24ae-ux3ebinux5funzip}}

Returns the full path to your unzip binary, if found

\hyperdef{}{ux24ae-ux3ebinux5funlzma}{\subsection{\hyperref[ux5cux24ae-ux5ctextgreaterux7bux7dbinux5funlzma]{\$ae-\textgreater{}bin\_unlzma}}\label{ux24ae-ux3ebinux5funlzma}}

Returns the full path to your unlzma binary, if found

\hyperdef{}{ux24ae-ux3ebinux5funxz}{\subsection{\hyperref[ux5cux24ae-ux5ctextgreaterux7bux7dbinux5funxz]{\$ae-\textgreater{}bin\_unxz}}\label{ux24ae-ux3ebinux5funxz}}

Returns the full path to your unxz binary, if found

\hyperdef{}{ux24boolux5f=ux5fux24ae-ux3ehaveux5foldux5fbunzip2}{\subsection{\hyperref[ux5cux24boolux5f=ux5fux5cux24ae-ux5ctextgreaterux7bux7dhaveux5foldux5fbunzip2]{\$bool
=
\$ae-\textgreater{}have\_old\_bunzip2}}\label{ux24boolux5f=ux5fux24ae-ux3ehaveux5foldux5fbunzip2}}

Older versions of ``/bin/bunzip2'', from before the ``bunzip2 1.0''
release, require all archive names to end in ``.bz2'' or it will not
extract them. This method checks if you have a recent version of
``bunzip2'' that allows any extension, or an older one that doesn't.

\hyperdef{}{debugux28ux5fMESSAGEux5fux29}{\subsection{\hyperref[debugux28ux5fMESSAGEux5fux29]{debug(
MESSAGE )}}\label{debugux28ux5fMESSAGEux5fux29}}

This method outputs MESSAGE to the default filehandle if \$DEBUG is
true. It's a small method, but it's here if you'd like to subclass it so
you can so something else with any debugging output.

\hyperdef{}{HOWux5fITux5fWORKS}{\section{\hyperref[HOWux5fITux5fWORKS]{HOW
IT WORKS}}\label{HOWux5fITux5fWORKS}}

``Archive::Extract'' tries first to determine what type of archive you
are passing it, by inspecting its suffix. It does not do this by using
Mime magic, or something related. See ``CAVEATS'' below.

Once it has determined the file type, it knows which extraction methods
it can use on the archive. It will try a perl solution first, then fall
back to a commandline tool if that fails. If that also fails, it will
return false, indicating it was unable to extract the archive. See the
section on ``GLOBAL VARIABLES'' to see how to alter this order.

\hyperdef{}{CAVEATS}{\section{\hyperref[CAVEATS]{CAVEATS}}\label{CAVEATS}}

\hyperdef{}{Fileux5fExtensions}{\subsection{\hyperref[Fileux5fExtensions]{File
Extensions}}\label{Fileux5fExtensions}}

``Archive::Extract'' trusts on the extension of the archive to determine
what type it is, and what extractor methods therefore can be used. If
your archives do not have any of the extensions as described in the
``new()'' method, you will have to specify the type explicitly, or
``Archive::Extract'' will not be able to extract the archive for you.

\hyperdef{}{Supportingux5fVeryux5fLargeux5fFiles}{\subsection{\hyperref[Supportingux5fVeryux5fLargeux5fFiles]{Supporting
Very Large Files}}\label{Supportingux5fVeryux5fLargeux5fFiles}}

``Archive::Extract'' can use either pure perl modules or command line
programs under the hood. Some of the pure perl modules (like
``Archive::Tar'' and Compress::unLZMA) take the entire contents of the
archive into memory, which may not be feasible on your system. Consider
setting the global variable \$Archive::Extract::PREFER\_BIN to 1, which
will prefer the use of command line programs and won't consume so much
memory.

See the ``GLOBAL VARIABLES'' section below for details.

\hyperdef{}{Bunzip2ux5fsupportux5fofux5farbitraryux5fextensions.}{\subsection{\hyperref[Bunzip2ux5fsupportux5fofux5farbitraryux5fextensions.]{Bunzip2
support of arbitrary
extensions.}}\label{Bunzip2ux5fsupportux5fofux5farbitraryux5fextensions.}}

Older versions of ``/bin/bunzip2'' do not support arbitrary file
extensions and insist on a ``.bz2'' suffix. Although we do our best to
guard against this, if you experience a bunzip2 error, it may be related
to this. For details, please see the ``have\_old\_bunzip2'' method.

\hyperdef{}{GLOBALux5fVARIABLES}{\section{\hyperref[GLOBALux5fVARIABLES]{GLOBAL
VARIABLES}}\label{GLOBALux5fVARIABLES}}

\hyperdef{}{ux24Archive::Extract::DEBUG}{\subsection{\hyperref[ux5cux24Archive::Extract::DEBUG]{\$Archive::Extract::DEBUG}}\label{ux24Archive::Extract::DEBUG}}

Set this variable to ``true'' to have all calls to command line tools be
printed out, including all their output. This also enables
``Carp::longmess'' errors, instead of the regular ``carp'' errors.

Good for tracking down why things don't work with your particular setup.

Defaults to ``false''.

\hyperdef{}{ux24Archive::Extract::WARN}{\subsection{\hyperref[ux5cux24Archive::Extract::WARN]{\$Archive::Extract::WARN}}\label{ux24Archive::Extract::WARN}}

This variable controls whether errors encountered internally by
``Archive::Extract'' should be ``carp'''d or not.

Set to false to silence warnings. Inspect the output of the ``error()''
method manually to see what went wrong.

Defaults to ``true''.

\hyperdef{}{ux24Archive::Extract::PREFERux5fBIN}{\subsection{\hyperref[ux5cux24Archive::Extract::PREFERux5fBIN]{\$Archive::Extract::PREFER\_BIN}}\label{ux24Archive::Extract::PREFERux5fBIN}}

This variables controls whether ``Archive::Extract'' should prefer the
use of perl modules, or commandline tools to extract archives.

Set to ``true'' to have ``Archive::Extract'' prefer commandline tools.

Defaults to ``false''.

\hyperdef{}{TODOux5fux2fux5fCAVEATS}{\section{\hyperref[TODOux5fux2fux5fCAVEATS]{TODO
/ CAVEATS}}\label{TODOux5fux2fux5fCAVEATS}}

\begin{description}
\itemsep1pt\parskip0pt\parsep0pt
\item[Mime magic support]
Maybe this module should use something like ``File::Type'' to determine
the type, rather than blindly trust the suffix.
\end{description}

\begin{description}
\itemsep1pt\parskip0pt\parsep0pt
\item[Thread safety]
Currently, ``Archive::Extract'' does a ``chdir'' to the extraction dir
before extraction, and a ``chdir'' back again after. This is not
necessarily thread safe. See ``rt.cpan.org'' bug ``\#45671'' for
details.
\end{description}

\hyperdef{}{BUGux5fREPORTS}{\section{\hyperref[BUGux5fREPORTS]{BUG
REPORTS}}\label{BUGux5fREPORTS}}

Please report bugs or other issues to
\textless{}bug-archive-extract@rt.cpan.org\textgreater{}.

\hyperdef{}{AUTHOR}{\section{\hyperref[AUTHOR]{AUTHOR}}\label{AUTHOR}}

This module by Jos Boumans \textless{}kane@cpan.org\textgreater{}.

\hyperdef{}{COPYRIGHT}{\section{\hyperref[COPYRIGHT]{COPYRIGHT}}\label{COPYRIGHT}}

This library is free software; you may redistribute and/or modify it
under the same terms as Perl itself.

\begin{longtable}[c]{@{}ll@{}}
\toprule\addlinespace
2013-11-09 & perl v5.18.1
\\\addlinespace
\bottomrule
\end{longtable}

\end{document}
