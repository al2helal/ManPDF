\documentclass[]{article}
\usepackage[T1]{fontenc}
\usepackage{lmodern}
\usepackage{amssymb,amsmath}
\usepackage{ifxetex,ifluatex}
\usepackage{fixltx2e} % provides \textsubscript
% use upquote if available, for straight quotes in verbatim environments
\IfFileExists{upquote.sty}{\usepackage{upquote}}{}
\ifnum 0\ifxetex 1\fi\ifluatex 1\fi=0 % if pdftex
  \usepackage[utf8]{inputenc}
\else % if luatex or xelatex
  \ifxetex
    \usepackage{mathspec}
    \usepackage{xltxtra,xunicode}
  \else
    \usepackage{fontspec}
  \fi
  \defaultfontfeatures{Mapping=tex-text,Scale=MatchLowercase}
  \newcommand{\euro}{€}
\fi
% use microtype if available
\IfFileExists{microtype.sty}{\usepackage{microtype}}{}
\usepackage{longtable,booktabs}
\ifxetex
  \usepackage[setpagesize=false, % page size defined by xetex
              unicode=false, % unicode breaks when used with xetex
              xetex]{hyperref}
\else
  \usepackage[unicode=true]{hyperref}
\fi
\hypersetup{breaklinks=true,
            bookmarks=true,
            pdfauthor={},
            pdftitle={asn1\_der\_coding(3)},
            colorlinks=true,
            citecolor=blue,
            urlcolor=blue,
            linkcolor=magenta,
            pdfborder={0 0 0}}
\urlstyle{same}  % don't use monospace font for urls
\setlength{\parindent}{0pt}
\setlength{\parskip}{6pt plus 2pt minus 1pt}
\setlength{\emergencystretch}{3em}  % prevent overfull lines
\setcounter{secnumdepth}{0}
\usepackage{pagecolor}

% Set background colour (of the page)
\definecolor{weirdbgcolor}{HTML}{FCF4F0}
\pagecolor{weirdbgcolor}

% Make bold text appear in a particular colour
\definecolor{boldcolor}{HTML}{6E0002}
\let\realtextbf=\textbf
\renewcommand{\textbf}[1]{\textcolor{boldcolor}{\realtextbf{#1}}}

% Use underlines instead of emphasis (ugh)
\renewcommand{\emph}[1]{\underline{#1}}

% % Use fixed-width font by default
% \renewcommand*\familydefault{\ttdefault}

\title{asn1\_der\_coding(3)}
\author{}
\date{}

\begin{document}
\maketitle

\begin{longtable}[c]{@{}lll@{}}
\toprule\addlinespace
asn1\_der\_coding(3) & libtasn1 & asn1\_der\_coding(3)
\\\addlinespace
\bottomrule
\end{longtable}

\hyperdef{}{NAME}{\section{\hyperref[NAME]{NAME}}\label{NAME}}

asn1\_der\_coding - API function

\hyperdef{}{SYNOPSIS}{\section{\hyperref[SYNOPSIS]{SYNOPSIS}}\label{SYNOPSIS}}

\textbf{\#include \textless{}libtasn1.h\textgreater{}}

~

\textbf{int asn1\_der\_coding(asn1\_node}\emph{element}\textbf{, const
char *}\emph{name}\textbf{, void *}\emph{ider}\textbf{, int
*}\emph{len}\textbf{, char *}\emph{ErrorDescription}\textbf{);}

\hyperdef{}{ARGUMENTS}{\section{\hyperref[ARGUMENTS]{ARGUMENTS}}\label{ARGUMENTS}}

\begin{description}
\itemsep1pt\parskip0pt\parsep0pt
\item[asn1\_node element]
pointer to an ASN1 element
\end{description}

\begin{description}
\itemsep1pt\parskip0pt\parsep0pt
\item[const char * name]
the name of the structure you want to encode (it must be inside
*POINTER).
\end{description}

\begin{description}
\itemsep1pt\parskip0pt\parsep0pt
\item[void * ider]
vector that will contain the DER encoding. DER must be a pointer to
memory cells already allocated.
\end{description}

\begin{description}
\itemsep1pt\parskip0pt\parsep0pt
\item[int * len]
number of bytes of * \emph{ider} : \emph{ider} {[}0{]}.. \emph{ider}
{[}len-1{]}, Initialy holds the sizeof of der vector.
\end{description}

\begin{description}
\itemsep1pt\parskip0pt\parsep0pt
\item[char * ErrorDescription]
return the error description or an empty string if success.
\end{description}

\hyperdef{}{DESCRIPTION}{\section{\hyperref[DESCRIPTION]{DESCRIPTION}}\label{DESCRIPTION}}

Creates the DER encoding for the NAME structure (inside *POINTER
structure).

\hyperdef{}{RETURNS}{\section{\hyperref[RETURNS]{RETURNS}}\label{RETURNS}}

\textbf{ASN1\_SUCCESS} if DER encoding OK,
\textbf{ASN1\_ELEMENT\_NOT\_FOUND} if \emph{name} is not a valid
element, \textbf{ASN1\_VALUE\_NOT\_FOUND} if there is an element without
a value, \textbf{ASN1\_MEM\_ERROR} if the \emph{ider} vector isn't big
enough and in this case \emph{len} will contain the length needed.

\hyperdef{}{COPYRIGHT}{\section{\hyperref[COPYRIGHT]{COPYRIGHT}}\label{COPYRIGHT}}

Copyright © 2006-2013 Free Software Foundation, Inc..

~

Copying and distribution of this file, with or without modification, are
permitted in any medium without royalty provided the copyright notice
and this notice are preserved.

\hyperdef{}{SEEux5fALSO}{\section{\hyperref[SEEux5fALSO]{SEE
ALSO}}\label{SEEux5fALSO}}

The full documentation for \textbf{libtasn1} is maintained as a Texinfo
manual. If the \textbf{info} and \textbf{libtasn1} programs are properly
installed at your site, the command

\begin{description}
\itemsep1pt\parskip0pt\parsep0pt
\item[]
\textbf{info libtasn1}
\end{description}

should give you access to the complete manual. As an alternative you may
obtain the manual from:

\begin{description}
\itemsep1pt\parskip0pt\parsep0pt
\item[]
\textbf{http://www.gnu.org/software/libtasn1/manual/}
\end{description}

\begin{longtable}[c]{@{}ll@{}}
\toprule\addlinespace
3.4 & libtasn1
\\\addlinespace
\bottomrule
\end{longtable}

\end{document}
