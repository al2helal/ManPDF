\documentclass[]{article}
\usepackage[T1]{fontenc}
\usepackage{lmodern}
\usepackage{amssymb,amsmath}
\usepackage{ifxetex,ifluatex}
\usepackage{fixltx2e} % provides \textsubscript
% use upquote if available, for straight quotes in verbatim environments
\IfFileExists{upquote.sty}{\usepackage{upquote}}{}
\ifnum 0\ifxetex 1\fi\ifluatex 1\fi=0 % if pdftex
  \usepackage[utf8]{inputenc}
\else % if luatex or xelatex
  \ifxetex
    \usepackage{mathspec}
    \usepackage{xltxtra,xunicode}
  \else
    \usepackage{fontspec}
  \fi
  \defaultfontfeatures{Mapping=tex-text,Scale=MatchLowercase}
  \newcommand{\euro}{€}
\fi
% use microtype if available
\IfFileExists{microtype.sty}{\usepackage{microtype}}{}
\usepackage{longtable,booktabs}
\ifxetex
  \usepackage[setpagesize=false, % page size defined by xetex
              unicode=false, % unicode breaks when used with xetex
              xetex]{hyperref}
\else
  \usepackage[unicode=true]{hyperref}
\fi
\hypersetup{breaklinks=true,
            bookmarks=true,
            pdfauthor={},
            pdftitle={BACKTRACE(3)},
            colorlinks=true,
            citecolor=blue,
            urlcolor=blue,
            linkcolor=magenta,
            pdfborder={0 0 0}}
\urlstyle{same}  % don't use monospace font for urls
\setlength{\parindent}{0pt}
\setlength{\parskip}{6pt plus 2pt minus 1pt}
\setlength{\emergencystretch}{3em}  % prevent overfull lines
\setcounter{secnumdepth}{0}
\usepackage{pagecolor}

% Set background colour (of the page)
\definecolor{weirdbgcolor}{HTML}{FCF4F0}
\pagecolor{weirdbgcolor}

% Make bold text appear in a particular colour
\definecolor{boldcolor}{HTML}{6E0002}
\let\realtextbf=\textbf
\renewcommand{\textbf}[1]{\textcolor{boldcolor}{\realtextbf{#1}}}

% Use underlines instead of emphasis (ugh)
\renewcommand{\emph}[1]{\underline{#1}}

% % Use fixed-width font by default
% \renewcommand*\familydefault{\ttdefault}

\title{BACKTRACE(3)}
\author{}
\date{}

\begin{document}
\maketitle

\begin{longtable}[c]{@{}lll@{}}
\toprule\addlinespace
BACKTRACE(3) & Linux Programmer's Manual & BACKTRACE(3)
\\\addlinespace
\bottomrule
\end{longtable}

\hyperdef{}{NAME}{\section{\hyperref[NAME]{NAME}}\label{NAME}}

backtrace, backtrace\_symbols, backtrace\_symbols\_fd - support for
application self-debugging

\hyperdef{}{SYNOPSIS}{\section{\hyperref[SYNOPSIS]{SYNOPSIS}}\label{SYNOPSIS}}

\textbf{\#include \textless{}execinfo.h\textgreater{}}

~

\textbf{int backtrace(void} \textbf{**}\emph{buffer}\textbf{,}
\textbf{int} \emph{size}\textbf{);}

~

\textbf{char **backtrace\_symbols(void *const}
\textbf{*}\emph{buffer}\textbf{,} \textbf{int} \emph{size}\textbf{);}

~

\textbf{void backtrace\_symbols\_fd(void *const}
\textbf{*}\emph{buffer}\textbf{,} \textbf{int} \emph{size}\textbf{,}
\textbf{int} \emph{fd}\textbf{);}

\hyperdef{}{DESCRIPTION}{\section{\hyperref[DESCRIPTION]{DESCRIPTION}}\label{DESCRIPTION}}

\textbf{backtrace}() returns a backtrace for the calling program, in the
array pointed to by \emph{buffer}. A backtrace is the series of
currently active function calls for the program. Each item in the array
pointed to by \emph{buffer} is of type \emph{void~*}, and is the return
address from the corresponding stack frame. The \emph{size} argument
specifies the maximum number of addresses that can be stored in
\emph{buffer}. If the backtrace is larger than \emph{size}, then the
addresses corresponding to the \emph{size} most recent function calls
are returned; to obtain the complete backtrace, make sure that
\emph{buffer} and \emph{size} are large enough.

~

Given the set of addresses returned by \textbf{backtrace}() in
\emph{buffer}, \textbf{backtrace\_symbols}() translates the addresses
into an array of strings that describe the addresses symbolically. The
\emph{size} argument specifies the number of addresses in \emph{buffer}.
The symbolic representation of each address consists of the function
name (if this can be determined), a hexadecimal offset into the
function, and the actual return address (in hexadecimal). The address of
the array of string pointers is returned as the function result of
\textbf{backtrace\_symbols}(). This array is \textbf{malloc}(3)ed by
\textbf{backtrace\_symbols}(), and must be freed by the caller. (The
strings pointed to by the array of pointers need not and should not be
freed.)

~

\textbf{backtrace\_symbols\_fd}() takes the same \emph{buffer} and
\emph{size} arguments as \textbf{backtrace\_symbols}(), but instead of
returning an array of strings to the caller, it writes the strings, one
per line, to the file descriptor \emph{fd}.
\textbf{backtrace\_symbols\_fd}() does not call \textbf{malloc}(3), and
so can be employed in situations where the latter function might fail.

\hyperdef{}{RETURNux5fVALUE}{\section{\hyperref[RETURNux5fVALUE]{RETURN
VALUE}}\label{RETURNux5fVALUE}}

\textbf{backtrace}() returns the number of addresses returned in
\emph{buffer}, which is not greater than \emph{size}. If the return
value is less than \emph{size}, then the full backtrace was stored; if
it is equal to \emph{size}, then it may have been truncated, in which
case the addresses of the oldest stack frames are not returned.

~

On success, \textbf{backtrace\_symbols}() returns a pointer to the array
\textbf{malloc}(3)ed by the call; on error, NULL is returned.

\hyperdef{}{VERSIONS}{\section{\hyperref[VERSIONS]{VERSIONS}}\label{VERSIONS}}

\textbf{backtrace}(), \textbf{backtrace\_symbols}(), and
\textbf{backtrace\_symbols\_fd}() are provided in glibc since version
2.1.

\hyperdef{}{CONFORMINGux5fTO}{\section{\hyperref[CONFORMINGux5fTO]{CONFORMING
TO}}\label{CONFORMINGux5fTO}}

These functions are GNU extensions.

\hyperdef{}{NOTES}{\section{\hyperref[NOTES]{NOTES}}\label{NOTES}}

These functions make some assumptions about how a function's return
address is stored on the stack. Note the following:

\begin{description}
\itemsep1pt\parskip0pt\parsep0pt
\item[*]
Omission of the frame pointers (as implied by any of \textbf{gcc}(1)'s
nonzero optimization levels) may cause these assumptions to be violated.
\end{description}

\begin{description}
\itemsep1pt\parskip0pt\parsep0pt
\item[*]
Inlined functions do not have stack frames.
\end{description}

\begin{description}
\itemsep1pt\parskip0pt\parsep0pt
\item[*]
Tail-call optimization causes one stack frame to replace another.
\end{description}

The symbol names may be unavailable without the use of special linker
options. For systems using the GNU linker, it is necessary to use the
\emph{-rdynamic} linker option. Note that names of ``static'' functions
are not exposed, and won't be available in the backtrace.

\hyperdef{}{EXAMPLE}{\section{\hyperref[EXAMPLE]{EXAMPLE}}\label{EXAMPLE}}

The program below demonstrates the use of \textbf{backtrace}() and
\textbf{backtrace\_symbols}(). The following shell session shows what we
might see when running the program:

\begin{verbatim}


$ cc -rdynamic prog.c -o prog
$ ./prog 3
backtrace() returned 8 addresses
./prog(myfunc3+0x5c) [0x80487f0]
./prog [0x8048871]
./prog(myfunc+0x21) [0x8048894]
./prog(myfunc+0x1a) [0x804888d]
./prog(myfunc+0x1a) [0x804888d]
./prog(main+0x65) [0x80488fb]
/lib/libc.so.6(__libc_start_main+0xdc) [0xb7e38f9c]
./prog [0x8048711]
\end{verbatim}

\hyperdef{}{Programux5fsource}{\subsection{\hyperref[Programux5fsource]{Program
source}}\label{Programux5fsource}}

\begin{verbatim}
#include <execinfo.h>
#include <stdio.h>
#include <stdlib.h>
#include <unistd.h>

void
myfunc3(void)
{
    int j, nptrs;
#define SIZE 100
    void *buffer[100];
    char **strings;

    nptrs = backtrace(buffer, SIZE);
    printf("backtrace() returned %d addresses\n", nptrs);

    /* The call backtrace_symbols_fd(buffer, nptrs, STDOUT_FILENO)
       would produce similar output to the following: */

    strings = backtrace_symbols(buffer, nptrs);
    if (strings == NULL) {
        perror("backtrace_symbols");
        exit(EXIT_FAILURE);
    }

    for (j = 0; j < nptrs; j++)
        printf("%s\n", strings[j]);

    free(strings);
}

static void   /* "static" means don't export the symbol... */
myfunc2(void)
{
    myfunc3();
}

void
myfunc(int ncalls)
{
    if (ncalls > 1)
        myfunc(ncalls - 1);
    else
        myfunc2();
}

int
main(int argc, char *argv[])
{
    if (argc != 2) {
        fprintf(stderr, "%s num-calls\n", argv[0]);
        exit(EXIT_FAILURE);
    }

    myfunc(atoi(argv[1]));
    exit(EXIT_SUCCESS);
}
\end{verbatim}

\hyperdef{}{SEEux5fALSO}{\section{\hyperref[SEEux5fALSO]{SEE
ALSO}}\label{SEEux5fALSO}}

\textbf{gcc}(1), \textbf{ld}(1), \textbf{dlopen}(3), \textbf{malloc}(3)

\hyperdef{}{COLOPHON}{\section{\hyperref[COLOPHON]{COLOPHON}}\label{COLOPHON}}

This page is part of release 3.54 of the Linux \emph{man-pages} project.
A description of the project, and information about reporting bugs, can
be found at http://www.kernel.org/doc/man-pages/.

\begin{longtable}[c]{@{}ll@{}}
\toprule\addlinespace
2008-06-14 & GNU
\\\addlinespace
\bottomrule
\end{longtable}

\end{document}
