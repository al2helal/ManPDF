\documentclass[]{article}
\usepackage[T1]{fontenc}
\usepackage{lmodern}
\usepackage{amssymb,amsmath}
\usepackage{ifxetex,ifluatex}
\usepackage{fixltx2e} % provides \textsubscript
% use upquote if available, for straight quotes in verbatim environments
\IfFileExists{upquote.sty}{\usepackage{upquote}}{}
\ifnum 0\ifxetex 1\fi\ifluatex 1\fi=0 % if pdftex
  \usepackage[utf8]{inputenc}
\else % if luatex or xelatex
  \ifxetex
    \usepackage{mathspec}
    \usepackage{xltxtra,xunicode}
  \else
    \usepackage{fontspec}
  \fi
  \defaultfontfeatures{Mapping=tex-text,Scale=MatchLowercase}
  \newcommand{\euro}{€}
\fi
% use microtype if available
\IfFileExists{microtype.sty}{\usepackage{microtype}}{}
\usepackage{longtable,booktabs}
\ifxetex
  \usepackage[setpagesize=false, % page size defined by xetex
              unicode=false, % unicode breaks when used with xetex
              xetex]{hyperref}
\else
  \usepackage[unicode=true]{hyperref}
\fi
\hypersetup{breaklinks=true,
            bookmarks=true,
            pdfauthor={},
            pdftitle={ABS(3)},
            colorlinks=true,
            citecolor=blue,
            urlcolor=blue,
            linkcolor=magenta,
            pdfborder={0 0 0}}
\urlstyle{same}  % don't use monospace font for urls
\setlength{\parindent}{0pt}
\setlength{\parskip}{6pt plus 2pt minus 1pt}
\setlength{\emergencystretch}{3em}  % prevent overfull lines
\setcounter{secnumdepth}{0}
\usepackage{pagecolor}

% Set background colour (of the page)
\definecolor{weirdbgcolor}{HTML}{FCF4F0}
\pagecolor{weirdbgcolor}

% Make bold text appear in a particular colour
\definecolor{boldcolor}{HTML}{6E0002}
\let\realtextbf=\textbf
\renewcommand{\textbf}[1]{\textcolor{boldcolor}{\realtextbf{#1}}}

% Use underlines instead of emphasis (ugh)
\renewcommand{\emph}[1]{\underline{#1}}

% % Use fixed-width font by default
% \renewcommand*\familydefault{\ttdefault}

\title{ABS(3)}
\author{}
\date{}

\begin{document}
\maketitle

\begin{longtable}[c]{@{}lll@{}}
\toprule\addlinespace
ABS(3) & Linux Programmer's Manual & ABS(3)
\\\addlinespace
\bottomrule
\end{longtable}

\hyperdef{}{NAME}{\section{\hyperref[NAME]{NAME}}\label{NAME}}

abs, labs, llabs, imaxabs - compute the absolute value of an integer

\hyperdef{}{SYNOPSIS}{\section{\hyperref[SYNOPSIS]{SYNOPSIS}}\label{SYNOPSIS}}

\begin{verbatim}
#include <stdlib.h>
 
int abs(int j);
 
long int labs(long int j);
 
long long int llabs(long long int j);
 
#include <inttypes.h>
 
intmax_t imaxabs(intmax_t j);
\end{verbatim}

~

Feature Test Macro Requirements for glibc (see
\textbf{feature\_test\_macros}(7)): \\

~

\textbf{llabs}():

\_XOPEN\_SOURCE~\textgreater{}=~600 \textbar{}\textbar{}
\_ISOC99\_SOURCE \textbar{}\textbar{}
\_POSIX\_C\_SOURCE~\textgreater{}=~200112L;

~

or \emph{cc~-std=c99}

\hyperdef{}{DESCRIPTION}{\section{\hyperref[DESCRIPTION]{DESCRIPTION}}\label{DESCRIPTION}}

The \textbf{abs}() function computes the absolute value of the integer
argument \emph{j}. The \textbf{labs}(), \textbf{llabs}() and
\textbf{imaxabs}() functions compute the absolute value of the argument
\emph{j} of the appropriate integer type for the function.

\hyperdef{}{RETURNux5fVALUE}{\section{\hyperref[RETURNux5fVALUE]{RETURN
VALUE}}\label{RETURNux5fVALUE}}

Returns the absolute value of the integer argument, of the appropriate
integer type for the function.

\hyperdef{}{ATTRIBUTES}{\section{\hyperref[ATTRIBUTES]{ATTRIBUTES}}\label{ATTRIBUTES}}

\hyperdef{}{Multithreadingux5fux28seeux5fpthreadsux287ux29ux29}{\subsection{\hyperref[Multithreadingux5fux28seeux5fpthreadsux287ux29ux29]{Multithreading
(see
pthreads(7))}}\label{Multithreadingux5fux28seeux5fpthreadsux287ux29ux29}}

The \textbf{abs}(), \textbf{labs}(), \textbf{llabs}(), and
\textbf{imaxabs}() functions are thread-safe.

\hyperdef{}{CONFORMINGux5fTO}{\section{\hyperref[CONFORMINGux5fTO]{CONFORMING
TO}}\label{CONFORMINGux5fTO}}

SVr4, POSIX.1-2001, 4.3BSD, C99. C89 only includes the \textbf{abs}()
and \textbf{labs}() functions; the functions \textbf{llabs}() and
\textbf{imaxabs}() were added in C99.

\hyperdef{}{NOTES}{\section{\hyperref[NOTES]{NOTES}}\label{NOTES}}

Trying to take the absolute value of the most negative integer is not
defined.

The \textbf{llabs}() function is included in glibc since version 2.0,
but is not in libc5 or libc4. The \textbf{imaxabs}() function is
included in glibc since version 2.1.1.

For \textbf{llabs}() to be declared, it may be necessary to define
\textbf{\_ISOC99\_SOURCE} or \textbf{\_ISOC9X\_SOURCE} (depending on the
version of glibc) before including any standard headers.

GCC handles \textbf{abs}() and \textbf{labs}() as built-in functions.
GCC 3.0 also handles \textbf{llabs}() and \textbf{imaxabs}() as
built-ins.

\hyperdef{}{SEEux5fALSO}{\section{\hyperref[SEEux5fALSO]{SEE
ALSO}}\label{SEEux5fALSO}}

\textbf{cabs}(3), \textbf{ceil}(3), \textbf{fabs}(3), \textbf{floor}(3),
\textbf{rint}(3)

\hyperdef{}{COLOPHON}{\section{\hyperref[COLOPHON]{COLOPHON}}\label{COLOPHON}}

This page is part of release 3.54 of the Linux \emph{man-pages} project.
A description of the project, and information about reporting bugs, can
be found at http://www.kernel.org/doc/man-pages/.

\begin{longtable}[c]{@{}ll@{}}
\toprule\addlinespace
2013-06-21 & GNU
\\\addlinespace
\bottomrule
\end{longtable}

\end{document}
