\documentclass[]{article}
\usepackage[T1]{fontenc}
\usepackage{lmodern}
\usepackage{amssymb,amsmath}
\usepackage{ifxetex,ifluatex}
\usepackage{fixltx2e} % provides \textsubscript
% use upquote if available, for straight quotes in verbatim environments
\IfFileExists{upquote.sty}{\usepackage{upquote}}{}
\ifnum 0\ifxetex 1\fi\ifluatex 1\fi=0 % if pdftex
  \usepackage[utf8]{inputenc}
\else % if luatex or xelatex
  \ifxetex
    \usepackage{mathspec}
    \usepackage{xltxtra,xunicode}
  \else
    \usepackage{fontspec}
  \fi
  \defaultfontfeatures{Mapping=tex-text,Scale=MatchLowercase}
  \newcommand{\euro}{€}
\fi
% use microtype if available
\IfFileExists{microtype.sty}{\usepackage{microtype}}{}
\usepackage{longtable,booktabs}
\ifxetex
  \usepackage[setpagesize=false, % page size defined by xetex
              unicode=false, % unicode breaks when used with xetex
              xetex]{hyperref}
\else
  \usepackage[unicode=true]{hyperref}
\fi
\hypersetup{breaklinks=true,
            bookmarks=true,
            pdfauthor={},
            pdftitle={Authen::SASL::Perl(3pm)},
            colorlinks=true,
            citecolor=blue,
            urlcolor=blue,
            linkcolor=magenta,
            pdfborder={0 0 0}}
\urlstyle{same}  % don't use monospace font for urls
\setlength{\parindent}{0pt}
\setlength{\parskip}{6pt plus 2pt minus 1pt}
\setlength{\emergencystretch}{3em}  % prevent overfull lines
\setcounter{secnumdepth}{0}
\usepackage{pagecolor}

% Set background colour (of the page)
\definecolor{weirdbgcolor}{HTML}{FCF4F0}
\pagecolor{weirdbgcolor}

% Make bold text appear in a particular colour
\definecolor{boldcolor}{HTML}{6E0002}
\let\realtextbf=\textbf
\renewcommand{\textbf}[1]{\textcolor{boldcolor}{\realtextbf{#1}}}

% Use underlines instead of emphasis (ugh)
\renewcommand{\emph}[1]{\underline{#1}}

% % Use fixed-width font by default
% \renewcommand*\familydefault{\ttdefault}

\title{Authen::SASL::Perl(3pm)}
\author{}
\date{}

\begin{document}
\maketitle

\begin{longtable}[c]{@{}lll@{}}
\toprule\addlinespace
Authen::SASL::Perl(3pm) & User Contributed Perl Documentation &
Authen::SASL::Perl(3pm)
\\\addlinespace
\bottomrule
\end{longtable}

\hyperdef{}{NAME}{\section{\hyperref[NAME]{NAME}}\label{NAME}}

Authen::SASL::Perl -- Perl implementation of the SASL Authentication
framework

\hyperdef{}{SYNOPSIS}{\section{\hyperref[SYNOPSIS]{SYNOPSIS}}\label{SYNOPSIS}}

\begin{verbatim}
 use Authen::SASL qw(Perl);
 $sasl = Authen::SASL->new(
   mechanism => 'CRAM-MD5 PLAIN ANONYMOUS',
   callback => {
     user => $user,
     pass => \&fetch_password
   }
 );
\end{verbatim}

\hyperdef{}{DESCRIPTION}{\section{\hyperref[DESCRIPTION]{DESCRIPTION}}\label{DESCRIPTION}}

\textbf{Authen::SASL::Perl} is the pure Perl implementation of SASL
mechanisms in the \textbf{Authen::SASL} framework.

At the time of this writing it provides the client part implementation
for the following SASL mechanisms:

\begin{description}
\itemsep1pt\parskip0pt\parsep0pt
\item[ANONYMOUS]
The Anonymous SASL Mechanism as defined in RFC 2245 resp. in IETF Draft
draft-ietf-sasl-anon-03.txt from February 2004 provides a method to
anonymously access internet services.

~

Since it does no authentication it does not need to send any
confidential information such as passwords in plain text over the
network.
\end{description}

\begin{description}
\itemsep1pt\parskip0pt\parsep0pt
\item[CRAM-MD5]
The CRAM-MD5 SASL Mechanism as defined in RFC2195 resp. in IETF Draft
draft-ietf-sasl-crammd5-XX.txt offers a simple challenge-response
authentication mechanism.

~

Since it is a challenge-response authentication mechanism no passwords
are transferred in clear-text over the wire.

~

Due to the simplicity of the protocol CRAM-MD5 is susceptible to replay
and dictionary attacks, so DIGEST-MD5 should be used in preferrence.
\end{description}

\begin{description}
\itemsep1pt\parskip0pt\parsep0pt
\item[DIGEST-MD5]
The DIGEST-MD5 SASL Mechanism as defined in RFC 2831 resp. in IETF Draft
draft-ietf-sasl-rfc2831bis-XX.txt offers the HTTP Digest Access
Authentication as SASL mechanism.

~

Like CRAM-MD5 it is a challenge-response authentication method that does
not send plain text passwords over the network.

~

Compared to CRAM-MD5, DIGEST-MD5 prevents chosen plaintext attacks, and
permits the use of third party authentication servers, so that it is
recommended to use DIGEST-MD5 instead of CRAM-MD5 when possible.
\end{description}

\begin{description}
\itemsep1pt\parskip0pt\parsep0pt
\item[EXTERNAL]
The EXTERNAL SASL mechanism as defined in RFC 2222 allows the use of
external authentication systems as SASL mechanisms.
\end{description}

\begin{description}
\itemsep1pt\parskip0pt\parsep0pt
\item[GSSAPI]
The GSSAPI SASL mechanism as defined in RFC 2222 resp. IETF Draft
draft-ietf-sasl-gssapi-XX.txt allows using the Generic Security Service
Application Program Interface {[}GSSAPI{]} KERBEROS V5 as as SASL
mechanism.

~

Although GSSAPI is a general mechanism for authentication it is almost
exlusively used for Kerberos 5.
\end{description}

\begin{description}
\itemsep1pt\parskip0pt\parsep0pt
\item[LOGIN]
The LOGIN SASL Mechanism as defined in IETF Draft
draft-murchison-sasl-login-XX.txt allows the combination of username and
clear-text password to be used in a SASL mechanism.

~

It does does not provide a security layer and sends the credentials in
clear over the wire. Thus this mechanism should not be used without
adequate security protection.
\end{description}

\begin{description}
\itemsep1pt\parskip0pt\parsep0pt
\item[PLAIN]
The Plain SASL Mechanism as defined in RFC 2595 resp. IETF Draft
draft-ietf-sasl-plain-XX.txt is another SASL mechanism that allows
username and clear-text password combinations in SASL environments.

~

Like LOGIN it sends the credentials in clear over the network and should
not be used without sufficient security protection.
\end{description}

As for server support, only \emph{PLAIN}, \emph{LOGIN} and
\emph{DIGEST-MD5} are supported at the time of this writing.

``server\_new'' OPTIONS is a hashref that is only relevant for
\emph{DIGEST-MD5} for now and it supports the following options:

\begin{description}
\item[- no\_integrity]
\end{description}

\begin{description}
\item[- no\_confidentiality]
\end{description}

which configures how the security layers are negotiated with the client
(or rather imposed to the client).

\hyperdef{}{SEEux5fALSO}{\section{\hyperref[SEEux5fALSO]{SEE
ALSO}}\label{SEEux5fALSO}}

Authen::SASL, Authen::SASL::Perl::ANONYMOUS,
Authen::SASL::Perl::CRAM\_MD5, Authen::SASL::Perl::DIGEST\_MD5,
Authen::SASL::Perl::EXTERNAL, Authen::SASL::Perl::GSSAPI,
Authen::SASL::Perl::LOGIN, Authen::SASL::Perl::PLAIN

\hyperdef{}{AUTHOR}{\section{\hyperref[AUTHOR]{AUTHOR}}\label{AUTHOR}}

Peter Marschall \textless{}peter@adpm.de\textgreater{}

Please report any bugs, or post any suggestions, to the perl-ldap
mailing list \textless{}perl-ldap@perl.org\textgreater{}

\hyperdef{}{COPYRIGHT}{\section{\hyperref[COPYRIGHT]{COPYRIGHT}}\label{COPYRIGHT}}

Copyright (c) 2004-2006 Peter Marschall. All rights reserved. This
document is distributed, and may be redistributed, under the same terms
as Perl itself.

\begin{longtable}[c]{@{}ll@{}}
\toprule\addlinespace
2010-03-11 & perl v5.10.1
\\\addlinespace
\bottomrule
\end{longtable}

\end{document}
