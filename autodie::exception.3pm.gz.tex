\documentclass[]{article}
\usepackage[T1]{fontenc}
\usepackage{lmodern}
\usepackage{amssymb,amsmath}
\usepackage{ifxetex,ifluatex}
\usepackage{fixltx2e} % provides \textsubscript
% use upquote if available, for straight quotes in verbatim environments
\IfFileExists{upquote.sty}{\usepackage{upquote}}{}
\ifnum 0\ifxetex 1\fi\ifluatex 1\fi=0 % if pdftex
  \usepackage[utf8]{inputenc}
\else % if luatex or xelatex
  \ifxetex
    \usepackage{mathspec}
    \usepackage{xltxtra,xunicode}
  \else
    \usepackage{fontspec}
  \fi
  \defaultfontfeatures{Mapping=tex-text,Scale=MatchLowercase}
  \newcommand{\euro}{€}
\fi
% use microtype if available
\IfFileExists{microtype.sty}{\usepackage{microtype}}{}
\usepackage{longtable,booktabs}
\ifxetex
  \usepackage[setpagesize=false, % page size defined by xetex
              unicode=false, % unicode breaks when used with xetex
              xetex]{hyperref}
\else
  \usepackage[unicode=true]{hyperref}
\fi
\hypersetup{breaklinks=true,
            bookmarks=true,
            pdfauthor={},
            pdftitle={autodie::exception(3pm)},
            colorlinks=true,
            citecolor=blue,
            urlcolor=blue,
            linkcolor=magenta,
            pdfborder={0 0 0}}
\urlstyle{same}  % don't use monospace font for urls
\setlength{\parindent}{0pt}
\setlength{\parskip}{6pt plus 2pt minus 1pt}
\setlength{\emergencystretch}{3em}  % prevent overfull lines
\setcounter{secnumdepth}{0}
\usepackage{pagecolor}

% Set background colour (of the page)
\definecolor{weirdbgcolor}{HTML}{FCF4F0}
\pagecolor{weirdbgcolor}

% Make bold text appear in a particular colour
\definecolor{boldcolor}{HTML}{6E0002}
\let\realtextbf=\textbf
\renewcommand{\textbf}[1]{\textcolor{boldcolor}{\realtextbf{#1}}}

% Use underlines instead of emphasis (ugh)
\renewcommand{\emph}[1]{\underline{#1}}

% % Use fixed-width font by default
% \renewcommand*\familydefault{\ttdefault}

\title{autodie::exception(3pm)}
\author{}
\date{}

\begin{document}
\maketitle

\begin{longtable}[c]{@{}lll@{}}
\toprule\addlinespace
autodie::exception(3pm) & User Contributed Perl Documentation &
autodie::exception(3pm)
\\\addlinespace
\bottomrule
\end{longtable}

\hyperdef{}{NAME}{\section{\hyperref[NAME]{NAME}}\label{NAME}}

autodie::exception - Exceptions from autodying functions.

\hyperdef{}{SYNOPSIS}{\section{\hyperref[SYNOPSIS]{SYNOPSIS}}\label{SYNOPSIS}}

\begin{verbatim}
    eval {
        use autodie;
        open(my $fh, '<', 'some_file.txt');
        ...
    };
    if (my $E = $@) {
        say "Ooops!  ",$E->caller," had problems: $@";
    }
\end{verbatim}

\hyperdef{}{DESCRIPTION}{\section{\hyperref[DESCRIPTION]{DESCRIPTION}}\label{DESCRIPTION}}

When an autodie enabled function fails, it generates an
``autodie::exception'' object. This can be interrogated to determine
further information about the error that occurred.

This document is broken into two sections; those methods that are most
useful to the end-developer, and those methods for anyone wishing to
subclass or get very familiar with ``autodie::exception''.

\hyperdef{}{Commonux5fMethods}{\subsection{\hyperref[Commonux5fMethods]{Common
Methods}}\label{Commonux5fMethods}}

These methods are intended to be used in the everyday dealing of
exceptions.

The following assume that the error has been copied into a separate
scalar:

\begin{verbatim}
    if ($E = $@) {
        ...
    }
\end{verbatim}

This is not required, but is recommended in case any code is called
which may reset or alter \$@.

\emph{args}

\begin{verbatim}
    my $array_ref = $E->args;
\end{verbatim}

Provides a reference to the arguments passed to the subroutine that
died.

\emph{function}

\begin{verbatim}
    my $sub = $E->function;
\end{verbatim}

The subroutine (including package) that threw the exception.

\emph{file}

\begin{verbatim}
    my $file = $E->file;
\end{verbatim}

The file in which the error occurred (eg, ``myscript.pl'' or
``MyTest.pm'').

\emph{package}

\begin{verbatim}
    my $package = $E->package;
\end{verbatim}

The package from which the exceptional subroutine was called.

\emph{caller}

\begin{verbatim}
    my $caller = $E->caller;
\end{verbatim}

The subroutine that \emph{called} the exceptional code.

\emph{line}

\begin{verbatim}
    my $line = $E->line;
\end{verbatim}

The line in ``\$E-\textgreater{}file'' where the exceptional code was
called.

\emph{context}

\begin{verbatim}
    my $context = $E->context;
\end{verbatim}

The context in which the subroutine was called by autodie; usually the
same as the context in which you called the autodying subroutine. This
can be `list', `scalar', or undefined (unknown). It will never be
`void', as ``autodie'' always captures the return value in one way or
another.

For some core functions that always return a scalar value regardless of
their context (eg, ``chown''), this may be `scalar', even if you used a
list context.

\emph{return}

\begin{verbatim}
    my $return_value = $E->return;
\end{verbatim}

The value(s) returned by the failed subroutine. When the subroutine was
called in a list context, this will always be a reference to an array
containing the results. When the subroutine was called in a scalar
context, this will be the actual scalar returned.

\emph{errno}

\begin{verbatim}
    my $errno = $E->errno;
\end{verbatim}

The value of \$! at the time when the exception occurred.

\textbf{NOTE}: This method will leave the main ``autodie::exception''
class and become part of a role in the future. You should only call
``errno'' for exceptions where \$! would reasonably have been set on
failure.

\emph{eval\_error}

\begin{verbatim}
    my $old_eval_error = $E->eval_error;
\end{verbatim}

The contents of \$@ immediately after autodie triggered an exception.
This may be useful when dealing with modules such as Text::Balanced that
set (but do not throw) \$@ on error.

\emph{matches}

\begin{verbatim}
    if ( $e->matches('open') ) { ... }
    if ( $e ~~ 'open' ) { ... }
\end{verbatim}

``matches'' is used to determine whether a given exception matches a
particular role. On Perl 5.10, using smart-match
(``\textasciitilde{}\textasciitilde{}'') with an ``autodie::exception''
object will use ``matches'' underneath.

An exception is considered to match a string if:

\begin{description}
\itemsep1pt\parskip0pt\parsep0pt
\item[•]
For a string not starting with a colon, the string exactly matches the
package and subroutine that threw the exception. For example,
``MyModule::log''. If the string does not contain a package name,
``CORE::'' is assumed.
\end{description}

\begin{description}
\itemsep1pt\parskip0pt\parsep0pt
\item[•]
For a string that does start with a colon, if the subroutine throwing
the exception \emph{does} that behaviour. For example, the
``CORE::open'' subroutine does ``:file'', ``:io'' and ``:all''.

~

See ``CATEGORIES'' in autodie for further information.
\end{description}

\hyperdef{}{Advancedux5fmethods}{\subsection{\hyperref[Advancedux5fmethods]{Advanced
methods}}\label{Advancedux5fmethods}}

The following methods, while usable from anywhere, are primarily
intended for developers wishing to subclass ``autodie::exception'',
write code that registers custom error messages, or otherwise work
closely with the ``autodie::exception'' model.

\emph{register}

\begin{verbatim}
    autodie::exception->register( 'CORE::open' => \&mysub );
\end{verbatim}

The ``register'' method allows for the registration of a message handler
for a given subroutine. The full subroutine name including the package
should be used.

Registered message handlers will receive the ``autodie::exception''
object as the first parameter.

\emph{add\_file\_and\_line}

\begin{verbatim}
    say "Problem occurred",$@->add_file_and_line;
\end{verbatim}

Returns the string " at \%s line \%d", where \%s is replaced with the
filename, and \%d is replaced with the line number.

Primarily intended for use by format handlers.

\emph{stringify}

\begin{verbatim}
    say "The error was: ",$@->stringify;
\end{verbatim}

Formats the error as a human readable string. Usually there's no reason
to call this directly, as it is used automatically if an
``autodie::exception'' object is ever used as a string.

Child classes can override this method to change how they're
stringified.

\emph{format\_default}

\begin{verbatim}
    my $error_string = $E->format_default;
\end{verbatim}

This produces the default error string for the given exception,
\emph{without using any registered message handlers}. It is primarily
intended to be called from a message handler when they have been passed
an exception they don't want to format.

Child classes can override this method to change how default messages
are formatted.

\emph{new}

\begin{verbatim}
    my $error = autodie::exception->new(
        args => \@_,
        function => "CORE::open",
        errno => $!,
        context => 'scalar',
        return => undef,
    );
\end{verbatim}

Creates a new ``autodie::exception'' object. Normally called directly
from an autodying function. The ``function'' argument is required, its
the function we were trying to call that generated the exception. The
``args'' parameter is optional.

The ``errno'' value is optional. In versions of ``autodie::exception''
1.99 and earlier the code would try to automatically use the current
value of \$!, but this was unreliable and is no longer supported.

Atrributes such as package, file, and caller are determined
automatically, and cannot be specified.

\hyperdef{}{SEEux5fALSO}{\section{\hyperref[SEEux5fALSO]{SEE
ALSO}}\label{SEEux5fALSO}}

autodie, autodie::exception::system

\hyperdef{}{LICENSE}{\section{\hyperref[LICENSE]{LICENSE}}\label{LICENSE}}

Copyright (C)2008 Paul Fenwick

This is free software. You may modify and/or redistribute this code
under the same terms as Perl 5.10 itself, or, at your option, any later
version of Perl 5.

\hyperdef{}{AUTHOR}{\section{\hyperref[AUTHOR]{AUTHOR}}\label{AUTHOR}}

Paul Fenwick \textless{}pjf@perltraining.com.au\textgreater{}

\begin{longtable}[c]{@{}ll@{}}
\toprule\addlinespace
2014-01-27 & perl v5.18.2
\\\addlinespace
\bottomrule
\end{longtable}

\end{document}
