\documentclass[]{article}
\usepackage[T1]{fontenc}
\usepackage{lmodern}
\usepackage{amssymb,amsmath}
\usepackage{ifxetex,ifluatex}
\usepackage{fixltx2e} % provides \textsubscript
% use upquote if available, for straight quotes in verbatim environments
\IfFileExists{upquote.sty}{\usepackage{upquote}}{}
\ifnum 0\ifxetex 1\fi\ifluatex 1\fi=0 % if pdftex
  \usepackage[utf8]{inputenc}
\else % if luatex or xelatex
  \ifxetex
    \usepackage{mathspec}
    \usepackage{xltxtra,xunicode}
  \else
    \usepackage{fontspec}
  \fi
  \defaultfontfeatures{Mapping=tex-text,Scale=MatchLowercase}
  \newcommand{\euro}{€}
\fi
% use microtype if available
\IfFileExists{microtype.sty}{\usepackage{microtype}}{}
\usepackage{longtable,booktabs}
\ifxetex
  \usepackage[setpagesize=false, % page size defined by xetex
              unicode=false, % unicode breaks when used with xetex
              xetex]{hyperref}
\else
  \usepackage[unicode=true]{hyperref}
\fi
\hypersetup{breaklinks=true,
            bookmarks=true,
            pdfauthor={},
            pdftitle={BASENAME(3)},
            colorlinks=true,
            citecolor=blue,
            urlcolor=blue,
            linkcolor=magenta,
            pdfborder={0 0 0}}
\urlstyle{same}  % don't use monospace font for urls
\setlength{\parindent}{0pt}
\setlength{\parskip}{6pt plus 2pt minus 1pt}
\setlength{\emergencystretch}{3em}  % prevent overfull lines
\setcounter{secnumdepth}{0}
\usepackage{pagecolor}

% Set background colour (of the page)
\definecolor{weirdbgcolor}{HTML}{FCF4F0}
\pagecolor{weirdbgcolor}

% Make bold text appear in a particular colour
\definecolor{boldcolor}{HTML}{6E0002}
\let\realtextbf=\textbf
\renewcommand{\textbf}[1]{\textcolor{boldcolor}{\realtextbf{#1}}}

% Use underlines instead of emphasis (ugh)
\renewcommand{\emph}[1]{\underline{#1}}

% % Use fixed-width font by default
% \renewcommand*\familydefault{\ttdefault}

\title{BASENAME(3)}
\author{}
\date{}

\begin{document}
\maketitle

\begin{longtable}[c]{@{}lll@{}}
\toprule\addlinespace
BASENAME(3) & Linux Programmer's Manual & BASENAME(3)
\\\addlinespace
\bottomrule
\end{longtable}

\hyperdef{}{NAME}{\section{\hyperref[NAME]{NAME}}\label{NAME}}

basename, dirname - parse pathname components

\hyperdef{}{SYNOPSIS}{\section{\hyperref[SYNOPSIS]{SYNOPSIS}}\label{SYNOPSIS}}

\begin{verbatim}
#include <libgen.h>
 
char *dirname(char *path);

char *basename(char *path);
\end{verbatim}

\hyperdef{}{DESCRIPTION}{\section{\hyperref[DESCRIPTION]{DESCRIPTION}}\label{DESCRIPTION}}

Warning: there are two different functions \textbf{basename}() - see
below.

The functions \textbf{dirname}() and \textbf{basename}() break a
null-terminated pathname string into directory and filename components.
In the usual case, \textbf{dirname}() returns the string up to, but not
including, the final `/', and \textbf{basename}() returns the component
following the final `/'. Trailing `/' characters are not counted as part
of the pathname.

If \emph{path} does not contain a slash, \textbf{dirname}() returns the
string ``.'' while \textbf{basename}() returns a copy of \emph{path}. If
\emph{path} is the string ``/'', then both \textbf{dirname}() and
\textbf{basename}() return the string ``/''. If \emph{path} is a NULL
pointer or points to an empty string, then both \textbf{dirname}() and
\textbf{basename}() return the string ``.''.

Concatenating the string returned by \textbf{dirname}(), a ``/'', and
the string returned by \textbf{basename}() yields a complete pathname.

Both \textbf{dirname}() and \textbf{basename}() may modify the contents
of \emph{path}, so it may be desirable to pass a copy when calling one
of these functions.

These functions may return pointers to statically allocated memory which
may be overwritten by subsequent calls. Alternatively, they may return a
pointer to some part of \emph{path}, so that the string referred to by
\emph{path} should not be modified or freed until the pointer returned
by the function is no longer required.

The following list of examples (taken from SUSv2) shows the strings
returned by \textbf{dirname}() and \textbf{basename}() for different
paths:

~

\begin{longtable}[c]{@{}llll@{}}
\toprule\addlinespace
path & dirname & basename &
\\\addlinespace
/usr/lib & /usr & lib &
\\\addlinespace
/usr/ & / & usr &
\\\addlinespace
usr & . & usr &
\\\addlinespace
/ & / & / &
\\\addlinespace
& & &
\\\addlinespace
\bottomrule
\end{longtable}

\hyperdef{}{RETURNux5fVALUE}{\section{\hyperref[RETURNux5fVALUE]{RETURN
VALUE}}\label{RETURNux5fVALUE}}

Both \textbf{dirname}() and \textbf{basename}() return pointers to
null-terminated strings. (Do not pass these pointers to
\textbf{free}(3).)

\hyperdef{}{CONFORMINGux5fTO}{\section{\hyperref[CONFORMINGux5fTO]{CONFORMING
TO}}\label{CONFORMINGux5fTO}}

POSIX.1-2001.

\hyperdef{}{NOTES}{\section{\hyperref[NOTES]{NOTES}}\label{NOTES}}

There are two different versions of \textbf{basename}() - the POSIX
version described above, and the GNU version, which one gets after

~

\begin{verbatim}

    #define _GNU_SOURCE         /* See feature_test_macros(7) */
 
    #include <string.h>
\end{verbatim}

The GNU version never modifies its argument, and returns the empty
string when \emph{path} has a trailing slash, and in particular also
when it is ``/''. There is no GNU version of \textbf{dirname}().

With glibc, one gets the POSIX version of \textbf{basename}() when
\emph{\textless{}libgen.h\textgreater{}} is included, and the GNU
version otherwise.

\hyperdef{}{BUGS}{\section{\hyperref[BUGS]{BUGS}}\label{BUGS}}

In the glibc implementation of the POSIX versions of these functions
they modify their argument, and segfault when called with a static
string like ``/usr/''. Before glibc 2.2.1, the glibc version of
\textbf{dirname}() did not correctly handle pathnames with trailing `/'
characters, and generated a segfault if given a NULL argument.

\hyperdef{}{EXAMPLE}{\section{\hyperref[EXAMPLE]{EXAMPLE}}\label{EXAMPLE}}

\begin{verbatim}
char *dirc, *basec, *bname, *dname;
char *path = "/etc/passwd";

dirc = strdup(path);
basec = strdup(path);
dname = dirname(dirc);
bname = basename(basec);
printf("dirname=%s, basename=%s\n", dname, bname);
\end{verbatim}

\hyperdef{}{SEEux5fALSO}{\section{\hyperref[SEEux5fALSO]{SEE
ALSO}}\label{SEEux5fALSO}}

\textbf{basename}(1), \textbf{dirname}(1)

\hyperdef{}{COLOPHON}{\section{\hyperref[COLOPHON]{COLOPHON}}\label{COLOPHON}}

This page is part of release 3.54 of the Linux \emph{man-pages} project.
A description of the project, and information about reporting bugs, can
be found at http://www.kernel.org/doc/man-pages/.

\begin{longtable}[c]{@{}ll@{}}
\toprule\addlinespace
2009-03-30 & GNU
\\\addlinespace
\bottomrule
\end{longtable}

\end{document}
