\documentclass[]{article}
\usepackage[T1]{fontenc}
\usepackage{lmodern}
\usepackage{amssymb,amsmath}
\usepackage{ifxetex,ifluatex}
\usepackage{fixltx2e} % provides \textsubscript
% use upquote if available, for straight quotes in verbatim environments
\IfFileExists{upquote.sty}{\usepackage{upquote}}{}
\ifnum 0\ifxetex 1\fi\ifluatex 1\fi=0 % if pdftex
  \usepackage[utf8]{inputenc}
\else % if luatex or xelatex
  \ifxetex
    \usepackage{mathspec}
    \usepackage{xltxtra,xunicode}
  \else
    \usepackage{fontspec}
  \fi
  \defaultfontfeatures{Mapping=tex-text,Scale=MatchLowercase}
  \newcommand{\euro}{€}
\fi
% use microtype if available
\IfFileExists{microtype.sty}{\usepackage{microtype}}{}
\usepackage{longtable,booktabs}
\ifxetex
  \usepackage[setpagesize=false, % page size defined by xetex
              unicode=false, % unicode breaks when used with xetex
              xetex]{hyperref}
\else
  \usepackage[unicode=true]{hyperref}
\fi
\hypersetup{breaklinks=true,
            bookmarks=true,
            pdfauthor={},
            pdftitle={RPC(3)},
            colorlinks=true,
            citecolor=blue,
            urlcolor=blue,
            linkcolor=magenta,
            pdfborder={0 0 0}}
\urlstyle{same}  % don't use monospace font for urls
\setlength{\parindent}{0pt}
\setlength{\parskip}{6pt plus 2pt minus 1pt}
\setlength{\emergencystretch}{3em}  % prevent overfull lines
\setcounter{secnumdepth}{0}
\usepackage{pagecolor}

% Set background colour (of the page)
\definecolor{weirdbgcolor}{HTML}{FCF4F0}
\pagecolor{weirdbgcolor}

% Make bold text appear in a particular colour
\definecolor{boldcolor}{HTML}{6E0002}
\let\realtextbf=\textbf
\renewcommand{\textbf}[1]{\textcolor{boldcolor}{\realtextbf{#1}}}

% Use underlines instead of emphasis (ugh)
\renewcommand{\emph}[1]{\underline{#1}}

% % Use fixed-width font by default
% \renewcommand*\familydefault{\ttdefault}

\title{RPC(3)}
\author{}
\date{}

\begin{document}
\maketitle

\begin{longtable}[c]{@{}lll@{}}
\toprule\addlinespace
RPC(3) & Linux Programmer's Manual & RPC(3)
\\\addlinespace
\bottomrule
\end{longtable}

\hyperdef{}{NAME}{\section{\hyperref[NAME]{NAME}}\label{NAME}}

rpc - library routines for remote procedure calls

\hyperdef{}{SYNOPSISux5fANDux5fDESCRIPTION}{\section{\hyperref[SYNOPSISux5fANDux5fDESCRIPTION]{SYNOPSIS
AND DESCRIPTION}}\label{SYNOPSISux5fANDux5fDESCRIPTION}}

These routines allow C programs to make procedure calls on other
machines across the network. First, the client calls a procedure to send
a data packet to the server. Upon receipt of the packet, the server
calls a dispatch routine to perform the requested service, and then
sends back a reply. Finally, the procedure call returns to the client.

To take use of these routines, include the header file
\emph{\textless{}rpc/rpc.h\textgreater{}}.

~

The prototypes below make use of the following types: \\

\begin{verbatim}

typedef int bool_t;

typedef bool_t (*xdrproc_t) (XDR *, void *, ...);

typedef bool_t (*resultproc_t) (caddr_t resp,
                                struct sockaddr_in *raddr);
\end{verbatim}

See the header files for the declarations of the \emph{AUTH},
\emph{CLIENT}, \emph{SVCXPRT}, and \emph{XDR} types.

\begin{verbatim}
void auth_destroy(AUTH *auth);
\end{verbatim}

\begin{description}
\itemsep1pt\parskip0pt\parsep0pt
\item[]
A macro that destroys the authentication information associated with
\emph{auth}. Destruction usually involves deallocation of private data
structures. The use of \emph{auth} is undefined after calling
\textbf{auth\_destroy}().
\end{description}

\begin{verbatim}
AUTH *authnone_create(void);
\end{verbatim}

\begin{description}
\itemsep1pt\parskip0pt\parsep0pt
\item[]
Create and return an RPC authentication handle that passes nonusable
authentication information with each remote procedure call. This is the
default authentication used by RPC.
\end{description}

\begin{verbatim}
AUTH *authunix_create(char *host, int uid, int gid,
                      int len, int *aup_gids);
\end{verbatim}

\begin{description}
\itemsep1pt\parskip0pt\parsep0pt
\item[]
Create and return an RPC authentication handle that contains
authentication information. The parameter \emph{host} is the name of the
machine on which the information was created; \emph{uid} is the user's
user ID; \emph{gid} is the user's current group ID; \emph{len} and
\emph{aup\_gids} refer to a counted array of groups to which the user
belongs. It is easy to impersonate a user.
\end{description}

\begin{verbatim}
AUTH *authunix_create_default(void);
\end{verbatim}

\begin{description}
\itemsep1pt\parskip0pt\parsep0pt
\item[]
Calls \textbf{authunix\_create}() with the appropriate parameters.
\end{description}

\begin{verbatim}
int callrpc(char *host, unsigned long prognum,
            unsigned long versnum, unsigned long procnum,
            xdrproc_t inproc, char *in,
            xdrproc_t outproc, char *out);
\end{verbatim}

\begin{description}
\itemsep1pt\parskip0pt\parsep0pt
\item[]
Call the remote procedure associated with \emph{prognum},
\emph{versnum}, and \emph{procnum} on the machine, \emph{host}. The
parameter \emph{in} is the address of the procedure's argument(s), and
\emph{out} is the address of where to place the result(s); \emph{inproc}
is used to encode the procedure's parameters, and \emph{outproc} is used
to decode the procedure's results. This routine returns zero if it
succeeds, or the value of \textbf{enum clnt\_stat} cast to an integer if
it fails. The routine \textbf{clnt\_perrno}() is handy for translating
failure statuses into messages.
\end{description}

\begin{description}
\itemsep1pt\parskip0pt\parsep0pt
\item[]
Warning: calling remote procedures with this routine uses UDP/IP as a
transport; see \textbf{clntudp\_create}() for restrictions. You do not
have control of timeouts or authentication using this routine.
\end{description}

\begin{verbatim}
enum clnt_stat clnt_broadcast(unsigned long prognum,
                     unsigned long versnum, unsigned long procnum,
                     xdrproc_t inproc, char *in,
                     xdrproc_t outproc, char *out,
                     resultproc_t eachresult);
\end{verbatim}

\begin{description}
\itemsep1pt\parskip0pt\parsep0pt
\item[]
Like \textbf{callrpc}(), except the call message is broadcast to all
locally connected broadcast nets. Each time it receives a response, this
routine calls \textbf{eachresult}(), whose form is:
\end{description}

\begin{description}
\item[]
\begin{verbatim}
eachresult(char *out, struct sockaddr_in *addr);
    
\end{verbatim}
\end{description}

\begin{description}
\itemsep1pt\parskip0pt\parsep0pt
\item[]
where \emph{out} is the same as \emph{out} passed to
\textbf{clnt\_broadcast}(), except that the remote procedure's output is
decoded there; \emph{addr} points to the address of the machine that
sent the results. If \textbf{eachresult}() returns zero,
\textbf{clnt\_broadcast}() waits for more replies; otherwise it returns
with appropriate status.
\end{description}

\begin{description}
\itemsep1pt\parskip0pt\parsep0pt
\item[]
Warning: broadcast sockets are limited in size to the maximum transfer
unit of the data link. For ethernet, this value is 1500 bytes.
\end{description}

\begin{verbatim}
enum clnt_stat clnt_call(CLIENT *clnt, unsigned long procnum,
                    xdrproc_t inproc, char *in,
                    xdrproc_t outproc, char *out,
                    struct timeval tout);
\end{verbatim}

\begin{description}
\itemsep1pt\parskip0pt\parsep0pt
\item[]
A macro that calls the remote procedure \emph{procnum} associated with
the client handle, \emph{clnt}, which is obtained with an RPC client
creation routine such as \textbf{clnt\_create}(). The parameter
\emph{in} is the address of the procedure's argument(s), and \emph{out}
is the address of where to place the result(s); \emph{inproc} is used to
encode the procedure's parameters, and \emph{outproc} is used to decode
the procedure's results; \emph{tout} is the time allowed for results to
come back.
\end{description}

\begin{verbatim}
clnt_destroy(CLIENT *clnt);
\end{verbatim}

\begin{description}
\itemsep1pt\parskip0pt\parsep0pt
\item[]
A macro that destroys the client's RPC handle. Destruction usually
involves deallocation of private data structures, including \emph{clnt}
itself. Use of \emph{clnt} is undefined after calling
\textbf{clnt\_destroy}(). If the RPC library opened the associated
socket, it will close it also. Otherwise, the socket remains open.
\end{description}

\begin{verbatim}
CLIENT *clnt_create(char *host, unsigned long prog,
                    unsigned long vers, char *proto);
\end{verbatim}

\begin{description}
\itemsep1pt\parskip0pt\parsep0pt
\item[]
Generic client creation routine. \emph{host} identifies the name of the
remote host where the server is located. \emph{proto} indicates which
kind of transport protocol to use. The currently supported values for
this field are ``udp'' and ``tcp''. Default timeouts are set, but can be
modified using \textbf{clnt\_control}().
\end{description}

\begin{description}
\itemsep1pt\parskip0pt\parsep0pt
\item[]
Warning: Using UDP has its shortcomings. Since UDP-based RPC messages
can hold only up to 8 Kbytes of encoded data, this transport cannot be
used for procedures that take large arguments or return huge results.
\end{description}

\begin{verbatim}
bool_t clnt_control(CLIENT *cl, int req, char *info);
\end{verbatim}

\begin{description}
\itemsep1pt\parskip0pt\parsep0pt
\item[]
A macro used to change or retrieve various information about a client
object. \emph{req} indicates the type of operation, and \emph{info} is a
pointer to the information. For both UDP and TCP, the supported values
of \emph{req} and their argument types and what they do are:
\end{description}

\begin{description}
\item[]
\begin{verbatim}
CLSET_TIMEOUT  struct timeval // set total timeout
CLGET_TIMEOUT  struct timeval // get total timeout
    
\end{verbatim}
\end{description}

\begin{description}
\itemsep1pt\parskip0pt\parsep0pt
\item[]
Note: if you set the timeout using \textbf{clnt\_control}(), the timeout
parameter passed to \textbf{clnt\_call}() will be ignored in all future
calls.
\end{description}

\begin{description}
\item[]
\begin{verbatim}
CLGET_SERVER_ADDR  struct sockaddr_in  // get server's address
    
\end{verbatim}
\end{description}

\begin{description}
\itemsep1pt\parskip0pt\parsep0pt
\item[]
The following operations are valid for UDP only:
\end{description}

\begin{description}
\item[]
\begin{verbatim}
CLSET_RETRY_TIMEOUT  struct timeval // set the retry timeout
CLGET_RETRY_TIMEOUT  struct timeval // get the retry timeout
    
\end{verbatim}
\end{description}

\begin{description}
\itemsep1pt\parskip0pt\parsep0pt
\item[]
The retry timeout is the time that ``UDP RPC'' waits for the server to
reply before retransmitting the request.
\end{description}

\begin{verbatim}
clnt_freeres(CLIENT * clnt, xdrproc_t outproc, char *out);
\end{verbatim}

\begin{description}
\itemsep1pt\parskip0pt\parsep0pt
\item[]
A macro that frees any data allocated by the RPC/XDR system when it
decoded the results of an RPC call. The parameter \emph{out} is the
address of the results, and \emph{outproc} is the XDR routine describing
the results. This routine returns one if the results were successfully
freed, and zero otherwise.
\end{description}

\begin{verbatim}
void clnt_geterr(CLIENT *clnt, struct rpc_err *errp);
\end{verbatim}

\begin{description}
\itemsep1pt\parskip0pt\parsep0pt
\item[]
A macro that copies the error structure out of the client handle to the
structure at address \emph{errp}.
\end{description}

\begin{verbatim}
void clnt_pcreateerror(char *s);
\end{verbatim}

\begin{description}
\itemsep1pt\parskip0pt\parsep0pt
\item[]
Print a message to standard error indicating why a client RPC handle
could not be created. The message is prepended with string \emph{s} and
a colon. Used when a \textbf{clnt\_create}(),
\textbf{clntraw\_create}(), \textbf{clnttcp\_create}(), or
\textbf{clntudp\_create}() call fails.
\end{description}

\begin{verbatim}
void clnt_perrno(enum clnt_stat stat);
\end{verbatim}

\begin{description}
\itemsep1pt\parskip0pt\parsep0pt
\item[]
Print a message to standard error corresponding to the condition
indicated by \emph{stat}. Used after \textbf{callrpc}().
\end{description}

\begin{verbatim}
clnt_perror(CLIENT *clnt, char *s);
\end{verbatim}

\begin{description}
\itemsep1pt\parskip0pt\parsep0pt
\item[]
Print a message to standard error indicating why an RPC call failed;
\emph{clnt} is the handle used to do the call. The message is prepended
with string \emph{s} and a colon. Used after \textbf{clnt\_call}().
\end{description}

\begin{verbatim}
char *clnt_spcreateerror(char *s);
\end{verbatim}

\begin{description}
\itemsep1pt\parskip0pt\parsep0pt
\item[]
Like \textbf{clnt\_pcreateerror}(), except that it returns a string
instead of printing to the standard error.
\end{description}

\begin{description}
\itemsep1pt\parskip0pt\parsep0pt
\item[]
Bugs: returns pointer to static data that is overwritten on each call.
\end{description}

\begin{verbatim}
char *clnt_sperrno(enum clnt_stat stat);
\end{verbatim}

\begin{description}
\itemsep1pt\parskip0pt\parsep0pt
\item[]
Take the same arguments as \textbf{clnt\_perrno}(), but instead of
sending a message to the standard error indicating why an RPC call
failed, return a pointer to a string which contains the message. The
string ends with a NEWLINE.
\end{description}

\begin{description}
\itemsep1pt\parskip0pt\parsep0pt
\item[]
\textbf{clnt\_sperrno}() is used instead of \textbf{clnt\_perrno}() if
the program does not have a standard error (as a program running as a
server quite likely does not), or if the programmer does not want the
message to be output with \textbf{printf}(3), or if a message format
different than that supported by \textbf{clnt\_perrno}() is to be used.
Note: unlike \textbf{clnt\_sperror}() and \textbf{clnt\_spcreaterror}(),
\textbf{clnt\_sperrno}() returns pointer to static data, but the result
will not get overwritten on each call.
\end{description}

\begin{verbatim}
char *clnt_sperror(CLIENT *rpch, char *s);
\end{verbatim}

\begin{description}
\itemsep1pt\parskip0pt\parsep0pt
\item[]
Like \textbf{clnt\_perror}(), except that (like
\textbf{clnt\_sperrno}()) it returns a string instead of printing to
standard error.
\end{description}

\begin{description}
\itemsep1pt\parskip0pt\parsep0pt
\item[]
Bugs: returns pointer to static data that is overwritten on each call.
\end{description}

\begin{verbatim}
CLIENT *clntraw_create(unsigned long prognum, unsigned long versnum);
\end{verbatim}

\begin{description}
\itemsep1pt\parskip0pt\parsep0pt
\item[]
This routine creates a toy RPC client for the remote program
\emph{prognum}, version \emph{versnum}. The transport used to pass
messages to the service is actually a buffer within the process's
address space, so the corresponding RPC server should live in the same
address space; see \textbf{svcraw\_create}(). This allows simulation of
RPC and acquisition of RPC overheads, such as round trip times, without
any kernel interference. This routine returns NULL if it fails.
\end{description}

\begin{verbatim}
CLIENT *clnttcp_create(struct sockaddr_in *addr,
                unsigned long prognum, unsigned long versnum,
                int *sockp, unsigned int sendsz, unsigned int recvsz);
\end{verbatim}

\begin{description}
\itemsep1pt\parskip0pt\parsep0pt
\item[]
This routine creates an RPC client for the remote program
\emph{prognum}, version \emph{versnum}; the client uses TCP/IP as a
transport. The remote program is located at Internet address
\emph{*addr}. If \emph{addr-\textgreater{}sin\_port} is zero, then it is
set to the actual port that the remote program is listening on (the
remote \textbf{portmap} service is consulted for this information). The
parameter \emph{sockp} is a socket; if it is \textbf{RPC\_ANYSOCK}, then
this routine opens a new one and sets \emph{sockp}. Since TCP-based RPC
uses buffered I/O, the user may specify the size of the send and receive
buffers with the parameters \emph{sendsz} and \emph{recvsz}; values of
zero choose suitable defaults. This routine returns NULL if it fails.
\end{description}

\begin{verbatim}
CLIENT *clntudp_create(struct sockaddr_in *addr,
                unsigned long prognum, unsigned long versnum,
                struct timeval wait, int *sockp);
\end{verbatim}

\begin{description}
\itemsep1pt\parskip0pt\parsep0pt
\item[]
This routine creates an RPC client for the remote program
\emph{prognum}, version \emph{versnum}; the client uses use UDP/IP as a
transport. The remote program is located at Internet address
\emph{addr}. If \emph{addr-\textgreater{}sin\_port} is zero, then it is
set to actual port that the remote program is listening on (the remote
\textbf{portmap} service is consulted for this information). The
parameter \emph{sockp} is a socket; if it is \textbf{RPC\_ANYSOCK}, then
this routine opens a new one and sets \emph{sockp}. The UDP transport
resends the call message in intervals of \emph{wait} time until a
response is received or until the call times out. The total time for the
call to time out is specified by \textbf{clnt\_call}().
\end{description}

\begin{description}
\itemsep1pt\parskip0pt\parsep0pt
\item[]
Warning: since UDP-based RPC messages can hold only up to 8 Kbytes of
encoded data, this transport cannot be used for procedures that take
large arguments or return huge results.
\end{description}

\begin{verbatim}
CLIENT *clntudp_bufcreate(struct sockaddr_in *addr,
            unsigned long prognum, unsigned long versnum,
            struct timeval wait, int *sockp,
            unsigned int sendsize, unsigned int recosize);
\end{verbatim}

\begin{description}
\itemsep1pt\parskip0pt\parsep0pt
\item[]
This routine creates an RPC client for the remote program
\emph{prognum}, on \emph{versnum}; the client uses use UDP/IP as a
transport. The remote program is located at Internet address
\emph{addr}. If \emph{addr-\textgreater{}sin\_port} is zero, then it is
set to actual port that the remote program is listening on (the remote
\textbf{portmap} service is consulted for this information). The
parameter \emph{sockp} is a socket; if it is \textbf{RPC\_ANYSOCK}, then
this routine opens a new one and sets \emph{sockp}. The UDP transport
resends the call message in intervals of \emph{wait} time until a
response is received or until the call times out. The total time for the
call to time out is specified by \textbf{clnt\_call}().
\end{description}

\begin{description}
\itemsep1pt\parskip0pt\parsep0pt
\item[]
This allows the user to specify the maximum packet size for sending and
receiving UDP-based RPC messages.
\end{description}

\begin{verbatim}
void get_myaddress(struct sockaddr_in *addr);
\end{verbatim}

\begin{description}
\itemsep1pt\parskip0pt\parsep0pt
\item[]
Stuff the machine's IP address into \emph{*addr}, without consulting the
library routines that deal with \emph{/etc/hosts}. The port number is
always set to \textbf{htons(PMAPPORT)}.
\end{description}

\begin{verbatim}
struct pmaplist *pmap_getmaps(struct sockaddr_in *addr);
\end{verbatim}

\begin{description}
\itemsep1pt\parskip0pt\parsep0pt
\item[]
A user interface to the \textbf{portmap} service, which returns a list
of the current RPC program-to-port mappings on the host located at IP
address \emph{*addr}. This routine can return NULL. The command
\emph{rpcinfo~-p} uses this routine.
\end{description}

\begin{verbatim}
unsigned short pmap_getport(struct sockaddr_in *addr,
                    unsigned long prognum, unsigned long versnum,
                    unsigned int protocol);
\end{verbatim}

\begin{description}
\itemsep1pt\parskip0pt\parsep0pt
\item[]
A user interface to the \textbf{portmap} service, which returns the port
number on which waits a service that supports program number
\emph{prognum}, version \emph{versnum}, and speaks the transport
protocol associated with \emph{protocol}. The value of \emph{protocol}
is most likely \textbf{IPPROTO\_UDP} or \textbf{IPPROTO\_TCP}. A return
value of zero means that the mapping does not exist or that the RPC
system failed to contact the remote \textbf{portmap} service. In the
latter case, the global variable \emph{rpc\_createerr} contains the RPC
status.
\end{description}

\begin{verbatim}
enum clnt_stat pmap_rmtcall(struct sockaddr_in *addr,
                    unsigned long prognum, unsigned long versnum,
                    unsigned long procnum,
                    xdrproc_t inproc, char *in,
                    xdrproc_t outproc, char *out,
                    struct timeval tout, unsigned long *portp);
\end{verbatim}

\begin{description}
\itemsep1pt\parskip0pt\parsep0pt
\item[]
A user interface to the \textbf{portmap} service, which instructs
\textbf{portmap} on the host at IP address \emph{*addr} to make an RPC
call on your behalf to a procedure on that host. The parameter
\emph{*portp} will be modified to the program's port number if the
procedure succeeds. The definitions of other parameters are discussed in
\textbf{callrpc}() and \textbf{clnt\_call}(). This procedure should be
used for a ``ping'' and nothing else. See also
\textbf{clnt\_broadcast}().
\end{description}

\begin{verbatim}
bool_t pmap_set(unsigned long prognum, unsigned long versnum,
                unsigned int protocol, unsigned short port);
\end{verbatim}

\begin{description}
\itemsep1pt\parskip0pt\parsep0pt
\item[]
A user interface to the \textbf{portmap} service, which establishes a
mapping between the triple
{[}\emph{prognum},\emph{versnum},\emph{protocol}{]} and \emph{port} on
the machine's \textbf{portmap} service. The value of \emph{protocol} is
most likely \textbf{IPPROTO\_UDP} or \textbf{IPPROTO\_TCP}. This routine
returns one if it succeeds, zero otherwise. Automatically done by
\textbf{svc\_register}().
\end{description}

\begin{verbatim}
bool_t pmap_unset(unsigned long prognum, unsigned long versnum);
\end{verbatim}

\begin{description}
\itemsep1pt\parskip0pt\parsep0pt
\item[]
A user interface to the \textbf{portmap} service, which destroys all
mapping between the triple {[}\emph{prognum},\emph{versnum},\emph{*}{]}
and \textbf{ports} on the machine's \textbf{portmap} service. This
routine returns one if it succeeds, zero otherwise.
\end{description}

\begin{verbatim}
int registerrpc(unsigned long prognum, unsigned long versnum,
                unsigned long procnum, char *(*procname)(char *),
                xdrproc_t inproc, xdrproc_t outproc);
\end{verbatim}

\begin{description}
\itemsep1pt\parskip0pt\parsep0pt
\item[]
Register procedure \emph{procname} with the RPC service package. If a
request arrives for program \emph{prognum}, version \emph{versnum}, and
procedure \emph{procnum}, \emph{procname} is called with a pointer to
its parameter(s); \emph{progname} should return a pointer to its static
result(s); \emph{inproc} is used to decode the parameters while
\emph{outproc} is used to encode the results. This routine returns zero
if the registration succeeded, -1 otherwise.
\end{description}

\begin{description}
\itemsep1pt\parskip0pt\parsep0pt
\item[]
Warning: remote procedures registered in this form are accessed using
the UDP/IP transport; see \textbf{svcudp\_create}() for restrictions.
\end{description}

\begin{verbatim}
struct rpc_createerr rpc_createerr;
\end{verbatim}

\begin{description}
\itemsep1pt\parskip0pt\parsep0pt
\item[]
A global variable whose value is set by any RPC client creation routine
that does not succeed. Use the routine \textbf{clnt\_pcreateerror}() to
print the reason why.
\end{description}

\begin{verbatim}
void svc_destroy(SVCXPRT *xprt);
\end{verbatim}

\begin{description}
\itemsep1pt\parskip0pt\parsep0pt
\item[]
A macro that destroys the RPC service transport handle, \emph{xprt}.
Destruction usually involves deallocation of private data structures,
including \emph{xprt} itself. Use of \emph{xprt} is undefined after
calling this routine.
\end{description}

\begin{verbatim}
fd_set svc_fdset;
\end{verbatim}

\begin{description}
\itemsep1pt\parskip0pt\parsep0pt
\item[]
A global variable reflecting the RPC service side's read file descriptor
bit mask; it is suitable as a parameter to the \textbf{select}(2) system
call. This is of interest only if a service implementor does not call
\textbf{svc\_run}(), but rather does his own asynchronous event
processing. This variable is read-only (do not pass its address to
\textbf{select}(2)!), yet it may change after calls to
\textbf{svc\_getreqset}() or any creation routines.
\end{description}

\begin{verbatim}
int svc_fds;
\end{verbatim}

\begin{description}
\itemsep1pt\parskip0pt\parsep0pt
\item[]
Similar to \textbf{svc\_fdset}, but limited to 32 descriptors. This
interface is obsoleted by \textbf{svc\_fdset}.
\end{description}

\begin{verbatim}
svc_freeargs(SVCXPRT *xprt, xdrproc_t inproc, char *in);
\end{verbatim}

\begin{description}
\itemsep1pt\parskip0pt\parsep0pt
\item[]
A macro that frees any data allocated by the RPC/XDR system when it
decoded the arguments to a service procedure using
\textbf{svc\_getargs}(). This routine returns 1 if the results were
successfully freed, and zero otherwise.
\end{description}

\begin{verbatim}
svc_getargs(SVCXPRT *xprt, xdrproc_t inproc, char *in);
\end{verbatim}

\begin{description}
\itemsep1pt\parskip0pt\parsep0pt
\item[]
A macro that decodes the arguments of an RPC request associated with the
RPC service transport handle, \emph{xprt}. The parameter \emph{in} is
the address where the arguments will be placed; \emph{inproc} is the XDR
routine used to decode the arguments. This routine returns one if
decoding succeeds, and zero otherwise.
\end{description}

\begin{verbatim}
struct sockaddr_in *svc_getcaller(SVCXPRT *xprt);
\end{verbatim}

\begin{description}
\itemsep1pt\parskip0pt\parsep0pt
\item[]
The approved way of getting the network address of the caller of a
procedure associated with the RPC service transport handle, \emph{xprt}.
\end{description}

\begin{verbatim}
void svc_getreqset(fd_set *rdfds);
\end{verbatim}

\begin{description}
\itemsep1pt\parskip0pt\parsep0pt
\item[]
This routine is of interest only if a service implementor does not call
\textbf{svc\_run}(), but instead implements custom asynchronous event
processing. It is called when the \textbf{select}(2) system call has
determined that an RPC request has arrived on some RPC socket(s);
\emph{rdfds} is the resultant read file descriptor bit mask. The routine
returns when all sockets associated with the value of \emph{rdfds} have
been serviced.
\end{description}

\begin{verbatim}
void svc_getreq(int rdfds);
\end{verbatim}

\begin{description}
\itemsep1pt\parskip0pt\parsep0pt
\item[]
Similar to \textbf{svc\_getreqset}(), but limited to 32 descriptors.
This interface is obsoleted by \textbf{svc\_getreqset}().
\end{description}

\begin{verbatim}
bool_t svc_register(SVCXPRT *xprt, unsigned long prognum,
                    unsigned long versnum,
                    void (*dispatch)(svc_req *, SVCXPRT *),
                    unsigned long protocol);
\end{verbatim}

\begin{description}
\item[]
Associates \emph{prognum} and \emph{versnum} with the service dispatch
procedure, \emph{dispatch}. If \emph{protocol} is zero, the service is
not registered with the \textbf{portmap} service. If \emph{protocol} is
nonzero, then a mapping of the triple
{[}\emph{prognum},\emph{versnum},\emph{protocol}{]} to
\emph{xprt-\textgreater{}xp\_port} is established with the local
\textbf{portmap} service (generally \emph{protocol} is zero,
\textbf{IPPROTO\_UDP} or \textbf{IPPROTO\_TCP}). The procedure
\emph{dispatch} has the following form: \\

\begin{verbatim}

dispatch(struct svc_req *request, SVCXPRT *xprt);
    
\end{verbatim}
\end{description}

\begin{description}
\itemsep1pt\parskip0pt\parsep0pt
\item[]
The \textbf{svc\_register}() routine returns one if it succeeds, and
zero otherwise.
\end{description}

\begin{verbatim}
void svc_run(void);
\end{verbatim}

\begin{description}
\itemsep1pt\parskip0pt\parsep0pt
\item[]
This routine never returns. It waits for RPC requests to arrive, and
calls the appropriate service procedure using \textbf{svc\_getreq}()
when one arrives. This procedure is usually waiting for a
\textbf{select}(2) system call to return.
\end{description}

\begin{verbatim}
bool_t svc_sendreply(SVCXPRT *xprt, xdrproc_t outproc, char *out);
\end{verbatim}

\begin{description}
\itemsep1pt\parskip0pt\parsep0pt
\item[]
Called by an RPC service's dispatch routine to send the results of a
remote procedure call. The parameter \emph{xprt} is the request's
associated transport handle; \emph{outproc} is the XDR routine which is
used to encode the results; and \emph{out} is the address of the
results. This routine returns one if it succeeds, zero otherwise.
\end{description}

\begin{verbatim}
void svc_unregister(unsigned long prognum, unsigned long versnum);
\end{verbatim}

\begin{description}
\itemsep1pt\parskip0pt\parsep0pt
\item[]
Remove all mapping of the double {[}\emph{prognum},\emph{versnum}{]} to
dispatch routines, and of the triple
{[}\emph{prognum},\emph{versnum},\emph{*}{]} to port number.
\end{description}

\begin{verbatim}
void svcerr_auth(SVCXPRT *xprt, enum auth_stat why);
\end{verbatim}

\begin{description}
\itemsep1pt\parskip0pt\parsep0pt
\item[]
Called by a service dispatch routine that refuses to perform a remote
procedure call due to an authentication error.
\end{description}

\begin{verbatim}
void svcerr_decode(SVCXPRT *xprt);
\end{verbatim}

\begin{description}
\itemsep1pt\parskip0pt\parsep0pt
\item[]
Called by a service dispatch routine that cannot successfully decode its
parameters. See also \textbf{svc\_getargs}().
\end{description}

\begin{verbatim}
void svcerr_noproc(SVCXPRT *xprt);
\end{verbatim}

\begin{description}
\itemsep1pt\parskip0pt\parsep0pt
\item[]
Called by a service dispatch routine that does not implement the
procedure number that the caller requests.
\end{description}

\begin{verbatim}
void svcerr_noprog(SVCXPRT *xprt);
\end{verbatim}

\begin{description}
\itemsep1pt\parskip0pt\parsep0pt
\item[]
Called when the desired program is not registered with the RPC package.
Service implementors usually do not need this routine.
\end{description}

\begin{verbatim}
void svcerr_progvers(SVCXPRT *xprt);
\end{verbatim}

\begin{description}
\itemsep1pt\parskip0pt\parsep0pt
\item[]
Called when the desired version of a program is not registered with the
RPC package. Service implementors usually do not need this routine.
\end{description}

\begin{verbatim}
void svcerr_systemerr(SVCXPRT *xprt);
\end{verbatim}

\begin{description}
\itemsep1pt\parskip0pt\parsep0pt
\item[]
Called by a service dispatch routine when it detects a system error not
covered by any particular protocol. For example, if a service can no
longer allocate storage, it may call this routine.
\end{description}

\begin{verbatim}
void svcerr_weakauth(SVCXPRT *xprt);
\end{verbatim}

\begin{description}
\itemsep1pt\parskip0pt\parsep0pt
\item[]
Called by a service dispatch routine that refuses to perform a remote
procedure call due to insufficient authentication parameters. The
routine calls \textbf{svcerr\_auth(xprt, AUTH\_TOOWEAK)}.
\end{description}

\begin{verbatim}
SVCXPRT *svcfd_create(int fd, unsigned int sendsize,
                      unsigned int recvsize);
\end{verbatim}

\begin{description}
\itemsep1pt\parskip0pt\parsep0pt
\item[]
Create a service on top of any open descriptor. Typically, this
descriptor is a connected socket for a stream protocol such as TCP.
\emph{sendsize} and \emph{recvsize} indicate sizes for the send and
receive buffers. If they are zero, a reasonable default is chosen.
\end{description}

\begin{verbatim}
SVCXPRT *svcraw_create(void);
\end{verbatim}

\begin{description}
\itemsep1pt\parskip0pt\parsep0pt
\item[]
This routine creates a toy RPC service transport, to which it returns a
pointer. The transport is really a buffer within the process's address
space, so the corresponding RPC client should live in the same address
space; see \textbf{clntraw\_create}(). This routine allows simulation of
RPC and acquisition of RPC overheads (such as round trip times), without
any kernel interference. This routine returns NULL if it fails.
\end{description}

\begin{verbatim}
SVCXPRT *svctcp_create(int sock, unsigned int send_buf_size,
                       unsigned int recv_buf_size);
\end{verbatim}

\begin{description}
\itemsep1pt\parskip0pt\parsep0pt
\item[]
This routine creates a TCP/IP-based RPC service transport, to which it
returns a pointer. The transport is associated with the socket
\emph{sock}, which may be \textbf{RPC\_ANYSOCK}, in which case a new
socket is created. If the socket is not bound to a local TCP port, then
this routine binds it to an arbitrary port. Upon completion,
\emph{xprt-\textgreater{}xp\_sock} is the transport's socket descriptor,
and \emph{xprt-\textgreater{}xp\_port} is the transport's port number.
This routine returns NULL if it fails. Since TCP-based RPC uses buffered
I/O, users may specify the size of buffers; values of zero choose
suitable defaults.
\end{description}

\begin{verbatim}
SVCXPRT *svcudp_bufcreate(int sock, unsigned int sendsize,
                          unsigned int recosize);
\end{verbatim}

\begin{description}
\itemsep1pt\parskip0pt\parsep0pt
\item[]
This routine creates a UDP/IP-based RPC service transport, to which it
returns a pointer. The transport is associated with the socket
\emph{sock}, which may be \textbf{RPC\_ANYSOCK}, in which case a new
socket is created. If the socket is not bound to a local UDP port, then
this routine binds it to an arbitrary port. Upon completion,
\emph{xprt-\textgreater{}xp\_sock} is the transport's socket descriptor,
and \emph{xprt-\textgreater{}xp\_port} is the transport's port number.
This routine returns NULL if it fails.
\end{description}

\begin{description}
\itemsep1pt\parskip0pt\parsep0pt
\item[]
This allows the user to specify the maximum packet size for sending and
receiving UDP-based RPC messages.
\end{description}

\begin{verbatim}
SVCXPRT *svcudp_create(int sock);
\end{verbatim}

\begin{description}
\itemsep1pt\parskip0pt\parsep0pt
\item[]
This call is equivalent to \emph{svcudp\_bufcreate(sock,SZ,SZ)} for some
default size \emph{SZ}.
\end{description}

\begin{verbatim}
bool_t xdr_accepted_reply(XDR *xdrs, struct accepted_reply *ar);
\end{verbatim}

\begin{description}
\itemsep1pt\parskip0pt\parsep0pt
\item[]
Used for encoding RPC reply messages. This routine is useful for users
who wish to generate RPC-style messages without using the RPC package.
\end{description}

\begin{verbatim}
bool_t xdr_authunix_parms(XDR *xdrs, struct authunix_parms *aupp);
\end{verbatim}

\begin{description}
\itemsep1pt\parskip0pt\parsep0pt
\item[]
Used for describing UNIX credentials. This routine is useful for users
who wish to generate these credentials without using the RPC
authentication package.
\end{description}

\begin{verbatim}
void xdr_callhdr(XDR *xdrs, struct rpc_msg *chdr);
\end{verbatim}

\begin{description}
\itemsep1pt\parskip0pt\parsep0pt
\item[]
Used for describing RPC call header messages. This routine is useful for
users who wish to generate RPC-style messages without using the RPC
package.
\end{description}

\begin{verbatim}
bool_t xdr_callmsg(XDR *xdrs, struct rpc_msg *cmsg);
\end{verbatim}

\begin{description}
\itemsep1pt\parskip0pt\parsep0pt
\item[]
Used for describing RPC call messages. This routine is useful for users
who wish to generate RPC-style messages without using the RPC package.
\end{description}

\begin{verbatim}
bool_t xdr_opaque_auth(XDR *xdrs, struct opaque_auth *ap);
\end{verbatim}

\begin{description}
\itemsep1pt\parskip0pt\parsep0pt
\item[]
Used for describing RPC authentication information messages. This
routine is useful for users who wish to generate RPC-style messages
without using the RPC package.
\end{description}

\begin{verbatim}
bool_t xdr_pmap(XDR *xdrs, struct pmap *regs);
\end{verbatim}

\begin{description}
\itemsep1pt\parskip0pt\parsep0pt
\item[]
Used for describing parameters to various \textbf{portmap} procedures,
externally. This routine is useful for users who wish to generate these
parameters without using the \textbf{pmap} interface.
\end{description}

\begin{verbatim}
bool_t xdr_pmaplist(XDR *xdrs, struct pmaplist **rp);
\end{verbatim}

\begin{description}
\itemsep1pt\parskip0pt\parsep0pt
\item[]
Used for describing a list of port mappings, externally. This routine is
useful for users who wish to generate these parameters without using the
\textbf{pmap} interface.
\end{description}

\begin{verbatim}
bool_t xdr_rejected_reply(XDR *xdrs, struct rejected_reply *rr);
\end{verbatim}

\begin{description}
\itemsep1pt\parskip0pt\parsep0pt
\item[]
Used for describing RPC reply messages. This routine is useful for users
who wish to generate RPC-style messages without using the RPC package.
\end{description}

\begin{verbatim}
bool_t xdr_replymsg(XDR *xdrs, struct rpc_msg *rmsg);
\end{verbatim}

\begin{description}
\itemsep1pt\parskip0pt\parsep0pt
\item[]
Used for describing RPC reply messages. This routine is useful for users
who wish to generate RPC style messages without using the RPC package.
\end{description}

\begin{verbatim}
void xprt_register(SVCXPRT *xprt);
\end{verbatim}

\begin{description}
\itemsep1pt\parskip0pt\parsep0pt
\item[]
After RPC service transport handles are created, they should register
themselves with the RPC service package. This routine modifies the
global variable \emph{svc\_fds}. Service implementors usually do not
need this routine.
\end{description}

\begin{verbatim}
void xprt_unregister(SVCXPRT *xprt);
\end{verbatim}

\begin{description}
\itemsep1pt\parskip0pt\parsep0pt
\item[]
Before an RPC service transport handle is destroyed, it should
unregister itself with the RPC service package. This routine modifies
the global variable \emph{svc\_fds}. Service implementors usually do not
need this routine.
\end{description}

\hyperdef{}{SEEux5fALSO}{\section{\hyperref[SEEux5fALSO]{SEE
ALSO}}\label{SEEux5fALSO}}

\textbf{xdr}(3)

~

The following manuals:

Remote Procedure Calls: Protocol Specification

~

Remote Procedure Call Programming Guide

~

rpcgen Programming Guide

~

~

\emph{RPC: Remote Procedure Call Protocol Specification}, RFC~1050, Sun
Microsystems, Inc., USC-ISI.

\hyperdef{}{COLOPHON}{\section{\hyperref[COLOPHON]{COLOPHON}}\label{COLOPHON}}

This page is part of release 3.54 of the Linux \emph{man-pages} project.
A description of the project, and information about reporting bugs, can
be found at http://www.kernel.org/doc/man-pages/.

\begin{longtable}[c]{@{}ll@{}}
\toprule\addlinespace
2008-07-17 &
\\\addlinespace
\bottomrule
\end{longtable}

\end{document}
