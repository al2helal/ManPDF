\documentclass[]{article}
\usepackage[T1]{fontenc}
\usepackage{lmodern}
\usepackage{amssymb,amsmath}
\usepackage{ifxetex,ifluatex}
\usepackage{fixltx2e} % provides \textsubscript
% use upquote if available, for straight quotes in verbatim environments
\IfFileExists{upquote.sty}{\usepackage{upquote}}{}
\ifnum 0\ifxetex 1\fi\ifluatex 1\fi=0 % if pdftex
  \usepackage[utf8]{inputenc}
\else % if luatex or xelatex
  \ifxetex
    \usepackage{mathspec}
    \usepackage{xltxtra,xunicode}
  \else
    \usepackage{fontspec}
  \fi
  \defaultfontfeatures{Mapping=tex-text,Scale=MatchLowercase}
  \newcommand{\euro}{€}
\fi
% use microtype if available
\IfFileExists{microtype.sty}{\usepackage{microtype}}{}
\usepackage{longtable,booktabs}
\ifxetex
  \usepackage[setpagesize=false, % page size defined by xetex
              unicode=false, % unicode breaks when used with xetex
              xetex]{hyperref}
\else
  \usepackage[unicode=true]{hyperref}
\fi
\hypersetup{breaklinks=true,
            bookmarks=true,
            pdfauthor={},
            pdftitle={ATEXIT(3)},
            colorlinks=true,
            citecolor=blue,
            urlcolor=blue,
            linkcolor=magenta,
            pdfborder={0 0 0}}
\urlstyle{same}  % don't use monospace font for urls
\setlength{\parindent}{0pt}
\setlength{\parskip}{6pt plus 2pt minus 1pt}
\setlength{\emergencystretch}{3em}  % prevent overfull lines
\setcounter{secnumdepth}{0}
\usepackage{pagecolor}

% Set background colour (of the page)
\definecolor{weirdbgcolor}{HTML}{FCF4F0}
\pagecolor{weirdbgcolor}

% Make bold text appear in a particular colour
\definecolor{boldcolor}{HTML}{6E0002}
\let\realtextbf=\textbf
\renewcommand{\textbf}[1]{\textcolor{boldcolor}{\realtextbf{#1}}}

% Use underlines instead of emphasis (ugh)
\renewcommand{\emph}[1]{\underline{#1}}

% % Use fixed-width font by default
% \renewcommand*\familydefault{\ttdefault}

\title{ATEXIT(3)}
\author{}
\date{}

\begin{document}
\maketitle

\begin{longtable}[c]{@{}lll@{}}
\toprule\addlinespace
ATEXIT(3) & Linux Programmer's Manual & ATEXIT(3)
\\\addlinespace
\bottomrule
\end{longtable}

\hyperdef{}{NAME}{\section{\hyperref[NAME]{NAME}}\label{NAME}}

atexit - register a function to be called at normal process termination

\hyperdef{}{SYNOPSIS}{\section{\hyperref[SYNOPSIS]{SYNOPSIS}}\label{SYNOPSIS}}

\begin{verbatim}
#include <stdlib.h>
 
int atexit(void (*function)(void));
\end{verbatim}

\hyperdef{}{DESCRIPTION}{\section{\hyperref[DESCRIPTION]{DESCRIPTION}}\label{DESCRIPTION}}

The \textbf{atexit}() function registers the given \emph{function} to be
called at normal process termination, either via \textbf{exit}(3) or via
return from the program's \emph{main}(). Functions so registered are
called in the reverse order of their registration; no arguments are
passed.

~

The same function may be registered multiple times: it is called once
for each registration.

POSIX.1-2001 requires that an implementation allow at least
\textbf{ATEXIT\_MAX} (32) such functions to be registered. The actual
limit supported by an implementation can be obtained using
\textbf{sysconf}(3).

When a child process is created via \textbf{fork}(2), it inherits copies
of its parent's registrations. Upon a successful call to one of the
\textbf{exec}(3) functions, all registrations are removed.

\hyperdef{}{RETURNux5fVALUE}{\section{\hyperref[RETURNux5fVALUE]{RETURN
VALUE}}\label{RETURNux5fVALUE}}

The \textbf{atexit}() function returns the value 0 if successful;
otherwise it returns a nonzero value.

\hyperdef{}{CONFORMINGux5fTO}{\section{\hyperref[CONFORMINGux5fTO]{CONFORMING
TO}}\label{CONFORMINGux5fTO}}

SVr4, 4.3BSD, C89, C99, POSIX.1-2001.

\hyperdef{}{NOTES}{\section{\hyperref[NOTES]{NOTES}}\label{NOTES}}

Functions registered using \textbf{atexit}() (and \textbf{on\_exit}(3))
are not called if a process terminates abnormally because of the
delivery of a signal.

~

If one of the functions registered functions calls \textbf{\_exit}(2),
then any remaining functions are not invoked, and the other process
termination steps performed by \textbf{exit}(3) are not performed.

~

POSIX.1-2001 says that the result of calling \textbf{exit}(3) more than
once (i.e., calling \textbf{exit}(3) within a function registered using
\textbf{atexit}()) is undefined. On some systems (but not Linux), this
can result in an infinite recursion; portable programs should not invoke
\textbf{exit}(3) inside a function registered using \textbf{atexit}().

~

The \textbf{atexit}() and \textbf{on\_exit}(3) functions register
functions on the same list: at normal process termination, the
registered functions are invoked in reverse order of their registration
by these two functions.

~

POSIX.1-2001 says that the result is undefined if \textbf{longjmp}(3) is
used to terminate execution of one of the functions registered
\textbf{atexit}().

\hyperdef{}{Linuxux5fnotes}{\subsection{\hyperref[Linuxux5fnotes]{Linux
notes}}\label{Linuxux5fnotes}}

Since glibc 2.2.3, \textbf{atexit}() (and \textbf{on\_exit}(3)) can be
used within a shared library to establish functions that are called when
the shared library is unloaded.

\hyperdef{}{EXAMPLE}{\section{\hyperref[EXAMPLE]{EXAMPLE}}\label{EXAMPLE}}

\begin{verbatim}
#include <stdio.h>
#include <stdlib.h>
#include <unistd.h>

void
bye(void)
{
    printf("That was all, folks\n");
}

int
main(void)
{
    long a;
    int i;

    a = sysconf(_SC_ATEXIT_MAX);
    printf("ATEXIT_MAX = %ld\n", a);

    i = atexit(bye);
    if (i != 0) {
        fprintf(stderr, "cannot set exit function\n");
        exit(EXIT_FAILURE);
    }

    exit(EXIT_SUCCESS);
}
\end{verbatim}

\hyperdef{}{SEEux5fALSO}{\section{\hyperref[SEEux5fALSO]{SEE
ALSO}}\label{SEEux5fALSO}}

\textbf{\_exit}(2), \textbf{exit}(3), \textbf{on\_exit}(3)

\hyperdef{}{COLOPHON}{\section{\hyperref[COLOPHON]{COLOPHON}}\label{COLOPHON}}

This page is part of release 3.54 of the Linux \emph{man-pages} project.
A description of the project, and information about reporting bugs, can
be found at http://www.kernel.org/doc/man-pages/.

\begin{longtable}[c]{@{}ll@{}}
\toprule\addlinespace
2008-12-05 & Linux
\\\addlinespace
\bottomrule
\end{longtable}

\end{document}
