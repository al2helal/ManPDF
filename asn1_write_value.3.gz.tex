\documentclass[]{article}
\usepackage[T1]{fontenc}
\usepackage{lmodern}
\usepackage{amssymb,amsmath}
\usepackage{ifxetex,ifluatex}
\usepackage{fixltx2e} % provides \textsubscript
% use upquote if available, for straight quotes in verbatim environments
\IfFileExists{upquote.sty}{\usepackage{upquote}}{}
\ifnum 0\ifxetex 1\fi\ifluatex 1\fi=0 % if pdftex
  \usepackage[utf8]{inputenc}
\else % if luatex or xelatex
  \ifxetex
    \usepackage{mathspec}
    \usepackage{xltxtra,xunicode}
  \else
    \usepackage{fontspec}
  \fi
  \defaultfontfeatures{Mapping=tex-text,Scale=MatchLowercase}
  \newcommand{\euro}{€}
\fi
% use microtype if available
\IfFileExists{microtype.sty}{\usepackage{microtype}}{}
\usepackage{longtable,booktabs}
\ifxetex
  \usepackage[setpagesize=false, % page size defined by xetex
              unicode=false, % unicode breaks when used with xetex
              xetex]{hyperref}
\else
  \usepackage[unicode=true]{hyperref}
\fi
\hypersetup{breaklinks=true,
            bookmarks=true,
            pdfauthor={},
            pdftitle={asn1\_write\_value(3)},
            colorlinks=true,
            citecolor=blue,
            urlcolor=blue,
            linkcolor=magenta,
            pdfborder={0 0 0}}
\urlstyle{same}  % don't use monospace font for urls
\setlength{\parindent}{0pt}
\setlength{\parskip}{6pt plus 2pt minus 1pt}
\setlength{\emergencystretch}{3em}  % prevent overfull lines
\setcounter{secnumdepth}{0}
\usepackage{pagecolor}

% Set background colour (of the page)
\definecolor{weirdbgcolor}{HTML}{FCF4F0}
\pagecolor{weirdbgcolor}

% Make bold text appear in a particular colour
\definecolor{boldcolor}{HTML}{6E0002}
\let\realtextbf=\textbf
\renewcommand{\textbf}[1]{\textcolor{boldcolor}{\realtextbf{#1}}}

% Use underlines instead of emphasis (ugh)
\renewcommand{\emph}[1]{\underline{#1}}

% % Use fixed-width font by default
% \renewcommand*\familydefault{\ttdefault}

\title{asn1\_write\_value(3)}
\author{}
\date{}

\begin{document}
\maketitle

\begin{longtable}[c]{@{}lll@{}}
\toprule\addlinespace
asn1\_write\_value(3) & libtasn1 & asn1\_write\_value(3)
\\\addlinespace
\bottomrule
\end{longtable}

\hyperdef{}{NAME}{\section{\hyperref[NAME]{NAME}}\label{NAME}}

asn1\_write\_value - API function

\hyperdef{}{SYNOPSIS}{\section{\hyperref[SYNOPSIS]{SYNOPSIS}}\label{SYNOPSIS}}

\textbf{\#include \textless{}libtasn1.h\textgreater{}}

~

\textbf{int asn1\_write\_value(asn1\_node}\emph{node\_root}\textbf{,
const char *}\emph{name}\textbf{, const void *}\emph{ivalue}\textbf{,
int}\emph{len}\textbf{);}

\hyperdef{}{ARGUMENTS}{\section{\hyperref[ARGUMENTS]{ARGUMENTS}}\label{ARGUMENTS}}

\begin{description}
\itemsep1pt\parskip0pt\parsep0pt
\item[asn1\_node node\_root]
pointer to a structure
\end{description}

\begin{description}
\itemsep1pt\parskip0pt\parsep0pt
\item[const char * name]
the name of the element inside the structure that you want to set.
\end{description}

\begin{description}
\itemsep1pt\parskip0pt\parsep0pt
\item[const void * ivalue]
vector used to specify the value to set. If len is \textgreater{}0,
VALUE must be a two's complement form integer. if len=0 *VALUE must be a
null terminated string with an integer value.
\end{description}

\begin{description}
\itemsep1pt\parskip0pt\parsep0pt
\item[int len]
number of bytes of *value to use to set the value:
value{[}0{]}..value{[}len-1{]} or 0 if value is a null terminated string
\end{description}

\hyperdef{}{DESCRIPTION}{\section{\hyperref[DESCRIPTION]{DESCRIPTION}}\label{DESCRIPTION}}

Set the value of one element inside a structure.

~

If an element is OPTIONAL and you want to delete it, you must use the
value=NULL and len=0. Using ``pkix.asn'':

~

result=asn1\_write\_value(cert, ``tbsCertificate.issuerUniqueID'', NULL,
0);

~

Description for each type:

\hyperdef{}{INTEGER}{\section{\hyperref[INTEGER]{INTEGER}}\label{INTEGER}}

VALUE must contain a two's complement form integer.

~

value{[}0{]}=0xFF , len=1 -\textgreater{} integer=-1. value{[}0{]}=0xFF
value{[}1{]}=0xFF , len=2 -\textgreater{} integer=-1. value{[}0{]}=0x01
, len=1 -\textgreater{} integer= 1. value{[}0{]}=0x00 value{[}1{]}=0x01
, len=2 -\textgreater{} integer= 1. value=``123'' , len=0
-\textgreater{} integer= 123.

\hyperdef{}{ENUMERATED}{\section{\hyperref[ENUMERATED]{ENUMERATED}}\label{ENUMERATED}}

As INTEGER (but only with not negative numbers).

\hyperdef{}{BOOLEAN}{\section{\hyperref[BOOLEAN]{BOOLEAN}}\label{BOOLEAN}}

VALUE must be the null terminated string ``TRUE'' or ``FALSE'' and LEN
!= 0.

~

value=``TRUE'' , len=1 -\textgreater{} boolean=TRUE. value=``FALSE'' ,
len=1 -\textgreater{} boolean=FALSE.

~

OBJECT IDENTIFIER: VALUE must be a null terminated string with each
number separated by a dot (e.g. ``1.2.3.543.1''). LEN != 0.

~

value=``1 2 840 10040 4 3'' , len=1 -\textgreater{} OID=dsa-with-sha.

\hyperdef{}{UTCTIME}{\section{\hyperref[UTCTIME]{UTCTIME}}\label{UTCTIME}}

VALUE must be a null terminated string in one of these formats:
``YYMMDDhhmmssZ'', ``YYMMDDhhmmssZ'', ``YYMMDDhhmmss+hh'mm''',
``YYMMDDhhmmss-hh'mm''', ``YYMMDDhhmm+hh'mm''', or
``YYMMDDhhmm-hh'mm'''. LEN != 0.

~

value=``9801011200Z'' , len=1 -\textgreater{} time=Jannuary 1st, 1998 at
12h 00m Greenwich Mean Time

\hyperdef{}{GENERALIZEDTIME}{\section{\hyperref[GENERALIZEDTIME]{GENERALIZEDTIME}}\label{GENERALIZEDTIME}}

VALUE must be in one of this format: ``YYYYMMDDhhmmss.sZ'',
``YYYYMMDDhhmmss.sZ'', ``YYYYMMDDhhmmss.s+hh'mm''',
``YYYYMMDDhhmmss.s-hh'mm''', ``YYYYMMDDhhmm+hh'mm''', or
``YYYYMMDDhhmm-hh'mm''' where ss.s indicates the seconds with any
precision like ``10.1'' or ``01.02''. LEN != 0

~

value=``2001010112001.12-0700'' , len=1 -\textgreater{} time=Jannuary
1st, 2001 at 12h 00m 01.12s Pacific Daylight Time

~

OCTET STRING: VALUE contains the octet string and LEN is the number of
octets.

~

value=``\$sh\$x01\$sh\$x02\$sh\$x03'' , len=3 -\textgreater{} three
bytes octet string

\hyperdef{}{GENERALSTRING}{\section{\hyperref[GENERALSTRING]{GENERALSTRING}}\label{GENERALSTRING}}

VALUE contains the generalstring and LEN is the number of octets.

~

value=``\$sh\$x01\$sh\$x02\$sh\$x03'' , len=3 -\textgreater{} three
bytes generalstring

~

BIT STRING: VALUE contains the bit string organized by bytes and LEN is
the number of bits.

~

value=``\$sh\$xCF'' , len=6 -\textgreater{} bit string=``110011'' (six
bits)

\hyperdef{}{CHOICE}{\section{\hyperref[CHOICE]{CHOICE}}\label{CHOICE}}

if NAME indicates a choice type, VALUE must specify one of the
alternatives with a null terminated string. LEN != 0. Using ``pkix.asn''

~

result=asn1\_write\_value(cert, ``certificate1.tbsCertificate.subject'',
``rdnSequence'', 1);

\hyperdef{}{ANY}{\section{\hyperref[ANY]{ANY}}\label{ANY}}

VALUE indicates the der encoding of a structure. LEN != 0.

~

SEQUENCE OF: VALUE must be the null terminated string ``NEW'' and LEN !=
0. With this instruction another element is appended in the sequence.
The name of this element will be ``?1'' if it's the first one, ``?2''
for the second and so on.

~

Using ``pkix.asn''

~

result=asn1\_write\_value(cert,
``certificate1.tbsCertificate.subject.rdnSequence'', ``NEW'', 1);

~

SET OF: the same as SEQUENCE OF. Using ``pkix.asn'':

~

result=asn1\_write\_value(cert,
``tbsCertificate.subject.rdnSequence.?LAST'', ``NEW'', 1);

\hyperdef{}{RETURNS}{\section{\hyperref[RETURNS]{RETURNS}}\label{RETURNS}}

\textbf{ASN1\_SUCCESS} if the value was set,
\textbf{ASN1\_ELEMENT\_NOT\_FOUND} if \emph{name} is not a valid
element, and \textbf{ASN1\_VALUE\_NOT\_VALID} if \emph{ivalue} has a
wrong format.

\hyperdef{}{COPYRIGHT}{\section{\hyperref[COPYRIGHT]{COPYRIGHT}}\label{COPYRIGHT}}

Copyright © 2006-2013 Free Software Foundation, Inc..

~

Copying and distribution of this file, with or without modification, are
permitted in any medium without royalty provided the copyright notice
and this notice are preserved.

\hyperdef{}{SEEux5fALSO}{\section{\hyperref[SEEux5fALSO]{SEE
ALSO}}\label{SEEux5fALSO}}

The full documentation for \textbf{libtasn1} is maintained as a Texinfo
manual. If the \textbf{info} and \textbf{libtasn1} programs are properly
installed at your site, the command

\begin{description}
\itemsep1pt\parskip0pt\parsep0pt
\item[]
\textbf{info libtasn1}
\end{description}

should give you access to the complete manual. As an alternative you may
obtain the manual from:

\begin{description}
\itemsep1pt\parskip0pt\parsep0pt
\item[]
\textbf{http://www.gnu.org/software/libtasn1/manual/}
\end{description}

\begin{longtable}[c]{@{}ll@{}}
\toprule\addlinespace
3.4 & libtasn1
\\\addlinespace
\bottomrule
\end{longtable}

\end{document}
