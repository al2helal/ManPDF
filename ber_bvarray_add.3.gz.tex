\documentclass[]{article}
\usepackage[T1]{fontenc}
\usepackage{lmodern}
\usepackage{amssymb,amsmath}
\usepackage{ifxetex,ifluatex}
\usepackage{fixltx2e} % provides \textsubscript
% use upquote if available, for straight quotes in verbatim environments
\IfFileExists{upquote.sty}{\usepackage{upquote}}{}
\ifnum 0\ifxetex 1\fi\ifluatex 1\fi=0 % if pdftex
  \usepackage[utf8]{inputenc}
\else % if luatex or xelatex
  \ifxetex
    \usepackage{mathspec}
    \usepackage{xltxtra,xunicode}
  \else
    \usepackage{fontspec}
  \fi
  \defaultfontfeatures{Mapping=tex-text,Scale=MatchLowercase}
  \newcommand{\euro}{€}
\fi
% use microtype if available
\IfFileExists{microtype.sty}{\usepackage{microtype}}{}
\usepackage{longtable,booktabs}
\ifxetex
  \usepackage[setpagesize=false, % page size defined by xetex
              unicode=false, % unicode breaks when used with xetex
              xetex]{hyperref}
\else
  \usepackage[unicode=true]{hyperref}
\fi
\hypersetup{breaklinks=true,
            bookmarks=true,
            pdfauthor={},
            pdftitle={LBER\_TYPES(3)},
            colorlinks=true,
            citecolor=blue,
            urlcolor=blue,
            linkcolor=magenta,
            pdfborder={0 0 0}}
\urlstyle{same}  % don't use monospace font for urls
\setlength{\parindent}{0pt}
\setlength{\parskip}{6pt plus 2pt minus 1pt}
\setlength{\emergencystretch}{3em}  % prevent overfull lines
\setcounter{secnumdepth}{0}
\usepackage{pagecolor}

% Set background colour (of the page)
\definecolor{weirdbgcolor}{HTML}{FCF4F0}
\pagecolor{weirdbgcolor}

% Make bold text appear in a particular colour
\definecolor{boldcolor}{HTML}{6E0002}
\let\realtextbf=\textbf
\renewcommand{\textbf}[1]{\textcolor{boldcolor}{\realtextbf{#1}}}

% Use underlines instead of emphasis (ugh)
\renewcommand{\emph}[1]{\underline{#1}}

% % Use fixed-width font by default
% \renewcommand*\familydefault{\ttdefault}

\title{LBER\_TYPES(3)}
\author{}
\date{}

\begin{document}
\maketitle

\begin{longtable}[c]{@{}lll@{}}
\toprule\addlinespace
LBER\_TYPES(3) & Library Functions Manual & LBER\_TYPES(3)
\\\addlinespace
\bottomrule
\end{longtable}

\hyperdef{}{NAME}{\section{\hyperref[NAME]{NAME}}\label{NAME}}

ber\_int\_t, ber\_uint\_t, ber\_len\_t, ber\_slen\_t, ber\_tag\_t,
struct berval, BerValue, BerVarray, BerElement, ber\_bvfree,
ber\_bvecfree, ber\_bvecadd, ber\_bvarray\_free, ber\_bvarray\_add,
ber\_bvdup, ber\_dupbv, ber\_bvstr, ber\_bvstrdup, ber\_str2bv,
ber\_alloc\_t, ber\_init, ber\_init2, ber\_free - OpenLDAP LBER types
and allocation functions

\hyperdef{}{LIBRARY}{\section{\hyperref[LIBRARY]{LIBRARY}}\label{LIBRARY}}

OpenLDAP LBER (liblber, -llber)

\hyperdef{}{SYNOPSIS}{\section{\hyperref[SYNOPSIS]{SYNOPSIS}}\label{SYNOPSIS}}

\textbf{\#include \textless{}lber.h\textgreater{}}

\begin{verbatim}
typedef impl_tag_t ber_tag_t;
typedef impl_int_t ber_int_t;
typedef impl_uint_t ber_uint_t;
typedef impl_len_t ber_len_t;
typedef impl_slen_t ber_slen_t;

typedef struct berval {
    ber_len_t bv_len;
    char *bv_val;
} BerValue, *BerVarray;

typedef struct berelement BerElement;
\end{verbatim}

\textbf{void ber\_bvfree(struct berval *}\emph{bv}\textbf{);}

\textbf{void ber\_bvecfree(struct berval **}\emph{bvec}\textbf{);}

\textbf{void ber\_bvecadd(struct berval ***}\emph{bvec}\textbf{, struct
berval *}\emph{bv}\textbf{);}

\textbf{void ber\_bvarray\_free(struct berval
*}\emph{bvarray}\textbf{);}

\textbf{void ber\_bvarray\_add(BerVarray *}\emph{bvarray}\textbf{,
BerValue *}\emph{bv}\textbf{);}

\textbf{struct berval *ber\_bvdup(const struct berval
*}\emph{bv}\textbf{);}

\textbf{struct berval *ber\_dupbv(const struct berval
*}\emph{dst}\textbf{, struct berval *}\emph{src}\textbf{);}

\textbf{struct berval *ber\_bvstr(const char *}\emph{str}\textbf{);}

\textbf{struct berval *ber\_bvstrdup(const char *}\emph{str}\textbf{);}

\textbf{struct berval *ber\_str2bv(const char *}\emph{str}\textbf{,
ber\_len\_t}\emph{len}\textbf{, int}\emph{dup}\textbf{, struct berval
*}\emph{bv}\textbf{);}

\textbf{BerElement *ber\_alloc\_t(int}\emph{options}\textbf{);}

\textbf{BerElement *ber\_init(struct berval *}\emph{bv}\textbf{);}

\textbf{void ber\_init2(BerElement *}\emph{ber}\textbf{, struct berval
*}\emph{bv}\textbf{, int}\emph{options}\textbf{);}

\textbf{void ber\_free(BerElement *}\emph{ber}\textbf{,
int}\emph{freebuf}\textbf{);}

\hyperdef{}{DESCRIPTION}{\section{\hyperref[DESCRIPTION]{DESCRIPTION}}\label{DESCRIPTION}}

The following are the basic types and structures defined for use with
the Lightweight BER library.

\textbf{ber\_int\_t} is a signed integer of at least 32 bits. It is
commonly equivalent to \textbf{int}. \textbf{ber\_uint\_t} is the
unsigned variant of \textbf{ber\_int\_t}.

\textbf{ber\_len\_t} is an unsigned integer of at least 32 bits used to
represent a length. It is commonly equivalent to a \textbf{size\_t}.
\textbf{ber\_slen\_t} is the signed variant to \textbf{ber\_len\_t}.

\textbf{ber\_tag\_t} is an unsigned integer of at least 32 bits used to
represent a BER tag. It is commonly equivalent to a
\textbf{unsigned~long}.

The actual definitions of the integral impl\_TYPE\_t types are platform
specific.

\textbf{BerValue}, commonly used as \textbf{struct~berval}, is used to
hold an arbitrary sequence of octets. \textbf{bv\_val} points to
\textbf{bv\_len} octets. \textbf{bv\_val} is not necessarily terminated
by a NULL (zero) octet. \textbf{ber\_bvfree}() frees a BerValue, pointed
to by \emph{bv}, returned from this API. If \emph{bv} is NULL, the
routine does nothing.

\textbf{ber\_bvecfree}() frees an array of BerValues (and the array),
pointed to by \emph{bvec}, returned from this API. If \emph{bvec} is
NULL, the routine does nothing. \textbf{ber\_bvecadd}() appends the
\emph{bv} pointer to the \emph{bvec} array. Space for the array is
allocated as needed. The end of the array is marked by a NULL pointer.

\textbf{ber\_bvarray\_free}() frees an array of BerValues (and the
array), pointed to by \emph{bvarray}, returned from this API. If
\emph{bvarray} is NULL, the routine does nothing.
\textbf{ber\_bvarray\_add}() appends the contents of the BerValue
pointed to by \emph{bv} to the \emph{bvarray} array. Space for the new
element is allocated as needed. The end of the array is marked by a
BerValue with a NULL bv\_val field.

\textbf{ber\_bvdup}() returns a copy of a BerValue. The routine returns
NULL upon error (e.g. out of memory). The caller should use
\textbf{ber\_bvfree}() to deallocate the resulting BerValue.
\textbf{ber\_dupbv}() copies a BerValue from \emph{src} to \emph{dst}.
If \emph{dst} is NULL a new BerValue will be allocated to hold the copy.
The routine returns NULL upon error, otherwise it returns a pointer to
the copy. If \emph{dst} is NULL the caller should use
\textbf{ber\_bvfree}() to deallocate the resulting BerValue, otherwise
\textbf{ber\_memfree}() should be used to deallocate the
\emph{dst-\textgreater{}bv\_val}. (The \textbf{ber\_bvdup}() function is
internally implemented as ber\_dupbv(NULL, bv). \textbf{ber\_bvdup}() is
provided only for compatibility with an expired draft of the LDAP C API;
\textbf{ber\_dupbv}() is the preferred interface.)

\textbf{ber\_bvstr}() returns a BerValue containing the string pointed
to by \emph{str}. \textbf{ber\_bvstrdup}() returns a BerValue containing
a copy of the string pointed to by \emph{str}. \textbf{ber\_str2bv}()
returns a BerValue containing the string pointed to by \emph{str}, whose
length may be optionally specified in \emph{len}. If \emph{dup} is
non-zero, the BerValue will contain a copy of \emph{str}. If \emph{len}
is zero, the number of bytes to copy will be determined by
\textbf{strlen}(3), otherwise \emph{len} bytes will be copied. If
\emph{bv} is non-NULL, the result will be stored in the given BerValue,
otherwise a new BerValue will be allocated to store the result. NOTE:
Both \textbf{ber\_bvstr}() and \textbf{ber\_bvstrdup}() are implemented
as macros using \textbf{ber\_str2bv}() in this version of the library.

\textbf{BerElement} is an opaque structure used to maintain state
information used in encoding and decoding. \textbf{ber\_alloc\_t}() is
used to create an empty BerElement structure. If \textbf{LBER\_USE\_DER}
is specified for the \emph{options} parameter then data lengths for data
written to the BerElement will be encoded in the minimal number of
octets required, otherwise they will always be written as four byte
values. \textbf{ber\_init}() creates a BerElement structure that is
initialized with a copy of the data in its \emph{bv} parameter.
\textbf{ber\_init2}() initializes an existing BerElement \emph{ber}
using the data in the \emph{bv} parameter. The data is referenced
directly, not copied. The \emph{options} parameter is the same as for
\textbf{ber\_alloc\_t}(). \textbf{ber\_free}() frees a BerElement
pointed to by \emph{ber}. If \emph{ber} is NULL, the routine does
nothing. If \emph{freebuf} is zero, the internal buffer is not freed.

\hyperdef{}{SEEux5fALSO}{\section{\hyperref[SEEux5fALSO]{SEE
ALSO}}\label{SEEux5fALSO}}

\textbf{lber-encode}(3), \textbf{lber-decode}(3),
\textbf{lber-memory}(3)

\hyperdef{}{ACKNOWLEDGEMENTS}{\section{\hyperref[ACKNOWLEDGEMENTS]{ACKNOWLEDGEMENTS}}\label{ACKNOWLEDGEMENTS}}

\textbf{OpenLDAP Software} is developed and maintained by The OpenLDAP
Project \textless{}http://www.openldap.org/\textgreater{}.
\textbf{OpenLDAP Software} is derived from University of Michigan LDAP
3.3 Release.

\begin{longtable}[c]{@{}ll@{}}
\toprule\addlinespace
2012/04/23 & OpenLDAP
\\\addlinespace
\bottomrule
\end{longtable}

\end{document}
