\documentclass[]{article}
\usepackage[T1]{fontenc}
\usepackage{lmodern}
\usepackage{amssymb,amsmath}
\usepackage{ifxetex,ifluatex}
\usepackage{fixltx2e} % provides \textsubscript
% use upquote if available, for straight quotes in verbatim environments
\IfFileExists{upquote.sty}{\usepackage{upquote}}{}
\ifnum 0\ifxetex 1\fi\ifluatex 1\fi=0 % if pdftex
  \usepackage[utf8]{inputenc}
\else % if luatex or xelatex
  \ifxetex
    \usepackage{mathspec}
    \usepackage{xltxtra,xunicode}
  \else
    \usepackage{fontspec}
  \fi
  \defaultfontfeatures{Mapping=tex-text,Scale=MatchLowercase}
  \newcommand{\euro}{€}
\fi
% use microtype if available
\IfFileExists{microtype.sty}{\usepackage{microtype}}{}
\usepackage{longtable,booktabs}
\ifxetex
  \usepackage[setpagesize=false, % page size defined by xetex
              unicode=false, % unicode breaks when used with xetex
              xetex]{hyperref}
\else
  \usepackage[unicode=true]{hyperref}
\fi
\hypersetup{breaklinks=true,
            bookmarks=true,
            pdfauthor={},
            pdftitle={SCANDIR(3)},
            colorlinks=true,
            citecolor=blue,
            urlcolor=blue,
            linkcolor=magenta,
            pdfborder={0 0 0}}
\urlstyle{same}  % don't use monospace font for urls
\setlength{\parindent}{0pt}
\setlength{\parskip}{6pt plus 2pt minus 1pt}
\setlength{\emergencystretch}{3em}  % prevent overfull lines
\setcounter{secnumdepth}{0}
\usepackage{pagecolor}

% Set background colour (of the page)
\definecolor{weirdbgcolor}{HTML}{FCF4F0}
\pagecolor{weirdbgcolor}

% Make bold text appear in a particular colour
\definecolor{boldcolor}{HTML}{6E0002}
\let\realtextbf=\textbf
\renewcommand{\textbf}[1]{\textcolor{boldcolor}{\realtextbf{#1}}}

% Use underlines instead of emphasis (ugh)
\renewcommand{\emph}[1]{\underline{#1}}

% % Use fixed-width font by default
% \renewcommand*\familydefault{\ttdefault}

\title{SCANDIR(3)}
\author{}
\date{}

\begin{document}
\maketitle

\begin{longtable}[c]{@{}lll@{}}
\toprule\addlinespace
SCANDIR(3) & Linux Programmer's Manual & SCANDIR(3)
\\\addlinespace
\bottomrule
\end{longtable}

\hyperdef{}{NAME}{\section{\hyperref[NAME]{NAME}}\label{NAME}}

scandir, alphasort, versionsort - scan a directory for matching entries

\hyperdef{}{SYNOPSIS}{\section{\hyperref[SYNOPSIS]{SYNOPSIS}}\label{SYNOPSIS}}

\begin{verbatim}
#include <dirent.h>
 
int scandir(const char *dirp, struct dirent ***namelist,
\end{verbatim}

\begin{verbatim}
int (*filter)(const struct dirent *),
int (*compar)(const struct dirent **, const struct dirent **));
\end{verbatim}

\begin{verbatim}
 
int alphasort(const void *a, const void *b);
 
int versionsort(const void *a, const void *b);
\end{verbatim}

~

Feature Test Macro Requirements for glibc (see
\textbf{feature\_test\_macros}(7)): \\

~

\textbf{scandir}(), \textbf{alphasort}(): \_BSD\_SOURCE
\textbar{}\textbar{} \_SVID\_SOURCE

~

\textbf{versionsort}(): \_GNU\_SOURCE

\hyperdef{}{DESCRIPTION}{\section{\hyperref[DESCRIPTION]{DESCRIPTION}}\label{DESCRIPTION}}

The \textbf{scandir}() function scans the directory \emph{dirp}, calling
\emph{filter}() on each directory entry. Entries for which
\emph{filter}() returns nonzero are stored in strings allocated via
\textbf{malloc}(3), sorted using \textbf{qsort}(3) with the comparison
function \emph{compar}(), and collected in array \emph{namelist} which
is allocated via \textbf{malloc}(3). If \emph{filter} is NULL, all
entries are selected.

The \textbf{alphasort}() and \textbf{versionsort}() functions can be
used as the comparison function \emph{compar}(). The former sorts
directory entries using \textbf{strcoll}(3), the latter using
\textbf{strverscmp}(3) on the strings \emph{(*a)-\textgreater{}d\_name}
and \emph{(*b)-\textgreater{}d\_name}.

\hyperdef{}{RETURNux5fVALUE}{\section{\hyperref[RETURNux5fVALUE]{RETURN
VALUE}}\label{RETURNux5fVALUE}}

The \textbf{scandir}() function returns the number of directory entries
selected. On error, -1 is returned, with \emph{errno} set to indicate
the cause of the error.

The \textbf{alphasort}() and \textbf{versionsort}() functions return an
integer less than, equal to, or greater than zero if the first argument
is considered to be respectively less than, equal to, or greater than
the second.

\hyperdef{}{ERRORS}{\section{\hyperref[ERRORS]{ERRORS}}\label{ERRORS}}

\begin{description}
\itemsep1pt\parskip0pt\parsep0pt
\item[\textbf{ENOENT}]
The path in \emph{dirp} does not exist.
\end{description}

\begin{description}
\itemsep1pt\parskip0pt\parsep0pt
\item[\textbf{ENOMEM}]
Insufficient memory to complete the operation.
\end{description}

\begin{description}
\itemsep1pt\parskip0pt\parsep0pt
\item[\textbf{ENOTDIR}]
The path in \emph{dirp} is not a directory.
\end{description}

\hyperdef{}{VERSIONS}{\section{\hyperref[VERSIONS]{VERSIONS}}\label{VERSIONS}}

\textbf{versionsort}() was added to glibc in version 2.1.

\hyperdef{}{CONFORMINGux5fTO}{\section{\hyperref[CONFORMINGux5fTO]{CONFORMING
TO}}\label{CONFORMINGux5fTO}}

\textbf{alphasort}() and \textbf{scandir}() are specified in
POSIX.1-2008, and are widely available. \textbf{versionsort}() is a GNU
extension.

The functions \textbf{scandir}() and \textbf{alphasort}() are from
4.3BSD, and have been available under Linux since libc4. Libc4 and libc5
use the more precise prototype

~

\begin{verbatim}
    int alphasort(const struct dirent ** a,
                  const struct dirent **b);
\end{verbatim}

~

but glibc 2.0 returns to the imprecise BSD prototype.

The function \textbf{versionsort}() is a GNU extension, available since
glibc 2.1.

Since glibc 2.1, \textbf{alphasort}() calls \textbf{strcoll}(3); earlier
it used \textbf{strcmp}(3).

\hyperdef{}{EXAMPLE}{\section{\hyperref[EXAMPLE]{EXAMPLE}}\label{EXAMPLE}}

\begin{verbatim}
#define _SVID_SOURCE
/* print files in current directory in reverse order */
#include <dirent.h>

int
main(void)
{
    struct dirent **namelist;
    int n;

    n = scandir(".", &namelist, NULL, alphasort);
    if (n < 0)
        perror("scandir");
    else {
        while (n--) {
            printf("%s\n", namelist[n]->d_name);
            free(namelist[n]);
        }
        free(namelist);
    }
}
\end{verbatim}

\hyperdef{}{SEEux5fALSO}{\section{\hyperref[SEEux5fALSO]{SEE
ALSO}}\label{SEEux5fALSO}}

\textbf{closedir}(3), \textbf{fnmatch}(3), \textbf{opendir}(3),
\textbf{readdir}(3), \textbf{rewinddir}(3), \textbf{scandirat}(3),
\textbf{seekdir}(3), \textbf{strcmp}(3), \textbf{strcoll}(3),
\textbf{strverscmp}(3), \textbf{telldir}(3)

\hyperdef{}{COLOPHON}{\section{\hyperref[COLOPHON]{COLOPHON}}\label{COLOPHON}}

This page is part of release 3.54 of the Linux \emph{man-pages} project.
A description of the project, and information about reporting bugs, can
be found at http://www.kernel.org/doc/man-pages/.

\begin{longtable}[c]{@{}ll@{}}
\toprule\addlinespace
2013-04-19 & GNU
\\\addlinespace
\bottomrule
\end{longtable}

\end{document}
