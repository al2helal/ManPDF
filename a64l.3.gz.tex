\documentclass[]{article}
\usepackage[T1]{fontenc}
\usepackage{lmodern}
\usepackage{amssymb,amsmath}
\usepackage{ifxetex,ifluatex}
\usepackage{fixltx2e} % provides \textsubscript
% use upquote if available, for straight quotes in verbatim environments
\IfFileExists{upquote.sty}{\usepackage{upquote}}{}
\ifnum 0\ifxetex 1\fi\ifluatex 1\fi=0 % if pdftex
  \usepackage[utf8]{inputenc}
\else % if luatex or xelatex
  \ifxetex
    \usepackage{mathspec}
    \usepackage{xltxtra,xunicode}
  \else
    \usepackage{fontspec}
  \fi
  \defaultfontfeatures{Mapping=tex-text,Scale=MatchLowercase}
  \newcommand{\euro}{€}
\fi
% use microtype if available
\IfFileExists{microtype.sty}{\usepackage{microtype}}{}
\usepackage{longtable,booktabs}
\ifxetex
  \usepackage[setpagesize=false, % page size defined by xetex
              unicode=false, % unicode breaks when used with xetex
              xetex]{hyperref}
\else
  \usepackage[unicode=true]{hyperref}
\fi
\hypersetup{breaklinks=true,
            bookmarks=true,
            pdfauthor={},
            pdftitle={A64L(3)},
            colorlinks=true,
            citecolor=blue,
            urlcolor=blue,
            linkcolor=magenta,
            pdfborder={0 0 0}}
\urlstyle{same}  % don't use monospace font for urls
\setlength{\parindent}{0pt}
\setlength{\parskip}{6pt plus 2pt minus 1pt}
\setlength{\emergencystretch}{3em}  % prevent overfull lines
\setcounter{secnumdepth}{0}
\usepackage{pagecolor}

% Set background colour (of the page)
\definecolor{weirdbgcolor}{HTML}{FCF4F0}
\pagecolor{weirdbgcolor}

% Make bold text appear in a particular colour
\definecolor{boldcolor}{HTML}{6E0002}
\let\realtextbf=\textbf
\renewcommand{\textbf}[1]{\textcolor{boldcolor}{\realtextbf{#1}}}

% Use underlines instead of emphasis (ugh)
\renewcommand{\emph}[1]{\underline{#1}}

% % Use fixed-width font by default
% \renewcommand*\familydefault{\ttdefault}

\title{A64L(3)}
\author{}
\date{}

\begin{document}
\maketitle

\begin{longtable}[c]{@{}lll@{}}
\toprule\addlinespace
A64L(3) & Linux Programmer's Manual & A64L(3)
\\\addlinespace
\bottomrule
\end{longtable}

\hyperdef{}{NAME}{\section{\hyperref[NAME]{NAME}}\label{NAME}}

a64l, l64a - convert between long and base-64

\hyperdef{}{SYNOPSIS}{\section{\hyperref[SYNOPSIS]{SYNOPSIS}}\label{SYNOPSIS}}

\textbf{\#include \textless{}stdlib.h\textgreater{}}

~

\textbf{long a64l(char *}\emph{str64}\textbf{);}

~

\textbf{char *l64a(long}\emph{value}\textbf{);}

~

Feature Test Macro Requirements for glibc (see
\textbf{feature\_test\_macros}(7)): \\

~

\textbf{a64l}(), \textbf{l64a}():

~

\_SVID\_SOURCE \textbar{}\textbar{} \_XOPEN\_SOURCE~\textgreater{}=~500
\textbar{}\textbar{} \_XOPEN\_SOURCE~\&\&~\_XOPEN\_SOURCE\_EXTENDED

\hyperdef{}{DESCRIPTION}{\section{\hyperref[DESCRIPTION]{DESCRIPTION}}\label{DESCRIPTION}}

These functions provide a conversion between 32-bit long integers and
little-endian base-64 ASCII strings (of length zero to six). If the
string used as argument for \textbf{a64l}() has length greater than six,
only the first six bytes are used. If the type \emph{long} has more than
32 bits, then \textbf{l64a}() uses only the low order 32 bits of
\emph{value}, and \textbf{a64l}() sign-extends its 32-bit result.

The 64 digits in the base-64 system are:

\begin{verbatim}

'.' represents a 0
'/' represents a 1
0-9 represent  2-11
A-Z represent 12-37
a-z represent 38-63
\end{verbatim}

So 123 = 59*64\^{}0 + 1*64\^{}1 = ``v/''.

\hyperdef{}{ATTRIBUTES}{\section{\hyperref[ATTRIBUTES]{ATTRIBUTES}}\label{ATTRIBUTES}}

\hyperdef{}{Multithreadingux5fux28seeux5fpthreadsux287ux29ux29}{\subsection{\hyperref[Multithreadingux5fux28seeux5fpthreadsux287ux29ux29]{Multithreading
(see
pthreads(7))}}\label{Multithreadingux5fux28seeux5fpthreadsux287ux29ux29}}

The \textbf{l64a}() function is not thread-safe.

The \textbf{a64l}() function is thread-safe.

\hyperdef{}{CONFORMINGux5fTO}{\section{\hyperref[CONFORMINGux5fTO]{CONFORMING
TO}}\label{CONFORMINGux5fTO}}

POSIX.1-2001.

\hyperdef{}{NOTES}{\section{\hyperref[NOTES]{NOTES}}\label{NOTES}}

The value returned by \textbf{l64a}() may be a pointer to a static
buffer, possibly overwritten by later calls.

The behavior of \textbf{l64a}() is undefined when \emph{value} is
negative. If \emph{value} is zero, it returns an empty string.

These functions are broken in glibc before 2.2.5 (puts most significant
digit first).

This is not the encoding used by \textbf{uuencode}(1).

\hyperdef{}{SEEux5fALSO}{\section{\hyperref[SEEux5fALSO]{SEE
ALSO}}\label{SEEux5fALSO}}

\textbf{uuencode}(1), \textbf{strtoul}(3)

\hyperdef{}{COLOPHON}{\section{\hyperref[COLOPHON]{COLOPHON}}\label{COLOPHON}}

This page is part of release 3.54 of the Linux \emph{man-pages} project.
A description of the project, and information about reporting bugs, can
be found at http://www.kernel.org/doc/man-pages/.

\begin{longtable}[c]{@{}ll@{}}
\toprule\addlinespace
2013-06-21 &
\\\addlinespace
\bottomrule
\end{longtable}

\end{document}
