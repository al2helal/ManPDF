\documentclass[]{article}
\usepackage[T1]{fontenc}
\usepackage{lmodern}
\usepackage{amssymb,amsmath}
\usepackage{ifxetex,ifluatex}
\usepackage{fixltx2e} % provides \textsubscript
% use upquote if available, for straight quotes in verbatim environments
\IfFileExists{upquote.sty}{\usepackage{upquote}}{}
\ifnum 0\ifxetex 1\fi\ifluatex 1\fi=0 % if pdftex
  \usepackage[utf8]{inputenc}
\else % if luatex or xelatex
  \ifxetex
    \usepackage{mathspec}
    \usepackage{xltxtra,xunicode}
  \else
    \usepackage{fontspec}
  \fi
  \defaultfontfeatures{Mapping=tex-text,Scale=MatchLowercase}
  \newcommand{\euro}{€}
\fi
% use microtype if available
\IfFileExists{microtype.sty}{\usepackage{microtype}}{}
\usepackage{longtable,booktabs}
\ifxetex
  \usepackage[setpagesize=false, % page size defined by xetex
              unicode=false, % unicode breaks when used with xetex
              xetex]{hyperref}
\else
  \usepackage[unicode=true]{hyperref}
\fi
\hypersetup{breaklinks=true,
            bookmarks=true,
            pdfauthor={},
            pdftitle={MALLOC\_HOOK(3)},
            colorlinks=true,
            citecolor=blue,
            urlcolor=blue,
            linkcolor=magenta,
            pdfborder={0 0 0}}
\urlstyle{same}  % don't use monospace font for urls
\setlength{\parindent}{0pt}
\setlength{\parskip}{6pt plus 2pt minus 1pt}
\setlength{\emergencystretch}{3em}  % prevent overfull lines
\setcounter{secnumdepth}{0}
\usepackage{pagecolor}

% Set background colour (of the page)
\definecolor{weirdbgcolor}{HTML}{FCF4F0}
\pagecolor{weirdbgcolor}

% Make bold text appear in a particular colour
\definecolor{boldcolor}{HTML}{6E0002}
\let\realtextbf=\textbf
\renewcommand{\textbf}[1]{\textcolor{boldcolor}{\realtextbf{#1}}}

% Use underlines instead of emphasis (ugh)
\renewcommand{\emph}[1]{\underline{#1}}

% % Use fixed-width font by default
% \renewcommand*\familydefault{\ttdefault}

\title{MALLOC\_HOOK(3)}
\author{}
\date{}

\begin{document}
\maketitle

\begin{longtable}[c]{@{}lll@{}}
\toprule\addlinespace
MALLOC\_HOOK(3) & Linux Programmer's Manual & MALLOC\_HOOK(3)
\\\addlinespace
\bottomrule
\end{longtable}

\hyperdef{}{NAME}{\section{\hyperref[NAME]{NAME}}\label{NAME}}

\_\_malloc\_hook, \_\_malloc\_initialize\_hook, \_\_memalign\_hook,
\_\_free\_hook, \_\_realloc\_hook, \_\_after\_morecore\_hook - malloc
debugging variables

\hyperdef{}{SYNOPSIS}{\section{\hyperref[SYNOPSIS]{SYNOPSIS}}\label{SYNOPSIS}}

\begin{verbatim}
#include <malloc.h>
 
void *(*__malloc_hook)(size_t size, const void *caller);
 
void *(*__realloc_hook)(void *ptr, size_t size, const void *caller);
 
void *(*__memalign_hook)(size_t alignment, size_t size,
                         const void *caller);
 
void (*__free_hook)(void *ptr, const void *caller);
 
void (*__malloc_initialize_hook)(void);
 
void (*__after_morecore_hook)(void);
\end{verbatim}

\hyperdef{}{DESCRIPTION}{\section{\hyperref[DESCRIPTION]{DESCRIPTION}}\label{DESCRIPTION}}

The GNU C library lets you modify the behavior of \textbf{malloc}(3),
\textbf{realloc}(3), and \textbf{free}(3) by specifying appropriate hook
functions. You can use these hooks to help you debug programs that use
dynamic memory allocation, for example.

The variable \textbf{\_\_malloc\_initialize\_hook} points at a function
that is called once when the malloc implementation is initialized. This
is a weak variable, so it can be overridden in the application with a
definition like the following:

\begin{verbatim}

    void (*__malloc_initialize_hook)(void) = my_init_hook;
\end{verbatim}

Now the function \emph{my\_init\_hook}() can do the initialization of
all hooks.

The four functions pointed to by \textbf{\_\_malloc\_hook},
\textbf{\_\_realloc\_hook}, \textbf{\_\_memalign\_hook},
\textbf{\_\_free\_hook} have a prototype like the functions
\textbf{malloc}(3), \textbf{realloc}(3), \textbf{memalign}(3),
\textbf{free}(3), respectively, except that they have a final argument
\emph{caller} that gives the address of the caller of
\textbf{malloc}(3), etc.

The variable \textbf{\_\_after\_morecore\_hook} points at a function
that is called each time after \textbf{sbrk}(2) was asked for more
memory.

\hyperdef{}{CONFORMINGux5fTO}{\section{\hyperref[CONFORMINGux5fTO]{CONFORMING
TO}}\label{CONFORMINGux5fTO}}

These functions are GNU extensions.

\hyperdef{}{NOTES}{\section{\hyperref[NOTES]{NOTES}}\label{NOTES}}

The use of these hook functions is not safe in multithreaded programs,
and they are now deprecated. Programmers should instead preempt calls to
the relevant functions by defining and exporting functions such as
``malloc'' and ``free''.

\hyperdef{}{EXAMPLE}{\section{\hyperref[EXAMPLE]{EXAMPLE}}\label{EXAMPLE}}

Here is a short example of how to use these variables.

~

\begin{verbatim}
#include <stdio.h>
#include <malloc.h>

/* Prototypes for our hooks.  */
static void my_init_hook(void);
static void *my_malloc_hook(size_t, const void *);

/* Variables to save original hooks. */
static void *(*old_malloc_hook)(size_t, const void *);

/* Override initializing hook from the C library. */
void (*__malloc_initialize_hook) (void) = my_init_hook;

static void
my_init_hook(void)
{
    old_malloc_hook = __malloc_hook;
    __malloc_hook = my_malloc_hook;
}

static void *
my_malloc_hook(size_t size, const void *caller)
{
    void *result;

    /* Restore all old hooks */
    __malloc_hook = old_malloc_hook;

    /* Call recursively */
    result = malloc(size);

    /* Save underlying hooks */
    old_malloc_hook = __malloc_hook;

    /* printf() might call malloc(), so protect it too. */
    printf("malloc(%u) called from %p returns %p\n",
            (unsigned int) size, caller, result);

    /* Restore our own hooks */
    __malloc_hook = my_malloc_hook;

    return result;
}
\end{verbatim}

\hyperdef{}{SEEux5fALSO}{\section{\hyperref[SEEux5fALSO]{SEE
ALSO}}\label{SEEux5fALSO}}

\textbf{mallinfo}(3), \textbf{malloc}(3), \textbf{mcheck}(3),
\textbf{mtrace}(3)

\hyperdef{}{COLOPHON}{\section{\hyperref[COLOPHON]{COLOPHON}}\label{COLOPHON}}

This page is part of release 3.54 of the Linux \emph{man-pages} project.
A description of the project, and information about reporting bugs, can
be found at http://www.kernel.org/doc/man-pages/.

\begin{longtable}[c]{@{}ll@{}}
\toprule\addlinespace
2010-10-13 & GNU
\\\addlinespace
\bottomrule
\end{longtable}

\end{document}
