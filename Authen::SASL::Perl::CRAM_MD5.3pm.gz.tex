\documentclass[]{article}
\usepackage[T1]{fontenc}
\usepackage{lmodern}
\usepackage{amssymb,amsmath}
\usepackage{ifxetex,ifluatex}
\usepackage{fixltx2e} % provides \textsubscript
% use upquote if available, for straight quotes in verbatim environments
\IfFileExists{upquote.sty}{\usepackage{upquote}}{}
\ifnum 0\ifxetex 1\fi\ifluatex 1\fi=0 % if pdftex
  \usepackage[utf8]{inputenc}
\else % if luatex or xelatex
  \ifxetex
    \usepackage{mathspec}
    \usepackage{xltxtra,xunicode}
  \else
    \usepackage{fontspec}
  \fi
  \defaultfontfeatures{Mapping=tex-text,Scale=MatchLowercase}
  \newcommand{\euro}{€}
\fi
% use microtype if available
\IfFileExists{microtype.sty}{\usepackage{microtype}}{}
\usepackage{longtable,booktabs}
\ifxetex
  \usepackage[setpagesize=false, % page size defined by xetex
              unicode=false, % unicode breaks when used with xetex
              xetex]{hyperref}
\else
  \usepackage[unicode=true]{hyperref}
\fi
\hypersetup{breaklinks=true,
            bookmarks=true,
            pdfauthor={},
            pdftitle={Authen::SASL::Perl::CRAM\_MD5(3pm)},
            colorlinks=true,
            citecolor=blue,
            urlcolor=blue,
            linkcolor=magenta,
            pdfborder={0 0 0}}
\urlstyle{same}  % don't use monospace font for urls
\setlength{\parindent}{0pt}
\setlength{\parskip}{6pt plus 2pt minus 1pt}
\setlength{\emergencystretch}{3em}  % prevent overfull lines
\setcounter{secnumdepth}{0}
\usepackage{pagecolor}

% Set background colour (of the page)
\definecolor{weirdbgcolor}{HTML}{FCF4F0}
\pagecolor{weirdbgcolor}

% Make bold text appear in a particular colour
\definecolor{boldcolor}{HTML}{6E0002}
\let\realtextbf=\textbf
\renewcommand{\textbf}[1]{\textcolor{boldcolor}{\realtextbf{#1}}}

% Use underlines instead of emphasis (ugh)
\renewcommand{\emph}[1]{\underline{#1}}

% % Use fixed-width font by default
% \renewcommand*\familydefault{\ttdefault}

\title{Authen::SASL::Perl::CRAM\_MD5(3pm)}
\author{}
\date{}

\begin{document}
\maketitle

\begin{longtable}[c]{@{}lll@{}}
\toprule\addlinespace
Authen::SASL::Perl::CRAM\_MD5(3pm) & User Contributed Perl Documentation
& Authen::SASL::Perl::CRAM\_MD5(3pm)
\\\addlinespace
\bottomrule
\end{longtable}

\hyperdef{}{NAME}{\section{\hyperref[NAME]{NAME}}\label{NAME}}

Authen::SASL::Perl::CRAM\_MD5 - CRAM MD5 Authentication class

\hyperdef{}{SYNOPSIS}{\section{\hyperref[SYNOPSIS]{SYNOPSIS}}\label{SYNOPSIS}}

\begin{verbatim}
  use Authen::SASL qw(Perl);
  $sasl = Authen::SASL->new(
    mechanism => 'CRAM-MD5',
    callback  => {
      user => $user,
      pass => $pass
    },
  );
\end{verbatim}

\hyperdef{}{DESCRIPTION}{\section{\hyperref[DESCRIPTION]{DESCRIPTION}}\label{DESCRIPTION}}

This method implements the client part of the CRAM-MD5 SASL algorithm,
as described in RFC 2195 resp. in IETF Draft
draft-ietf-sasl-crammd5-XX.txt.

\hyperdef{}{CALLBACK}{\subsection{\hyperref[CALLBACK]{CALLBACK}}\label{CALLBACK}}

The callbacks used are:

\begin{description}
\itemsep1pt\parskip0pt\parsep0pt
\item[user]
The username to be used for authentication
\end{description}

\begin{description}
\itemsep1pt\parskip0pt\parsep0pt
\item[pass]
The user's password to be used for authentication
\end{description}

\hyperdef{}{SEEux5fALSO}{\section{\hyperref[SEEux5fALSO]{SEE
ALSO}}\label{SEEux5fALSO}}

Authen::SASL, Authen::SASL::Perl

\hyperdef{}{AUTHORS}{\section{\hyperref[AUTHORS]{AUTHORS}}\label{AUTHORS}}

Software written by Graham Barr
\textless{}gbarr@pobox.com\textgreater{}, documentation written by Peter
Marschall \textless{}peter@adpm.de\textgreater{}.

Please report any bugs, or post any suggestions, to the perl-ldap
mailing list \textless{}perl-ldap@perl.org\textgreater{}

\hyperdef{}{COPYRIGHT}{\section{\hyperref[COPYRIGHT]{COPYRIGHT}}\label{COPYRIGHT}}

Copyright (c) 2002-2004 Graham Barr. All rights reserved. This program
is free software; you can redistribute it and/or modify it under the
same terms as Perl itself.

Documentation Copyright (c) 2004 Peter Marschall. All rights reserved.
This documentation is distributed, and may be redistributed, under the
same terms as Perl itself.

\begin{longtable}[c]{@{}ll@{}}
\toprule\addlinespace
2010-03-11 & perl v5.10.1
\\\addlinespace
\bottomrule
\end{longtable}

\end{document}
