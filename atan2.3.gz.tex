\documentclass[]{article}
\usepackage[T1]{fontenc}
\usepackage{lmodern}
\usepackage{amssymb,amsmath}
\usepackage{ifxetex,ifluatex}
\usepackage{fixltx2e} % provides \textsubscript
% use upquote if available, for straight quotes in verbatim environments
\IfFileExists{upquote.sty}{\usepackage{upquote}}{}
\ifnum 0\ifxetex 1\fi\ifluatex 1\fi=0 % if pdftex
  \usepackage[utf8]{inputenc}
\else % if luatex or xelatex
  \ifxetex
    \usepackage{mathspec}
    \usepackage{xltxtra,xunicode}
  \else
    \usepackage{fontspec}
  \fi
  \defaultfontfeatures{Mapping=tex-text,Scale=MatchLowercase}
  \newcommand{\euro}{€}
\fi
% use microtype if available
\IfFileExists{microtype.sty}{\usepackage{microtype}}{}
\usepackage{longtable,booktabs}
\ifxetex
  \usepackage[setpagesize=false, % page size defined by xetex
              unicode=false, % unicode breaks when used with xetex
              xetex]{hyperref}
\else
  \usepackage[unicode=true]{hyperref}
\fi
\hypersetup{breaklinks=true,
            bookmarks=true,
            pdfauthor={},
            pdftitle={ATAN2(3)},
            colorlinks=true,
            citecolor=blue,
            urlcolor=blue,
            linkcolor=magenta,
            pdfborder={0 0 0}}
\urlstyle{same}  % don't use monospace font for urls
\setlength{\parindent}{0pt}
\setlength{\parskip}{6pt plus 2pt minus 1pt}
\setlength{\emergencystretch}{3em}  % prevent overfull lines
\setcounter{secnumdepth}{0}
\usepackage{pagecolor}

% Set background colour (of the page)
\definecolor{weirdbgcolor}{HTML}{FCF4F0}
\pagecolor{weirdbgcolor}

% Make bold text appear in a particular colour
\definecolor{boldcolor}{HTML}{6E0002}
\let\realtextbf=\textbf
\renewcommand{\textbf}[1]{\textcolor{boldcolor}{\realtextbf{#1}}}

% Use underlines instead of emphasis (ugh)
\renewcommand{\emph}[1]{\underline{#1}}

% % Use fixed-width font by default
% \renewcommand*\familydefault{\ttdefault}

\title{ATAN2(3)}
\author{}
\date{}

\begin{document}
\maketitle

\begin{longtable}[c]{@{}lll@{}}
\toprule\addlinespace
ATAN2(3) & Linux Programmer's Manual & ATAN2(3)
\\\addlinespace
\bottomrule
\end{longtable}

\hyperdef{}{NAME}{\section{\hyperref[NAME]{NAME}}\label{NAME}}

atan2, atan2f, atan2l - arc tangent function of two variables

\hyperdef{}{SYNOPSIS}{\section{\hyperref[SYNOPSIS]{SYNOPSIS}}\label{SYNOPSIS}}

\begin{verbatim}
#include <math.h>

double atan2(double y, double x);
float atan2f(float y, float x);
long double atan2l(long double y, long double x);
\end{verbatim}

Link with \emph{-lm}.

~

Feature Test Macro Requirements for glibc (see
\textbf{feature\_test\_macros}(7)): \\

~

\textbf{atan2f}(), \textbf{atan2l}():

\_BSD\_SOURCE \textbar{}\textbar{} \_SVID\_SOURCE \textbar{}\textbar{}
\_XOPEN\_SOURCE~\textgreater{}=~600 \textbar{}\textbar{}
\_ISOC99\_SOURCE \textbar{}\textbar{}
\_POSIX\_C\_SOURCE~\textgreater{}=~200112L;

~

or \emph{cc~-std=c99}

\hyperdef{}{DESCRIPTION}{\section{\hyperref[DESCRIPTION]{DESCRIPTION}}\label{DESCRIPTION}}

The \textbf{atan2}() function calculates the principal value of the arc
tangent of \emph{y/x}, using the signs of the two arguments to determine
the quadrant of the result.

\hyperdef{}{RETURNux5fVALUE}{\section{\hyperref[RETURNux5fVALUE]{RETURN
VALUE}}\label{RETURNux5fVALUE}}

On success, these functions return the principal value of the arc
tangent of \emph{y/x} in radians; the return value is in the range
{[}-pi,~pi{]}.

~

If \emph{y} is +0 (-0) and \emph{x} is less than 0, +pi (-pi) is
returned.

~

If \emph{y} is +0 (-0) and \emph{x} is greater than 0, +0 (-0) is
returned.

~

If \emph{y} is less than 0 and \emph{x} is +0 or -0, -pi/2 is returned.

~

If \emph{y} is greater than 0 and \emph{x} is +0 or -0, pi/2 is
returned.

~

If either \emph{x} or \emph{y} is NaN, a NaN is returned.

~

If \emph{y} is +0 (-0) and \emph{x} is -0, +pi (-pi) is returned.

~

If \emph{y} is +0 (-0) and \emph{x} is +0, +0 (-0) is returned.

~

If \emph{y} is a finite value greater (less) than 0, and \emph{x} is
negative infinity, +pi (-pi) is returned.

~

If \emph{y} is a finite value greater (less) than 0, and \emph{x} is
positive infinity, +0 (-0) is returned.

~

If \emph{y} is positive infinity (negative infinity), and \emph{x} is
finite, pi/2 (-pi/2) is returned.

~

If \emph{y} is positive infinity (negative infinity) and \emph{x} is
negative infinity, +3*pi/4 (-3*pi/4) is returned.

~

If \emph{y} is positive infinity (negative infinity) and \emph{x} is
positive infinity, +pi/4 (-pi/4) is returned.

\hyperdef{}{ERRORS}{\section{\hyperref[ERRORS]{ERRORS}}\label{ERRORS}}

No errors occur.

\hyperdef{}{CONFORMINGux5fTO}{\section{\hyperref[CONFORMINGux5fTO]{CONFORMING
TO}}\label{CONFORMINGux5fTO}}

C99, POSIX.1-2001. The variant returning \emph{double} also conforms to
SVr4, 4.3BSD, C89.

\hyperdef{}{SEEux5fALSO}{\section{\hyperref[SEEux5fALSO]{SEE
ALSO}}\label{SEEux5fALSO}}

\textbf{acos}(3), \textbf{asin}(3), \textbf{atan}(3), \textbf{carg}(3),
\textbf{cos}(3), \textbf{sin}(3), \textbf{tan}(3)

\hyperdef{}{COLOPHON}{\section{\hyperref[COLOPHON]{COLOPHON}}\label{COLOPHON}}

This page is part of release 3.54 of the Linux \emph{man-pages} project.
A description of the project, and information about reporting bugs, can
be found at http://www.kernel.org/doc/man-pages/.

\begin{longtable}[c]{@{}ll@{}}
\toprule\addlinespace
2010-09-20 &
\\\addlinespace
\bottomrule
\end{longtable}

\end{document}
