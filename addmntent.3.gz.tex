\documentclass[]{article}
\usepackage[T1]{fontenc}
\usepackage{lmodern}
\usepackage{amssymb,amsmath}
\usepackage{ifxetex,ifluatex}
\usepackage{fixltx2e} % provides \textsubscript
% use upquote if available, for straight quotes in verbatim environments
\IfFileExists{upquote.sty}{\usepackage{upquote}}{}
\ifnum 0\ifxetex 1\fi\ifluatex 1\fi=0 % if pdftex
  \usepackage[utf8]{inputenc}
\else % if luatex or xelatex
  \ifxetex
    \usepackage{mathspec}
    \usepackage{xltxtra,xunicode}
  \else
    \usepackage{fontspec}
  \fi
  \defaultfontfeatures{Mapping=tex-text,Scale=MatchLowercase}
  \newcommand{\euro}{€}
\fi
% use microtype if available
\IfFileExists{microtype.sty}{\usepackage{microtype}}{}
\usepackage{longtable,booktabs}
\ifxetex
  \usepackage[setpagesize=false, % page size defined by xetex
              unicode=false, % unicode breaks when used with xetex
              xetex]{hyperref}
\else
  \usepackage[unicode=true]{hyperref}
\fi
\hypersetup{breaklinks=true,
            bookmarks=true,
            pdfauthor={},
            pdftitle={GETMNTENT(3)},
            colorlinks=true,
            citecolor=blue,
            urlcolor=blue,
            linkcolor=magenta,
            pdfborder={0 0 0}}
\urlstyle{same}  % don't use monospace font for urls
\setlength{\parindent}{0pt}
\setlength{\parskip}{6pt plus 2pt minus 1pt}
\setlength{\emergencystretch}{3em}  % prevent overfull lines
\setcounter{secnumdepth}{0}
\usepackage{pagecolor}

% Set background colour (of the page)
\definecolor{weirdbgcolor}{HTML}{FCF4F0}
\pagecolor{weirdbgcolor}

% Make bold text appear in a particular colour
\definecolor{boldcolor}{HTML}{6E0002}
\let\realtextbf=\textbf
\renewcommand{\textbf}[1]{\textcolor{boldcolor}{\realtextbf{#1}}}

% Use underlines instead of emphasis (ugh)
\renewcommand{\emph}[1]{\underline{#1}}

% % Use fixed-width font by default
% \renewcommand*\familydefault{\ttdefault}

\title{GETMNTENT(3)}
\author{}
\date{}

\begin{document}
\maketitle

\begin{longtable}[c]{@{}lll@{}}
\toprule\addlinespace
GETMNTENT(3) & Linux Programmer's Manual & GETMNTENT(3)
\\\addlinespace
\bottomrule
\end{longtable}

\hyperdef{}{NAME}{\section{\hyperref[NAME]{NAME}}\label{NAME}}

getmntent, setmntent, addmntent, endmntent, hasmntopt, getmntent\_r -
get filesystem descriptor file entry

\hyperdef{}{SYNOPSIS}{\section{\hyperref[SYNOPSIS]{SYNOPSIS}}\label{SYNOPSIS}}

\begin{verbatim}
#include <stdio.h>
#include <mntent.h>
 
FILE *setmntent(const char *filename, const char *type);
 
struct mntent *getmntent(FILE *fp);
 
int addmntent(FILE *fp, const struct mntent *mnt);
 
int endmntent(FILE *fp);
 
char *hasmntopt(const struct mntent *mnt, const char *opt);
 
/* GNU extension */
#include <mntent.h>
 
struct mntent *getmntent_r(FILE *fp, struct mntent *mntbuf,
                           char *buf, int buflen);
\end{verbatim}

~

Feature Test Macro Requirements for glibc (see
\textbf{feature\_test\_macros}(7)): \\

~

\textbf{getmntent\_r}(): \_BSD\_SOURCE \textbar{}\textbar{}
\_SVID\_SOURCE

\hyperdef{}{DESCRIPTION}{\section{\hyperref[DESCRIPTION]{DESCRIPTION}}\label{DESCRIPTION}}

These routines are used to access the filesystem description file
\emph{/etc/fstab} and the mounted filesystem description file
\emph{/etc/mtab}.

The \textbf{setmntent}() function opens the filesystem description file
\emph{filename} and returns a file pointer which can be used by
\textbf{getmntent}(). The argument \emph{type} is the type of access
required and can take the same values as the \emph{mode} argument of
\textbf{fopen}(3).

The \textbf{getmntent}() function reads the next line from the
filesystem description file \emph{fp} and returns a pointer to a
structure containing the broken out fields from a line in the file. The
pointer points to a static area of memory which is overwritten by
subsequent calls to \textbf{getmntent}().

The \textbf{addmntent}() function adds the \emph{mntent} structure
\emph{mnt} to the end of the open file \emph{fp}.

The \textbf{endmntent}() function closes the filesystem description file
\emph{fp}.

The \textbf{hasmntopt}() function scans the \emph{mnt\_opts} field (see
below) of the \emph{mntent} structure \emph{mnt} for a substring that
matches \emph{opt}. See \emph{\textless{}mntent.h\textgreater{}} and
\textbf{mount}(8) for valid mount options.

The reentrant \textbf{getmntent\_r}() function is similar to
\textbf{getmntent}(), but stores the \emph{struct mount} in the provided
\emph{*mntbuf} and stores the strings pointed to by the entries in that
struct in the provided array \emph{buf} of size \emph{buflen}.

The \emph{mntent} structure is defined in
\emph{\textless{}mntent.h\textgreater{}} as follows:

~

\begin{verbatim}
struct mntent {
    char *mnt_fsname;   /* name of mounted filesystem */
    char *mnt_dir;      /* filesystem path prefix */
    char *mnt_type;     /* mount type (see mntent.h) */
    char *mnt_opts;     /* mount options (see mntent.h) */
    int   mnt_freq;     /* dump frequency in days */
    int   mnt_passno;   /* pass number on parallel fsck */
};
\end{verbatim}

~

Since fields in the mtab and fstab files are separated by whitespace,
octal escapes are used to represent the four characters space
(\textbackslash{}040), tab (\textbackslash{}011), newline
(\textbackslash{}012) and backslash (\textbackslash{}134) in those files
when they occur in one of the four strings in a \emph{mntent} structure.
The routines \textbf{addmntent}() and \textbf{getmntent}() will convert
from string representation to escaped representation and back.

\hyperdef{}{RETURNux5fVALUE}{\section{\hyperref[RETURNux5fVALUE]{RETURN
VALUE}}\label{RETURNux5fVALUE}}

The \textbf{getmntent}() and \textbf{getmntent\_r}() functions return a
pointer to the \emph{mntent} structure or NULL on failure.

The \textbf{addmntent}() function returns 0 on success and 1 on failure.

The \textbf{endmntent}() function always returns 1.

The \textbf{hasmntopt}() function returns the address of the substring
if a match is found and NULL otherwise.

\hyperdef{}{FILES}{\section{\hyperref[FILES]{FILES}}\label{FILES}}

\begin{verbatim}
/etc/fstab          filesystem description file
/etc/mtab           mounted filesystem description file
\end{verbatim}

\hyperdef{}{CONFORMINGux5fTO}{\section{\hyperref[CONFORMINGux5fTO]{CONFORMING
TO}}\label{CONFORMINGux5fTO}}

The nonreentrant functions are from SunOS 4.1.3. A routine
\textbf{getmntent\_r}() was introduced in HP-UX 10, but it returns an
int. The prototype shown above is glibc-only.

\hyperdef{}{NOTES}{\section{\hyperref[NOTES]{NOTES}}\label{NOTES}}

System V also has a \textbf{getmntent}() function but the calling
sequence differs, and the returned structure is different. Under System
V \emph{/etc/mnttab} is used. 4.4BSD and Digital UNIX have a routine
\textbf{getmntinfo}(), a wrapper around the system call
\textbf{getfsstat}().

\hyperdef{}{SEEux5fALSO}{\section{\hyperref[SEEux5fALSO]{SEE
ALSO}}\label{SEEux5fALSO}}

\textbf{fopen}(3), \textbf{fstab}(5), \textbf{mount}(8)

\hyperdef{}{COLOPHON}{\section{\hyperref[COLOPHON]{COLOPHON}}\label{COLOPHON}}

This page is part of release 3.54 of the Linux \emph{man-pages} project.
A description of the project, and information about reporting bugs, can
be found at http://www.kernel.org/doc/man-pages/.

\begin{longtable}[c]{@{}ll@{}}
\toprule\addlinespace
2009-09-15 &
\\\addlinespace
\bottomrule
\end{longtable}

\end{document}
