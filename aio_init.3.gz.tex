\documentclass[]{article}
\usepackage[T1]{fontenc}
\usepackage{lmodern}
\usepackage{amssymb,amsmath}
\usepackage{ifxetex,ifluatex}
\usepackage{fixltx2e} % provides \textsubscript
% use upquote if available, for straight quotes in verbatim environments
\IfFileExists{upquote.sty}{\usepackage{upquote}}{}
\ifnum 0\ifxetex 1\fi\ifluatex 1\fi=0 % if pdftex
  \usepackage[utf8]{inputenc}
\else % if luatex or xelatex
  \ifxetex
    \usepackage{mathspec}
    \usepackage{xltxtra,xunicode}
  \else
    \usepackage{fontspec}
  \fi
  \defaultfontfeatures{Mapping=tex-text,Scale=MatchLowercase}
  \newcommand{\euro}{€}
\fi
% use microtype if available
\IfFileExists{microtype.sty}{\usepackage{microtype}}{}
\usepackage{longtable,booktabs}
\ifxetex
  \usepackage[setpagesize=false, % page size defined by xetex
              unicode=false, % unicode breaks when used with xetex
              xetex]{hyperref}
\else
  \usepackage[unicode=true]{hyperref}
\fi
\hypersetup{breaklinks=true,
            bookmarks=true,
            pdfauthor={},
            pdftitle={AIO\_INIT(3)},
            colorlinks=true,
            citecolor=blue,
            urlcolor=blue,
            linkcolor=magenta,
            pdfborder={0 0 0}}
\urlstyle{same}  % don't use monospace font for urls
\setlength{\parindent}{0pt}
\setlength{\parskip}{6pt plus 2pt minus 1pt}
\setlength{\emergencystretch}{3em}  % prevent overfull lines
\setcounter{secnumdepth}{0}
\usepackage{pagecolor}

% Set background colour (of the page)
\definecolor{weirdbgcolor}{HTML}{FCF4F0}
\pagecolor{weirdbgcolor}

% Make bold text appear in a particular colour
\definecolor{boldcolor}{HTML}{6E0002}
\let\realtextbf=\textbf
\renewcommand{\textbf}[1]{\textcolor{boldcolor}{\realtextbf{#1}}}

% Use underlines instead of emphasis (ugh)
\renewcommand{\emph}[1]{\underline{#1}}

% % Use fixed-width font by default
% \renewcommand*\familydefault{\ttdefault}

\title{AIO\_INIT(3)}
\author{}
\date{}

\begin{document}
\maketitle

\begin{longtable}[c]{@{}lll@{}}
\toprule\addlinespace
AIO\_INIT(3) & Linux Programmer's Manual & AIO\_INIT(3)
\\\addlinespace
\bottomrule
\end{longtable}

\hyperdef{}{NAME}{\section{\hyperref[NAME]{NAME}}\label{NAME}}

aio\_init - asynchronous I/O initialization

\hyperdef{}{SYNOPSIS}{\section{\hyperref[SYNOPSIS]{SYNOPSIS}}\label{SYNOPSIS}}

\begin{verbatim}
#define _GNU_SOURCE         /* See feature_test_macros(7) */
#include <aio.h>

void aio_init(const struct aioinit *init);
\end{verbatim}

~

Link with \emph{-lrt}.

\hyperdef{}{DESCRIPTION}{\section{\hyperref[DESCRIPTION]{DESCRIPTION}}\label{DESCRIPTION}}

The GNU-specific \textbf{aio\_init}() function allows the caller to
provide tuning hints to the glibc POSIX AIO implementation. Use of this
function is optional, but to be effective, it must be called before
employing any other functions in the POSIX AIO API.

~

The tuning information is provided in the buffer pointed to by the
argument \emph{init}. This buffer is a structure of the following form:

\begin{verbatim}
struct aioinit {
    int aio_threads;    /* Maximum number of threads */
    int aio_num;        /* Number of expected simultaneous
                           requests */
    int aio_locks;      /* Not used */
    int aio_usedba;     /* Not used */
    int aio_debug;      /* Not used */
    int aio_numusers;   /* Not used */
    int aio_idle_time;  /* Number of seconds before idle thread
                           terminates (since glibc 2.2) */
    int aio_reserved;
};
\end{verbatim}

The following fields are used in the \emph{aioinit} structure:

\begin{description}
\itemsep1pt\parskip0pt\parsep0pt
\item[\emph{aio\_threads}]
This field specifies the maximum number of worker threads that may be
used by the implementation. If the number of outstanding I/O operations
exceeds this limit, then excess operations will be queued until a worker
thread becomes free. If this field is specified with a value less than
1, the value 1 is used. The default value is 20.
\end{description}

\begin{description}
\itemsep1pt\parskip0pt\parsep0pt
\item[\emph{aio\_num}]
This field should specify the maximum number of simultaneous I/O
requests that the caller expects to enqueue. If a value less than 32 is
specified for this field, it is rounded up to 32. The default value is
64.
\end{description}

\begin{description}
\itemsep1pt\parskip0pt\parsep0pt
\item[\emph{aio\_idle\_time}]
This field specifies the amount of time in seconds that a worker thread
should wait for further requests before terminating, after having
completed a previous request. The default value is 1.
\end{description}

\hyperdef{}{VERSIONS}{\section{\hyperref[VERSIONS]{VERSIONS}}\label{VERSIONS}}

The \textbf{aio\_init}() function is available since glibc 2.1.

\hyperdef{}{CONFORMINGux5fTO}{\section{\hyperref[CONFORMINGux5fTO]{CONFORMING
TO}}\label{CONFORMINGux5fTO}}

This function is a GNU extension.

\hyperdef{}{SEEux5fALSO}{\section{\hyperref[SEEux5fALSO]{SEE
ALSO}}\label{SEEux5fALSO}}

\textbf{aio}(7)

\hyperdef{}{COLOPHON}{\section{\hyperref[COLOPHON]{COLOPHON}}\label{COLOPHON}}

This page is part of release 3.54 of the Linux \emph{man-pages} project.
A description of the project, and information about reporting bugs, can
be found at http://www.kernel.org/doc/man-pages/.

\begin{longtable}[c]{@{}ll@{}}
\toprule\addlinespace
2012-04-26 & Linux
\\\addlinespace
\bottomrule
\end{longtable}

\end{document}
