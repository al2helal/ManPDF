\documentclass[]{article}
\usepackage[T1]{fontenc}
\usepackage{lmodern}
\usepackage{amssymb,amsmath}
\usepackage{ifxetex,ifluatex}
\usepackage{fixltx2e} % provides \textsubscript
% use upquote if available, for straight quotes in verbatim environments
\IfFileExists{upquote.sty}{\usepackage{upquote}}{}
\ifnum 0\ifxetex 1\fi\ifluatex 1\fi=0 % if pdftex
  \usepackage[utf8]{inputenc}
\else % if luatex or xelatex
  \ifxetex
    \usepackage{mathspec}
    \usepackage{xltxtra,xunicode}
  \else
    \usepackage{fontspec}
  \fi
  \defaultfontfeatures{Mapping=tex-text,Scale=MatchLowercase}
  \newcommand{\euro}{€}
\fi
% use microtype if available
\IfFileExists{microtype.sty}{\usepackage{microtype}}{}
\usepackage{longtable,booktabs}
\ifxetex
  \usepackage[setpagesize=false, % page size defined by xetex
              unicode=false, % unicode breaks when used with xetex
              xetex]{hyperref}
\else
  \usepackage[unicode=true]{hyperref}
\fi
\hypersetup{breaklinks=true,
            bookmarks=true,
            pdfauthor={},
            pdftitle={ATANH(3)},
            colorlinks=true,
            citecolor=blue,
            urlcolor=blue,
            linkcolor=magenta,
            pdfborder={0 0 0}}
\urlstyle{same}  % don't use monospace font for urls
\setlength{\parindent}{0pt}
\setlength{\parskip}{6pt plus 2pt minus 1pt}
\setlength{\emergencystretch}{3em}  % prevent overfull lines
\setcounter{secnumdepth}{0}
\usepackage{pagecolor}

% Set background colour (of the page)
\definecolor{weirdbgcolor}{HTML}{FCF4F0}
\pagecolor{weirdbgcolor}

% Make bold text appear in a particular colour
\definecolor{boldcolor}{HTML}{6E0002}
\let\realtextbf=\textbf
\renewcommand{\textbf}[1]{\textcolor{boldcolor}{\realtextbf{#1}}}

% Use underlines instead of emphasis (ugh)
\renewcommand{\emph}[1]{\underline{#1}}

% % Use fixed-width font by default
% \renewcommand*\familydefault{\ttdefault}

\title{ATANH(3)}
\author{}
\date{}

\begin{document}
\maketitle

\begin{longtable}[c]{@{}lll@{}}
\toprule\addlinespace
ATANH(3) & Linux Programmer's Manual & ATANH(3)
\\\addlinespace
\bottomrule
\end{longtable}

\hyperdef{}{NAME}{\section{\hyperref[NAME]{NAME}}\label{NAME}}

atanh, atanhf, atanhl - inverse hyperbolic tangent function

\hyperdef{}{SYNOPSIS}{\section{\hyperref[SYNOPSIS]{SYNOPSIS}}\label{SYNOPSIS}}

\begin{verbatim}
#include <math.h>
 
double atanh(double x);
 
float atanhf(float x);
 
long double atanhl(long double x);
 
\end{verbatim}

Link with \emph{-lm}.

~

Feature Test Macro Requirements for glibc (see
\textbf{feature\_test\_macros}(7)): \\

~

\textbf{atanh}():

\_BSD\_SOURCE \textbar{}\textbar{} \_SVID\_SOURCE \textbar{}\textbar{}
\_XOPEN\_SOURCE~\textgreater{}=~500 \textbar{}\textbar{}
\_XOPEN\_SOURCE~\&\&~\_XOPEN\_SOURCE\_EXTENDED \textbar{}\textbar{}
\_ISOC99\_SOURCE \textbar{}\textbar{}
\_POSIX\_C\_SOURCE~\textgreater{}=~200112L;

~

or \emph{cc~-std=c99}

~

\textbf{atanhf}(), \textbf{atanhl}():

\_BSD\_SOURCE \textbar{}\textbar{} \_SVID\_SOURCE \textbar{}\textbar{}
\_XOPEN\_SOURCE~\textgreater{}=~600 \textbar{}\textbar{}
\_ISOC99\_SOURCE \textbar{}\textbar{}
\_POSIX\_C\_SOURCE~\textgreater{}=~200112L;

~

or \emph{cc~-std=c99}

\hyperdef{}{DESCRIPTION}{\section{\hyperref[DESCRIPTION]{DESCRIPTION}}\label{DESCRIPTION}}

The \textbf{atanh}() function calculates the inverse hyperbolic tangent
of \emph{x}; that is the value whose hyperbolic tangent is \emph{x}.

\hyperdef{}{RETURNux5fVALUE}{\section{\hyperref[RETURNux5fVALUE]{RETURN
VALUE}}\label{RETURNux5fVALUE}}

On success, these functions return the inverse hyperbolic tangent of
\emph{x}.

~

If \emph{x} is a NaN, a NaN is returned.

~

If \emph{x} is +0 (-0), +0 (-0) is returned.

~

If \emph{x} is +1 or -1, a pole error occurs, and the functions return
\textbf{HUGE\_VAL}, \textbf{HUGE\_VALF}, or \textbf{HUGE\_VALL},
respectively, with the mathematically correct sign.

~

If the absolute value of \emph{x} is greater than 1, a domain error
occurs, and a NaN is returned.

\hyperdef{}{ERRORS}{\section{\hyperref[ERRORS]{ERRORS}}\label{ERRORS}}

See \textbf{math\_error}(7) for information on how to determine whether
an error has occurred when calling these functions.

The following errors can occur:

\begin{description}
\itemsep1pt\parskip0pt\parsep0pt
\item[Domain error: \emph{x} less than -1 or greater than +1]
\emph{errno} is set to \textbf{EDOM}. An invalid floating-point
exception (\textbf{FE\_INVALID}) is raised.
\end{description}

\begin{description}
\itemsep1pt\parskip0pt\parsep0pt
\item[Pole error: \emph{x} is +1 or -1]
\emph{errno} is set to \textbf{ERANGE} (but see BUGS). A divide-by-zero
floating-point exception (\textbf{FE\_DIVBYZERO}) is raised.
\end{description}

\hyperdef{}{CONFORMINGux5fTO}{\section{\hyperref[CONFORMINGux5fTO]{CONFORMING
TO}}\label{CONFORMINGux5fTO}}

C99, POSIX.1-2001. The variant returning \emph{double} also conforms to
SVr4, 4.3BSD, C89.

\hyperdef{}{BUGS}{\section{\hyperref[BUGS]{BUGS}}\label{BUGS}}

In glibc 2.9 and earlier, when a pole error occurs, \emph{errno} as set
to \textbf{EDOM} instead of the POSIX-mandated \textbf{ERANGE}. Since
version 2.10, glibc does the right thing.

\hyperdef{}{SEEux5fALSO}{\section{\hyperref[SEEux5fALSO]{SEE
ALSO}}\label{SEEux5fALSO}}

\textbf{acosh}(3), \textbf{asinh}(3), \textbf{catanh}(3),
\textbf{cosh}(3), \textbf{sinh}(3), \textbf{tanh}(3)

\hyperdef{}{COLOPHON}{\section{\hyperref[COLOPHON]{COLOPHON}}\label{COLOPHON}}

This page is part of release 3.54 of the Linux \emph{man-pages} project.
A description of the project, and information about reporting bugs, can
be found at http://www.kernel.org/doc/man-pages/.

\begin{longtable}[c]{@{}ll@{}}
\toprule\addlinespace
2010-09-11 &
\\\addlinespace
\bottomrule
\end{longtable}

\end{document}
