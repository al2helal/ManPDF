\documentclass[]{article}
\usepackage[T1]{fontenc}
\usepackage{lmodern}
\usepackage{amssymb,amsmath}
\usepackage{ifxetex,ifluatex}
\usepackage{fixltx2e} % provides \textsubscript
% use upquote if available, for straight quotes in verbatim environments
\IfFileExists{upquote.sty}{\usepackage{upquote}}{}
\ifnum 0\ifxetex 1\fi\ifluatex 1\fi=0 % if pdftex
  \usepackage[utf8]{inputenc}
\else % if luatex or xelatex
  \ifxetex
    \usepackage{mathspec}
    \usepackage{xltxtra,xunicode}
  \else
    \usepackage{fontspec}
  \fi
  \defaultfontfeatures{Mapping=tex-text,Scale=MatchLowercase}
  \newcommand{\euro}{€}
\fi
% use microtype if available
\IfFileExists{microtype.sty}{\usepackage{microtype}}{}
\usepackage{longtable,booktabs}
\ifxetex
  \usepackage[setpagesize=false, % page size defined by xetex
              unicode=false, % unicode breaks when used with xetex
              xetex]{hyperref}
\else
  \usepackage[unicode=true]{hyperref}
\fi
\hypersetup{breaklinks=true,
            bookmarks=true,
            pdfauthor={},
            pdftitle={AIO\_SUSPEND(3)},
            colorlinks=true,
            citecolor=blue,
            urlcolor=blue,
            linkcolor=magenta,
            pdfborder={0 0 0}}
\urlstyle{same}  % don't use monospace font for urls
\setlength{\parindent}{0pt}
\setlength{\parskip}{6pt plus 2pt minus 1pt}
\setlength{\emergencystretch}{3em}  % prevent overfull lines
\setcounter{secnumdepth}{0}
\usepackage{pagecolor}

% Set background colour (of the page)
\definecolor{weirdbgcolor}{HTML}{FCF4F0}
\pagecolor{weirdbgcolor}

% Make bold text appear in a particular colour
\definecolor{boldcolor}{HTML}{6E0002}
\let\realtextbf=\textbf
\renewcommand{\textbf}[1]{\textcolor{boldcolor}{\realtextbf{#1}}}

% Use underlines instead of emphasis (ugh)
\renewcommand{\emph}[1]{\underline{#1}}

% % Use fixed-width font by default
% \renewcommand*\familydefault{\ttdefault}

\title{AIO\_SUSPEND(3)}
\author{}
\date{}

\begin{document}
\maketitle

\begin{longtable}[c]{@{}lll@{}}
\toprule\addlinespace
AIO\_SUSPEND(3) & Linux Programmer's Manual & AIO\_SUSPEND(3)
\\\addlinespace
\bottomrule
\end{longtable}

\hyperdef{}{NAME}{\section{\hyperref[NAME]{NAME}}\label{NAME}}

aio\_suspend - wait for asynchronous I/O operation or timeout

\hyperdef{}{SYNOPSIS}{\section{\hyperref[SYNOPSIS]{SYNOPSIS}}\label{SYNOPSIS}}

\begin{verbatim}
 
#include <aio.h>
 
int aio_suspend(const struct aiocb * const aiocb_list[],
 
                int nitems, const struct timespec *timeout);
 
Link with  -lrt.
\end{verbatim}

\hyperdef{}{DESCRIPTION}{\section{\hyperref[DESCRIPTION]{DESCRIPTION}}\label{DESCRIPTION}}

The \textbf{aio\_suspend}() function suspends the calling thread until
one of the following occurs:

\begin{description}
\itemsep1pt\parskip0pt\parsep0pt
\item[*]
One or more of the asynchronous I/O requests in the list
\emph{aiocb\_list} has completed.
\end{description}

\begin{description}
\itemsep1pt\parskip0pt\parsep0pt
\item[*]
A signal is delivered.
\end{description}

\begin{description}
\itemsep1pt\parskip0pt\parsep0pt
\item[*]
\emph{timeout} is not NULL and the specified time interval has passed.
(For details of the \emph{timespec} structure, see
\textbf{nanosleep}(2).)
\end{description}

The \emph{nitems} argument specifies the number of items in
\emph{aiocb\_list}. Each item in the list pointed to by
\emph{aiocb\_list} must be either NULL (and then is ignored), or a
pointer to a control block on which I/O was initiated using
\textbf{aio\_read}(3), \textbf{aio\_write}(3), or
\textbf{lio\_listio}(3). (See \textbf{aio}(7) for a description of the
\emph{aiocb} structure.)

If \textbf{CLOCK\_MONOTONIC} is supported, this clock is used to measure
the timeout interval (see \textbf{clock\_gettime}(3)).

\hyperdef{}{RETURNux5fVALUE}{\section{\hyperref[RETURNux5fVALUE]{RETURN
VALUE}}\label{RETURNux5fVALUE}}

If this function returns after completion of one of the I/O requests
specified in \emph{aiocb\_list}, 0 is returned. Otherwise, -1 is
returned, and \emph{errno} is set to indicate the error.

\hyperdef{}{ERRORS}{\section{\hyperref[ERRORS]{ERRORS}}\label{ERRORS}}

\begin{description}
\itemsep1pt\parskip0pt\parsep0pt
\item[\textbf{EAGAIN}]
The call timed out before any of the indicated operations had completed.
\end{description}

\begin{description}
\itemsep1pt\parskip0pt\parsep0pt
\item[\textbf{EINTR}]
The call was ended by signal (possibly the completion signal of one of
the operations we were waiting for); see \textbf{signal}(7).
\end{description}

\begin{description}
\itemsep1pt\parskip0pt\parsep0pt
\item[\textbf{ENOSYS}]
\textbf{aio\_suspend}() is not implemented.
\end{description}

\hyperdef{}{VERSIONS}{\section{\hyperref[VERSIONS]{VERSIONS}}\label{VERSIONS}}

The \textbf{aio\_suspend}() function is available since glibc 2.1.

\hyperdef{}{CONFORMINGux5fTO}{\section{\hyperref[CONFORMINGux5fTO]{CONFORMING
TO}}\label{CONFORMINGux5fTO}}

POSIX.1-2001, POSIX.1-2008.

\hyperdef{}{NOTES}{\section{\hyperref[NOTES]{NOTES}}\label{NOTES}}

One can achieve polling by using a non-NULL \emph{timeout} that
specifies a zero time interval.

~

If one or more of the asynchronous I/O operations specified in
\emph{aiocb\_list} has already completed at the time of the call to
\textbf{aio\_suspend}(), then the call returns immediately.

~

To determine which I/O operations have completed after a successful
return from \textbf{aio\_suspend}(), use \textbf{aio\_error}(3) to scan
the list of \emph{aiocb} structures pointed to by \emph{aiocb\_list}.

\hyperdef{}{SEEux5fALSO}{\section{\hyperref[SEEux5fALSO]{SEE
ALSO}}\label{SEEux5fALSO}}

\textbf{aio\_cancel}(3), \textbf{aio\_error}(3), \textbf{aio\_fsync}(3),
\textbf{aio\_read}(3), \textbf{aio\_return}(3), \textbf{aio\_write}(3),
\textbf{lio\_listio}(3), \textbf{aio}(7), \textbf{time}(7)

\hyperdef{}{COLOPHON}{\section{\hyperref[COLOPHON]{COLOPHON}}\label{COLOPHON}}

This page is part of release 3.54 of the Linux \emph{man-pages} project.
A description of the project, and information about reporting bugs, can
be found at http://www.kernel.org/doc/man-pages/.

\begin{longtable}[c]{@{}ll@{}}
\toprule\addlinespace
2012-05-08 &
\\\addlinespace
\bottomrule
\end{longtable}

\end{document}
