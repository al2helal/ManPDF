\documentclass[]{article}
\usepackage[T1]{fontenc}
\usepackage{lmodern}
\usepackage{amssymb,amsmath}
\usepackage{ifxetex,ifluatex}
\usepackage{fixltx2e} % provides \textsubscript
% use upquote if available, for straight quotes in verbatim environments
\IfFileExists{upquote.sty}{\usepackage{upquote}}{}
\ifnum 0\ifxetex 1\fi\ifluatex 1\fi=0 % if pdftex
  \usepackage[utf8]{inputenc}
\else % if luatex or xelatex
  \ifxetex
    \usepackage{mathspec}
    \usepackage{xltxtra,xunicode}
  \else
    \usepackage{fontspec}
  \fi
  \defaultfontfeatures{Mapping=tex-text,Scale=MatchLowercase}
  \newcommand{\euro}{€}
\fi
% use microtype if available
\IfFileExists{microtype.sty}{\usepackage{microtype}}{}
\usepackage{longtable,booktabs}
\ifxetex
  \usepackage[setpagesize=false, % page size defined by xetex
              unicode=false, % unicode breaks when used with xetex
              xetex]{hyperref}
\else
  \usepackage[unicode=true]{hyperref}
\fi
\hypersetup{breaklinks=true,
            bookmarks=true,
            pdfauthor={},
            pdftitle={AptPkg::Version(3pm)},
            colorlinks=true,
            citecolor=blue,
            urlcolor=blue,
            linkcolor=magenta,
            pdfborder={0 0 0}}
\urlstyle{same}  % don't use monospace font for urls
\setlength{\parindent}{0pt}
\setlength{\parskip}{6pt plus 2pt minus 1pt}
\setlength{\emergencystretch}{3em}  % prevent overfull lines
\setcounter{secnumdepth}{0}
\usepackage{pagecolor}

% Set background colour (of the page)
\definecolor{weirdbgcolor}{HTML}{FCF4F0}
\pagecolor{weirdbgcolor}

% Make bold text appear in a particular colour
\definecolor{boldcolor}{HTML}{6E0002}
\let\realtextbf=\textbf
\renewcommand{\textbf}[1]{\textcolor{boldcolor}{\realtextbf{#1}}}

% Use underlines instead of emphasis (ugh)
\renewcommand{\emph}[1]{\underline{#1}}

% % Use fixed-width font by default
% \renewcommand*\familydefault{\ttdefault}

\title{AptPkg::Version(3pm)}
\author{}
\date{}

\begin{document}
\maketitle

\begin{longtable}[c]{@{}lll@{}}
\toprule\addlinespace
AptPkg::Version(3pm) & User Contributed Perl Documentation &
AptPkg::Version(3pm)
\\\addlinespace
\bottomrule
\end{longtable}

\hyperdef{}{NAME}{\section{\hyperref[NAME]{NAME}}\label{NAME}}

AptPkg::Version - APT package versioning class

\hyperdef{}{SYNOPSIS}{\section{\hyperref[SYNOPSIS]{SYNOPSIS}}\label{SYNOPSIS}}

use AptPkg::Version;

\hyperdef{}{DESCRIPTION}{\section{\hyperref[DESCRIPTION]{DESCRIPTION}}\label{DESCRIPTION}}

The AptPkg::Version module provides an interface to \textbf{APT}'s
package version handling.

\hyperdef{}{AptPkg::Version}{\subsection{\hyperref[AptPkg::Version]{AptPkg::Version}}\label{AptPkg::Version}}

The AptPkg::Version package implements the \textbf{APT}
pkgVersioningSystem class.

An instance of the AptPkg::Version class may be fetched using the
``versioning'' method from an AptPkg::System object.

The following methods are implemented:

\begin{description}
\item[label]
Return the description of the versioning system, for example:

~

\begin{verbatim}
    Standard .deb
    
\end{verbatim}

~

for Debian systems.
\end{description}

\begin{description}
\itemsep1pt\parskip0pt\parsep0pt
\item[compare(\emph{A}, \emph{B})]
Compare package version \emph{A} with \emph{B}, returning a negative
value if \emph{A} is an earlier version than \emph{B}, zero if the same
or a positive value if \emph{A} is later.
\end{description}

\begin{description}
\itemsep1pt\parskip0pt\parsep0pt
\item[rel\_compare(\emph{A}, \emph{B})]
Compare distribution release numbers.
\end{description}

\begin{description}
\item[check\_dep(\emph{PKG}, \emph{OP}, \emph{DEP})]
Check that the package version \emph{PKG} satisfies the relation
\emph{OP} to the dependency version \emph{DEP}.

~

The relation \emph{OP} is specified in the Debian syntax regardless of
the versioning system:

~

\begin{verbatim}
    <<  strictly earlier
    <=  earlier or equal
    =   exactly equal
    >=  later or equal
    >>  strictly later
    
\end{verbatim}
\end{description}

\begin{description}
\itemsep1pt\parskip0pt\parsep0pt
\item[upstream(\emph{VER})]
Return the upstream component of the given version string.
\end{description}

\hyperdef{}{SEEux5fALSO}{\section{\hyperref[SEEux5fALSO]{SEE
ALSO}}\label{SEEux5fALSO}}

\emph{AptPkg::Config}(3pm), \emph{AptPkg::System}(3pm),
\emph{AptPkg}(3pm).

\hyperdef{}{AUTHOR}{\section{\hyperref[AUTHOR]{AUTHOR}}\label{AUTHOR}}

Brendan O'Dea \textless{}bod@debian.org\textgreater{}

\begin{longtable}[c]{@{}ll@{}}
\toprule\addlinespace
2013-03-01 & perl v5.18.1
\\\addlinespace
\bottomrule
\end{longtable}

\end{document}
