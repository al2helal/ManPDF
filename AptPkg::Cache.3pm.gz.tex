\documentclass[]{article}
\usepackage[T1]{fontenc}
\usepackage{lmodern}
\usepackage{amssymb,amsmath}
\usepackage{ifxetex,ifluatex}
\usepackage{fixltx2e} % provides \textsubscript
% use upquote if available, for straight quotes in verbatim environments
\IfFileExists{upquote.sty}{\usepackage{upquote}}{}
\ifnum 0\ifxetex 1\fi\ifluatex 1\fi=0 % if pdftex
  \usepackage[utf8]{inputenc}
\else % if luatex or xelatex
  \ifxetex
    \usepackage{mathspec}
    \usepackage{xltxtra,xunicode}
  \else
    \usepackage{fontspec}
  \fi
  \defaultfontfeatures{Mapping=tex-text,Scale=MatchLowercase}
  \newcommand{\euro}{€}
\fi
% use microtype if available
\IfFileExists{microtype.sty}{\usepackage{microtype}}{}
\usepackage{longtable,booktabs}
\ifxetex
  \usepackage[setpagesize=false, % page size defined by xetex
              unicode=false, % unicode breaks when used with xetex
              xetex]{hyperref}
\else
  \usepackage[unicode=true]{hyperref}
\fi
\hypersetup{breaklinks=true,
            bookmarks=true,
            pdfauthor={},
            pdftitle={AptPkg::Cache(3pm)},
            colorlinks=true,
            citecolor=blue,
            urlcolor=blue,
            linkcolor=magenta,
            pdfborder={0 0 0}}
\urlstyle{same}  % don't use monospace font for urls
\setlength{\parindent}{0pt}
\setlength{\parskip}{6pt plus 2pt minus 1pt}
\setlength{\emergencystretch}{3em}  % prevent overfull lines
\setcounter{secnumdepth}{0}
\usepackage{pagecolor}

% Set background colour (of the page)
\definecolor{weirdbgcolor}{HTML}{FCF4F0}
\pagecolor{weirdbgcolor}

% Make bold text appear in a particular colour
\definecolor{boldcolor}{HTML}{6E0002}
\let\realtextbf=\textbf
\renewcommand{\textbf}[1]{\textcolor{boldcolor}{\realtextbf{#1}}}

% Use underlines instead of emphasis (ugh)
\renewcommand{\emph}[1]{\underline{#1}}

% % Use fixed-width font by default
% \renewcommand*\familydefault{\ttdefault}

\title{AptPkg::Cache(3pm)}
\author{}
\date{}

\begin{document}
\maketitle

\begin{longtable}[c]{@{}lll@{}}
\toprule\addlinespace
AptPkg::Cache(3pm) & User Contributed Perl Documentation &
AptPkg::Cache(3pm)
\\\addlinespace
\bottomrule
\end{longtable}

\hyperdef{}{NAME}{\section{\hyperref[NAME]{NAME}}\label{NAME}}

AptPkg::Cache - APT package cache interface

\hyperdef{}{SYNOPSIS}{\section{\hyperref[SYNOPSIS]{SYNOPSIS}}\label{SYNOPSIS}}

use AptPkg::Cache;

\hyperdef{}{DESCRIPTION}{\section{\hyperref[DESCRIPTION]{DESCRIPTION}}\label{DESCRIPTION}}

The AptPkg::Cache module provides an interface to \textbf{APT}'s package
cache.

\hyperdef{}{AptPkg::Cache}{\subsection{\hyperref[AptPkg::Cache]{AptPkg::Cache}}\label{AptPkg::Cache}}

The AptPkg::Cache package implements the \textbf{APT} pkgCacheFile class
as a hash reference (inherits from AptPkg::hash). The hash keys are the
names of packages in the cache, and the values are
AptPkg::Cache::Package objects (which in turn appear as hash references,
see below).

\emph{Constructor}

\begin{description}
\itemsep1pt\parskip0pt\parsep0pt
\item[new({[}\emph{LOCK}{]})]
Instantiation of the object uses configuration from the
\$AptPkg::Config::\_config and \$AptPkg::System::\_system objects
(automatically initialised if not done explicitly).

~

The cache initialisation can be quite verbose--controlled by the value
of \$\_config-\textgreater{}\{quiet\}, which is set to ``2'' (quiet) if
the \$\_config object is auto-initialised.

~

The cache directory is locked if LOCK is true.

~

It is important to note that the structure of the returned object
contains self-referential elements, so some care must be taken if
attempting to traverse it recursively.
\end{description}

\emph{Methods}

\begin{description}
\itemsep1pt\parskip0pt\parsep0pt
\item[files]
Return a list of AptPkg::Cache::PkgFile objects describing the package
files.
\end{description}

\begin{description}
\itemsep1pt\parskip0pt\parsep0pt
\item[packages]
Return an AptPkg::PkgRecords object which may be used to retrieve
additional information about packages.
\end{description}

\begin{description}
\itemsep1pt\parskip0pt\parsep0pt
\item[get, exists, keys]
These methods are used to implement the hashref abstraction:
\$obj-\textgreater{}get(\$pack) and \$obj-\textgreater{}\{\$pack\} are
equivalent.
\end{description}

\begin{description}
\itemsep1pt\parskip0pt\parsep0pt
\item[is\_multi\_arch]
Cache is multi-arch enabled.
\end{description}

\begin{description}
\itemsep1pt\parskip0pt\parsep0pt
\item[native\_arch]
Native architecture.
\end{description}

\hyperdef{}{AptPkg::Cache::Package}{\subsection{\hyperref[AptPkg::Cache::Package]{AptPkg::Cache::Package}}\label{AptPkg::Cache::Package}}

Implements the \textbf{APT} pkgCache::PkgIterator class as a hash
reference.

\emph{Keys}

\begin{description}
\item[Name]
\end{description}

\begin{description}
\item[Section]
\end{description}

\begin{description}
\itemsep1pt\parskip0pt\parsep0pt
\item[Arch]
Package name, section and architecture.
\end{description}

\begin{description}
\itemsep1pt\parskip0pt\parsep0pt
\item[FullName]
Fully qualified name, including architecture.
\end{description}

\begin{description}
\itemsep1pt\parskip0pt\parsep0pt
\item[ShortName]
The shortest unambiguous package name: the same as ``Name'' for native
packages, and ``FullName'' for non-native.
\end{description}

\begin{description}
\item[SelectedState]
\end{description}

\begin{description}
\item[InstState]
\end{description}

\begin{description}
\itemsep1pt\parskip0pt\parsep0pt
\item[CurrentState]
Package state from the status file.

~

SelectedState may be ``Unknown'', ``Install'', ``Hold'', ``DeInstall''
or ``Purge''.

~

InstState may be ``Ok'', ``ReInstReq'', ``HoldInst'' or
``HoldReInstReq''.

~

CurrentState may be ``NotInstalled'', ``UnPacked'', ``HalfConfigured'',
``HalfInstalled'', ``ConfigFiles'' or ``Installed''.

~

In a numeric context, SelectedState, InstState and CurrentState evaluate
to an AptPkg::State:: constant.
\end{description}

\begin{description}
\itemsep1pt\parskip0pt\parsep0pt
\item[VersionList]
A reference to an array of AptPkg::Cache::Version objects describing
available versions of the package.
\end{description}

\begin{description}
\itemsep1pt\parskip0pt\parsep0pt
\item[CurrentVer]
An AptPkg::Cache::Version object describing the currently installed
version (if any) of the package.
\end{description}

\begin{description}
\itemsep1pt\parskip0pt\parsep0pt
\item[RevDependsList]
A reference to an array of AptPkg::Cache::Depends objects describing
packages which depend upon the current package.
\end{description}

\begin{description}
\itemsep1pt\parskip0pt\parsep0pt
\item[ProvidesList]
For virtual packages, this is a reference to an array of
AptPkg::Cache::Provides objects describing packages which provide the
current package.
\end{description}

\begin{description}
\itemsep1pt\parskip0pt\parsep0pt
\item[Flags]
A comma separated list if flags: ``Auto'', ``Essential'' or
``Important''.

~

In a numeric context, evaluates to a combination of AptPkg::Flag::
constants.

~

{[}Note: the only one of these you need worry about is ``Essential'',
which is set based on the package's header of the same name. ``Auto''
seems to be used internally by \textbf{APT}, and ``Important'' seems to
only be set on the apt package.{]}
\end{description}

\begin{description}
\itemsep1pt\parskip0pt\parsep0pt
\item[Index]
Internal \textbf{APT} unique reference for the package record.
\end{description}

\hyperdef{}{AptPkg::Cache::Version}{\subsection{\hyperref[AptPkg::Cache::Version]{AptPkg::Cache::Version}}\label{AptPkg::Cache::Version}}

Implements the \textbf{APT} pkgCache::VerIterator class as a hash
reference.

\emph{Keys}

\begin{description}
\item[VerStr]
\end{description}

\begin{description}
\item[Section]
\end{description}

\begin{description}
\itemsep1pt\parskip0pt\parsep0pt
\item[Arch]
The package version, section and architecture.
\end{description}

\begin{description}
\itemsep1pt\parskip0pt\parsep0pt
\item[MultiArch]
Multi-arch state: ``None'', ``All'', ``Foreign'', ``Same'', ``Allowed'',
``AllForeign'' or ``AllAllowed''.

~

In a numeric context, evaluates to an AptPkg::Version:: constant.
\end{description}

\begin{description}
\itemsep1pt\parskip0pt\parsep0pt
\item[ParentPkg]
An AptPkg::Cache::Package objct describing the package providing this
version.
\end{description}

\begin{description}
\itemsep1pt\parskip0pt\parsep0pt
\item[DescriptionList]
A list of AptCache::Cache::Description objects describing the files
which descrie a package version. The list includes both Package files
and Translation files, which contain translated Description fields.
\end{description}

\begin{description}
\itemsep1pt\parskip0pt\parsep0pt
\item[TranslatedDescription]
An AptCache::Cache::Description object for the current locale, which
will generally be a Translation file.
\end{description}

\begin{description}
\itemsep1pt\parskip0pt\parsep0pt
\item[DependsList]
A reference to an array of AptPkg::Cache::Depends objects describing
packages which the current package depends upon.
\end{description}

\begin{description}
\itemsep1pt\parskip0pt\parsep0pt
\item[ProvidesList]
A reference to an array of AptPkg::Cache::Provides objects describing
virtual packages provided by this version.
\end{description}

\begin{description}
\itemsep1pt\parskip0pt\parsep0pt
\item[FileList]
A reference to an array of AptPkg::Cache::VerFile objects describing the
packages files which include the current version.
\end{description}

\begin{description}
\itemsep1pt\parskip0pt\parsep0pt
\item[Size]
The \emph{.deb} file size, in bytes.
\end{description}

\begin{description}
\itemsep1pt\parskip0pt\parsep0pt
\item[InstalledSize]
The disk space used when installed, in bytes.
\end{description}

\begin{description}
\itemsep1pt\parskip0pt\parsep0pt
\item[Index]
Internal \textbf{APT} unique reference for the version record.
\end{description}

\begin{description}
\itemsep1pt\parskip0pt\parsep0pt
\item[Priority]
Package priority: ``important'', ``required'', ``standard'',
``optional'' or ``extra''.

~

In a numeric context, evaluates to an AptPkg::VerPriority:: constant.
\end{description}

\hyperdef{}{AptPkg::Cache::Depends}{\subsection{\hyperref[AptPkg::Cache::Depends]{AptPkg::Cache::Depends}}\label{AptPkg::Cache::Depends}}

Implements the \textbf{APT} pkgCache::DepIterator class as a hash
reference.

\emph{Keys}

\begin{description}
\itemsep1pt\parskip0pt\parsep0pt
\item[DepType]
Type of dependency: ``Depends'', ``PreDepends'', ``Suggests'',
``Recommends'', ``Conflicts'', ``Replaces'' or ``Obsoletes''.

~

In a numeric context, evaluates to an AptPkg::Dep:: constant.
\end{description}

\begin{description}
\item[ParentPkg]
\end{description}

\begin{description}
\itemsep1pt\parskip0pt\parsep0pt
\item[ParentVer]
AptPkg::Cache::Package and AptPkg::Cache::Version objects describing the
package declaring the dependency.
\end{description}

\begin{description}
\itemsep1pt\parskip0pt\parsep0pt
\item[TargetPkg]
AptPkg::Cache::Package object describing the depended package.
\end{description}

\begin{description}
\itemsep1pt\parskip0pt\parsep0pt
\item[TargetVer]
For versioned dependencies, TargetVer is a string giving the version of
the target package required.
\end{description}

\begin{description}
\item[CompType]
\end{description}

\begin{description}
\itemsep1pt\parskip0pt\parsep0pt
\item[CompTypeDeb]
CompType gives a normalised comparison operator (\textgreater{},
\textgreater{}=, etc) describing the relationship to TargetVer (``'' if
none).

~

CompTypeDeb returns Debian-style operators (\textgreater{}\textgreater{}
rather than \textgreater{}).

~

In a numeric context, both CompType and CompTypeDeb evaluate to an
AptPkg::Dep:: constant.

~

Alternative dependencies (Depends: a \textbar{} b) are identified by all
but the last having the AptPkg::Dep::Or bit set in the numeric
representation of CompType (and CompTypeDeb).
\end{description}

\begin{description}
\itemsep1pt\parskip0pt\parsep0pt
\item[Index]
Internal \textbf{APT} unique reference for the dependency record.
\end{description}

\hyperdef{}{AptPkg::Cache::Provides}{\subsection{\hyperref[AptPkg::Cache::Provides]{AptPkg::Cache::Provides}}\label{AptPkg::Cache::Provides}}

Implements the \textbf{APT} pkgCache::PrvIterator class as a hash
reference.

\emph{Keys}

\begin{description}
\itemsep1pt\parskip0pt\parsep0pt
\item[Name]
Name of virtual package.
\end{description}

\begin{description}
\item[OwnerPkg]
\end{description}

\begin{description}
\itemsep1pt\parskip0pt\parsep0pt
\item[OwnerVer]
AptPkg::Cache::Package and AptPkg::Cache::Version objects describing the
providing package.
\end{description}

\begin{description}
\itemsep1pt\parskip0pt\parsep0pt
\item[ProvideVersion]
Version of the virtual package. {[}Not currently supported by dpkg{]}
\end{description}

\begin{description}
\itemsep1pt\parskip0pt\parsep0pt
\item[Index]
Internal \textbf{APT} unique reference for the provides record.
\end{description}

\hyperdef{}{AptPkg::Cache::VerFile}{\subsection{\hyperref[AptPkg::Cache::VerFile]{AptPkg::Cache::VerFile}}\label{AptPkg::Cache::VerFile}}

Implements the \textbf{APT} pkgCache::VerFileIterator class as a hash
reference.

\emph{Keys}

\begin{description}
\itemsep1pt\parskip0pt\parsep0pt
\item[File]
An AptPkg::Cache::PkgFile object describing the packages file.
\end{description}

\begin{description}
\item[Offset]
\end{description}

\begin{description}
\itemsep1pt\parskip0pt\parsep0pt
\item[Size]
The byte offset and length of the entry in the file.
\end{description}

\begin{description}
\itemsep1pt\parskip0pt\parsep0pt
\item[Index]
Internal \textbf{APT} unique reference for the version file record.
\end{description}

\hyperdef{}{AptPkg::Cache::PkgFile}{\subsection{\hyperref[AptPkg::Cache::PkgFile]{AptPkg::Cache::PkgFile}}\label{AptPkg::Cache::PkgFile}}

Implements the \textbf{APT} pkgCache::PkgFileIterator class as a hash
reference.

\emph{Keys}

\begin{description}
\itemsep1pt\parskip0pt\parsep0pt
\item[FileName]
Packages file path.
\end{description}

\begin{description}
\itemsep1pt\parskip0pt\parsep0pt
\item[IndexType]
File type: ``Debian Package Index'', ``Debian dpkg status file''.
\end{description}

\begin{description}
\item[Archive]
\end{description}

\begin{description}
\item[Component]
\end{description}

\begin{description}
\item[Version]
\end{description}

\begin{description}
\item[Origin]
\end{description}

\begin{description}
\item[Label]
\end{description}

\begin{description}
\itemsep1pt\parskip0pt\parsep0pt
\item[Site]
Fields from the Release file.
\end{description}

\begin{description}
\itemsep1pt\parskip0pt\parsep0pt
\item[IsOk]
True if the cache is in sync with this file.
\end{description}

\begin{description}
\itemsep1pt\parskip0pt\parsep0pt
\item[Index]
Internal \textbf{APT} unique reference for the package file record.
\end{description}

\hyperdef{}{AptPkg::Cache::DescFile}{\subsection{\hyperref[AptPkg::Cache::DescFile]{AptPkg::Cache::DescFile}}\label{AptPkg::Cache::DescFile}}

Implements the \textbf{APT} pkgCache::DescFileIterator class as a hash
reference.

\emph{Keys}

\begin{description}
\itemsep1pt\parskip0pt\parsep0pt
\item[File]
An AptPkg::Cache::PkgFile object describing the packages file.
\end{description}

\hyperdef{}{SEEux5fALSO}{\section{\hyperref[SEEux5fALSO]{SEE
ALSO}}\label{SEEux5fALSO}}

\emph{AptPkg::Config}(3pm), \emph{AptPkg::System}(3pm),
\emph{AptPkg}(3pm), \emph{AptPkg::hash}(3pm),
\emph{AptPkg::PkgRecords}(3pm), \emph{AptPkg::Policy}(3pm).

\hyperdef{}{AUTHOR}{\section{\hyperref[AUTHOR]{AUTHOR}}\label{AUTHOR}}

Brendan O'Dea \textless{}bod@debian.org\textgreater{}

\begin{longtable}[c]{@{}ll@{}}
\toprule\addlinespace
2013-06-18 & perl v5.18.1
\\\addlinespace
\bottomrule
\end{longtable}

\end{document}
