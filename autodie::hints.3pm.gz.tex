\documentclass[]{article}
\usepackage[T1]{fontenc}
\usepackage{lmodern}
\usepackage{amssymb,amsmath}
\usepackage{ifxetex,ifluatex}
\usepackage{fixltx2e} % provides \textsubscript
% use upquote if available, for straight quotes in verbatim environments
\IfFileExists{upquote.sty}{\usepackage{upquote}}{}
\ifnum 0\ifxetex 1\fi\ifluatex 1\fi=0 % if pdftex
  \usepackage[utf8]{inputenc}
\else % if luatex or xelatex
  \ifxetex
    \usepackage{mathspec}
    \usepackage{xltxtra,xunicode}
  \else
    \usepackage{fontspec}
  \fi
  \defaultfontfeatures{Mapping=tex-text,Scale=MatchLowercase}
  \newcommand{\euro}{€}
\fi
% use microtype if available
\IfFileExists{microtype.sty}{\usepackage{microtype}}{}
\usepackage{longtable,booktabs}
\ifxetex
  \usepackage[setpagesize=false, % page size defined by xetex
              unicode=false, % unicode breaks when used with xetex
              xetex]{hyperref}
\else
  \usepackage[unicode=true]{hyperref}
\fi
\hypersetup{breaklinks=true,
            bookmarks=true,
            pdfauthor={},
            pdftitle={autodie::hints(3pm)},
            colorlinks=true,
            citecolor=blue,
            urlcolor=blue,
            linkcolor=magenta,
            pdfborder={0 0 0}}
\urlstyle{same}  % don't use monospace font for urls
\setlength{\parindent}{0pt}
\setlength{\parskip}{6pt plus 2pt minus 1pt}
\setlength{\emergencystretch}{3em}  % prevent overfull lines
\setcounter{secnumdepth}{0}
\usepackage{pagecolor}

% Set background colour (of the page)
\definecolor{weirdbgcolor}{HTML}{FCF4F0}
\pagecolor{weirdbgcolor}

% Make bold text appear in a particular colour
\definecolor{boldcolor}{HTML}{6E0002}
\let\realtextbf=\textbf
\renewcommand{\textbf}[1]{\textcolor{boldcolor}{\realtextbf{#1}}}

% Use underlines instead of emphasis (ugh)
\renewcommand{\emph}[1]{\underline{#1}}

% % Use fixed-width font by default
% \renewcommand*\familydefault{\ttdefault}

\title{autodie::hints(3pm)}
\author{}
\date{}

\begin{document}
\maketitle

\begin{longtable}[c]{@{}lll@{}}
\toprule\addlinespace
autodie::hints(3pm) & User Contributed Perl Documentation &
autodie::hints(3pm)
\\\addlinespace
\bottomrule
\end{longtable}

\hyperdef{}{NAME}{\section{\hyperref[NAME]{NAME}}\label{NAME}}

autodie::hints - Provide hints about user subroutines to autodie

\hyperdef{}{SYNOPSIS}{\section{\hyperref[SYNOPSIS]{SYNOPSIS}}\label{SYNOPSIS}}

\begin{verbatim}
    package Your::Module;
    our %DOES = ( 'autodie::hints::provider' => 1 );
    sub AUTODIE_HINTS {
        return {
            foo => { scalar => HINTS, list => SOME_HINTS },
            bar => { scalar => HINTS, list => MORE_HINTS },
        }
    }
    # Later, in your main program...
    use Your::Module qw(foo bar);
    use autodie      qw(:default foo bar);
    foo();         # succeeds or dies based on scalar hints
    # Alternatively, hints can be set on subroutines we've
    # imported.
    use autodie::hints;
    use Some::Module qw(think_positive);
    BEGIN {
        autodie::hints->set_hints_for(
            \&think_positive,
            {
                fail => sub { $_[0] <= 0 }
            }
        )
    }
    use autodie qw(think_positive);
    think_positive(...);    # Returns positive or dies.
\end{verbatim}

\hyperdef{}{DESCRIPTION}{\section{\hyperref[DESCRIPTION]{DESCRIPTION}}\label{DESCRIPTION}}

\hyperdef{}{Introduction}{\subsection{\hyperref[Introduction]{Introduction}}\label{Introduction}}

The autodie pragma is very smart when it comes to working with Perl's
built-in functions. The behaviour for these functions are fixed, and
``autodie'' knows exactly how they try to signal failure.

But what about user-defined subroutines from modules? If you use
``autodie'' on a user-defined subroutine then it assumes the following
behaviour to demonstrate failure:

\begin{description}
\itemsep1pt\parskip0pt\parsep0pt
\item[•]
A false value, in scalar context
\end{description}

\begin{description}
\itemsep1pt\parskip0pt\parsep0pt
\item[•]
An empty list, in list context
\end{description}

\begin{description}
\itemsep1pt\parskip0pt\parsep0pt
\item[•]
A list containing a single undef, in list context
\end{description}

All other return values (including the list of the single zero, and the
list containing a single empty string) are considered successful.
However, real-world code isn't always that easy. Perhaps the code you're
working with returns a string containing the word ``FAIL'' upon failure,
or a two element list containing ``(undef,''human error message``)''. To
make autodie work with these sorts of subroutines, we have the
\emph{hinting interface}.

The hinting interface allows \emph{hints} to be provided to ``autodie''
on how it should detect failure from user-defined subroutines. While
these \emph{can} be provided by the end-user of ``autodie'', they are
ideally written into the module itself, or into a helper module or
sub-class of ``autodie'' itself.

\hyperdef{}{Whatux5fareux5fhintsux3f}{\subsection{\hyperref[Whatux5fareux5fhintsux3f]{What
are hints?}}\label{Whatux5fareux5fhintsux3f}}

A \emph{hint} is a subroutine or value that is checked against the
return value of an autodying subroutine. If the match returns true,
``autodie'' considers the subroutine to have failed.

If the hint provided is a subroutine, then ``autodie'' will pass the
complete return value to that subroutine. If the hint is any other
value, then ``autodie'' will smart-match against the value provided. In
Perl 5.8.x there is no smart-match operator, and as such only subroutine
hints are supported in these versions.

Hints can be provided for both scalar and list contexts. Note that an
autodying subroutine will never see a void context, as ``autodie''
always needs to capture the return value for examination. Autodying
subroutines called in void context act as if they're called in a scalar
context, but their return value is discarded after it has been checked.

\hyperdef{}{Exampleux5fhints}{\subsection{\hyperref[Exampleux5fhints]{Example
hints}}\label{Exampleux5fhints}}

Hints may consist of scalars, array references, regular expressions and
subroutine references. You can specify different hints for how failure
should be identified in scalar and list contexts.

These examples apply for use in the ``AUTODIE\_HINTS'' subroutine and
when calling ``autodie::hints-'' \emph{set\_hints\_for()}\textgreater{}.

The most common context-specific hints are:

\begin{verbatim}
        # Scalar failures always return undef:
            {  scalar => undef  }
        # Scalar failures return any false value [default expectation]:
            {  scalar => sub { ! $_[0] }  }
        # Scalar failures always return zero explicitly:
            {  scalar => '0'  }
        # List failures always return an empty list:
            {  list => []  }
        # List failures return () or (undef) [default expectation]:
            {  list => sub { ! @_ || @_ == 1 && !defined $_[0] }  }
        # List failures return () or a single false value:
            {  list => sub { ! @_ || @_ == 1 && !$_[0] }  }
        # List failures return (undef, "some string")
            {  list => sub { @_ == 2 && !defined $_[0] }  }
        # Unsuccessful foo() returns 'FAIL' or '_FAIL' in scalar context,
        #                    returns (-1) in list context...
        autodie::hints->set_hints_for(
            \&foo,
            {
                scalar => qr/^ _? FAIL $/xms,
                list   => [-1],
            }
        );
        # Unsuccessful foo() returns 0 in all contexts...
        autodie::hints->set_hints_for(
            \&foo,
            {
                scalar => 0,
                list   => [0],
            }
        );
\end{verbatim}

This ``in all contexts'' construction is very common, and can be
abbreviated, using the `fail' key. This sets both the ``scalar'' and
``list'' hints to the same value:

\begin{verbatim}
        # Unsuccessful foo() returns 0 in all contexts...
        autodie::hints->set_hints_for(
            \&foo,
            {
                fail => sub { @_ == 1 and defined $_[0] and $_[0] == 0 }
            }
        );
        # Unsuccessful think_positive() returns negative number on failure...
        autodie::hints->set_hints_for(
            \&think_positive,
            {
                fail => sub { $_[0] < 0 }
            }
        );
        # Unsuccessful my_system() returns non-zero on failure...
        autodie::hints->set_hints_for(
            \&my_system,
            {
                fail => sub { $_[0] != 0 }
            }
        );
\end{verbatim}

\hyperdef{}{Manuallyux5fsettingux5fhintsux5ffromux5fwithinux5fyourux5fprogram}{\section{\hyperref[Manuallyux5fsettingux5fhintsux5ffromux5fwithinux5fyourux5fprogram]{Manually
setting hints from within your
program}}\label{Manuallyux5fsettingux5fhintsux5ffromux5fwithinux5fyourux5fprogram}}

If you are using a module which returns something special on failure,
then you can manually create hints for each of the desired subroutines.
Once the hints are specified, they are available for all files and
modules loaded thereafter, thus you can move this work into a module and
it will still work.

\begin{verbatim}
        use Some::Module qw(foo bar);
        use autodie::hints;
        autodie::hints->set_hints_for(
                \&foo,
                {
                        scalar => SCALAR_HINT,
                        list   => LIST_HINT,
                }
        );
        autodie::hints->set_hints_for(
                \&bar,
                { fail => SOME_HINT, }
        );
\end{verbatim}

It is possible to pass either a subroutine reference (recommended) or a
fully qualified subroutine name as the first argument. This means you
can set hints on modules that \emph{might} get loaded:

\begin{verbatim}
        use autodie::hints;
        autodie::hints->set_hints_for(
                'Some::Module:bar', { fail => SCALAR_HINT, }
        );
\end{verbatim}

This technique is most useful when you have a project that uses a lot of
third-party modules. You can define all your possible hints in
one-place. This can even be in a sub-class of autodie. For example:

\begin{verbatim}
        package my::autodie;
        use parent qw(autodie);
        use autodie::hints;
        autodie::hints->set_hints_for(...);
        1;
\end{verbatim}

You can now ``use my::autodie'', which will work just like the standard
``autodie'', but is now aware of any hints that you've set.

\hyperdef{}{Addingux5fhintsux5ftoux5fyourux5fmodule}{\section{\hyperref[Addingux5fhintsux5ftoux5fyourux5fmodule]{Adding
hints to your module}}\label{Addingux5fhintsux5ftoux5fyourux5fmodule}}

``autodie'' provides a passive interface to allow you to declare hints
for your module. These hints will be found and used by ``autodie'' if it
is loaded, but otherwise have no effect (or dependencies) without
autodie. To set these, your module needs to declare that it \emph{does}
the ``autodie::hints::provider'' role. This can be done by writing your
own ``DOES'' method, using a system such as ``Class::DOES'' to handle
the heavy-lifting for you, or declaring a \%DOES package variable with a
``autodie::hints::provider'' key and a corresponding true value.

Note that checking for a \%DOES hash is an ``autodie''-only short-cut.
Other modules do not use this mechanism for checking roles, although you
can use the ``Class::DOES'' module from the CPAN to allow it.

In addition, you must define a ``AUTODIE\_HINTS'' subroutine that
returns a hash-reference containing the hints for your subroutines:

\begin{verbatim}
        package Your::Module;
        # We can use the Class::DOES from the CPAN to declare adherence
        # to a role.
        use Class::DOES 'autodie::hints::provider' => 1;
        # Alternatively, we can declare the role in %DOES.  Note that
        # this is an autodie specific optimisation, although Class::DOES
        # can be used to promote this to a true role declaration.
        our %DOES = ( 'autodie::hints::provider' => 1 );
        # Finally, we must define the hints themselves.
        sub AUTODIE_HINTS {
            return {
                foo => { scalar => HINTS, list => SOME_HINTS },
                bar => { scalar => HINTS, list => MORE_HINTS },
                baz => { fail => HINTS },
            }
        }
\end{verbatim}

This allows your code to set hints without relying on ``autodie'' and
``autodie::hints'' being loaded, or even installed. In this way your
code can do the right thing when ``autodie'' is installed, but does not
need to depend upon it to function.

\hyperdef{}{Insistingux5fonux5fhints}{\section{\hyperref[Insistingux5fonux5fhints]{Insisting
on hints}}\label{Insistingux5fonux5fhints}}

When a user-defined subroutine is wrapped by ``autodie'', it will use
hints if they are available, and otherwise reverts to the \emph{default
behaviour} described in the introduction of this document. This can be
problematic if we expect a hint to exist, but (for whatever reason) it
has not been loaded.

We can ask autodie to \emph{insist} that a hint be used by prefixing an
exclamation mark to the start of the subroutine name. A lone exclamation
mark indicates that \emph{all} subroutines after it must have hints
declared.

\begin{verbatim}
        # foo() and bar() must have their hints defined
        use autodie qw( !foo !bar baz );
        # Everything must have hints (recommended).
        use autodie qw( ! foo bar baz );
        # bar() and baz() must have their hints defined
        use autodie qw( foo ! bar baz );
        # Enable autodie for all of Perl's supported built-ins,
        # as well as for foo(), bar() and baz().  Everything must
        # have hints.
        use autodie qw( ! :all foo bar baz );
\end{verbatim}

If hints are not available for the specified subroutines, this will
cause a compile-time error. Insisting on hints for Perl's built-in
functions (eg, ``open'' and ``close'') is always successful.

Insisting on hints is \emph{strongly} recommended.

\hyperdef{}{Diagnostics}{\section{\hyperref[Diagnostics]{Diagnostics}}\label{Diagnostics}}

\begin{description}
\itemsep1pt\parskip0pt\parsep0pt
\item[Attempts to set\_hints\_for unidentifiable subroutine]
You've called ``autodie::hints-\textgreater{}set\_hints\_for()'' using a
subroutine reference, but that reference could not be resolved back to a
subroutine name. It may be an anonymous subroutine (which can't be made
autodying), or may lack a name for other reasons.

~

If you receive this error with a subroutine that has a real name, then
you may have found a bug in autodie. See ``BUGS'' in autodie for how to
report this.
\end{description}

\begin{description}
\itemsep1pt\parskip0pt\parsep0pt
\item[fail hints cannot be provided with either scalar or list hints for
\%s]
When defining hints, you can either supply both ``list'' and ``scalar''
keywords, \emph{or} you can provide a single ``fail'' keyword. You can't
mix and match them.
\end{description}

\begin{description}
\itemsep1pt\parskip0pt\parsep0pt
\item[\%s hint missing for \%s]
You've provided either a ``scalar'' hint without supplying a ``list''
hint, or vice-versa. You \emph{must} supply both ``scalar'' and ``list''
hints, \emph{or} a single ``fail'' hint.
\end{description}

\hyperdef{}{ACKNOWLEDGEMENTS}{\section{\hyperref[ACKNOWLEDGEMENTS]{ACKNOWLEDGEMENTS}}\label{ACKNOWLEDGEMENTS}}

\begin{description}
\itemsep1pt\parskip0pt\parsep0pt
\item[•]
Dr Damian Conway for suggesting the hinting interface and providing the
example usage.
\end{description}

\begin{description}
\itemsep1pt\parskip0pt\parsep0pt
\item[•]
Jacinta Richardson for translating much of my ideas into this
documentation.
\end{description}

\hyperdef{}{AUTHOR}{\section{\hyperref[AUTHOR]{AUTHOR}}\label{AUTHOR}}

Copyright 2009, Paul Fenwick
\textless{}pjf@perltraining.com.au\textgreater{}

\hyperdef{}{LICENSE}{\section{\hyperref[LICENSE]{LICENSE}}\label{LICENSE}}

This module is free software. You may distribute it under the same terms
as Perl itself.

\hyperdef{}{SEEux5fALSO}{\section{\hyperref[SEEux5fALSO]{SEE
ALSO}}\label{SEEux5fALSO}}

autodie, Class::DOES

\begin{longtable}[c]{@{}ll@{}}
\toprule\addlinespace
2014-01-27 & perl v5.18.2
\\\addlinespace
\bottomrule
\end{longtable}

\end{document}
