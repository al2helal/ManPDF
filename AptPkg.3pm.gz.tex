\documentclass[]{article}
\usepackage[T1]{fontenc}
\usepackage{lmodern}
\usepackage{amssymb,amsmath}
\usepackage{ifxetex,ifluatex}
\usepackage{fixltx2e} % provides \textsubscript
% use upquote if available, for straight quotes in verbatim environments
\IfFileExists{upquote.sty}{\usepackage{upquote}}{}
\ifnum 0\ifxetex 1\fi\ifluatex 1\fi=0 % if pdftex
  \usepackage[utf8]{inputenc}
\else % if luatex or xelatex
  \ifxetex
    \usepackage{mathspec}
    \usepackage{xltxtra,xunicode}
  \else
    \usepackage{fontspec}
  \fi
  \defaultfontfeatures{Mapping=tex-text,Scale=MatchLowercase}
  \newcommand{\euro}{€}
\fi
% use microtype if available
\IfFileExists{microtype.sty}{\usepackage{microtype}}{}
\usepackage{longtable,booktabs}
\ifxetex
  \usepackage[setpagesize=false, % page size defined by xetex
              unicode=false, % unicode breaks when used with xetex
              xetex]{hyperref}
\else
  \usepackage[unicode=true]{hyperref}
\fi
\hypersetup{breaklinks=true,
            bookmarks=true,
            pdfauthor={},
            pdftitle={AptPkg(3pm)},
            colorlinks=true,
            citecolor=blue,
            urlcolor=blue,
            linkcolor=magenta,
            pdfborder={0 0 0}}
\urlstyle{same}  % don't use monospace font for urls
\setlength{\parindent}{0pt}
\setlength{\parskip}{6pt plus 2pt minus 1pt}
\setlength{\emergencystretch}{3em}  % prevent overfull lines
\setcounter{secnumdepth}{0}
\usepackage{pagecolor}

% Set background colour (of the page)
\definecolor{weirdbgcolor}{HTML}{FCF4F0}
\pagecolor{weirdbgcolor}

% Make bold text appear in a particular colour
\definecolor{boldcolor}{HTML}{6E0002}
\let\realtextbf=\textbf
\renewcommand{\textbf}[1]{\textcolor{boldcolor}{\realtextbf{#1}}}

% Use underlines instead of emphasis (ugh)
\renewcommand{\emph}[1]{\underline{#1}}

% % Use fixed-width font by default
% \renewcommand*\familydefault{\ttdefault}

\title{AptPkg(3pm)}
\author{}
\date{}

\begin{document}
\maketitle

\begin{longtable}[c]{@{}lll@{}}
\toprule\addlinespace
AptPkg(3pm) & User Contributed Perl Documentation & AptPkg(3pm)
\\\addlinespace
\bottomrule
\end{longtable}

\hyperdef{}{NAME}{\section{\hyperref[NAME]{NAME}}\label{NAME}}

AptPkg - interface to libapt-pkg

\hyperdef{}{SYNOPSIS}{\section{\hyperref[SYNOPSIS]{SYNOPSIS}}\label{SYNOPSIS}}

use AptPkg;

\hyperdef{}{DESCRIPTION}{\section{\hyperref[DESCRIPTION]{DESCRIPTION}}\label{DESCRIPTION}}

The AptPkg module provides a low-level XS interface to libapt-pkg.

Note that this interface is intended to be internal, and may change, see
the AptPkg::Config, AptPkg::System, AptPkg::Version, AptPkg::Cache,
Apt::Policy and AptPkg::Source classes for a higher level interface.

\hyperdef{}{AptPkg}{\subsection{\hyperref[AptPkg]{AptPkg}}\label{AptPkg}}

The AptPkg package provides the following functions:

\begin{description}
\itemsep1pt\parskip0pt\parsep0pt
\item[\_init\_config(\emph{CONF})]
Initialise a Configuration object (pkgInitConfig). See the init method
in AptPkg::Config.
\end{description}

\begin{description}
\itemsep1pt\parskip0pt\parsep0pt
\item[\_init\_system(\emph{CONF})]
Return a pointer to the system object (pkgInitSystem). See the system
method in AptPkg::Config.
\end{description}

\begin{description}
\itemsep1pt\parskip0pt\parsep0pt
\item[\_parse\_cmdline(\emph{CONF}, \emph{ARG\_DEFS}, \ldots{})]
Constructs a CommandLine instance, invokes the Parse method and returns
the remaining arguments. See the parse\_cmdline method in
AptPkg::Config.
\end{description}

\hyperdef{}{AptPkg::ux5fconfig}{\subsection{\hyperref[AptPkg::ux5fconfig]{AptPkg::\_config}}\label{AptPkg::ux5fconfig}}

The AptPkg::\_config package wraps a Perl class around the Configuration
class. It provides an instance of the global \_config object, and
exposes the following methods:

\begin{verbatim}
    Find, FindFile, FindDir, FindB, FindAny, Set, Exists,
    ExistsAny, Tree and Dump.
\end{verbatim}

The functions ReadConfigFile and ReadConfigDir are also provided within
the package and may be used as methods.

\hyperdef{}{AptPkg::ux5fconfig::item}{\subsection{\hyperref[AptPkg::ux5fconfig::item]{AptPkg::\_config::item}}\label{AptPkg::ux5fconfig::item}}

The AptPkg::\_config::item package wraps a Perl class around the
Configuration::Item class. The AptPkg::\_config Tree method returns an
instance of this class.

Methods:

\begin{verbatim}
    Value, Tag, FullTag, Parent, Child and Next.
\end{verbatim}

\hyperdef{}{AptPkg::System}{\subsection{\hyperref[AptPkg::System]{AptPkg::System}}\label{AptPkg::System}}

The AptPkg::System package wraps a Perl class around the pkgSystem
class. It provides an instance of the global \_system object, and
exposes the following methods:

\begin{verbatim}
    Label, VS, Lock and UnLock.
\end{verbatim}

\hyperdef{}{AptPkg::Version}{\subsection{\hyperref[AptPkg::Version]{AptPkg::Version}}\label{AptPkg::Version}}

The AptPkg::Version package wraps a Perl class around the
pkgVersioningSystem class. It exposes the following methods:

\begin{verbatim}
    Label, CmpVersion, CmpReleaseVer, CheckDep and UpstreamVersion.
\end{verbatim}

\hyperdef{}{AptPkg::ux5fcache}{\subsection{\hyperref[AptPkg::ux5fcache]{AptPkg::\_cache}}\label{AptPkg::ux5fcache}}

The AptPkg::\_cache package wraps a Perl class around the pkgCacheFile
class. It exposes the following methods:

\begin{verbatim}
    Open, Close, FindPkg, PkgBegin, FileList, Packages, Policy, MultiArchCache
    and NativeArch.
\end{verbatim}

\hyperdef{}{AptPkg::Cache::ux5fpackage}{\subsection{\hyperref[AptPkg::Cache::ux5fpackage]{AptPkg::Cache::\_package}}\label{AptPkg::Cache::ux5fpackage}}

The AptPkg::Cache::\_package package wraps a Perl class around the
pkgCache::PkgIterator class. It exposes the following methods:

\begin{verbatim}
    Next, Name, FullName, Arch, Section, VersionList, CurrentVer,
    RevDependsList, ProvidesList, Index, SelectedState, InstState,
    CurrentState and Flags.
\end{verbatim}

\hyperdef{}{AptPkg::Cache::ux5fversion}{\subsection{\hyperref[AptPkg::Cache::ux5fversion]{AptPkg::Cache::\_version}}\label{AptPkg::Cache::ux5fversion}}

The AptPkg::Cache::\_version package wraps a Perl class around the
pkgCache::VerIterator class. It exposes the following methods:

\begin{verbatim}
    VerStr, Section, MultiArch, Arch, ParentPkg, DescriptionList,
    TranslatedDescription, DependsList, ProvidesList, FileList, Index
    and Priority.
\end{verbatim}

\hyperdef{}{AptPkg::Cache::ux5fdepends}{\subsection{\hyperref[AptPkg::Cache::ux5fdepends]{AptPkg::Cache::\_depends}}\label{AptPkg::Cache::ux5fdepends}}

The AptPkg::Cache::\_depends package wraps a Perl class around the
pkgCache::DepIterator class. It exposes the following methods:

\begin{verbatim}
    TargetVer, TargetPkg, ParentVer, ParentPkg, Index, CompType and
    DepType.
\end{verbatim}

\hyperdef{}{AptPkg::Cache::ux5fprovides}{\subsection{\hyperref[AptPkg::Cache::ux5fprovides]{AptPkg::Cache::\_provides}}\label{AptPkg::Cache::ux5fprovides}}

The AptPkg::Cache::\_provides package wraps a Perl class around the
pkgCache::PrvIterator class. It exposes the following methods:

\begin{verbatim}
    Name, ProvideVersion, OwnerVer, OwnerPkg and Index.
\end{verbatim}

\hyperdef{}{AptPkg::Cache::ux5fdescription}{\subsection{\hyperref[AptPkg::Cache::ux5fdescription]{AptPkg::Cache::\_description}}\label{AptPkg::Cache::ux5fdescription}}

The AptPkg::Cache::\_description package wraps a Perl class around the
pkgCache::DescIterator class. It exposes the following methods:

\begin{verbatim}
    LanguageCode, md5 and FileList.
\end{verbatim}

\hyperdef{}{AptPkg::Cache::ux5fpkgux5ffile}{\subsection{\hyperref[AptPkg::Cache::ux5fpkgux5ffile]{AptPkg::Cache::\_pkg\_file}}\label{AptPkg::Cache::ux5fpkgux5ffile}}

The AptPkg::Cache::\_pkg\_file package wraps a Perl class around the
pkgCache::PkgFileIterator class. It exposes the following methods:

\begin{verbatim}
    FileName, Archive, Component, Version, Origin, Label, Site,
    IndexType and Index.
\end{verbatim}

\hyperdef{}{AptPkg::Cache::ux5fverux5ffile}{\subsection{\hyperref[AptPkg::Cache::ux5fverux5ffile]{AptPkg::Cache::\_ver\_file}}\label{AptPkg::Cache::ux5fverux5ffile}}

The AptPkg::Cache::\_ver\_file package wraps a Perl class around the
pkgCache::VerFileIterator class. It exposes the following methods:

\begin{verbatim}
    File, Index, Offset and Size.
\end{verbatim}

\hyperdef{}{AptPkg::Cache::ux5fdescux5ffile}{\subsection{\hyperref[AptPkg::Cache::ux5fdescux5ffile]{AptPkg::Cache::\_desc\_file}}\label{AptPkg::Cache::ux5fdescux5ffile}}

The AptPkg::Cache::\_desc\_file package wraps a Perl class around the
pkgCache::DescFileIterator class. It exposes the following methods:

\begin{verbatim}
    File
\end{verbatim}

\hyperdef{}{AptPkg::Cache::ux5fpkgux5frecords}{\subsection{\hyperref[AptPkg::Cache::ux5fpkgux5frecords]{AptPkg::Cache::\_pkg\_records}}\label{AptPkg::Cache::ux5fpkgux5frecords}}

The AptPkg::Cache::\_pkg\_records package wraps a Perl class around the
pkgRecords class. It exposes the following methods:

\begin{verbatim}
    Lookup.
\end{verbatim}

\hyperdef{}{AptPkg::ux5fpolicy}{\subsection{\hyperref[AptPkg::ux5fpolicy]{AptPkg::\_policy}}\label{AptPkg::ux5fpolicy}}

The AptPkg::\_policy package wraps a Perl class around the pkgPolicy
class. It exposes the following methods:

\begin{verbatim}
    GetPriority, GetMatch and GetCandidateVer.
\end{verbatim}

\hyperdef{}{AptPkg::ux5fsourceux5flist}{\subsection{\hyperref[AptPkg::ux5fsourceux5flist]{AptPkg::\_source\_list}}\label{AptPkg::ux5fsourceux5flist}}

The AptPkg::\_source\_list package wraps a Perl class around the
pkgSourceList class. Required as an argument to the
AptPkg::\_src\_records constructor.

\hyperdef{}{AptPkg::ux5fsrcux5frecords}{\subsection{\hyperref[AptPkg::ux5fsrcux5frecords]{AptPkg::\_src\_records}}\label{AptPkg::ux5fsrcux5frecords}}

The AptPkg::\_src\_records package wraps a Perl class around the
pkgSrcRecords class. It exposes the following methods:

\begin{verbatim}
    Restart, Find.
\end{verbatim}

\hyperdef{}{Constants}{\subsection{\hyperref[Constants]{Constants}}\label{Constants}}

The following \textbf{APT} enumerations are included, used by attributes
of AptPkg::Cache.

\emph{pkgCache::Version::VerMultiArch}

``AptPkg::Version::None'', ``AptPkg::Version::All'',
``AptPkg::Version::Foreign'', ``AptPkg::Version::Same'',
``AptPkg::Version::Allowed'', ``AptPkg::Version::AllForeign'' and
``AptPkg::Version::AllAllowed''.

\emph{pkgCache::Dep::DepType}

``AptPkg::Dep::Depends'', ``AptPkg::Dep::PreDepends'',
``AptPkg::Dep::Suggests'', ``AptPkg::Dep::Recommends'',
``AptPkg::Dep::Conflicts'', ``AptPkg::Dep::Replaces'' and
``AptPkg::Dep::Obsoletes''.

\emph{pkgCache::Dep::DepCompareOp}

``AptPkg::Dep::Or'', ``AptPkg::Dep::NoOp'', ``AptPkg::Dep::LessEq'',
``AptPkg::Dep::GreaterEq'', ``AptPkg::Dep::Less'',
``AptPkg::Dep::Greater'', ``AptPkg::Dep::Equals'' and
``AptPkg::Dep::NotEquals''.

\emph{pkgCache::State::VerPriority}

``AptPkg::State::Important'', ``AptPkg::State::Required'',
``AptPkg::State::Standard'', ``AptPkg::State::Optional'' and
``AptPkg::State::Extra''.

\emph{pkgCache::State::PkgSelectedState}

``AptPkg::State::Unknown'', ``AptPkg::State::Install'',
``AptPkg::State::Hold'', ``AptPkg::State::DeInstall'' and
``AptPkg::State::Purge''.

\emph{pkgCache::State::PkgInstState}

``AptPkg::State::Ok'', ``AptPkg::State::ReInstReq'',
``AptPkg::State::HoldInst'' and ``AptPkg::State::HoldReInstReq''.

\emph{pkgCache::State::PkgCurrentState}

``AptPkg::State::NotInstalled'', ``AptPkg::State::UnPacked'',
``AptPkg::State::HalfConfigured'', ``AptPkg::State::HalfInstalled'',
``AptPkg::State::ConfigFiles'' and ``AptPkg::State::Installed''.

\emph{pkgCache::Flag::PkgFlags}

``AptPkg::Flag::Auto'', ``AptPkg::Flag::Essential'' and
``AptPkg::Flag::Important''.

\hyperdef{}{SEEux5fALSO}{\section{\hyperref[SEEux5fALSO]{SEE
ALSO}}\label{SEEux5fALSO}}

\emph{AptPkg::Config}(3pm), \emph{AptPkg::System}(3pm),
\emph{AptPkg::Version}(3pm), \emph{AptPkg::Cache}(3pm),
\emph{AptPkg::Source}(3pm).

\hyperdef{}{AUTHOR}{\section{\hyperref[AUTHOR]{AUTHOR}}\label{AUTHOR}}

Brendan O'Dea \textless{}bod@debian.org\textgreater{}

\begin{longtable}[c]{@{}ll@{}}
\toprule\addlinespace
2013-05-11 & perl v5.18.1
\\\addlinespace
\bottomrule
\end{longtable}

\end{document}
