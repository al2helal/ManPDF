\documentclass[]{article}
\usepackage[T1]{fontenc}
\usepackage{lmodern}
\usepackage{amssymb,amsmath}
\usepackage{ifxetex,ifluatex}
\usepackage{fixltx2e} % provides \textsubscript
% use upquote if available, for straight quotes in verbatim environments
\IfFileExists{upquote.sty}{\usepackage{upquote}}{}
\ifnum 0\ifxetex 1\fi\ifluatex 1\fi=0 % if pdftex
  \usepackage[utf8]{inputenc}
\else % if luatex or xelatex
  \ifxetex
    \usepackage{mathspec}
    \usepackage{xltxtra,xunicode}
  \else
    \usepackage{fontspec}
  \fi
  \defaultfontfeatures{Mapping=tex-text,Scale=MatchLowercase}
  \newcommand{\euro}{€}
\fi
% use microtype if available
\IfFileExists{microtype.sty}{\usepackage{microtype}}{}
\usepackage{longtable,booktabs}
\ifxetex
  \usepackage[setpagesize=false, % page size defined by xetex
              unicode=false, % unicode breaks when used with xetex
              xetex]{hyperref}
\else
  \usepackage[unicode=true]{hyperref}
\fi
\hypersetup{breaklinks=true,
            bookmarks=true,
            pdfauthor={},
            pdftitle={asn1\_encode\_simple\_der(3)},
            colorlinks=true,
            citecolor=blue,
            urlcolor=blue,
            linkcolor=magenta,
            pdfborder={0 0 0}}
\urlstyle{same}  % don't use monospace font for urls
\setlength{\parindent}{0pt}
\setlength{\parskip}{6pt plus 2pt minus 1pt}
\setlength{\emergencystretch}{3em}  % prevent overfull lines
\setcounter{secnumdepth}{0}
\usepackage{pagecolor}

% Set background colour (of the page)
\definecolor{weirdbgcolor}{HTML}{FCF4F0}
\pagecolor{weirdbgcolor}

% Make bold text appear in a particular colour
\definecolor{boldcolor}{HTML}{6E0002}
\let\realtextbf=\textbf
\renewcommand{\textbf}[1]{\textcolor{boldcolor}{\realtextbf{#1}}}

% Use underlines instead of emphasis (ugh)
\renewcommand{\emph}[1]{\underline{#1}}

% % Use fixed-width font by default
% \renewcommand*\familydefault{\ttdefault}

\title{asn1\_encode\_simple\_der(3)}
\author{}
\date{}

\begin{document}
\maketitle

\begin{longtable}[c]{@{}lll@{}}
\toprule\addlinespace
asn1\_encode\_simple\_der(3) & libtasn1 & asn1\_encode\_simple\_der(3)
\\\addlinespace
\bottomrule
\end{longtable}

\hyperdef{}{NAME}{\section{\hyperref[NAME]{NAME}}\label{NAME}}

asn1\_encode\_simple\_der - API function

\hyperdef{}{SYNOPSIS}{\section{\hyperref[SYNOPSIS]{SYNOPSIS}}\label{SYNOPSIS}}

\textbf{\#include \textless{}libtasn1.h\textgreater{}}

~

\textbf{int asn1\_encode\_simple\_der(unsigned int}\emph{etype}\textbf{,
const unsigned char *}\emph{str}\textbf{, unsigned
int}\emph{str\_len}\textbf{, unsigned char *}\emph{tl}\textbf{, unsigned
int *}\emph{tl\_len}\textbf{);}

\hyperdef{}{ARGUMENTS}{\section{\hyperref[ARGUMENTS]{ARGUMENTS}}\label{ARGUMENTS}}

\begin{description}
\itemsep1pt\parskip0pt\parsep0pt
\item[unsigned int etype]
The type of the string to be encoded (ASN1\_ETYPE\_)
\end{description}

\begin{description}
\itemsep1pt\parskip0pt\parsep0pt
\item[const unsigned char * str]
the string data.
\end{description}

\begin{description}
\itemsep1pt\parskip0pt\parsep0pt
\item[unsigned int str\_len]
the string length
\end{description}

\begin{description}
\itemsep1pt\parskip0pt\parsep0pt
\item[unsigned char * tl]
the encoded tag and length
\end{description}

\begin{description}
\itemsep1pt\parskip0pt\parsep0pt
\item[unsigned int * tl\_len]
the bytes of the \emph{tl} field
\end{description}

\hyperdef{}{DESCRIPTION}{\section{\hyperref[DESCRIPTION]{DESCRIPTION}}\label{DESCRIPTION}}

Creates the DER encoding for various simple ASN.1 types like strings
etc. It stores the tag and length in \emph{tl} , which should have space
for at least \textbf{ASN1\_MAX\_TL\_SIZE} bytes. Initially
\emph{tl\_len} should contain the size of \emph{tl} .

~

The complete DER encoding should consist of the value in \emph{tl}
appended with the provided \emph{str} .

\hyperdef{}{RETURNS}{\section{\hyperref[RETURNS]{RETURNS}}\label{RETURNS}}

\textbf{ASN1\_SUCCESS} if successful or an error value.

\hyperdef{}{COPYRIGHT}{\section{\hyperref[COPYRIGHT]{COPYRIGHT}}\label{COPYRIGHT}}

Copyright © 2006-2013 Free Software Foundation, Inc..

~

Copying and distribution of this file, with or without modification, are
permitted in any medium without royalty provided the copyright notice
and this notice are preserved.

\hyperdef{}{SEEux5fALSO}{\section{\hyperref[SEEux5fALSO]{SEE
ALSO}}\label{SEEux5fALSO}}

The full documentation for \textbf{libtasn1} is maintained as a Texinfo
manual. If the \textbf{info} and \textbf{libtasn1} programs are properly
installed at your site, the command

\begin{description}
\itemsep1pt\parskip0pt\parsep0pt
\item[]
\textbf{info libtasn1}
\end{description}

should give you access to the complete manual. As an alternative you may
obtain the manual from:

\begin{description}
\itemsep1pt\parskip0pt\parsep0pt
\item[]
\textbf{http://www.gnu.org/software/libtasn1/manual/}
\end{description}

\begin{longtable}[c]{@{}ll@{}}
\toprule\addlinespace
3.4 & libtasn1
\\\addlinespace
\bottomrule
\end{longtable}

\end{document}
