\documentclass[]{article}
\usepackage[T1]{fontenc}
\usepackage{lmodern}
\usepackage{amssymb,amsmath}
\usepackage{ifxetex,ifluatex}
\usepackage{fixltx2e} % provides \textsubscript
% use upquote if available, for straight quotes in verbatim environments
\IfFileExists{upquote.sty}{\usepackage{upquote}}{}
\ifnum 0\ifxetex 1\fi\ifluatex 1\fi=0 % if pdftex
  \usepackage[utf8]{inputenc}
\else % if luatex or xelatex
  \ifxetex
    \usepackage{mathspec}
    \usepackage{xltxtra,xunicode}
  \else
    \usepackage{fontspec}
  \fi
  \defaultfontfeatures{Mapping=tex-text,Scale=MatchLowercase}
  \newcommand{\euro}{€}
\fi
% use microtype if available
\IfFileExists{microtype.sty}{\usepackage{microtype}}{}
\usepackage{longtable,booktabs}
\ifxetex
  \usepackage[setpagesize=false, % page size defined by xetex
              unicode=false, % unicode breaks when used with xetex
              xetex]{hyperref}
\else
  \usepackage[unicode=true]{hyperref}
\fi
\hypersetup{breaklinks=true,
            bookmarks=true,
            pdfauthor={},
            pdftitle={ARGZ\_ADD(3)},
            colorlinks=true,
            citecolor=blue,
            urlcolor=blue,
            linkcolor=magenta,
            pdfborder={0 0 0}}
\urlstyle{same}  % don't use monospace font for urls
\setlength{\parindent}{0pt}
\setlength{\parskip}{6pt plus 2pt minus 1pt}
\setlength{\emergencystretch}{3em}  % prevent overfull lines
\setcounter{secnumdepth}{0}
\usepackage{pagecolor}

% Set background colour (of the page)
\definecolor{weirdbgcolor}{HTML}{FCF4F0}
\pagecolor{weirdbgcolor}

% Make bold text appear in a particular colour
\definecolor{boldcolor}{HTML}{6E0002}
\let\realtextbf=\textbf
\renewcommand{\textbf}[1]{\textcolor{boldcolor}{\realtextbf{#1}}}

% Use underlines instead of emphasis (ugh)
\renewcommand{\emph}[1]{\underline{#1}}

% % Use fixed-width font by default
% \renewcommand*\familydefault{\ttdefault}

\title{ARGZ\_ADD(3)}
\author{}
\date{}

\begin{document}
\maketitle

\begin{longtable}[c]{@{}lll@{}}
\toprule\addlinespace
ARGZ\_ADD(3) & Linux Programmer's Manual & ARGZ\_ADD(3)
\\\addlinespace
\bottomrule
\end{longtable}

\hyperdef{}{NAME}{\section{\hyperref[NAME]{NAME}}\label{NAME}}

argz\_add, argz\_add\_sep, argz\_append, argz\_count, argz\_create,
argz\_create\_sep, argz\_delete, argz\_extract, argz\_insert,
argz\_next, argz\_replace, argz\_stringify - functions to handle an argz
list

\hyperdef{}{SYNOPSIS}{\section{\hyperref[SYNOPSIS]{SYNOPSIS}}\label{SYNOPSIS}}

\begin{verbatim}
#include <argz.h>
 
error_t argz_add(char **argz, size_t *argz_len, const char *str);
 
error_t argz_add_sep(char **argz, size_t *argz_len,
                     const char *str, int delim);
 
error_t argz_append(char **argz, size_t *argz_len,
                     const char *buf, size_t buf_len);
 
size_t argz_count(const char *argz, size_t argz_len);
 
error_t argz_create(char * const argv[], char **argz,
                     size_t *argz_len);
 
error_t argz_create_sep(const char *str, int sep, char **argz,
                     size_t *argz_len);
 
error_t argz_delete(char **argz, size_t *argz_len, char *entry);
 
void argz_extract(char *argz, size_t argz_len, char  **argv);
 
error_t argz_insert(char **argz, size_t *argz_len, char *before,
                     const char *entry);
 
char *argz_next(char *argz, size_t argz_len, const char *entry);
 
error_t argz_replace(char **argz, size_t *argz_len, const char *str,
                     const char *with, unsigned int *replace_count);
 
void argz_stringify(char *argz, size_t len, int sep);
\end{verbatim}

\hyperdef{}{DESCRIPTION}{\section{\hyperref[DESCRIPTION]{DESCRIPTION}}\label{DESCRIPTION}}

These functions are glibc-specific.

An argz vector is a pointer to a character buffer together with a
length. The intended interpretation of the character buffer is an array
of strings, where the strings are separated by null bytes
('\textbackslash{}0'). If the length is nonzero, the last byte of the
buffer must be a null byte.

These functions are for handling argz vectors. The pair (NULL,0) is an
argz vector, and, conversely, argz vectors of length 0 must have NULL
pointer. Allocation of nonempty argz vectors is done using
\textbf{malloc}(3), so that \textbf{free}(3) can be used to dispose of
them again.

\textbf{argz\_add}() adds the string \emph{str} at the end of the array
\emph{*argz}, and updates \emph{*argz} and \emph{*argz\_len}.

\textbf{argz\_add\_sep}() is similar, but splits the string \emph{str}
into substrings separated by the delimiter \emph{delim}. For example,
one might use this on a UNIX search path with delimiter `:'.

\textbf{argz\_append}() appends the argz vector
(\emph{buf},~\emph{buf\_len}) after (\emph{*argz},~\emph{*argz\_len})
and updates \emph{*argz} and \emph{*argz\_len}. (Thus, \emph{*argz\_len}
will be increased by \emph{buf\_len}.)

\textbf{argz\_count}() counts the number of strings, that is, the number
of null bytes ('\textbackslash{}0'), in (\emph{argz},~\emph{argz\_len}).

\textbf{argz\_create}() converts a UNIX-style argument vector
\emph{argv}, terminated by \emph{(char~*)~0}, into an argz vector
(\emph{*argz},~\emph{*argz\_len}).

\textbf{argz\_create\_sep}() converts the null-terminated string
\emph{str} into an argz vector (\emph{*argz},~\emph{*argz\_len}) by
breaking it up at every occurrence of the separator \emph{sep}.

\textbf{argz\_delete}() removes the substring pointed to by \emph{entry}
from the argz vector (\emph{*argz},~\emph{*argz\_len}) and updates
\emph{*argz} and \emph{*argz\_len}.

\textbf{argz\_extract}() is the opposite of \textbf{argz\_create}(). It
takes the argz vector (\emph{argz},~\emph{argz\_len}) and fills the
array starting at \emph{argv} with pointers to the substrings, and a
final NULL, making a UNIX-style argv vector. The array \emph{argv} must
have room for \emph{argz\_count}(\emph{argz}, \emph{argz\_len}) + 1
pointers.

\textbf{argz\_insert}() is the opposite of \textbf{argz\_delete}(). It
inserts the argument \emph{entry} at position \emph{before} into the
argz vector (\emph{*argz},~\emph{*argz\_len}) and updates \emph{*argz}
and \emph{*argz\_len}. If \emph{before} is NULL, then \emph{entry} will
inserted at the end.

\textbf{argz\_next}() is a function to step trough the argz vector. If
\emph{entry} is NULL, the first entry is returned. Otherwise, the entry
following is returned. It returns NULL if there is no following entry.

\textbf{argz\_replace}() replaces each occurrence of \emph{str} with
\emph{with}, reallocating argz as necessary. If \emph{replace\_count} is
non-NULL, \emph{*replace\_count} will be incremented by the number of
replacements.

\textbf{argz\_stringify}() is the opposite of
\textbf{argz\_create\_sep}(). It transforms the argz vector into a
normal string by replacing all null bytes ('\textbackslash{}0') except
the last by \emph{sep}.

\hyperdef{}{RETURNux5fVALUE}{\section{\hyperref[RETURNux5fVALUE]{RETURN
VALUE}}\label{RETURNux5fVALUE}}

All argz functions that do memory allocation have a return type of
\emph{error\_t}, and return 0 for success, and \textbf{ENOMEM} if an
allocation error occurs.

\hyperdef{}{CONFORMINGux5fTO}{\section{\hyperref[CONFORMINGux5fTO]{CONFORMING
TO}}\label{CONFORMINGux5fTO}}

These functions are a GNU extension. Handle with care.

\hyperdef{}{BUGS}{\section{\hyperref[BUGS]{BUGS}}\label{BUGS}}

Argz vectors without a terminating null byte may lead to Segmentation
Faults.

\hyperdef{}{SEEux5fALSO}{\section{\hyperref[SEEux5fALSO]{SEE
ALSO}}\label{SEEux5fALSO}}

\textbf{envz\_add}(3)

\hyperdef{}{COLOPHON}{\section{\hyperref[COLOPHON]{COLOPHON}}\label{COLOPHON}}

This page is part of release 3.54 of the Linux \emph{man-pages} project.
A description of the project, and information about reporting bugs, can
be found at http://www.kernel.org/doc/man-pages/.

\begin{longtable}[c]{@{}ll@{}}
\toprule\addlinespace
2007-05-18 &
\\\addlinespace
\bottomrule
\end{longtable}

\end{document}
