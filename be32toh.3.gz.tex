\documentclass[]{article}
\usepackage[T1]{fontenc}
\usepackage{lmodern}
\usepackage{amssymb,amsmath}
\usepackage{ifxetex,ifluatex}
\usepackage{fixltx2e} % provides \textsubscript
% use upquote if available, for straight quotes in verbatim environments
\IfFileExists{upquote.sty}{\usepackage{upquote}}{}
\ifnum 0\ifxetex 1\fi\ifluatex 1\fi=0 % if pdftex
  \usepackage[utf8]{inputenc}
\else % if luatex or xelatex
  \ifxetex
    \usepackage{mathspec}
    \usepackage{xltxtra,xunicode}
  \else
    \usepackage{fontspec}
  \fi
  \defaultfontfeatures{Mapping=tex-text,Scale=MatchLowercase}
  \newcommand{\euro}{€}
\fi
% use microtype if available
\IfFileExists{microtype.sty}{\usepackage{microtype}}{}
\usepackage{longtable,booktabs}
\ifxetex
  \usepackage[setpagesize=false, % page size defined by xetex
              unicode=false, % unicode breaks when used with xetex
              xetex]{hyperref}
\else
  \usepackage[unicode=true]{hyperref}
\fi
\hypersetup{breaklinks=true,
            bookmarks=true,
            pdfauthor={},
            pdftitle={ENDIAN(3)},
            colorlinks=true,
            citecolor=blue,
            urlcolor=blue,
            linkcolor=magenta,
            pdfborder={0 0 0}}
\urlstyle{same}  % don't use monospace font for urls
\setlength{\parindent}{0pt}
\setlength{\parskip}{6pt plus 2pt minus 1pt}
\setlength{\emergencystretch}{3em}  % prevent overfull lines
\setcounter{secnumdepth}{0}
\usepackage{pagecolor}

% Set background colour (of the page)
\definecolor{weirdbgcolor}{HTML}{FCF4F0}
\pagecolor{weirdbgcolor}

% Make bold text appear in a particular colour
\definecolor{boldcolor}{HTML}{6E0002}
\let\realtextbf=\textbf
\renewcommand{\textbf}[1]{\textcolor{boldcolor}{\realtextbf{#1}}}

% Use underlines instead of emphasis (ugh)
\renewcommand{\emph}[1]{\underline{#1}}

% % Use fixed-width font by default
% \renewcommand*\familydefault{\ttdefault}

\title{ENDIAN(3)}
\author{}
\date{}

\begin{document}
\maketitle

\begin{longtable}[c]{@{}lll@{}}
\toprule\addlinespace
ENDIAN(3) & Linux Programmer's Manual & ENDIAN(3)
\\\addlinespace
\bottomrule
\end{longtable}

\hyperdef{}{NAME}{\section{\hyperref[NAME]{NAME}}\label{NAME}}

htobe16, htole16, be16toh, le16toh, htobe32, htole32, be32toh, le32toh,
htobe64, htole64, be64toh, le64toh - convert values between host and
big-/little-endian byte order

\hyperdef{}{SYNOPSIS}{\section{\hyperref[SYNOPSIS]{SYNOPSIS}}\label{SYNOPSIS}}

\begin{verbatim}
#define _BSD_SOURCE             /* See feature_test_macros(7) */
#include <endian.h>

uint16_t htobe16(uint16_t host_16bits);
uint16_t htole16(uint16_t host_16bits);
uint16_t be16toh(uint16_t big_endian_16bits);
uint16_t le16toh(uint16_t little_endian_16bits);

uint32_t htobe32(uint32_t host_32bits);
uint32_t htole32(uint32_t host_32bits);
uint32_t be32toh(uint32_t big_endian_32bits);
uint32_t le32toh(uint32_t little_endian_32bits);

uint64_t htobe64(uint64_t host_64bits);
uint64_t htole64(uint64_t host_64bits);
uint64_t be64toh(uint64_t big_endian_64bits);
uint64_t le64toh(uint64_t little_endian_64bits);
\end{verbatim}

\hyperdef{}{DESCRIPTION}{\section{\hyperref[DESCRIPTION]{DESCRIPTION}}\label{DESCRIPTION}}

These functions convert the byte encoding of integer values from the
byte order that the current CPU (the ``host'') uses, to and from
little-endian and big-endian byte order.

~

The number, \emph{nn}, in the name of each function indicates the size
of integer handled by the function, either 16, 32, or 64 bits.

~

The functions with names of the form "htobe \emph{nn}" convert from host
byte order to big-endian order.

~

The functions with names of the form "htole \emph{nn}" convert from host
byte order to little-endian order.

~

The functions with names of the form "be \emph{nn}toh" convert from
big-endian order to host byte order.

~

The functions with names of the form "le \emph{nn}toh" convert from
little-endian order to host byte order.

\hyperdef{}{VERSIONS}{\section{\hyperref[VERSIONS]{VERSIONS}}\label{VERSIONS}}

These functions were added to glibc in version 2.9.

\hyperdef{}{CONFORMINGux5fTO}{\section{\hyperref[CONFORMINGux5fTO]{CONFORMING
TO}}\label{CONFORMINGux5fTO}}

These functions are nonstandard. Similar functions are present on the
BSDs, where the required header file is
\emph{\textless{}sys/endian.h\textgreater{}} instead of
\emph{\textless{}endian.h\textgreater{}}. Unfortunately, NetBSD,
FreeBSD, and glibc haven't followed the original OpenBSD naming
convention for these functions, whereby the \emph{nn} component always
appears at the end of the function name (thus, for example, in NetBSD,
FreeBSD, and glibc, the equivalent of OpenBSDs ``betoh32'' is
``be32toh'').

\hyperdef{}{NOTES}{\section{\hyperref[NOTES]{NOTES}}\label{NOTES}}

These functions are similar to the older \textbf{byteorder}(3) family of
functions. For example, \textbf{be32toh}() is identical to
\textbf{ntohl}()\textbf{.}

~

The advantage of the \textbf{byteorder}(3) functions is that they are
standard functions available on all UNIX systems. On the other hand, the
fact that they were designed for use in the context of TCP/IP means that
they lack the 64-bit and little-endian variants described in this page.

\hyperdef{}{EXAMPLE}{\section{\hyperref[EXAMPLE]{EXAMPLE}}\label{EXAMPLE}}

The program below display the results of converting an integer from host
byte order to both little-endian and big-endian byte order. Since host
byte order is either little-endian or big-endian, only one of these
conversions will have an effect. When we run this program on a
little-endian system such as x86-32, we see the following: \\

\begin{verbatim}

$  ./a.out
x.u32 = 0x44332211
htole32(x.u32) = 0x44332211
htobe32(x.u32) = 0x11223344
\end{verbatim}

\hyperdef{}{Programux5fsource}{\subsection{\hyperref[Programux5fsource]{Program
source}}\label{Programux5fsource}}

\begin{verbatim}
#include <endian.h>
#include <stdint.h>
#include <stdio.h>
#include <stdlib.h>

int
main(int argc, char *argv[])
{
    union {
    uint32_t u32;
    uint8_t arr[4];
    } x;

    x.arr[0] = 0x11;    /* Lowest-address byte */
    x.arr[1] = 0x22;
    x.arr[2] = 0x33;
    x.arr[3] = 0x44;    /* Highest-address byte */

    printf("x.u32 = 0x%x\n", x.u32);
    printf("htole32(x.u32) = 0x%x\n", htole32(x.u32));
    printf("htobe32(x.u32) = 0x%x\n", htobe32(x.u32));

    exit(EXIT_SUCCESS);
}
\end{verbatim}

\hyperdef{}{SEEux5fALSO}{\section{\hyperref[SEEux5fALSO]{SEE
ALSO}}\label{SEEux5fALSO}}

\textbf{byteorder}(3)

\hyperdef{}{COLOPHON}{\section{\hyperref[COLOPHON]{COLOPHON}}\label{COLOPHON}}

This page is part of release 3.54 of the Linux \emph{man-pages} project.
A description of the project, and information about reporting bugs, can
be found at http://www.kernel.org/doc/man-pages/.

\begin{longtable}[c]{@{}ll@{}}
\toprule\addlinespace
2010-09-10 & GNU
\\\addlinespace
\bottomrule
\end{longtable}

\end{document}
