\documentclass[]{article}
\usepackage[T1]{fontenc}
\usepackage{lmodern}
\usepackage{amssymb,amsmath}
\usepackage{ifxetex,ifluatex}
\usepackage{fixltx2e} % provides \textsubscript
% use upquote if available, for straight quotes in verbatim environments
\IfFileExists{upquote.sty}{\usepackage{upquote}}{}
\ifnum 0\ifxetex 1\fi\ifluatex 1\fi=0 % if pdftex
  \usepackage[utf8]{inputenc}
\else % if luatex or xelatex
  \ifxetex
    \usepackage{mathspec}
    \usepackage{xltxtra,xunicode}
  \else
    \usepackage{fontspec}
  \fi
  \defaultfontfeatures{Mapping=tex-text,Scale=MatchLowercase}
  \newcommand{\euro}{€}
\fi
% use microtype if available
\IfFileExists{microtype.sty}{\usepackage{microtype}}{}
\usepackage{longtable,booktabs}
\ifxetex
  \usepackage[setpagesize=false, % page size defined by xetex
              unicode=false, % unicode breaks when used with xetex
              xetex]{hyperref}
\else
  \usepackage[unicode=true]{hyperref}
\fi
\hypersetup{breaklinks=true,
            bookmarks=true,
            pdfauthor={},
            pdftitle={AIO\_WRITE(3)},
            colorlinks=true,
            citecolor=blue,
            urlcolor=blue,
            linkcolor=magenta,
            pdfborder={0 0 0}}
\urlstyle{same}  % don't use monospace font for urls
\setlength{\parindent}{0pt}
\setlength{\parskip}{6pt plus 2pt minus 1pt}
\setlength{\emergencystretch}{3em}  % prevent overfull lines
\setcounter{secnumdepth}{0}
\usepackage{pagecolor}

% Set background colour (of the page)
\definecolor{weirdbgcolor}{HTML}{FCF4F0}
\pagecolor{weirdbgcolor}

% Make bold text appear in a particular colour
\definecolor{boldcolor}{HTML}{6E0002}
\let\realtextbf=\textbf
\renewcommand{\textbf}[1]{\textcolor{boldcolor}{\realtextbf{#1}}}

% Use underlines instead of emphasis (ugh)
\renewcommand{\emph}[1]{\underline{#1}}

% % Use fixed-width font by default
% \renewcommand*\familydefault{\ttdefault}

\title{AIO\_WRITE(3)}
\author{}
\date{}

\begin{document}
\maketitle

\begin{longtable}[c]{@{}lll@{}}
\toprule\addlinespace
AIO\_WRITE(3) & Linux Programmer's Manual & AIO\_WRITE(3)
\\\addlinespace
\bottomrule
\end{longtable}

\hyperdef{}{NAME}{\section{\hyperref[NAME]{NAME}}\label{NAME}}

aio\_write - asynchronous write

\hyperdef{}{SYNOPSIS}{\section{\hyperref[SYNOPSIS]{SYNOPSIS}}\label{SYNOPSIS}}

\textbf{\#include \textless{}aio.h\textgreater{}}

~

\textbf{int aio\_write(struct aiocb *}\emph{aiocbp}\textbf{);}

~

Link with \emph{-lrt}.

\hyperdef{}{DESCRIPTION}{\section{\hyperref[DESCRIPTION]{DESCRIPTION}}\label{DESCRIPTION}}

The \textbf{aio\_write}() function queues the I/O request described by
the buffer pointed to by \emph{aiocbp}. This function is the
asynchronous analog of \textbf{write}(2). The arguments of the call

~

write(fd, buf, count)

~

correspond (in order) to the fields \emph{aio\_fildes}, \emph{aio\_buf},
and \emph{aio\_nbytes} of the structure pointed to by \emph{aiocbp}.
(See \textbf{aio}(7) for a description of the \emph{aiocb} structure.)

If \textbf{O\_APPEND} is not set, the data is written starting at the
absolute file offset \emph{aiocbp-\textgreater{}aio\_offset}, regardless
of the current file offset. If \textbf{O\_APPEND} is set, data is
written at the end of the file in the same order as
\textbf{aio\_write}() calls are made. After the call, the value of the
current file offset is unspecified.

The ``asynchronous'' means that this call returns as soon as the request
has been enqueued; the write may or may not have completed when the call
returns. One tests for completion using \textbf{aio\_error}(3). The
return status of a completed I/O operation can be obtained
\textbf{aio\_return}(3). Asynchronous notification of I/O completion can
be obtained by setting \emph{aiocbp-\textgreater{}aio\_sigevent}
appropriately; see \textbf{sigevent}(7) for details.

If \textbf{\_POSIX\_PRIORITIZED\_IO} is defined, and this file supports
it, then the asynchronous operation is submitted at a priority equal to
that of the calling process minus
\emph{aiocbp-\textgreater{}aio\_reqprio}.

The field \emph{aiocbp-\textgreater{}aio\_lio\_opcode} is ignored.

No data is written to a regular file beyond its maximum offset.

\hyperdef{}{RETURNux5fVALUE}{\section{\hyperref[RETURNux5fVALUE]{RETURN
VALUE}}\label{RETURNux5fVALUE}}

On success, 0 is returned. On error the request is not enqueued, -1 is
returned, and \emph{errno} is set appropriately. If an error is detected
only later, it will be reported via \textbf{aio\_return}(3) (returns
status -1) and \textbf{aio\_error}(3) (error status---whatever one would
have gotten in \emph{errno}, such as \textbf{EBADF}).

\hyperdef{}{ERRORS}{\section{\hyperref[ERRORS]{ERRORS}}\label{ERRORS}}

\begin{description}
\itemsep1pt\parskip0pt\parsep0pt
\item[\textbf{EAGAIN}]
Out of resources.
\end{description}

\begin{description}
\itemsep1pt\parskip0pt\parsep0pt
\item[\textbf{EBADF}]
\emph{aio\_fildes} is not a valid file descriptor open for writing.
\end{description}

\begin{description}
\itemsep1pt\parskip0pt\parsep0pt
\item[\textbf{EFBIG}]
The file is a regular file, we want to write at least one byte, but the
starting position is at or beyond the maximum offset for this file.
\end{description}

\begin{description}
\itemsep1pt\parskip0pt\parsep0pt
\item[\textbf{EINVAL}]
One or more of \emph{aio\_offset}, \emph{aio\_reqprio},
\emph{aio\_nbytes} are invalid.
\end{description}

\begin{description}
\itemsep1pt\parskip0pt\parsep0pt
\item[\textbf{ENOSYS}]
\textbf{aio\_write}() is not implemented.
\end{description}

\hyperdef{}{VERSIONS}{\section{\hyperref[VERSIONS]{VERSIONS}}\label{VERSIONS}}

The \textbf{aio\_write}() function is available since glibc 2.1.

\hyperdef{}{CONFORMINGux5fTO}{\section{\hyperref[CONFORMINGux5fTO]{CONFORMING
TO}}\label{CONFORMINGux5fTO}}

POSIX.1-2001, POSIX.1-2008.

\hyperdef{}{NOTES}{\section{\hyperref[NOTES]{NOTES}}\label{NOTES}}

It is a good idea to zero out the control block before use. The control
block must not be changed while the write operation is in progress. The
buffer area being written out must not be accessed during the operation
or undefined results may occur. The memory areas involved must remain
valid.

~

Simultaneous I/O operations specifying the same \emph{aiocb} structure
produce undefined results.

\hyperdef{}{SEEux5fALSO}{\section{\hyperref[SEEux5fALSO]{SEE
ALSO}}\label{SEEux5fALSO}}

\textbf{aio\_cancel}(3), \textbf{aio\_error}(3), \textbf{aio\_fsync}(3),
\textbf{aio\_read}(3), \textbf{aio\_return}(3),
\textbf{aio\_suspend}(3), \textbf{lio\_listio}(3), \textbf{aio}(7)

\hyperdef{}{COLOPHON}{\section{\hyperref[COLOPHON]{COLOPHON}}\label{COLOPHON}}

This page is part of release 3.54 of the Linux \emph{man-pages} project.
A description of the project, and information about reporting bugs, can
be found at http://www.kernel.org/doc/man-pages/.

\begin{longtable}[c]{@{}ll@{}}
\toprule\addlinespace
2012-05-08 &
\\\addlinespace
\bottomrule
\end{longtable}

\end{document}
