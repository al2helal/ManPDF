\documentclass[]{article}
\usepackage[T1]{fontenc}
\usepackage{lmodern}
\usepackage{amssymb,amsmath}
\usepackage{ifxetex,ifluatex}
\usepackage{fixltx2e} % provides \textsubscript
% use upquote if available, for straight quotes in verbatim environments
\IfFileExists{upquote.sty}{\usepackage{upquote}}{}
\ifnum 0\ifxetex 1\fi\ifluatex 1\fi=0 % if pdftex
  \usepackage[utf8]{inputenc}
\else % if luatex or xelatex
  \ifxetex
    \usepackage{mathspec}
    \usepackage{xltxtra,xunicode}
  \else
    \usepackage{fontspec}
  \fi
  \defaultfontfeatures{Mapping=tex-text,Scale=MatchLowercase}
  \newcommand{\euro}{€}
\fi
% use microtype if available
\IfFileExists{microtype.sty}{\usepackage{microtype}}{}
\usepackage{longtable,booktabs}
\ifxetex
  \usepackage[setpagesize=false, % page size defined by xetex
              unicode=false, % unicode breaks when used with xetex
              xetex]{hyperref}
\else
  \usepackage[unicode=true]{hyperref}
\fi
\hypersetup{breaklinks=true,
            bookmarks=true,
            pdfauthor={},
            pdftitle={ALLOCA(3)},
            colorlinks=true,
            citecolor=blue,
            urlcolor=blue,
            linkcolor=magenta,
            pdfborder={0 0 0}}
\urlstyle{same}  % don't use monospace font for urls
\setlength{\parindent}{0pt}
\setlength{\parskip}{6pt plus 2pt minus 1pt}
\setlength{\emergencystretch}{3em}  % prevent overfull lines
\setcounter{secnumdepth}{0}
\usepackage{pagecolor}

% Set background colour (of the page)
\definecolor{weirdbgcolor}{HTML}{FCF4F0}
\pagecolor{weirdbgcolor}

% Make bold text appear in a particular colour
\definecolor{boldcolor}{HTML}{6E0002}
\let\realtextbf=\textbf
\renewcommand{\textbf}[1]{\textcolor{boldcolor}{\realtextbf{#1}}}

% Use underlines instead of emphasis (ugh)
\renewcommand{\emph}[1]{\underline{#1}}

% % Use fixed-width font by default
% \renewcommand*\familydefault{\ttdefault}

\title{ALLOCA(3)}
\author{}
\date{}

\begin{document}
\maketitle

\begin{longtable}[c]{@{}lll@{}}
\toprule\addlinespace
ALLOCA(3) & Linux Programmer's Manual & ALLOCA(3)
\\\addlinespace
\bottomrule
\end{longtable}

\hyperdef{}{NAME}{\section{\hyperref[NAME]{NAME}}\label{NAME}}

alloca - allocate memory that is automatically freed

\hyperdef{}{SYNOPSIS}{\section{\hyperref[SYNOPSIS]{SYNOPSIS}}\label{SYNOPSIS}}

\textbf{\#include \textless{}alloca.h\textgreater{}}

~

\textbf{void *alloca(size\_t}\emph{size}\textbf{);}

\hyperdef{}{DESCRIPTION}{\section{\hyperref[DESCRIPTION]{DESCRIPTION}}\label{DESCRIPTION}}

The \textbf{alloca}() function allocates \emph{size} bytes of space in
the stack frame of the caller. This temporary space is automatically
freed when the function that called \textbf{alloca}() returns to its
caller.

\hyperdef{}{RETURNux5fVALUE}{\section{\hyperref[RETURNux5fVALUE]{RETURN
VALUE}}\label{RETURNux5fVALUE}}

The \textbf{alloca}() function returns a pointer to the beginning of the
allocated space. If the allocation causes stack overflow, program
behavior is undefined.

\hyperdef{}{CONFORMINGux5fTO}{\section{\hyperref[CONFORMINGux5fTO]{CONFORMING
TO}}\label{CONFORMINGux5fTO}}

This function is not in POSIX.1-2001.

~

There is evidence that the \textbf{alloca}() function appeared in 32V,
PWB, PWB.2, 3BSD, and 4BSD. There is a man page for it in 4.3BSD. Linux
uses the GNU version.

\hyperdef{}{NOTES}{\section{\hyperref[NOTES]{NOTES}}\label{NOTES}}

The \textbf{alloca}() function is machine- and compiler-dependent. For
certain applications, its use can improve efficiency compared to the use
of \textbf{malloc}(3) plus \textbf{free}(3). In certain cases, it can
also simplify memory deallocation in applications that use
\textbf{longjmp}(3) or \textbf{siglongjmp}(3). Otherwise, its use is
discouraged.

~

Because the space allocated by \textbf{alloca}() is allocated within the
stack frame, that space is automatically freed if the function return is
jumped over by a call to \textbf{longjmp}(3) or \textbf{siglongjmp}(3).

~

Do not attempt to \textbf{free}(3) space allocated by \textbf{alloca}()!

\hyperdef{}{Notesux5fonux5ftheux5fGNUux5fversion}{\subsection{\hyperref[Notesux5fonux5ftheux5fGNUux5fversion]{Notes
on the GNU version}}\label{Notesux5fonux5ftheux5fGNUux5fversion}}

Normally, \textbf{gcc}(1) translates calls to \textbf{alloca}() with
inlined code. This is not done when either the \emph{-ansi},
\emph{-std=c89}, \emph{-std=c99}, or the \emph{-std=c11} option is given
\textbf{and} the header \emph{\textless{}alloca.h\textgreater{}} is not
included. Otherwise (without an -ansi or -std=c* option) the glibc
version of \emph{\textless{}stdlib.h\textgreater{}} includes
\emph{\textless{}alloca.h\textgreater{}} and that contains the lines:

\begin{verbatim}

    #ifdef  __GNUC__
    #define alloca(size)   __builtin_alloca (size)
    #endif
\end{verbatim}

with messy consequences if one has a private version of this function.

The fact that the code is inlined means that it is impossible to take
the address of this function, or to change its behavior by linking with
a different library.

The inlined code often consists of a single instruction adjusting the
stack pointer, and does not check for stack overflow. Thus, there is no
NULL error return.

\hyperdef{}{BUGS}{\section{\hyperref[BUGS]{BUGS}}\label{BUGS}}

There is no error indication if the stack frame cannot be extended.
(However, after a failed allocation, the program is likely to receive a
\textbf{SIGSEGV} signal if it attempts to access the unallocated space.)

~

On many systems \textbf{alloca}() cannot be used inside the list of
arguments of a function call, because the stack space reserved by
\textbf{alloca}() would appear on the stack in the middle of the space
for the function arguments.

\hyperdef{}{SEEux5fALSO}{\section{\hyperref[SEEux5fALSO]{SEE
ALSO}}\label{SEEux5fALSO}}

\textbf{brk}(2), \textbf{longjmp}(3), \textbf{malloc}(3)

\hyperdef{}{COLOPHON}{\section{\hyperref[COLOPHON]{COLOPHON}}\label{COLOPHON}}

This page is part of release 3.54 of the Linux \emph{man-pages} project.
A description of the project, and information about reporting bugs, can
be found at http://www.kernel.org/doc/man-pages/.

\begin{longtable}[c]{@{}ll@{}}
\toprule\addlinespace
2013-05-12 & GNU
\\\addlinespace
\bottomrule
\end{longtable}

\end{document}
