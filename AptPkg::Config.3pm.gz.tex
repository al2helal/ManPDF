\documentclass[]{article}
\usepackage[T1]{fontenc}
\usepackage{lmodern}
\usepackage{amssymb,amsmath}
\usepackage{ifxetex,ifluatex}
\usepackage{fixltx2e} % provides \textsubscript
% use upquote if available, for straight quotes in verbatim environments
\IfFileExists{upquote.sty}{\usepackage{upquote}}{}
\ifnum 0\ifxetex 1\fi\ifluatex 1\fi=0 % if pdftex
  \usepackage[utf8]{inputenc}
\else % if luatex or xelatex
  \ifxetex
    \usepackage{mathspec}
    \usepackage{xltxtra,xunicode}
  \else
    \usepackage{fontspec}
  \fi
  \defaultfontfeatures{Mapping=tex-text,Scale=MatchLowercase}
  \newcommand{\euro}{€}
\fi
% use microtype if available
\IfFileExists{microtype.sty}{\usepackage{microtype}}{}
\usepackage{longtable,booktabs}
\ifxetex
  \usepackage[setpagesize=false, % page size defined by xetex
              unicode=false, % unicode breaks when used with xetex
              xetex]{hyperref}
\else
  \usepackage[unicode=true]{hyperref}
\fi
\hypersetup{breaklinks=true,
            bookmarks=true,
            pdfauthor={},
            pdftitle={AptPkg::Config(3pm)},
            colorlinks=true,
            citecolor=blue,
            urlcolor=blue,
            linkcolor=magenta,
            pdfborder={0 0 0}}
\urlstyle{same}  % don't use monospace font for urls
\setlength{\parindent}{0pt}
\setlength{\parskip}{6pt plus 2pt minus 1pt}
\setlength{\emergencystretch}{3em}  % prevent overfull lines
\setcounter{secnumdepth}{0}
\usepackage{pagecolor}

% Set background colour (of the page)
\definecolor{weirdbgcolor}{HTML}{FCF4F0}
\pagecolor{weirdbgcolor}

% Make bold text appear in a particular colour
\definecolor{boldcolor}{HTML}{6E0002}
\let\realtextbf=\textbf
\renewcommand{\textbf}[1]{\textcolor{boldcolor}{\realtextbf{#1}}}

% Use underlines instead of emphasis (ugh)
\renewcommand{\emph}[1]{\underline{#1}}

% % Use fixed-width font by default
% \renewcommand*\familydefault{\ttdefault}

\title{AptPkg::Config(3pm)}
\author{}
\date{}

\begin{document}
\maketitle

\begin{longtable}[c]{@{}lll@{}}
\toprule\addlinespace
AptPkg::Config(3pm) & User Contributed Perl Documentation &
AptPkg::Config(3pm)
\\\addlinespace
\bottomrule
\end{longtable}

\hyperdef{}{NAME}{\section{\hyperref[NAME]{NAME}}\label{NAME}}

AptPkg::Config - APT configuration interface

\hyperdef{}{SYNOPSIS}{\section{\hyperref[SYNOPSIS]{SYNOPSIS}}\label{SYNOPSIS}}

use AptPkg::Config;

\hyperdef{}{DESCRIPTION}{\section{\hyperref[DESCRIPTION]{DESCRIPTION}}\label{DESCRIPTION}}

The AptPkg::Config module provides an interface to \textbf{APT}'s
configuration mechanism.

Provides a configuration file and command line parser for a
tree-oriented configuration environment.

\hyperdef{}{AptPkg::Config}{\subsection{\hyperref[AptPkg::Config]{AptPkg::Config}}\label{AptPkg::Config}}

The AptPkg::Config package implements the \textbf{APT} Configuration
class.

A global instance of the libapt-pkg \_config instance is provided as
\$AptPkg::Config::\_config, and may be imported.

The following methods are implemented:

\begin{description}
\itemsep1pt\parskip0pt\parsep0pt
\item[get(\emph{KEY}, {[}\emph{DEFAULT}{]})]
Fetch the value of \emph{KEY} from the configuration object, returning
undef if not found (or \emph{DEFAULT} if given).

~

If the key ends in ::, an array of values is returned in an array
context, or a string containing the values separated by spaces in a
scalar context.

~

A trailing /f, /d, /b or /i causes file, directory, boolean or integer
interpretation (the underlying XS call is FindAny).
\end{description}

\begin{description}
\item[get\_file(\emph{KEY}, {[}\emph{DEFAULT}{]}), get\_dir(\emph{KEY},
{[} \emph{DEFAULT}{]})]
Variants of get which prepend the directory value from the parent key.
The get\_dir method additionally appends a `/'.

~

For example, given the configuration file:

~

\begin{verbatim}
    foo "/some/dir/" { bar "value"; }
    
\end{verbatim}

~

then:

~

\begin{verbatim}
    $conf->get("foo::bar")      # "value"
    $conf->get_file("foo::bar") # "/some/dir/value"
    $conf->get_dir("foo::bar")  # "/some/dir/value/"
    
\end{verbatim}
\end{description}

\begin{description}
\item[get\_bool(\emph{KEY}, {[}\emph{DEFAULT}{]})]
Another get varient, which returns true (1) if the value contains any
of:

~

\begin{verbatim}
    1 yes true with on enable
    
\end{verbatim}

~

otherwise false ('').
\end{description}

\begin{description}
\itemsep1pt\parskip0pt\parsep0pt
\item[set(\emph{KEY}, \emph{VALUE})]
Set configuration entry \emph{KEY} to \emph{VALUE}. Returns
\emph{VALUE}. Note that empty parent entries may be created for
\emph{KEY}s containing ::.
\end{description}

\begin{description}
\itemsep1pt\parskip0pt\parsep0pt
\item[exists(\emph{KEY})]
Test if \emph{KEY} exists in the configuration.
\end{description}

\begin{description}
\itemsep1pt\parskip0pt\parsep0pt
\item[dump]
Principally for debugging, output the contents of the configuration
object to stderr.
\end{description}

\begin{description}
\item[read\_file(\emph{FILE}, {[}\emph{AS\_SECTIONAL},
{[}\emph{DEPTH}{]}{]})]
Load the contents of \emph{FILE} into the object. The format of the file
is described in \emph{apt.conf}(5).

~

If the \emph{AS\_SECTIONAL} argument is true, then the file is parsed as
a BIND-style config. That is:

~

\begin{verbatim}
    foo "bar" { baz "quux"; }
    
\end{verbatim}

~

is interpreted as if it were:

~

\begin{verbatim}
    foo::bar { baz "quux"; }
    
\end{verbatim}

~

The \emph{DEPTH} argument may be used to restrict the number of nested
include directives processed.
\end{description}

\begin{description}
\itemsep1pt\parskip0pt\parsep0pt
\item[read\_dir(\emph{DIR}, {[}\emph{AS\_SECTIONAL},
{[}\emph{DEPTH}{]}{]})]
Load configuration from all files in \emph{DIR}.
\end{description}

\begin{description}
\itemsep1pt\parskip0pt\parsep0pt
\item[init]
Initialise the configuration object with some default values for the
libapt-pkg library and reads the default configuration file
/etc/apt/apt.conf (or as given by the environment variable APT\_CONFIG)
if it exists.
\end{description}

\begin{description}
\itemsep1pt\parskip0pt\parsep0pt
\item[system]
Return the AptPkg::System object appropriate for this system.
\end{description}

\begin{description}
\item[parse\_cmdline(\emph{DEFS}, {[}\emph{ARG}, \ldots{}{]})]
Parse the arguments given by \emph{ARG}s based on the contents of
\emph{DEFS} and returns the list of remaining arguments.

~

Note, the function does not return if there are errors processing the
args. Use eval to trap such errors.

~

\emph{DEFS} is a reference to an array containing a set argument
definition arrays. The elements of each definition define: the short
argument character, the long argument string, the configuration key and
the optional argument type (defaults to Boolean).

~

Valid argument types are defined by the strings:

~

\begin{verbatim}
    HasArg      takes an argument value (-f foo)
    IntLevel    defines an integer value (-q -q, -qq, -q2, -q=2)
    Boolean     true/false (-d, -d=true, -d=yes, --no-d, -d=false, etc)
    InvBoolean  same as Boolean but false with no specified sense (-d)
    ConfigFile  load the specified configuration file
    ArbItem     arbitrary configuration string of the form key=value
    
\end{verbatim}

~

The configuration key in the last two cases is ignored, and for the rest
gives the key into which the value is placed.

~

Single case equivalents also work (has\_arg == HasArg).

~

Example:

~

\begin{verbatim}
    @files = $conf->parse_cmndline([
            [ 'h', 'help', 'help' ],
            [ 'v', 'version', 'version' ],
            [ 'c', 'config-file', '', ConfigFile ],
            [ 'o', 'option', '', ArbItem ],
        ], @ARGV);
    
\end{verbatim}
\end{description}

The module uses AptPkg::hash to provide a hash-like access to the
object, so that \$conf-\textgreater{}\{key\} is equivalent to using the
get/set methods.

Additionally inherits the constructor (new) and keys methods from that
module.

Methods of the internal XS object (AptPkg::\_config) such as Find may
also be used. See AptPkg.

\hyperdef{}{AptPkg::Config::Iter}{\subsection{\hyperref[AptPkg::Config::Iter]{AptPkg::Config::Iter}}\label{AptPkg::Config::Iter}}

Iterator object for AptPkg::Config which is returned by the keys method.

\begin{description}
\itemsep1pt\parskip0pt\parsep0pt
\item[new(\emph{XS\_OBJ}, {[}\emph{ROOT}{]})]
Constructor, which is invoked by the keys method. \emph{ROOT}, if given
determines the subset of the tree to walk (may be given as an argument
to keys).
\end{description}

\begin{description}
\itemsep1pt\parskip0pt\parsep0pt
\item[next]
Returns the current key and advances to the next.
\end{description}

\hyperdef{}{SEEux5fALSO}{\section{\hyperref[SEEux5fALSO]{SEE
ALSO}}\label{SEEux5fALSO}}

\emph{AptPkg::System}(3pm), \emph{AptPkg}(3pm),
\emph{AptPkg::hash}(3pm).

\hyperdef{}{AUTHOR}{\section{\hyperref[AUTHOR]{AUTHOR}}\label{AUTHOR}}

Brendan O'Dea \textless{}bod@debian.org\textgreater{}

\begin{longtable}[c]{@{}ll@{}}
\toprule\addlinespace
2013-03-01 & perl v5.18.1
\\\addlinespace
\bottomrule
\end{longtable}

\end{document}
