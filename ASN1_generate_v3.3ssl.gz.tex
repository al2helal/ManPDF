\documentclass[]{article}
\usepackage[T1]{fontenc}
\usepackage{lmodern}
\usepackage{amssymb,amsmath}
\usepackage{ifxetex,ifluatex}
\usepackage{fixltx2e} % provides \textsubscript
% use upquote if available, for straight quotes in verbatim environments
\IfFileExists{upquote.sty}{\usepackage{upquote}}{}
\ifnum 0\ifxetex 1\fi\ifluatex 1\fi=0 % if pdftex
  \usepackage[utf8]{inputenc}
\else % if luatex or xelatex
  \ifxetex
    \usepackage{mathspec}
    \usepackage{xltxtra,xunicode}
  \else
    \usepackage{fontspec}
  \fi
  \defaultfontfeatures{Mapping=tex-text,Scale=MatchLowercase}
  \newcommand{\euro}{€}
\fi
% use microtype if available
\IfFileExists{microtype.sty}{\usepackage{microtype}}{}
\usepackage{longtable,booktabs}
\ifxetex
  \usepackage[setpagesize=false, % page size defined by xetex
              unicode=false, % unicode breaks when used with xetex
              xetex]{hyperref}
\else
  \usepackage[unicode=true]{hyperref}
\fi
\hypersetup{breaklinks=true,
            bookmarks=true,
            pdfauthor={},
            pdftitle={ASN1\_generate\_nconf(3SSL)},
            colorlinks=true,
            citecolor=blue,
            urlcolor=blue,
            linkcolor=magenta,
            pdfborder={0 0 0}}
\urlstyle{same}  % don't use monospace font for urls
\setlength{\parindent}{0pt}
\setlength{\parskip}{6pt plus 2pt minus 1pt}
\setlength{\emergencystretch}{3em}  % prevent overfull lines
\setcounter{secnumdepth}{0}
\usepackage{pagecolor}

% Set background colour (of the page)
\definecolor{weirdbgcolor}{HTML}{FCF4F0}
\pagecolor{weirdbgcolor}

% Make bold text appear in a particular colour
\definecolor{boldcolor}{HTML}{6E0002}
\let\realtextbf=\textbf
\renewcommand{\textbf}[1]{\textcolor{boldcolor}{\realtextbf{#1}}}

% Use underlines instead of emphasis (ugh)
\renewcommand{\emph}[1]{\underline{#1}}

% % Use fixed-width font by default
% \renewcommand*\familydefault{\ttdefault}

\title{ASN1\_generate\_nconf(3SSL)}
\author{}
\date{}

\begin{document}
\maketitle

\begin{longtable}[c]{@{}lll@{}}
\toprule\addlinespace
ASN1\_generate\_nconf(3SSL) & OpenSSL & ASN1\_generate\_nconf(3SSL)
\\\addlinespace
\bottomrule
\end{longtable}

\hyperdef{}{NAME}{\section{\hyperref[NAME]{NAME}}\label{NAME}}

ASN1\_generate\_nconf, ASN1\_generate\_v3 - ASN1 generation functions

\hyperdef{}{SYNOPSIS}{\section{\hyperref[SYNOPSIS]{SYNOPSIS}}\label{SYNOPSIS}}

\begin{verbatim}
 #include <openssl/asn1.h>
 ASN1_TYPE *ASN1_generate_nconf(char *str, CONF *nconf);
 ASN1_TYPE *ASN1_generate_v3(char *str, X509V3_CTX *cnf);
\end{verbatim}

\hyperdef{}{DESCRIPTION}{\section{\hyperref[DESCRIPTION]{DESCRIPTION}}\label{DESCRIPTION}}

These functions generate the ASN1 encoding of a string in an
\textbf{ASN1\_TYPE} structure.

\textbf{str} contains the string to encode \textbf{nconf} or
\textbf{cnf} contains the optional configuration information where
additional strings will be read from. \textbf{nconf} will typically come
from a config file wherease \textbf{cnf} is obtained from an
\textbf{X509V3\_CTX} structure which will typically be used by X509 v3
certificate extension functions. \textbf{cnf} or \textbf{nconf} can be
set to \textbf{NULL} if no additional configuration will be used.

\hyperdef{}{GENERATIONux5fSTRINGux5fFORMAT}{\section{\hyperref[GENERATIONux5fSTRINGux5fFORMAT]{GENERATION
STRING FORMAT}}\label{GENERATIONux5fSTRINGux5fFORMAT}}

The actual data encoded is determined by the string \textbf{str} and the
configuration information. The general format of the string is:

\begin{description}
\item[\textbf{{[}modifier,{]}type{[}:value{]}}]
\end{description}

That is zero or more comma separated modifiers followed by a type
followed by an optional colon and a value. The formats of \textbf{type},
\textbf{value} and \textbf{modifier} are explained below.

\hyperdef{}{SUPPORTEDux5fTYPES}{\subsection{\hyperref[SUPPORTEDux5fTYPES]{SUPPORTED
TYPES}}\label{SUPPORTEDux5fTYPES}}

The supported types are listed below. Unless otherwise specified only
the \textbf{ASCII} format is permissible.

\begin{description}
\itemsep1pt\parskip0pt\parsep0pt
\item[\textbf{BOOLEAN}, \textbf{BOOL}]
This encodes a boolean type. The \textbf{value} string is mandatory and
should be \textbf{TRUE} or \textbf{FALSE}. Additionally \textbf{TRUE},
\textbf{true}, \textbf{Y}, \textbf{y}, \textbf{YES}, \textbf{yes},
\textbf{FALSE}, \textbf{false}, \textbf{N}, \textbf{n}, \textbf{NO} and
\textbf{no} are acceptable.
\end{description}

\begin{description}
\itemsep1pt\parskip0pt\parsep0pt
\item[\textbf{NULL}]
Encode the \textbf{NULL} type, the \textbf{value} string must not be
present.
\end{description}

\begin{description}
\itemsep1pt\parskip0pt\parsep0pt
\item[\textbf{INTEGER}, \textbf{INT}]
Encodes an ASN1 \textbf{INTEGER} type. The \textbf{value} string
represents the value of the integer, it can be preceded by a minus sign
and is normally interpreted as a decimal value unless the prefix
\textbf{0x} is included.
\end{description}

\begin{description}
\itemsep1pt\parskip0pt\parsep0pt
\item[\textbf{ENUMERATED}, \textbf{ENUM}]
Encodes the ASN1 \textbf{ENUMERATED} type, it is otherwise identical to
\textbf{INTEGER}.
\end{description}

\begin{description}
\itemsep1pt\parskip0pt\parsep0pt
\item[\textbf{OBJECT}, \textbf{OID}]
Encodes an ASN1 \textbf{OBJECT IDENTIFIER}, the \textbf{value} string
can be a short name, a long name or numerical format.
\end{description}

\begin{description}
\itemsep1pt\parskip0pt\parsep0pt
\item[\textbf{UTCTIME}, \textbf{UTC}]
Encodes an ASN1 \textbf{UTCTime} structure, the value should be in the
format \textbf{YYMMDDHHMMSSZ}.
\end{description}

\begin{description}
\itemsep1pt\parskip0pt\parsep0pt
\item[\textbf{GENERALIZEDTIME}, \textbf{GENTIME}]
Encodes an ASN1 \textbf{GeneralizedTime} structure, the value should be
in the format \textbf{YYYYMMDDHHMMSSZ}.
\end{description}

\begin{description}
\itemsep1pt\parskip0pt\parsep0pt
\item[\textbf{OCTETSTRING}, \textbf{OCT}]
Encodes an ASN1 \textbf{OCTET STRING}. \textbf{value} represents the
contents of this structure, the format strings \textbf{ASCII} and
\textbf{HEX} can be used to specify the format of \textbf{value}.
\end{description}

\begin{description}
\itemsep1pt\parskip0pt\parsep0pt
\item[\textbf{BITSTRING}, \textbf{BITSTR}]
Encodes an ASN1 \textbf{BIT STRING}. \textbf{value} represents the
contents of this structure, the format strings \textbf{ASCII},
\textbf{HEX} and \textbf{BITLIST} can be used to specify the format of
\textbf{value}.

~

If the format is anything other than \textbf{BITLIST} the number of
unused bits is set to zero.
\end{description}

\begin{description}
\itemsep1pt\parskip0pt\parsep0pt
\item[\textbf{UNIVERSALSTRING}, \textbf{UNIV}, \textbf{IA5},
\textbf{IA5STRING}, \textbf{UTF8}, \textbf{UTF8String}, \textbf{BMP},
\textbf{BMPSTRING}, \textbf{VISIBLESTRING}, \textbf{VISIBLE},
\textbf{PRINTABLESTRING}, \textbf{PRINTABLE}, \textbf{T61},
\textbf{T61STRING}, \textbf{TELETEXSTRING}, \textbf{GeneralString},
\textbf{NUMERICSTRING}, \textbf{NUMERIC}]
These encode the corresponding string types. \textbf{value} represents
the contents of this structure. The format can be \textbf{ASCII} or
\textbf{UTF8}.
\end{description}

\begin{description}
\itemsep1pt\parskip0pt\parsep0pt
\item[\textbf{SEQUENCE}, \textbf{SEQ}, \textbf{SET}]
Formats the result as an ASN1 \textbf{SEQUENCE} or \textbf{SET} type.
\textbf{value} should be a section name which will contain the contents.
The field names in the section are ignored and the values are in the
generated string format. If \textbf{value} is absent then an empty
SEQUENCE will be encoded.
\end{description}

\hyperdef{}{MODIFIERS}{\subsection{\hyperref[MODIFIERS]{MODIFIERS}}\label{MODIFIERS}}

Modifiers affect the following structure, they can be used to add
EXPLICIT or IMPLICIT tagging, add wrappers or to change the string
format of the final type and value. The supported formats are documented
below.

\begin{description}
\itemsep1pt\parskip0pt\parsep0pt
\item[\textbf{EXPLICIT}, \textbf{EXP}]
Add an explicit tag to the following structure. This string should be
followed by a colon and the tag value to use as a decimal value.

~

By following the number with \textbf{U}, \textbf{A}, \textbf{P} or
\textbf{C} UNIVERSAL, APPLICATION, PRIVATE or CONTEXT SPECIFIC tagging
can be used, the default is CONTEXT SPECIFIC.
\end{description}

\begin{description}
\itemsep1pt\parskip0pt\parsep0pt
\item[\textbf{IMPLICIT}, \textbf{IMP}]
This is the same as \textbf{EXPLICIT} except IMPLICIT tagging is used
instead.
\end{description}

\begin{description}
\itemsep1pt\parskip0pt\parsep0pt
\item[\textbf{OCTWRAP}, \textbf{SEQWRAP}, \textbf{SETWRAP},
\textbf{BITWRAP}]
The following structure is surrounded by an OCTET STRING, a SEQUENCE, a
SET or a BIT STRING respectively. For a BIT STRING the number of unused
bits is set to zero.
\end{description}

\begin{description}
\itemsep1pt\parskip0pt\parsep0pt
\item[\textbf{FORMAT}]
This specifies the format of the ultimate value. It should be followed
by a colon and one of the strings \textbf{ASCII}, \textbf{UTF8},
\textbf{HEX} or \textbf{BITLIST}.

~

If no format specifier is included then \textbf{ASCII} is used. If
\textbf{UTF8} is specified then the value string must be a valid
\textbf{UTF8} string. For \textbf{HEX} the output must be a set of hex
digits. \textbf{BITLIST} (which is only valid for a BIT STRING) is a
comma separated list of the indices of the set bits, all other bits are
zero.
\end{description}

\hyperdef{}{EXAMPLES}{\section{\hyperref[EXAMPLES]{EXAMPLES}}\label{EXAMPLES}}

A simple IA5String:

\begin{verbatim}
 IA5STRING:Hello World
\end{verbatim}

An IA5String explicitly tagged:

\begin{verbatim}
 EXPLICIT:0,IA5STRING:Hello World
\end{verbatim}

An IA5String explicitly tagged using APPLICATION tagging:

\begin{verbatim}
 EXPLICIT:0A,IA5STRING:Hello World
\end{verbatim}

A BITSTRING with bits 1 and 5 set and all others zero:

\begin{verbatim}
 FORMAT:BITLIST,BITSTRING:1,5
\end{verbatim}

A more complex example using a config file to produce a SEQUENCE
consiting of a BOOL an OID and a UTF8String:

\begin{verbatim}
 asn1 = SEQUENCE:seq_section
 [seq_section]
 field1 = BOOLEAN:TRUE
 field2 = OID:commonName
 field3 = UTF8:Third field
\end{verbatim}

This example produces an RSAPrivateKey structure, this is the key
contained in the file client.pem in all OpenSSL distributions (note: the
field names such as `coeff' are ignored and are present just for
clarity):

\begin{verbatim}
 asn1=SEQUENCE:private_key
 [private_key]
 version=INTEGER:0
 n=INTEGER:0xBB6FE79432CC6EA2D8F970675A5A87BFBE1AFF0BE63E879F2AFFB93644\
 D4D2C6D000430DEC66ABF47829E74B8C5108623A1C0EE8BE217B3AD8D36D5EB4FCA1D9
 e=INTEGER:0x010001
 d=INTEGER:0x6F05EAD2F27FFAEC84BEC360C4B928FD5F3A9865D0FCAAD291E2A52F4A\
 F810DC6373278C006A0ABBA27DC8C63BF97F7E666E27C5284D7D3B1FFFE16B7A87B51D
 p=INTEGER:0xF3929B9435608F8A22C208D86795271D54EBDFB09DDEF539AB083DA912\
 D4BD57
 q=INTEGER:0xC50016F89DFF2561347ED1186A46E150E28BF2D0F539A1594BBD7FE467\
 46EC4F
 exp1=INTEGER:0x9E7D4326C924AFC1DEA40B45650134966D6F9DFA3A7F9D698CD4ABEA\
 9C0A39B9
 exp2=INTEGER:0xBA84003BB95355AFB7C50DF140C60513D0BA51D637272E355E397779\
 E7B2458F
 coeff=INTEGER:0x30B9E4F2AFA5AC679F920FC83F1F2DF1BAF1779CF989447FABC2F5\
 628657053A
\end{verbatim}

This example is the corresponding public key in a SubjectPublicKeyInfo
structure:

\begin{verbatim}
 # Start with a SEQUENCE
 asn1=SEQUENCE:pubkeyinfo
 # pubkeyinfo contains an algorithm identifier and the public key wrapped
 # in a BIT STRING
 [pubkeyinfo]
 algorithm=SEQUENCE:rsa_alg
 pubkey=BITWRAP,SEQUENCE:rsapubkey
 # algorithm ID for RSA is just an OID and a NULL
 [rsa_alg]
 algorithm=OID:rsaEncryption
 parameter=NULL
 # Actual public key: modulus and exponent
 [rsapubkey]
 n=INTEGER:0xBB6FE79432CC6EA2D8F970675A5A87BFBE1AFF0BE63E879F2AFFB93644\
 D4D2C6D000430DEC66ABF47829E74B8C5108623A1C0EE8BE217B3AD8D36D5EB4FCA1D9
 e=INTEGER:0x010001
\end{verbatim}

\hyperdef{}{RETURNux5fVALUES}{\section{\hyperref[RETURNux5fVALUES]{RETURN
VALUES}}\label{RETURNux5fVALUES}}

\emph{ASN1\_generate\_nconf()} and \emph{ASN1\_generate\_v3()} return
the encoded data as an \textbf{ASN1\_TYPE} structure or \textbf{NULL} if
an error occurred.

The error codes that can be obtained by \emph{ERR\_get\_error}(3).

\hyperdef{}{SEEux5fALSO}{\section{\hyperref[SEEux5fALSO]{SEE
ALSO}}\label{SEEux5fALSO}}

\emph{ERR\_get\_error}(3)

\hyperdef{}{HISTORY}{\section{\hyperref[HISTORY]{HISTORY}}\label{HISTORY}}

\emph{ASN1\_generate\_nconf()} and \emph{ASN1\_generate\_v3()} were
added to OpenSSL 0.9.8

\begin{longtable}[c]{@{}ll@{}}
\toprule\addlinespace
2017-11-02 & 1.0.1f
\\\addlinespace
\bottomrule
\end{longtable}

\end{document}
