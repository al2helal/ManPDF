\documentclass[]{article}
\usepackage[T1]{fontenc}
\usepackage{lmodern}
\usepackage{amssymb,amsmath}
\usepackage{ifxetex,ifluatex}
\usepackage{fixltx2e} % provides \textsubscript
% use upquote if available, for straight quotes in verbatim environments
\IfFileExists{upquote.sty}{\usepackage{upquote}}{}
\ifnum 0\ifxetex 1\fi\ifluatex 1\fi=0 % if pdftex
  \usepackage[utf8]{inputenc}
\else % if luatex or xelatex
  \ifxetex
    \usepackage{mathspec}
    \usepackage{xltxtra,xunicode}
  \else
    \usepackage{fontspec}
  \fi
  \defaultfontfeatures{Mapping=tex-text,Scale=MatchLowercase}
  \newcommand{\euro}{€}
\fi
% use microtype if available
\IfFileExists{microtype.sty}{\usepackage{microtype}}{}
\usepackage{longtable,booktabs}
\ifxetex
  \usepackage[setpagesize=false, % page size defined by xetex
              unicode=false, % unicode breaks when used with xetex
              xetex]{hyperref}
\else
  \usepackage[unicode=true]{hyperref}
\fi
\hypersetup{breaklinks=true,
            bookmarks=true,
            pdfauthor={},
            pdftitle={AptPkg::hash(3pm)},
            colorlinks=true,
            citecolor=blue,
            urlcolor=blue,
            linkcolor=magenta,
            pdfborder={0 0 0}}
\urlstyle{same}  % don't use monospace font for urls
\setlength{\parindent}{0pt}
\setlength{\parskip}{6pt plus 2pt minus 1pt}
\setlength{\emergencystretch}{3em}  % prevent overfull lines
\setcounter{secnumdepth}{0}
\usepackage{pagecolor}

% Set background colour (of the page)
\definecolor{weirdbgcolor}{HTML}{FCF4F0}
\pagecolor{weirdbgcolor}

% Make bold text appear in a particular colour
\definecolor{boldcolor}{HTML}{6E0002}
\let\realtextbf=\textbf
\renewcommand{\textbf}[1]{\textcolor{boldcolor}{\realtextbf{#1}}}

% Use underlines instead of emphasis (ugh)
\renewcommand{\emph}[1]{\underline{#1}}

% % Use fixed-width font by default
% \renewcommand*\familydefault{\ttdefault}

\title{AptPkg::hash(3pm)}
\author{}
\date{}

\begin{document}
\maketitle

\begin{longtable}[c]{@{}lll@{}}
\toprule\addlinespace
AptPkg::hash(3pm) & User Contributed Perl Documentation &
AptPkg::hash(3pm)
\\\addlinespace
\bottomrule
\end{longtable}

\hyperdef{}{NAME}{\section{\hyperref[NAME]{NAME}}\label{NAME}}

AptPkg::hash - a helper class for implementing tied hashes

\hyperdef{}{SYNOPSIS}{\section{\hyperref[SYNOPSIS]{SYNOPSIS}}\label{SYNOPSIS}}

use AptPkg::hash;

\hyperdef{}{DESCRIPTION}{\section{\hyperref[DESCRIPTION]{DESCRIPTION}}\label{DESCRIPTION}}

The AptPkg::hash class provides hash-like access for objects which have
an underlying XS implementation.

Such objects need to add AptPkg::hash to @ISA, provide get, set and
exists methods, and an iterator class.

\hyperdef{}{AptPkg::hash}{\subsection{\hyperref[AptPkg::hash]{AptPkg::hash}}\label{AptPkg::hash}}

\begin{description}
\itemsep1pt\parskip0pt\parsep0pt
\item[new({[}\emph{XS\_OBJECT}{]})]
Create a object as a tied hash. The object is implemented as a hash
reference blessed into the class, which in turn is tied to AptPkg::hash.

~

This means that both \$obj-\textgreater{} \emph{method()} and
\$obj-\textgreater{}\{key\} valid, the latter invoking get/set (through
FETCH/STORE).

~

The tie associates an array reference with the hash, which initially
contains a reference to the hash, the XS object and an anon hash which
may be used by subclasses to store state information.

~

If no XS object is passed, one is created via new in the XS class. The
name of that class is constructed from the class name, by lower-casing
the last component and prefixing it with an underscore (eg.
AptPkg::Config becomes AptPkg::\_config).

~

If the module contains a @KEYS array, then the private hash will be
populated with those entries as keys (see the description below of the
AptPkg::hash::method class).
\end{description}

\begin{description}
\itemsep1pt\parskip0pt\parsep0pt
\item[\_self, \_xs, \_priv]
Accessors which may be used in subclass methods to fetch the three array
elements associated with the hash reference.
\end{description}

\begin{description}
\itemsep1pt\parskip0pt\parsep0pt
\item[keys(\emph{ARGS})]
In a scalar context, creates and returns a new iterator object (the
class name with the suffix ::Iter appended).

~

The XS object, the private hash and any arguments are passed to the
constructor.

~

In an array context, the iterator is used to generate a list of keys
which are then returned.

~

The iterator class must implement a next method, which returns the
current key and advances to the next.
\end{description}

\hyperdef{}{AptPkg::hash::method}{\subsection{\hyperref[AptPkg::hash::method]{AptPkg::hash::method}}\label{AptPkg::hash::method}}

The AptPkg::hash::method class extends AptPkg::hash, providing a simple
way to map a fixed set of keys (defined by the @KEYS array) into method
calls on either the object, or the internal XS object.

Classes inheriting from AptPkg::hash::method should provide an iterator
class which inherits from AptPkg::hash::method::iter.

\hyperdef{}{AUTHOR}{\section{\hyperref[AUTHOR]{AUTHOR}}\label{AUTHOR}}

Brendan O'Dea \textless{}bod@debian.org\textgreater{}

\begin{longtable}[c]{@{}ll@{}}
\toprule\addlinespace
2013-05-10 & perl v5.18.1
\\\addlinespace
\bottomrule
\end{longtable}

\end{document}
