\documentclass[]{article}
\usepackage[T1]{fontenc}
\usepackage{lmodern}
\usepackage{amssymb,amsmath}
\usepackage{ifxetex,ifluatex}
\usepackage{fixltx2e} % provides \textsubscript
% use upquote if available, for straight quotes in verbatim environments
\IfFileExists{upquote.sty}{\usepackage{upquote}}{}
\ifnum 0\ifxetex 1\fi\ifluatex 1\fi=0 % if pdftex
  \usepackage[utf8]{inputenc}
\else % if luatex or xelatex
  \ifxetex
    \usepackage{mathspec}
    \usepackage{xltxtra,xunicode}
  \else
    \usepackage{fontspec}
  \fi
  \defaultfontfeatures{Mapping=tex-text,Scale=MatchLowercase}
  \newcommand{\euro}{€}
\fi
% use microtype if available
\IfFileExists{microtype.sty}{\usepackage{microtype}}{}
\usepackage{longtable,booktabs}
\ifxetex
  \usepackage[setpagesize=false, % page size defined by xetex
              unicode=false, % unicode breaks when used with xetex
              xetex]{hyperref}
\else
  \usepackage[unicode=true]{hyperref}
\fi
\hypersetup{breaklinks=true,
            bookmarks=true,
            pdfauthor={},
            pdftitle={ASN1\_STRING\_length(3SSL)},
            colorlinks=true,
            citecolor=blue,
            urlcolor=blue,
            linkcolor=magenta,
            pdfborder={0 0 0}}
\urlstyle{same}  % don't use monospace font for urls
\setlength{\parindent}{0pt}
\setlength{\parskip}{6pt plus 2pt minus 1pt}
\setlength{\emergencystretch}{3em}  % prevent overfull lines
\setcounter{secnumdepth}{0}
\usepackage{pagecolor}

% Set background colour (of the page)
\definecolor{weirdbgcolor}{HTML}{FCF4F0}
\pagecolor{weirdbgcolor}

% Make bold text appear in a particular colour
\definecolor{boldcolor}{HTML}{6E0002}
\let\realtextbf=\textbf
\renewcommand{\textbf}[1]{\textcolor{boldcolor}{\realtextbf{#1}}}

% Use underlines instead of emphasis (ugh)
\renewcommand{\emph}[1]{\underline{#1}}

% % Use fixed-width font by default
% \renewcommand*\familydefault{\ttdefault}

\title{ASN1\_STRING\_length(3SSL)}
\author{}
\date{}

\begin{document}
\maketitle

\begin{longtable}[c]{@{}lll@{}}
\toprule\addlinespace
ASN1\_STRING\_length(3SSL) & OpenSSL & ASN1\_STRING\_length(3SSL)
\\\addlinespace
\bottomrule
\end{longtable}

\hyperdef{}{NAME}{\section{\hyperref[NAME]{NAME}}\label{NAME}}

ASN1\_STRING\_dup, ASN1\_STRING\_cmp, ASN1\_STRING\_set,
ASN1\_STRING\_length, ASN1\_STRING\_length\_set, ASN1\_STRING\_type,
ASN1\_STRING\_data - ASN1\_STRING utility functions

\hyperdef{}{SYNOPSIS}{\section{\hyperref[SYNOPSIS]{SYNOPSIS}}\label{SYNOPSIS}}

\begin{verbatim}
 #include <openssl/asn1.h>
 int ASN1_STRING_length(ASN1_STRING *x);
 unsigned char * ASN1_STRING_data(ASN1_STRING *x);
 ASN1_STRING * ASN1_STRING_dup(ASN1_STRING *a);
 int ASN1_STRING_cmp(ASN1_STRING *a, ASN1_STRING *b);
 int ASN1_STRING_set(ASN1_STRING *str, const void *data, int len);
 int ASN1_STRING_type(ASN1_STRING *x);
 int ASN1_STRING_to_UTF8(unsigned char **out, ASN1_STRING *in);
\end{verbatim}

\hyperdef{}{DESCRIPTION}{\section{\hyperref[DESCRIPTION]{DESCRIPTION}}\label{DESCRIPTION}}

These functions allow an \textbf{ASN1\_STRING} structure to be
manipulated.

\emph{ASN1\_STRING\_length()} returns the length of the content of
\textbf{x}.

\emph{ASN1\_STRING\_data()} returns an internal pointer to the data of
\textbf{x}. Since this is an internal pointer it should \textbf{not} be
freed or modified in any way.

\emph{ASN1\_STRING\_dup()} returns a copy of the structure \textbf{a}.

\emph{ASN1\_STRING\_cmp()} compares \textbf{a} and \textbf{b} returning
0 if the two are identical. The string types and content are compared.

\emph{ASN1\_STRING\_set()} sets the data of string \textbf{str} to the
buffer \textbf{data} or length \textbf{len}. The supplied data is
copied. If \textbf{len} is -1 then the length is determined by
strlen(data).

\emph{ASN1\_STRING\_type()} returns the type of \textbf{x}, using
standard constants such as \textbf{V\_ASN1\_OCTET\_STRING}.

\emph{ASN1\_STRING\_to\_UTF8()} converts the string \textbf{in} to UTF8
format, the converted data is allocated in a buffer in \textbf{*out}.
The length of \textbf{out} is returned or a negative error code. The
buffer \textbf{*out} should be free using \emph{OPENSSL\_free()}.

\hyperdef{}{NOTES}{\section{\hyperref[NOTES]{NOTES}}\label{NOTES}}

Almost all ASN1 types in OpenSSL are represented as an
\textbf{ASN1\_STRING} structure. Other types such as
\textbf{ASN1\_OCTET\_STRING} are simply typedefed to
\textbf{ASN1\_STRING} and the functions call the \textbf{ASN1\_STRING}
equivalents. \textbf{ASN1\_STRING} is also used for some \textbf{CHOICE}
types which consist entirely of primitive string types such as
\textbf{DirectoryString} and \textbf{Time}.

These functions should \textbf{not} be used to examine or modify
\textbf{ASN1\_INTEGER} or \textbf{ASN1\_ENUMERATED} types: the relevant
\textbf{INTEGER} or \textbf{ENUMERATED} utility functions should be used
instead.

In general it cannot be assumed that the data returned by
\emph{ASN1\_STRING\_data()} is null terminated or does not contain
embedded nulls. The actual format of the data will depend on the actual
string type itself: for example for and IA5String the data will be
ASCII, for a BMPString two bytes per character in big endian format,
UTF8String will be in UTF8 format.

Similar care should be take to ensure the data is in the correct format
when calling \emph{ASN1\_STRING\_set()}.

\hyperdef{}{RETURNux5fVALUES}{\section{\hyperref[RETURNux5fVALUES]{RETURN
VALUES}}\label{RETURNux5fVALUES}}

\hyperdef{}{SEEux5fALSO}{\section{\hyperref[SEEux5fALSO]{SEE
ALSO}}\label{SEEux5fALSO}}

\emph{ERR\_get\_error}(3)

\hyperdef{}{HISTORY}{\section{\hyperref[HISTORY]{HISTORY}}\label{HISTORY}}

\begin{longtable}[c]{@{}ll@{}}
\toprule\addlinespace
2014-01-06 & 1.0.1f
\\\addlinespace
\bottomrule
\end{longtable}

\end{document}
