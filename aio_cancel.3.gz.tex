\documentclass[]{article}
\usepackage[T1]{fontenc}
\usepackage{lmodern}
\usepackage{amssymb,amsmath}
\usepackage{ifxetex,ifluatex}
\usepackage{fixltx2e} % provides \textsubscript
% use upquote if available, for straight quotes in verbatim environments
\IfFileExists{upquote.sty}{\usepackage{upquote}}{}
\ifnum 0\ifxetex 1\fi\ifluatex 1\fi=0 % if pdftex
  \usepackage[utf8]{inputenc}
\else % if luatex or xelatex
  \ifxetex
    \usepackage{mathspec}
    \usepackage{xltxtra,xunicode}
  \else
    \usepackage{fontspec}
  \fi
  \defaultfontfeatures{Mapping=tex-text,Scale=MatchLowercase}
  \newcommand{\euro}{€}
\fi
% use microtype if available
\IfFileExists{microtype.sty}{\usepackage{microtype}}{}
\usepackage{longtable,booktabs}
\ifxetex
  \usepackage[setpagesize=false, % page size defined by xetex
              unicode=false, % unicode breaks when used with xetex
              xetex]{hyperref}
\else
  \usepackage[unicode=true]{hyperref}
\fi
\hypersetup{breaklinks=true,
            bookmarks=true,
            pdfauthor={},
            pdftitle={AIO\_CANCEL(3)},
            colorlinks=true,
            citecolor=blue,
            urlcolor=blue,
            linkcolor=magenta,
            pdfborder={0 0 0}}
\urlstyle{same}  % don't use monospace font for urls
\setlength{\parindent}{0pt}
\setlength{\parskip}{6pt plus 2pt minus 1pt}
\setlength{\emergencystretch}{3em}  % prevent overfull lines
\setcounter{secnumdepth}{0}
\usepackage{pagecolor}

% Set background colour (of the page)
\definecolor{weirdbgcolor}{HTML}{FCF4F0}
\pagecolor{weirdbgcolor}

% Make bold text appear in a particular colour
\definecolor{boldcolor}{HTML}{6E0002}
\let\realtextbf=\textbf
\renewcommand{\textbf}[1]{\textcolor{boldcolor}{\realtextbf{#1}}}

% Use underlines instead of emphasis (ugh)
\renewcommand{\emph}[1]{\underline{#1}}

% % Use fixed-width font by default
% \renewcommand*\familydefault{\ttdefault}

\title{AIO\_CANCEL(3)}
\author{}
\date{}

\begin{document}
\maketitle

\begin{longtable}[c]{@{}lll@{}}
\toprule\addlinespace
AIO\_CANCEL(3) & Linux Programmer's Manual & AIO\_CANCEL(3)
\\\addlinespace
\bottomrule
\end{longtable}

\hyperdef{}{NAME}{\section{\hyperref[NAME]{NAME}}\label{NAME}}

aio\_cancel - cancel an outstanding asynchronous I/O request

\hyperdef{}{SYNOPSIS}{\section{\hyperref[SYNOPSIS]{SYNOPSIS}}\label{SYNOPSIS}}

\textbf{\#include \textless{}aio.h\textgreater{}}

~

\textbf{int aio\_cancel(int}\emph{fd}\textbf{, struct aiocb
*}\emph{aiocbp}\textbf{);}

~

Link with \emph{-lrt}.

\hyperdef{}{DESCRIPTION}{\section{\hyperref[DESCRIPTION]{DESCRIPTION}}\label{DESCRIPTION}}

The \textbf{aio\_cancel}() function attempts to cancel outstanding
asynchronous I/O requests for the file descriptor \emph{fd}. If
\emph{aiocbp} is NULL, all such requests are canceled. Otherwise, only
the request described by the control block pointed to by \emph{aiocbp}
is canceled. (See \textbf{aio}(7) for a description of the \emph{aiocb}
structure.)

Normal asynchronous notification occurs for canceled requests (see
\textbf{aio}(7) and \textbf{sigevent}(7)). The request return status
(\textbf{aio\_return}(3)) is set to -1, and the request error status
(\textbf{aio\_error}(3)) is set to \textbf{ECANCELED}. The control block
of requests that cannot be canceled is not changed.

If the request could not be canceled, then it will terminate in the
usual way after performing the I/O operation. (In this case,
\textbf{aio\_error}(3) will return the status \textbf{EINPROGRESSS}.)

If \emph{aiocbp} is not NULL, and \emph{fd} differs from the file
descriptor with which the asynchronous operation was initiated,
unspecified results occur.

Which operations are cancelable is implementation-defined.

\hyperdef{}{RETURNux5fVALUE}{\section{\hyperref[RETURNux5fVALUE]{RETURN
VALUE}}\label{RETURNux5fVALUE}}

The \textbf{aio\_cancel}() function returns one of the following values:

\begin{description}
\itemsep1pt\parskip0pt\parsep0pt
\item[\textbf{AIO\_CANCELED}]
All requests were successfully canceled.
\end{description}

\begin{description}
\itemsep1pt\parskip0pt\parsep0pt
\item[\textbf{AIO\_NOTCANCELED}]
At least one of the requests specified was not canceled because it was
in progress. In this case, one may check the status of individual
requests using \textbf{aio\_error}(3).
\end{description}

\begin{description}
\itemsep1pt\parskip0pt\parsep0pt
\item[\textbf{AIO\_ALLDONE}]
All requests had already been completed before the call.
\end{description}

\begin{description}
\itemsep1pt\parskip0pt\parsep0pt
\item[-1]
An error occurred. The cause of the error can be found by inspecting
\emph{errno}.
\end{description}

\hyperdef{}{ERRORS}{\section{\hyperref[ERRORS]{ERRORS}}\label{ERRORS}}

\begin{description}
\itemsep1pt\parskip0pt\parsep0pt
\item[\textbf{EBADF}]
\emph{fd} is not a valid file descriptor.
\end{description}

\begin{description}
\itemsep1pt\parskip0pt\parsep0pt
\item[\textbf{ENOSYS}]
\textbf{aio\_cancel}() is not implemented.
\end{description}

\hyperdef{}{VERSIONS}{\section{\hyperref[VERSIONS]{VERSIONS}}\label{VERSIONS}}

The \textbf{aio\_cancel}() function is available since glibc 2.1.

\hyperdef{}{CONFORMINGux5fTO}{\section{\hyperref[CONFORMINGux5fTO]{CONFORMING
TO}}\label{CONFORMINGux5fTO}}

POSIX.1-2001, POSIX.1-2008.

\hyperdef{}{EXAMPLE}{\section{\hyperref[EXAMPLE]{EXAMPLE}}\label{EXAMPLE}}

See \textbf{aio}(7).

\hyperdef{}{SEEux5fALSO}{\section{\hyperref[SEEux5fALSO]{SEE
ALSO}}\label{SEEux5fALSO}}

\textbf{aio\_error}(3), \textbf{aio\_fsync}(3), \textbf{aio\_read}(3),
\textbf{aio\_return}(3), \textbf{aio\_suspend}(3),
\textbf{aio\_write}(3), \textbf{lio\_listio}(3), \textbf{aio}(7)

\hyperdef{}{COLOPHON}{\section{\hyperref[COLOPHON]{COLOPHON}}\label{COLOPHON}}

This page is part of release 3.54 of the Linux \emph{man-pages} project.
A description of the project, and information about reporting bugs, can
be found at http://www.kernel.org/doc/man-pages/.

\begin{longtable}[c]{@{}ll@{}}
\toprule\addlinespace
2012-05-08 &
\\\addlinespace
\bottomrule
\end{longtable}

\end{document}
