\documentclass[]{article}
\usepackage[T1]{fontenc}
\usepackage{lmodern}
\usepackage{amssymb,amsmath}
\usepackage{ifxetex,ifluatex}
\usepackage{fixltx2e} % provides \textsubscript
% use upquote if available, for straight quotes in verbatim environments
\IfFileExists{upquote.sty}{\usepackage{upquote}}{}
\ifnum 0\ifxetex 1\fi\ifluatex 1\fi=0 % if pdftex
  \usepackage[utf8]{inputenc}
\else % if luatex or xelatex
  \ifxetex
    \usepackage{mathspec}
    \usepackage{xltxtra,xunicode}
  \else
    \usepackage{fontspec}
  \fi
  \defaultfontfeatures{Mapping=tex-text,Scale=MatchLowercase}
  \newcommand{\euro}{€}
\fi
% use microtype if available
\IfFileExists{microtype.sty}{\usepackage{microtype}}{}
\usepackage{longtable,booktabs}
\ifxetex
  \usepackage[setpagesize=false, % page size defined by xetex
              unicode=false, % unicode breaks when used with xetex
              xetex]{hyperref}
\else
  \usepackage[unicode=true]{hyperref}
\fi
\hypersetup{breaklinks=true,
            bookmarks=true,
            pdfauthor={},
            pdftitle={ADJTIME(3)},
            colorlinks=true,
            citecolor=blue,
            urlcolor=blue,
            linkcolor=magenta,
            pdfborder={0 0 0}}
\urlstyle{same}  % don't use monospace font for urls
\setlength{\parindent}{0pt}
\setlength{\parskip}{6pt plus 2pt minus 1pt}
\setlength{\emergencystretch}{3em}  % prevent overfull lines
\setcounter{secnumdepth}{0}
\usepackage{pagecolor}

% Set background colour (of the page)
\definecolor{weirdbgcolor}{HTML}{FCF4F0}
\pagecolor{weirdbgcolor}

% Make bold text appear in a particular colour
\definecolor{boldcolor}{HTML}{6E0002}
\let\realtextbf=\textbf
\renewcommand{\textbf}[1]{\textcolor{boldcolor}{\realtextbf{#1}}}

% Use underlines instead of emphasis (ugh)
\renewcommand{\emph}[1]{\underline{#1}}

% % Use fixed-width font by default
% \renewcommand*\familydefault{\ttdefault}

\title{ADJTIME(3)}
\author{}
\date{}

\begin{document}
\maketitle

\begin{longtable}[c]{@{}lll@{}}
\toprule\addlinespace
ADJTIME(3) & Linux Programmer's Manual & ADJTIME(3)
\\\addlinespace
\bottomrule
\end{longtable}

\hyperdef{}{NAME}{\section{\hyperref[NAME]{NAME}}\label{NAME}}

adjtime - correct the time to synchronize the system clock

\hyperdef{}{SYNOPSIS}{\section{\hyperref[SYNOPSIS]{SYNOPSIS}}\label{SYNOPSIS}}

\begin{verbatim}
int adjtime(const struct timeval *delta, struct timeval *olddelta);
\end{verbatim}

~

Feature Test Macro Requirements for glibc (see
\textbf{feature\_test\_macros}(7)): \\

~

\textbf{adjtime}(): \_BSD\_SOURCE

\hyperdef{}{DESCRIPTION}{\section{\hyperref[DESCRIPTION]{DESCRIPTION}}\label{DESCRIPTION}}

The \textbf{adjtime}() function gradually adjusts the system clock (as
returned by \textbf{gettimeofday}(2)). The amount of time by which the
clock is to be adjusted is specified in the structure pointed to by
\emph{delta}. This structure has the following form: \\

\begin{verbatim}

struct timeval {
    time_t      tv_sec;     /* seconds */
    suseconds_t tv_usec;    /* microseconds */
};
\end{verbatim}

If the adjustment in \emph{delta} is positive, then the system clock is
speeded up by some small percentage (i.e., by adding a small amount of
time to the clock value in each second) until the adjustment has been
completed. If the adjustment in \emph{delta} is negative, then the clock
is slowed down in a similar fashion.

~

If a clock adjustment from an earlier \textbf{adjtime}() call is already
in progress at the time of a later \textbf{adjtime}() call, and
\emph{delta} is not NULL for the later call, then the earlier adjustment
is stopped, but any already completed part of that adjustment is not
undone.

~

If \emph{olddelta} is not NULL, then the buffer that it points to is
used to return the amount of time remaining from any previous adjustment
that has not yet been completed.

\hyperdef{}{RETURNux5fVALUE}{\section{\hyperref[RETURNux5fVALUE]{RETURN
VALUE}}\label{RETURNux5fVALUE}}

On success, \textbf{adjtime}() returns 0. On failure, -1 is returned,
and \emph{errno} is set to indicate the error.

\hyperdef{}{ERRORS}{\section{\hyperref[ERRORS]{ERRORS}}\label{ERRORS}}

\begin{description}
\itemsep1pt\parskip0pt\parsep0pt
\item[\textbf{EINVAL}]
The adjustment in \emph{delta} is outside the permitted range.
\end{description}

\begin{description}
\itemsep1pt\parskip0pt\parsep0pt
\item[\textbf{EPERM}]
The caller does not have sufficient privilege to adjust the time. Under
Linux the \textbf{CAP\_SYS\_TIME} capability is required.
\end{description}

\hyperdef{}{CONFORMINGux5fTO}{\section{\hyperref[CONFORMINGux5fTO]{CONFORMING
TO}}\label{CONFORMINGux5fTO}}

4.3BSD, System V.

\hyperdef{}{NOTES}{\section{\hyperref[NOTES]{NOTES}}\label{NOTES}}

The adjustment that \textbf{adjtime}() makes to the clock is carried out
in such a manner that the clock is always monotonically increasing.
Using \textbf{adjtime}() to adjust the time prevents the problems that
can be caused for certain applications (e.g., \textbf{make}(1)) by
abrupt positive or negative jumps in the system time.

~

\textbf{adjtime}() is intended to be used to make small adjustments to
the system time. Most systems impose a limit on the adjustment that can
be specified in \emph{delta}. In the glibc implementation, \emph{delta}
must be less than or equal to (INT\_MAX / 1000000 - 2) and greater than
or equal to (INT\_MIN / 1000000 + 2) (respectively 2145 and -2145
seconds on i386).

\hyperdef{}{BUGS}{\section{\hyperref[BUGS]{BUGS}}\label{BUGS}}

A longstanding bug meant that if \emph{delta} was specified as NULL, no
valid information about the outstanding clock adjustment was returned in
\emph{olddelta}. (In this circumstance, \textbf{adjtime}() should return
the outstanding clock adjustment, without changing it.) This bug is
fixed on systems with glibc 2.8 or later and Linux kernel 2.6.26 or
later.

\hyperdef{}{SEEux5fALSO}{\section{\hyperref[SEEux5fALSO]{SEE
ALSO}}\label{SEEux5fALSO}}

\textbf{adjtimex}(2), \textbf{gettimeofday}(2), \textbf{time}(7)

\hyperdef{}{COLOPHON}{\section{\hyperref[COLOPHON]{COLOPHON}}\label{COLOPHON}}

This page is part of release 3.54 of the Linux \emph{man-pages} project.
A description of the project, and information about reporting bugs, can
be found at http://www.kernel.org/doc/man-pages/.

\begin{longtable}[c]{@{}ll@{}}
\toprule\addlinespace
2008-06-22 & Linux
\\\addlinespace
\bottomrule
\end{longtable}

\end{document}
