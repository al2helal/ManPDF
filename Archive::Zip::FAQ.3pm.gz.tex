\documentclass[]{article}
\usepackage[T1]{fontenc}
\usepackage{lmodern}
\usepackage{amssymb,amsmath}
\usepackage{ifxetex,ifluatex}
\usepackage{fixltx2e} % provides \textsubscript
% use upquote if available, for straight quotes in verbatim environments
\IfFileExists{upquote.sty}{\usepackage{upquote}}{}
\ifnum 0\ifxetex 1\fi\ifluatex 1\fi=0 % if pdftex
  \usepackage[utf8]{inputenc}
\else % if luatex or xelatex
  \ifxetex
    \usepackage{mathspec}
    \usepackage{xltxtra,xunicode}
  \else
    \usepackage{fontspec}
  \fi
  \defaultfontfeatures{Mapping=tex-text,Scale=MatchLowercase}
  \newcommand{\euro}{€}
\fi
% use microtype if available
\IfFileExists{microtype.sty}{\usepackage{microtype}}{}
\usepackage{longtable,booktabs}
\ifxetex
  \usepackage[setpagesize=false, % page size defined by xetex
              unicode=false, % unicode breaks when used with xetex
              xetex]{hyperref}
\else
  \usepackage[unicode=true]{hyperref}
\fi
\hypersetup{breaklinks=true,
            bookmarks=true,
            pdfauthor={},
            pdftitle={Archive::Zip::FAQ(3pm)},
            colorlinks=true,
            citecolor=blue,
            urlcolor=blue,
            linkcolor=magenta,
            pdfborder={0 0 0}}
\urlstyle{same}  % don't use monospace font for urls
\setlength{\parindent}{0pt}
\setlength{\parskip}{6pt plus 2pt minus 1pt}
\setlength{\emergencystretch}{3em}  % prevent overfull lines
\setcounter{secnumdepth}{0}
\usepackage{pagecolor}

% Set background colour (of the page)
\definecolor{weirdbgcolor}{HTML}{FCF4F0}
\pagecolor{weirdbgcolor}

% Make bold text appear in a particular colour
\definecolor{boldcolor}{HTML}{6E0002}
\let\realtextbf=\textbf
\renewcommand{\textbf}[1]{\textcolor{boldcolor}{\realtextbf{#1}}}

% Use underlines instead of emphasis (ugh)
\renewcommand{\emph}[1]{\underline{#1}}

% % Use fixed-width font by default
% \renewcommand*\familydefault{\ttdefault}

\title{Archive::Zip::FAQ(3pm)}
\author{}
\date{}

\begin{document}
\maketitle

\begin{longtable}[c]{@{}lll@{}}
\toprule\addlinespace
Archive::Zip::FAQ(3pm) & User Contributed Perl Documentation &
Archive::Zip::FAQ(3pm)
\\\addlinespace
\bottomrule
\end{longtable}

\hyperdef{}{NAME}{\section{\hyperref[NAME]{NAME}}\label{NAME}}

Archive::Zip::FAQ - Answers to a few frequently asked questions about
Archive::Zip

\hyperdef{}{DESCRIPTION}{\section{\hyperref[DESCRIPTION]{DESCRIPTION}}\label{DESCRIPTION}}

It seems that I keep answering the same questions over and over again. I
assume that this is because my documentation is deficient, rather than
that people don't read the documentation.

So this FAQ is an attempt to cut down on the number of personal answers
I have to give. At least I can now say "You \emph{did} read the FAQ,
right?".

The questions are not in any particular order. The answers assume the
current version of Archive::Zip; some of the answers depend on newly
added/fixed functionality.

\hyperdef{}{Installux5fproblemsux5fonux5fRedHatux5f8ux5forux5f9ux5fwithux5fPerlux5f5.8.0}{\section{\hyperref[Installux5fproblemsux5fonux5fRedHatux5f8ux5forux5f9ux5fwithux5fPerlux5f5.8.0]{Install
problems on RedHat 8 or 9 with Perl
5.8.0}}\label{Installux5fproblemsux5fonux5fRedHatux5f8ux5forux5f9ux5fwithux5fPerlux5f5.8.0}}

\textbf{Q:} Archive::Zip won't install on my RedHat 9 system! It's
broke!

\textbf{A:} This has become something of a FAQ. Basically, RedHat broke
some versions of Perl by setting LANG to UTF8. They apparently have a
fixed version out as an update.

You might try running CPAN or creating your Makefile after exporting the
LANG environment variable as

``LANG=C''

\textless{}https://bugzilla.redhat.com/bugzilla/show\_bug.cgi?id=87682\textgreater{}

\hyperdef{}{Whyux5fisux5fmyux5fzipux5ffileux5fsoux5fbigux3f}{\section{\hyperref[Whyux5fisux5fmyux5fzipux5ffileux5fsoux5fbigux3f]{Why
is my zip file so
big?}}\label{Whyux5fisux5fmyux5fzipux5ffileux5fsoux5fbigux3f}}

\textbf{Q:} My zip file is actually bigger than what I stored in it!
Why?

\textbf{A:} Some things to make sure of:

\begin{description}
\itemsep1pt\parskip0pt\parsep0pt
\item[Make sure that you are requesting COMPRESSION\_DEFLATED if you are
storing strings.]
\$member-\textgreater{}desiredCompressionMethod( COMPRESSION\_DEFLATED
);
\end{description}

\begin{description}
\itemsep1pt\parskip0pt\parsep0pt
\item[Don't make lots of little files if you can help it.]
Since zip computes the compression tables for each member, small members
without much entropy won't compress well. Instead, if you've got lots of
repeated strings in your data, try to combine them into one big member.
\end{description}

\begin{description}
\itemsep1pt\parskip0pt\parsep0pt
\item[Make sure that you are requesting COMPRESSION\_STORED if you are
storing things that are already compressed.]
If you're storing a .zip, .jpg, .mp3, or other compressed file in a zip,
then don't compress them again. They'll get bigger.
\end{description}

\hyperdef{}{Sampleux5fcodeux3f}{\section{\hyperref[Sampleux5fcodeux3f]{Sample
code?}}\label{Sampleux5fcodeux3f}}

\textbf{Q:} Can you send me code to do (whatever)?

\textbf{A:} Have you looked in the ``examples/'' directory yet? It
contains:

\begin{description}
\item[examples/calcSizes.pl -- How to find out how big a Zip file will
be before writing it]
\end{description}

\begin{description}
\item[examples/copy.pl -- Copies one Zip file to another]
\end{description}

\begin{description}
\item[examples/extract.pl -- extract file(s) from a Zip]
\end{description}

\begin{description}
\item[examples/mailZip.pl -- make and mail a zip file]
\end{description}

\begin{description}
\item[examples/mfh.pl -- demo for use of MockFileHandle]
\end{description}

\begin{description}
\item[examples/readScalar.pl -- shows how to use IO::Scalar as the
source of a Zip read]
\end{description}

\begin{description}
\item[examples/selfex.pl -- a brief example of a self-extracting Zip]
\end{description}

\begin{description}
\item[examples/unzipAll.pl -- uses Archive::Zip::Tree to unzip an entire
Zip]
\end{description}

\begin{description}
\item[examples/updateZip.pl -- shows how to read/modify/write a Zip]
\end{description}

\begin{description}
\item[examples/updateTree.pl -- shows how to update a Zip in place]
\end{description}

\begin{description}
\item[examples/writeScalar.pl -- shows how to use IO::Scalar as the
destination of a Zip write]
\end{description}

\begin{description}
\item[examples/writeScalar2.pl -- shows how to use IO::String as the
destination of a Zip write]
\end{description}

\begin{description}
\item[examples/zip.pl -- Constructs a Zip file]
\end{description}

\begin{description}
\item[examples/zipcheck.pl -- One way to check a Zip file for validity]
\end{description}

\begin{description}
\item[examples/zipinfo.pl -- Prints out information about a Zip archive
file]
\end{description}

\begin{description}
\item[examples/zipGrep.pl -- Searches for text in Zip files]
\end{description}

\begin{description}
\item[examples/ziptest.pl -- Lists a Zip file and checks member CRCs]
\end{description}

\begin{description}
\item[examples/ziprecent.pl -- Puts recent files into a zipfile]
\end{description}

\begin{description}
\item[examples/ziptest.pl -- Another way to check a Zip file for
validity]
\end{description}

\hyperdef{}{Canux27tux5fReadux2fmodifyux2fwriteux5fsameux5fZipux5ffile}{\section{\hyperref[Canux27tux5fReadux2fmodifyux2fwriteux5fsameux5fZipux5ffile]{Can't
Read/modify/write same Zip
file}}\label{Canux27tux5fReadux2fmodifyux2fwriteux5fsameux5fZipux5ffile}}

\textbf{Q:} Why can't I open a Zip file, add a member, and write it
back? I get an error message when I try.

\textbf{A:} Because Archive::Zip doesn't (and can't, generally) read
file contents into memory, the original Zip file is required to stay
around until the writing of the new file is completed.

The best way to do this is to write the Zip to a temporary file and then
rename the temporary file to have the old name (possibly after deleting
the old one).

Archive::Zip v1.02 added the archive methods ``overwrite()'' and
``overwriteAs()'' to do this simply and carefully.

See ``examples/updateZip.pl'' for an example of this technique.

\hyperdef{}{Fileux5fcreationux5ftimeux5fnotux5fset}{\section{\hyperref[Fileux5fcreationux5ftimeux5fnotux5fset]{File
creation time not set}}\label{Fileux5fcreationux5ftimeux5fnotux5fset}}

\textbf{Q:} Upon extracting files, I see that their modification (and
access) times are set to the time in the Zip archive. However, their
creation time is not set to the same time. Why?

\textbf{A:} Mostly because Perl doesn't give cross-platform access to
\emph{creation time}. Indeed, many systems (like Unix) don't support
such a concept. However, if yours does, you can easily set it. Get the
modification time from the member using ``lastModTime()''.

\hyperdef{}{Canux27tux5fuseux5fArchive::Zipux5fonux5fgzipux5ffiles}{\section{\hyperref[Canux27tux5fuseux5fArchive::Zipux5fonux5fgzipux5ffiles]{Can't
use Archive::Zip on gzip
files}}\label{Canux27tux5fuseux5fArchive::Zipux5fonux5fgzipux5ffiles}}

\textbf{Q:} Can I use Archive::Zip to extract Unix gzip files?

\textbf{A:} No.

There is a distinction between Unix gzip files, and Zip archives that
also can use the gzip compression.

Depending on the format of the gzip file, you can use
Compress::Raw::Zlib, or Archive::Tar to decompress it (and de-archive it
in the case of Tar files).

You can unzip PKZIP/WinZip/etc/ archives using Archive::Zip (that's what
it's for) as long as any compressed members are compressed using Deflate
compression.

\hyperdef{}{Addux5faux5fdirectoryux2ftreeux5ftoux5faux5fZip}{\section{\hyperref[Addux5faux5fdirectoryux2ftreeux5ftoux5faux5fZip]{Add
a directory/tree to a
Zip}}\label{Addux5faux5fdirectoryux2ftreeux5ftoux5faux5fZip}}

\textbf{Q:} How can I add a directory (or tree) full of files to a Zip?

\textbf{A:} You can use the Archive::Zip::addTree*() methods:

\begin{verbatim}
   use Archive::Zip;
   my $zip = Archive::Zip->new();
   # add all readable files and directories below . as xyz/*
   $zip->addTree( '.', 'xyz' ); 
   # add all readable plain files below /abc as def/*
   $zip->addTree( '/abc', 'def', sub { -f && -r } );    
   # add all .c files below /tmp as stuff/*
   $zip->addTreeMatching( '/tmp', 'stuff', '\.c$' );
   # add all .o files below /tmp as stuff/* if they aren't writable
   $zip->addTreeMatching( '/tmp', 'stuff', '\.o$', sub { ! -w } );
   # add all .so files below /tmp that are smaller than 200 bytes as stuff/*
   $zip->addTreeMatching( '/tmp', 'stuff', '\.o$', sub { -s < 200 } );
   # and write them into a file
   $zip->writeToFileNamed('xxx.zip');
\end{verbatim}

\hyperdef{}{Extractux5faux5fdirectoryux2ftree}{\section{\hyperref[Extractux5faux5fdirectoryux2ftree]{Extract
a directory/tree}}\label{Extractux5faux5fdirectoryux2ftree}}

\textbf{Q:} How can I extract some (or all) files from a Zip into a
different directory?

\textbf{A:} You can use the \emph{Archive::Zip::extractTree()} method:
??? \textbar{}\textbar{}

\begin{verbatim}
   # now extract the same files into /tmpx
   $zip->extractTree( 'stuff', '/tmpx' );
\end{verbatim}

\hyperdef{}{Updateux5faux5fdirectoryux2ftree}{\section{\hyperref[Updateux5faux5fdirectoryux2ftree]{Update
a directory/tree}}\label{Updateux5faux5fdirectoryux2ftree}}

\textbf{Q:} How can I update a Zip from a directory tree, adding or
replacing only the newer files?

\textbf{A:} You can use the \emph{Archive::Zip::updateTree()} method
that was added in version 1.09.

\hyperdef{}{Zipux5ftimesux5fmightux5fbeux5foffux5fbyux5f1ux5fsecond}{\section{\hyperref[Zipux5ftimesux5fmightux5fbeux5foffux5fbyux5f1ux5fsecond]{Zip
times might be off by 1
second}}\label{Zipux5ftimesux5fmightux5fbeux5foffux5fbyux5f1ux5fsecond}}

\textbf{Q:} It bothers me greatly that my file times are wrong by one
second about half the time. Why don't you do something about it?

\textbf{A:} Get over it. This is a result of the Zip format storing
times in DOS format, which has a resolution of only two seconds.

\hyperdef{}{Zipux5ftimesux5fdonux27tux5fincludeux5ftimeux5fzoneux5finformation}{\section{\hyperref[Zipux5ftimesux5fdonux27tux5fincludeux5ftimeux5fzoneux5finformation]{Zip
times don't include time zone
information}}\label{Zipux5ftimesux5fdonux27tux5fincludeux5ftimeux5fzoneux5finformation}}

\textbf{Q:} My file times don't respect time zones. What gives?

\textbf{A:} If this is important to you, please submit patches to read
the various Extra Fields that encode times with time zones. I'm just
using the DOS Date/Time, which doesn't have a time zone.

\hyperdef{}{Howux5fdoux5fIux5fmakeux5faux5fself-extractingux5fZip}{\section{\hyperref[Howux5fdoux5fIux5fmakeux5faux5fself-extractingux5fZip]{How
do I make a self-extracting
Zip}}\label{Howux5fdoux5fIux5fmakeux5faux5fself-extractingux5fZip}}

\textbf{Q:} I want to make a self-extracting Zip file. Can I do this?

\textbf{A:} Yes. You can write a self-extracting archive stub (that is,
a version of unzip) to the output filehandle that you pass to
\emph{writeToFileHandle()}. See examples/selfex.pl for how to write a
self-extracting archive.

However, you should understand that this will only work on one kind of
platform (the one for which the stub was compiled).

\hyperdef{}{Howux5fcanux5fIux5fdealux5fwithux5fZipsux5fwithux5fprependedux5fgarbageux5fux28i.e.ux5ffromux5fSircamux29}{\section{\hyperref[Howux5fcanux5fIux5fdealux5fwithux5fZipsux5fwithux5fprependedux5fgarbageux5fux28i.e.ux5ffromux5fSircamux29]{How
can I deal with Zips with prepended garbage (i.e. from
Sircam)}}\label{Howux5fcanux5fIux5fdealux5fwithux5fZipsux5fwithux5fprependedux5fgarbageux5fux28i.e.ux5ffromux5fSircamux29}}

\textbf{Q:} How can I tell if a Zip has been damaged by adding garbage
to the beginning or inside the file?

\textbf{A:} I added code for this for the Amavis virus scanner. You can
query archives for their `eocdOffset' property, which should be 0:

\begin{verbatim}
  if ($zip->eocdOffset > 0)
    { warn($zip->eocdOffset . " bytes of garbage at beginning or within Zip") }
\end{verbatim}

When members are extracted, this offset will be used to adjust the start
of the member if necessary.

\hyperdef{}{Canux27tux5fextractux5fShrunkux5ffiles}{\section{\hyperref[Canux27tux5fextractux5fShrunkux5ffiles]{Can't
extract Shrunk files}}\label{Canux27tux5fextractux5fShrunkux5ffiles}}

\textbf{Q:} I'm trying to extract a file out of a Zip produced by PKZIP,
and keep getting this error message:

\begin{verbatim}
  error: Unsupported compression combination: read 6, write 0
\end{verbatim}

\textbf{A:} You can't uncompress this archive member. Archive::Zip only
supports uncompressed members, and compressed members that are
compressed using the compression supported by Compress::Raw::Zlib. That
means only Deflated and Stored members.

Your file is compressed using the Shrink format, which isn't supported
by Compress::Raw::Zlib.

You could, perhaps, use a command-line UnZip program (like the Info-Zip
one) to extract this.

\hyperdef{}{Canux27tux5fdoux5fdecryption}{\section{\hyperref[Canux27tux5fdoux5fdecryption]{Can't
do decryption}}\label{Canux27tux5fdoux5fdecryption}}

\textbf{Q:} How do I decrypt encrypted Zip members?

\textbf{A:} With some other program or library. Archive::Zip doesn't
support decryption, and probably never will (unless \emph{you} write
it).

\hyperdef{}{Howux5ftoux5ftestux5ffileux5fintegrityux3f}{\section{\hyperref[Howux5ftoux5ftestux5ffileux5fintegrityux3f]{How
to test file
integrity?}}\label{Howux5ftoux5ftestux5ffileux5fintegrityux3f}}

\textbf{Q:} How can Archive::Zip can test the validity of a Zip file?

\textbf{A:} If you try to decompress the file, the gzip streams will
report errors if you have garbage. Most of the time.

If you try to open the file and a central directory structure can't be
found, an error will be reported.

When a file is being read, if we can't find a proper PK.. signature in
the right places we report a format error.

If there is added garbage at the beginning of a Zip file (as inserted by
some viruses), you can find out about it, but Archive::Zip will ignore
it, and you can still use the archive. When it gets written back out the
added stuff will be gone.

There are two ready-to-use utilities in the examples directory that can
be used to test file integrity, or that you can use as examples for your
own code:

\begin{description}
\item[examples/zipcheck.pl shows how to use an attempted extraction to
test a file.]
\end{description}

\begin{description}
\item[examples/ziptest.pl shows how to test CRCs in a file.]
\end{description}

\hyperdef{}{Duplicateux5ffilesux5finux5fZipux3f}{\section{\hyperref[Duplicateux5ffilesux5finux5fZipux3f]{Duplicate
files in Zip?}}\label{Duplicateux5ffilesux5finux5fZipux3f}}

\textbf{Q:} Archive::Zip let me put the same file in my Zip twice! Why
don't you prevent this?

\textbf{A:} As far as I can tell, this is not disallowed by the Zip
spec. If you think it's a bad idea, check for it yourself:

\begin{verbatim}
  $zip->addFile($someFile, $someName) unless $zip->memberNamed($someName);
\end{verbatim}

I can even imagine cases where this might be useful (for instance,
multiple versions of files).

\hyperdef{}{Fileux5fownershipux2fpermissionsux2fACLSux2fetc}{\section{\hyperref[Fileux5fownershipux2fpermissionsux2fACLSux2fetc]{File
ownership/permissions/ACLS/etc}}\label{Fileux5fownershipux2fpermissionsux2fACLSux2fetc}}

\textbf{Q:} Why doesn't Archive::Zip deal with file ownership, ACLs,
etc.?

\textbf{A:} There is no standard way to represent these in the Zip file
format. If you want to send me code to properly handle the various extra
fields that have been used to represent these through the years, I'll
look at it.

\hyperdef{}{Iux5fcanux27tux5fcompileux5fbutux5fActiveStateux5fonlyux5fhasux5fanux5foldux5fversionux5fofux5fArchive::Zip}{\section{\hyperref[Iux5fcanux27tux5fcompileux5fbutux5fActiveStateux5fonlyux5fhasux5fanux5foldux5fversionux5fofux5fArchive::Zip]{I
can't compile but ActiveState only has an old version of
Archive::Zip}}\label{Iux5fcanux27tux5fcompileux5fbutux5fActiveStateux5fonlyux5fhasux5fanux5foldux5fversionux5fofux5fArchive::Zip}}

\textbf{Q:} I've only installed modules using ActiveState's PPM program
and repository. But they have a much older version of Archive::Zip than
is in CPAN. Will you send me a newer PPM?

\textbf{A:} Probably not, unless I get lots of extra time. But there's
no reason you can't install the version from CPAN. Archive::Zip is pure
Perl, so all you need is NMAKE, which you can get for free from
Microsoft (see the FAQ in the ActiveState documentation for details on
how to install CPAN modules).

\hyperdef{}{Myux5fJPEGsux5fux28orux5fMP3ux27sux29ux5fdonux27tux5fcompressux5fwhenux5fIux5fputux5fthemux5fintoux5fZipsux21}{\section{\hyperref[Myux5fJPEGsux5fux28orux5fMP3ux27sux29ux5fdonux27tux5fcompressux5fwhenux5fIux5fputux5fthemux5fintoux5fZipsux21]{My
JPEGs (or MP3's) don't compress when I put them into
Zips!}}\label{Myux5fJPEGsux5fux28orux5fMP3ux27sux29ux5fdonux27tux5fcompressux5fwhenux5fIux5fputux5fthemux5fintoux5fZipsux21}}

\textbf{Q:} How come my JPEGs and MP3's don't compress much when I put
them into Zips?

\textbf{A:} Because they're already compressed.

\hyperdef{}{Underux5fWindowsux2cux5fthingsux5flockux5fupux2fgetux5fdamaged}{\section{\hyperref[Underux5fWindowsux2cux5fthingsux5flockux5fupux2fgetux5fdamaged]{Under
Windows, things lock up/get
damaged}}\label{Underux5fWindowsux2cux5fthingsux5flockux5fupux2fgetux5fdamaged}}

\textbf{Q:} I'm using Windows. When I try to use Archive::Zip, my
machine locks up/makes funny sounds/displays a BSOD/corrupts data. How
can I fix this?

\textbf{A:} First, try the newest version of Compress::Raw::Zlib. I know
of Windows-related problems prior to v1.14 of that library.

If that doesn't get rid of the problem, fix your computer or get rid of
Windows.

\hyperdef{}{Zipux5fcontentsux5finux5faux5fscalar}{\section{\hyperref[Zipux5fcontentsux5finux5faux5fscalar]{Zip
contents in a scalar}}\label{Zipux5fcontentsux5finux5faux5fscalar}}

\textbf{Q:} I want to read a Zip file from (or write one to) a scalar
variable instead of a file. How can I do this?

\textbf{A:} Use ``IO::String'' and the ``readFromFileHandle()'' and
``writeToFileHandle()'' methods. See ``examples/readScalar.pl'' and
``examples/writeScalar.pl''.

\hyperdef{}{Readingux5ffromux5fstreams}{\section{\hyperref[Readingux5ffromux5fstreams]{Reading
from streams}}\label{Readingux5ffromux5fstreams}}

\textbf{Q:} How do I read from a stream (like for the Info-Zip
``funzip'' program)?

\textbf{A:} This isn't currently supported, though writing to a stream
is.

\begin{longtable}[c]{@{}ll@{}}
\toprule\addlinespace
2009-06-30 & perl v5.14.2
\\\addlinespace
\bottomrule
\end{longtable}

\end{document}
