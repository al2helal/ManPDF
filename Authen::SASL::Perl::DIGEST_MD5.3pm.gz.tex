\documentclass[]{article}
\usepackage[T1]{fontenc}
\usepackage{lmodern}
\usepackage{amssymb,amsmath}
\usepackage{ifxetex,ifluatex}
\usepackage{fixltx2e} % provides \textsubscript
% use upquote if available, for straight quotes in verbatim environments
\IfFileExists{upquote.sty}{\usepackage{upquote}}{}
\ifnum 0\ifxetex 1\fi\ifluatex 1\fi=0 % if pdftex
  \usepackage[utf8]{inputenc}
\else % if luatex or xelatex
  \ifxetex
    \usepackage{mathspec}
    \usepackage{xltxtra,xunicode}
  \else
    \usepackage{fontspec}
  \fi
  \defaultfontfeatures{Mapping=tex-text,Scale=MatchLowercase}
  \newcommand{\euro}{€}
\fi
% use microtype if available
\IfFileExists{microtype.sty}{\usepackage{microtype}}{}
\usepackage{longtable,booktabs}
\ifxetex
  \usepackage[setpagesize=false, % page size defined by xetex
              unicode=false, % unicode breaks when used with xetex
              xetex]{hyperref}
\else
  \usepackage[unicode=true]{hyperref}
\fi
\hypersetup{breaklinks=true,
            bookmarks=true,
            pdfauthor={},
            pdftitle={Authen::SASL::Perl::DIGEST\_MD5(3pm)},
            colorlinks=true,
            citecolor=blue,
            urlcolor=blue,
            linkcolor=magenta,
            pdfborder={0 0 0}}
\urlstyle{same}  % don't use monospace font for urls
\setlength{\parindent}{0pt}
\setlength{\parskip}{6pt plus 2pt minus 1pt}
\setlength{\emergencystretch}{3em}  % prevent overfull lines
\setcounter{secnumdepth}{0}
\usepackage{pagecolor}

% Set background colour (of the page)
\definecolor{weirdbgcolor}{HTML}{FCF4F0}
\pagecolor{weirdbgcolor}

% Make bold text appear in a particular colour
\definecolor{boldcolor}{HTML}{6E0002}
\let\realtextbf=\textbf
\renewcommand{\textbf}[1]{\textcolor{boldcolor}{\realtextbf{#1}}}

% Use underlines instead of emphasis (ugh)
\renewcommand{\emph}[1]{\underline{#1}}

% % Use fixed-width font by default
% \renewcommand*\familydefault{\ttdefault}

\title{Authen::SASL::Perl::DIGEST\_MD5(3pm)}
\author{}
\date{}

\begin{document}
\maketitle

\begin{longtable}[c]{@{}lll@{}}
\toprule\addlinespace
Authen::SASL::Perl::DIGEST\_MD5(3pm) & User Contributed Perl
Documentation & Authen::SASL::Perl::DIGEST\_MD5(3pm)
\\\addlinespace
\bottomrule
\end{longtable}

\hyperdef{}{NAME}{\section{\hyperref[NAME]{NAME}}\label{NAME}}

Authen::SASL::Perl::DIGEST\_MD5 - Digest MD5 Authentication class

\hyperdef{}{SYNOPSIS}{\section{\hyperref[SYNOPSIS]{SYNOPSIS}}\label{SYNOPSIS}}

\begin{verbatim}
  use Authen::SASL qw(Perl);
  $sasl = Authen::SASL->new(
    mechanism => 'DIGEST-MD5',
    callback  => {
      user => $user, 
      pass => $pass,
      serv => $serv
    },
  );
\end{verbatim}

\hyperdef{}{DESCRIPTION}{\section{\hyperref[DESCRIPTION]{DESCRIPTION}}\label{DESCRIPTION}}

This method implements the client and server parts of the DIGEST-MD5
SASL algorithm, as described in RFC 2831.

\hyperdef{}{CALLBACK}{\subsection{\hyperref[CALLBACK]{CALLBACK}}\label{CALLBACK}}

The callbacks used are:

\emph{client}

\begin{description}
\itemsep1pt\parskip0pt\parsep0pt
\item[authname]
The authorization id to use after successful authentication
\end{description}

\begin{description}
\itemsep1pt\parskip0pt\parsep0pt
\item[user]
The username to be used in the response
\end{description}

\begin{description}
\itemsep1pt\parskip0pt\parsep0pt
\item[pass]
The password to be used to compute the response.
\end{description}

\begin{description}
\itemsep1pt\parskip0pt\parsep0pt
\item[serv]
The service name when authenticating to a replicated service
\end{description}

\begin{description}
\itemsep1pt\parskip0pt\parsep0pt
\item[realm]
The authentication realm when overriding the server-provided default. If
not given the server-provided value is used.

~

The callback will be passed the list of realms that the server provided
in the initial response.
\end{description}

\emph{server}

\begin{description}
\itemsep1pt\parskip0pt\parsep0pt
\item[realm]
The default realm to provide to the client
\end{description}

\begin{description}
\itemsep1pt\parskip0pt\parsep0pt
\item[getsecret(username, realm, authzid)]
returns the password associated with ``username'' and ``realm''
\end{description}

\hyperdef{}{PROPERTIES}{\subsection{\hyperref[PROPERTIES]{PROPERTIES}}\label{PROPERTIES}}

The properties used are:

\begin{description}
\itemsep1pt\parskip0pt\parsep0pt
\item[maxbuf]
The maximum buffer size for receiving cipher text
\end{description}

\begin{description}
\itemsep1pt\parskip0pt\parsep0pt
\item[minssf]
The minimum SSF value that should be provided by the SASL security
layer. The default is 0
\end{description}

\begin{description}
\itemsep1pt\parskip0pt\parsep0pt
\item[maxssf]
The maximum SSF value that should be provided by the SASL security
layer. The default is 2**31
\end{description}

\begin{description}
\itemsep1pt\parskip0pt\parsep0pt
\item[externalssf]
The SSF value provided by an underlying external security layer. The
default is 0
\end{description}

\begin{description}
\itemsep1pt\parskip0pt\parsep0pt
\item[ssf]
The actual SSF value provided by the SASL security layer after the SASL
authentication phase has been completed. This value is read-only and set
by the implementation after the SASL authentication phase has been
completed.
\end{description}

\begin{description}
\itemsep1pt\parskip0pt\parsep0pt
\item[maxout]
The maximum plaintext buffer size for sending data to the peer. This
value is set by the implementation after the SASL authentication phase
has been completed and a SASL security layer is in effect.
\end{description}

\hyperdef{}{SEEux5fALSO}{\section{\hyperref[SEEux5fALSO]{SEE
ALSO}}\label{SEEux5fALSO}}

Authen::SASL, Authen::SASL::Perl

\hyperdef{}{AUTHORS}{\section{\hyperref[AUTHORS]{AUTHORS}}\label{AUTHORS}}

Graham Barr, Djamel Boudjerda (NEXOR), Paul Connolly, Julian Onions
(NEXOR), Yann Kerherve.

Please report any bugs, or post any suggestions, to the perl-ldap
mailing list \textless{}perl-ldap@perl.org\textgreater{}

\hyperdef{}{COPYRIGHT}{\section{\hyperref[COPYRIGHT]{COPYRIGHT}}\label{COPYRIGHT}}

Copyright (c) 2003-2009 Graham Barr, Djamel Boudjerda, Paul Connolly,
Julian Onions, Nexor, Peter Marschall and Yann Kerherve. All rights
reserved. This program is free software; you can redistribute it and/or
modify it under the same terms as Perl itself.

\begin{longtable}[c]{@{}ll@{}}
\toprule\addlinespace
2010-06-07 & perl v5.10.1
\\\addlinespace
\bottomrule
\end{longtable}

\end{document}
