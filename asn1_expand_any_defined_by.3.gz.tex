\documentclass[]{article}
\usepackage[T1]{fontenc}
\usepackage{lmodern}
\usepackage{amssymb,amsmath}
\usepackage{ifxetex,ifluatex}
\usepackage{fixltx2e} % provides \textsubscript
% use upquote if available, for straight quotes in verbatim environments
\IfFileExists{upquote.sty}{\usepackage{upquote}}{}
\ifnum 0\ifxetex 1\fi\ifluatex 1\fi=0 % if pdftex
  \usepackage[utf8]{inputenc}
\else % if luatex or xelatex
  \ifxetex
    \usepackage{mathspec}
    \usepackage{xltxtra,xunicode}
  \else
    \usepackage{fontspec}
  \fi
  \defaultfontfeatures{Mapping=tex-text,Scale=MatchLowercase}
  \newcommand{\euro}{€}
\fi
% use microtype if available
\IfFileExists{microtype.sty}{\usepackage{microtype}}{}
\usepackage{longtable,booktabs}
\ifxetex
  \usepackage[setpagesize=false, % page size defined by xetex
              unicode=false, % unicode breaks when used with xetex
              xetex]{hyperref}
\else
  \usepackage[unicode=true]{hyperref}
\fi
\hypersetup{breaklinks=true,
            bookmarks=true,
            pdfauthor={},
            pdftitle={asn1\_expand\_any\_defined\_by(3)},
            colorlinks=true,
            citecolor=blue,
            urlcolor=blue,
            linkcolor=magenta,
            pdfborder={0 0 0}}
\urlstyle{same}  % don't use monospace font for urls
\setlength{\parindent}{0pt}
\setlength{\parskip}{6pt plus 2pt minus 1pt}
\setlength{\emergencystretch}{3em}  % prevent overfull lines
\setcounter{secnumdepth}{0}
\usepackage{pagecolor}

% Set background colour (of the page)
\definecolor{weirdbgcolor}{HTML}{FCF4F0}
\pagecolor{weirdbgcolor}

% Make bold text appear in a particular colour
\definecolor{boldcolor}{HTML}{6E0002}
\let\realtextbf=\textbf
\renewcommand{\textbf}[1]{\textcolor{boldcolor}{\realtextbf{#1}}}

% Use underlines instead of emphasis (ugh)
\renewcommand{\emph}[1]{\underline{#1}}

% % Use fixed-width font by default
% \renewcommand*\familydefault{\ttdefault}

\title{asn1\_expand\_any\_defined\_by(3)}
\author{}
\date{}

\begin{document}
\maketitle

\begin{longtable}[c]{@{}lll@{}}
\toprule\addlinespace
asn1\_expand\_any\_defined\_by(3) & libtasn1 &
asn1\_expand\_any\_defined\_by(3)
\\\addlinespace
\bottomrule
\end{longtable}

\hyperdef{}{NAME}{\section{\hyperref[NAME]{NAME}}\label{NAME}}

asn1\_expand\_any\_defined\_by - API function

\hyperdef{}{SYNOPSIS}{\section{\hyperref[SYNOPSIS]{SYNOPSIS}}\label{SYNOPSIS}}

\textbf{\#include \textless{}libtasn1.h\textgreater{}}

~

\textbf{int
asn1\_expand\_any\_defined\_by(asn1\_node}\emph{definitions}\textbf{,
asn1\_node *}\emph{element}\textbf{);}

\hyperdef{}{ARGUMENTS}{\section{\hyperref[ARGUMENTS]{ARGUMENTS}}\label{ARGUMENTS}}

\begin{description}
\itemsep1pt\parskip0pt\parsep0pt
\item[asn1\_node definitions]
ASN1 definitions
\end{description}

\begin{description}
\itemsep1pt\parskip0pt\parsep0pt
\item[asn1\_node * element]
pointer to an ASN1 structure
\end{description}

\hyperdef{}{DESCRIPTION}{\section{\hyperref[DESCRIPTION]{DESCRIPTION}}\label{DESCRIPTION}}

Expands every ``ANY DEFINED BY'' element of a structure created from a
DER decoding process (asn1\_der\_decoding function). The element ANY
must be defined by an OBJECT IDENTIFIER. The type used to expand the
element ANY is the first one following the definition of the actual
value of the OBJECT IDENTIFIER.

\hyperdef{}{RETURNS}{\section{\hyperref[RETURNS]{RETURNS}}\label{RETURNS}}

\textbf{ASN1\_SUCCESS} if Substitution OK,
\textbf{ASN1\_ERROR\_TYPE\_ANY} if some ``ANY DEFINED BY'' element
couldn't be expanded due to a problem in OBJECT\_ID -\textgreater{} TYPE
association, or other error codes depending on DER decoding.

\hyperdef{}{COPYRIGHT}{\section{\hyperref[COPYRIGHT]{COPYRIGHT}}\label{COPYRIGHT}}

Copyright © 2006-2013 Free Software Foundation, Inc..

~

Copying and distribution of this file, with or without modification, are
permitted in any medium without royalty provided the copyright notice
and this notice are preserved.

\hyperdef{}{SEEux5fALSO}{\section{\hyperref[SEEux5fALSO]{SEE
ALSO}}\label{SEEux5fALSO}}

The full documentation for \textbf{libtasn1} is maintained as a Texinfo
manual. If the \textbf{info} and \textbf{libtasn1} programs are properly
installed at your site, the command

\begin{description}
\itemsep1pt\parskip0pt\parsep0pt
\item[]
\textbf{info libtasn1}
\end{description}

should give you access to the complete manual. As an alternative you may
obtain the manual from:

\begin{description}
\itemsep1pt\parskip0pt\parsep0pt
\item[]
\textbf{http://www.gnu.org/software/libtasn1/manual/}
\end{description}

\begin{longtable}[c]{@{}ll@{}}
\toprule\addlinespace
3.4 & libtasn1
\\\addlinespace
\bottomrule
\end{longtable}

\end{document}
