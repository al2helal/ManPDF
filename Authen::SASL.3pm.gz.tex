\documentclass[]{article}
\usepackage[T1]{fontenc}
\usepackage{lmodern}
\usepackage{amssymb,amsmath}
\usepackage{ifxetex,ifluatex}
\usepackage{fixltx2e} % provides \textsubscript
% use upquote if available, for straight quotes in verbatim environments
\IfFileExists{upquote.sty}{\usepackage{upquote}}{}
\ifnum 0\ifxetex 1\fi\ifluatex 1\fi=0 % if pdftex
  \usepackage[utf8]{inputenc}
\else % if luatex or xelatex
  \ifxetex
    \usepackage{mathspec}
    \usepackage{xltxtra,xunicode}
  \else
    \usepackage{fontspec}
  \fi
  \defaultfontfeatures{Mapping=tex-text,Scale=MatchLowercase}
  \newcommand{\euro}{€}
\fi
% use microtype if available
\IfFileExists{microtype.sty}{\usepackage{microtype}}{}
\usepackage{longtable,booktabs}
\ifxetex
  \usepackage[setpagesize=false, % page size defined by xetex
              unicode=false, % unicode breaks when used with xetex
              xetex]{hyperref}
\else
  \usepackage[unicode=true]{hyperref}
\fi
\hypersetup{breaklinks=true,
            bookmarks=true,
            pdfauthor={},
            pdftitle={Authen::SASL(3pm)},
            colorlinks=true,
            citecolor=blue,
            urlcolor=blue,
            linkcolor=magenta,
            pdfborder={0 0 0}}
\urlstyle{same}  % don't use monospace font for urls
\setlength{\parindent}{0pt}
\setlength{\parskip}{6pt plus 2pt minus 1pt}
\setlength{\emergencystretch}{3em}  % prevent overfull lines
\setcounter{secnumdepth}{0}
\usepackage{pagecolor}

% Set background colour (of the page)
\definecolor{weirdbgcolor}{HTML}{FCF4F0}
\pagecolor{weirdbgcolor}

% Make bold text appear in a particular colour
\definecolor{boldcolor}{HTML}{6E0002}
\let\realtextbf=\textbf
\renewcommand{\textbf}[1]{\textcolor{boldcolor}{\realtextbf{#1}}}

% Use underlines instead of emphasis (ugh)
\renewcommand{\emph}[1]{\underline{#1}}

% % Use fixed-width font by default
% \renewcommand*\familydefault{\ttdefault}

\title{Authen::SASL(3pm)}
\author{}
\date{}

\begin{document}
\maketitle

\begin{longtable}[c]{@{}lll@{}}
\toprule\addlinespace
Authen::SASL(3pm) & User Contributed Perl Documentation &
Authen::SASL(3pm)
\\\addlinespace
\bottomrule
\end{longtable}

\hyperdef{}{NAME}{\section{\hyperref[NAME]{NAME}}\label{NAME}}

Authen::SASL - SASL Authentication framework

\hyperdef{}{SYNOPSIS}{\section{\hyperref[SYNOPSIS]{SYNOPSIS}}\label{SYNOPSIS}}

\begin{verbatim}
 use Authen::SASL;
 $sasl = Authen::SASL->new(
   mechanism => 'CRAM-MD5 PLAIN ANONYMOUS',
   callback => {
     pass => \&fetch_password,
     user => $user,
   }
 );
\end{verbatim}

\hyperdef{}{DESCRIPTION}{\section{\hyperref[DESCRIPTION]{DESCRIPTION}}\label{DESCRIPTION}}

SASL is a generic mechanism for authentication used by several network
protocols. \textbf{Authen::SASL} provides an implementation framework
that all protocols should be able to share.

The framework allows different implementations of the connection class
to be plugged in. At the time of writing there were two such plugins.

\begin{description}
\itemsep1pt\parskip0pt\parsep0pt
\item[Authen::SASL::Perl]
This module implements several mechanisms and is implemented entirely in
Perl.
\end{description}

\begin{description}
\itemsep1pt\parskip0pt\parsep0pt
\item[Authen::SASL::XS]
This module uses the Cyrus SASL C-library (both version 1 and 2 are
supported).
\end{description}

\begin{description}
\itemsep1pt\parskip0pt\parsep0pt
\item[Authen::SASL::Cyrus]
This module is the predecessor to Authen::SASL::XS. It is reccomended to
use Authen::SASL::XS
\end{description}

By default the order in which these plugins are selected is
Authen::SASL::XS, Authen::SASL::Cyrus and then Authen::SASL::Perl.

If you want to change it or want to specifically use one implementation
only simply do

\begin{verbatim}
 use Authen::SASL qw(Perl);
\end{verbatim}

or if you have another plugin module that supports the Authen::SASL API

\begin{verbatim}
 use Authen::SASL qw(My::SASL::Plugin);
\end{verbatim}

\hyperdef{}{CONTRUCTOR}{\subsection{\hyperref[CONTRUCTOR]{CONTRUCTOR}}\label{CONTRUCTOR}}

\begin{description}
\itemsep1pt\parskip0pt\parsep0pt
\item[new ( OPTIONS )]
The contructor may be called with or without arguments. Passing
arguments is just a short cut to calling the ``mechanism'' and
``callback'' methods.
\end{description}

\begin{description}
\itemsep1pt\parskip0pt\parsep0pt
\item[callback =\textgreater{} \{ NAME =\textgreater{} VALUE, NAME
=\textgreater{} VALUE, \ldots{} \}]
Set the callbacks. See the callback method for details.
\end{description}

\begin{description}
\item[mechanism =\textgreater{} NAMES]
\end{description}

\begin{description}
\itemsep1pt\parskip0pt\parsep0pt
\item[mech =\textgreater{} NAMES]
Set the list of mechanisms to choose from. See the mechanism method for
details.
\end{description}

\begin{description}
\item[debug =\textgreater{} VALUE]
Set the debug level bit-value to ``VALUE''

~

Debug output will be sent to ``STDERR''. The bits of this value are:

~

\begin{verbatim}
 1   Show debug messages in the Perl modules for the mechanisms.
     (Currently only used in GSSAPI)
 4   With security layers in place show information on packages read.
 8   With security layers in place show information on packages written.
    
\end{verbatim}

~

The default value is 0.
\end{description}

\hyperdef{}{METHODS}{\subsection{\hyperref[METHODS]{METHODS}}\label{METHODS}}

\begin{description}
\itemsep1pt\parskip0pt\parsep0pt
\item[mechanism ( )]
Returns the current list of mechanisms
\end{description}

\begin{description}
\itemsep1pt\parskip0pt\parsep0pt
\item[mechanism ( NAMES )]
Set the list of mechanisms to choose from. ``NAMES'' should be a space
separated string of the names.
\end{description}

\begin{description}
\itemsep1pt\parskip0pt\parsep0pt
\item[callback ( NAME )]
Returns the current callback associated with ``NAME''.
\end{description}

\begin{description}
\itemsep1pt\parskip0pt\parsep0pt
\item[callback ( NAME =\textgreater{} VALUE, NAME =\textgreater{} VALUE,
\ldots{} )]
Sets the given callbacks to the given values
\end{description}

\begin{description}
\itemsep1pt\parskip0pt\parsep0pt
\item[client\_new ( SERVICE, HOST, SECURITY )]
Creates and returns a new connection object for a client-side
connection.
\end{description}

\begin{description}
\itemsep1pt\parskip0pt\parsep0pt
\item[server\_new ( SERVICE, HOST, OPTIONS )]
Creates and returns a new connection object for a server-side
connection.
\end{description}

\begin{description}
\itemsep1pt\parskip0pt\parsep0pt
\item[error ( )]
Returns any error from the last connection
\end{description}

\hyperdef{}{Theux5fConnectionux5fClass}{\section{\hyperref[Theux5fConnectionux5fClass]{The
Connection Class}}\label{Theux5fConnectionux5fClass}}

\begin{description}
\itemsep1pt\parskip0pt\parsep0pt
\item[server\_start ( CHALLENGE )]
server\_start begins the authentication using the chosen mechanism. If
the mechanism is not supported by the installed SASL it fails. Because
for some mechanisms the client has to start the negotiation, you can
give the client challenge as a parameter.
\end{description}

\begin{description}
\itemsep1pt\parskip0pt\parsep0pt
\item[server\_step ( CHALLENGE )]
server\_step performs the next step in the negotiation process. The
first parameter you give is the clients challenge/response.
\end{description}

\begin{description}
\itemsep1pt\parskip0pt\parsep0pt
\item[client\_start ( )]
The initial step to be performed. Returns the initial value to pass to
the server or an empty list on error.
\end{description}

\begin{description}
\itemsep1pt\parskip0pt\parsep0pt
\item[client\_step ( CHALLENGE )]
This method is called when a response from the server requires it.
CHALLENGE is the value from the server. Returns the next value to pass
to the server or an empty list on error.
\end{description}

\begin{description}
\itemsep1pt\parskip0pt\parsep0pt
\item[need\_step ( )]
Returns true if the selected mechanism requires another step before
completion (error or success).
\end{description}

\begin{description}
\itemsep1pt\parskip0pt\parsep0pt
\item[answer ( NAME )]
The method will return the value returned from the last call to the
callback NAME
\end{description}

\begin{description}
\itemsep1pt\parskip0pt\parsep0pt
\item[property ( NAME )]
Returns the property value associated with ``NAME''.
\end{description}

\begin{description}
\itemsep1pt\parskip0pt\parsep0pt
\item[property ( NAME =\textgreater{} VALUE, NAME =\textgreater{} VALUE,
\ldots{} )]
Sets the named properties to their associated values.
\end{description}

\begin{description}
\itemsep1pt\parskip0pt\parsep0pt
\item[service ( )]
Returns the service argument that was passed to *\_new-methods.
\end{description}

\begin{description}
\itemsep1pt\parskip0pt\parsep0pt
\item[host ( )]
Returns the host argument that was passed to *\_new-methods.
\end{description}

\begin{description}
\itemsep1pt\parskip0pt\parsep0pt
\item[mechanism ( )]
Returns the name of the chosen mechanism.
\end{description}

\begin{description}
\itemsep1pt\parskip0pt\parsep0pt
\item[is\_success ( )]
Once \emph{need\_step()} returns false, then you can check if the
authentication succeeded by calling this method which returns a boolean
value.
\end{description}

\hyperdef{}{Callbacks}{\subsection{\hyperref[Callbacks]{Callbacks}}\label{Callbacks}}

There are three different ways in which a callback may be passed

\begin{description}
\itemsep1pt\parskip0pt\parsep0pt
\item[CODEREF]
If the value passed is a code reference then, when needed, it will be
called and the connection object will be passed as the first argument.
In addition some callbacks may be passed additional arguments.
\end{description}

\begin{description}
\itemsep1pt\parskip0pt\parsep0pt
\item[ARRAYREF]
If the value passed is an array reference, the first element in the
array must be a code reference. When the callback is called the code
reference will be called with the connection object passed as the first
argument and all other values from the array passed after.
\end{description}

\begin{description}
\itemsep1pt\parskip0pt\parsep0pt
\item[SCALAR]
All other values passed will be used directly. ie it is the same as
passing an code reference that, when called, returns the value.
\end{description}

\hyperdef{}{SEEux5fALSO}{\section{\hyperref[SEEux5fALSO]{SEE
ALSO}}\label{SEEux5fALSO}}

Authen::SASL::Perl, Authen::SASL::XS, Authen::SASL::Cyrus

\hyperdef{}{AUTHOR}{\section{\hyperref[AUTHOR]{AUTHOR}}\label{AUTHOR}}

Graham Barr \textless{}gbarr@pobox.com\textgreater{}

Please report any bugs, or post any suggestions, to the perl-ldap
mailing list \textless{}perl-ldap@perl.org\textgreater{}

\hyperdef{}{COPYRIGHT}{\section{\hyperref[COPYRIGHT]{COPYRIGHT}}\label{COPYRIGHT}}

Copyright (c) 1998-2005 Graham Barr. All rights reserved. This program
is free software; you can redistribute it and/or modify it under the
same terms as Perl itself.

\begin{longtable}[c]{@{}ll@{}}
\toprule\addlinespace
2010-03-11 & perl v5.10.1
\\\addlinespace
\bottomrule
\end{longtable}

\end{document}
