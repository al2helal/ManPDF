\documentclass[]{article}
\usepackage[T1]{fontenc}
\usepackage{lmodern}
\usepackage{amssymb,amsmath}
\usepackage{ifxetex,ifluatex}
\usepackage{fixltx2e} % provides \textsubscript
% use upquote if available, for straight quotes in verbatim environments
\IfFileExists{upquote.sty}{\usepackage{upquote}}{}
\ifnum 0\ifxetex 1\fi\ifluatex 1\fi=0 % if pdftex
  \usepackage[utf8]{inputenc}
\else % if luatex or xelatex
  \ifxetex
    \usepackage{mathspec}
    \usepackage{xltxtra,xunicode}
  \else
    \usepackage{fontspec}
  \fi
  \defaultfontfeatures{Mapping=tex-text,Scale=MatchLowercase}
  \newcommand{\euro}{€}
\fi
% use microtype if available
\IfFileExists{microtype.sty}{\usepackage{microtype}}{}
\usepackage{longtable,booktabs}
\ifxetex
  \usepackage[setpagesize=false, % page size defined by xetex
              unicode=false, % unicode breaks when used with xetex
              xetex]{hyperref}
\else
  \usepackage[unicode=true]{hyperref}
\fi
\hypersetup{breaklinks=true,
            bookmarks=true,
            pdfauthor={},
            pdftitle={ASN1\_STRING\_print\_ex(3SSL)},
            colorlinks=true,
            citecolor=blue,
            urlcolor=blue,
            linkcolor=magenta,
            pdfborder={0 0 0}}
\urlstyle{same}  % don't use monospace font for urls
\setlength{\parindent}{0pt}
\setlength{\parskip}{6pt plus 2pt minus 1pt}
\setlength{\emergencystretch}{3em}  % prevent overfull lines
\setcounter{secnumdepth}{0}
\usepackage{pagecolor}

% Set background colour (of the page)
\definecolor{weirdbgcolor}{HTML}{FCF4F0}
\pagecolor{weirdbgcolor}

% Make bold text appear in a particular colour
\definecolor{boldcolor}{HTML}{6E0002}
\let\realtextbf=\textbf
\renewcommand{\textbf}[1]{\textcolor{boldcolor}{\realtextbf{#1}}}

% Use underlines instead of emphasis (ugh)
\renewcommand{\emph}[1]{\underline{#1}}

% % Use fixed-width font by default
% \renewcommand*\familydefault{\ttdefault}

\title{ASN1\_STRING\_print\_ex(3SSL)}
\author{}
\date{}

\begin{document}
\maketitle

\begin{longtable}[c]{@{}lll@{}}
\toprule\addlinespace
ASN1\_STRING\_print\_ex(3SSL) & OpenSSL & ASN1\_STRING\_print\_ex(3SSL)
\\\addlinespace
\bottomrule
\end{longtable}

\hyperdef{}{NAME}{\section{\hyperref[NAME]{NAME}}\label{NAME}}

ASN1\_STRING\_print\_ex, ASN1\_STRING\_print\_ex\_fp - ASN1\_STRING
output routines.

\hyperdef{}{SYNOPSIS}{\section{\hyperref[SYNOPSIS]{SYNOPSIS}}\label{SYNOPSIS}}

\begin{verbatim}
 #include <openssl/asn1.h>
 int ASN1_STRING_print_ex(BIO *out, ASN1_STRING *str, unsigned long flags);
 int ASN1_STRING_print_ex_fp(FILE *fp, ASN1_STRING *str, unsigned long flags);
 int ASN1_STRING_print(BIO *out, ASN1_STRING *str);
\end{verbatim}

\hyperdef{}{DESCRIPTION}{\section{\hyperref[DESCRIPTION]{DESCRIPTION}}\label{DESCRIPTION}}

These functions output an \textbf{ASN1\_STRING} structure.
\textbf{ASN1\_STRING} is used to represent all the ASN1 string types.

\emph{ASN1\_STRING\_print\_ex()} outputs \textbf{str} to \textbf{out},
the format is determined by the options \textbf{flags}.
\emph{ASN1\_STRING\_print\_ex\_fp()} is identical except it outputs to
\textbf{fp} instead.

\emph{ASN1\_STRING\_print()} prints \textbf{str} to \textbf{out} but
using a different format to \emph{ASN1\_STRING\_print\_ex()}. It
replaces unprintable characters (other than CR, LF) with `.'.

\hyperdef{}{NOTES}{\section{\hyperref[NOTES]{NOTES}}\label{NOTES}}

\emph{ASN1\_STRING\_print()} is a legacy function which should be
avoided in new applications.

Although there are a large number of options frequently
\textbf{ASN1\_STRFLGS\_RFC2253} is suitable, or on UTF8 terminals
\textbf{ASN1\_STRFLGS\_RFC2253 \&
\textasciitilde{}ASN1\_STRFLGS\_ESC\_MSB}.

The complete set of supported options for \textbf{flags} is listed
below.

Various characters can be escaped. If \textbf{ASN1\_STRFLGS\_ESC\_2253}
is set the characters determined by RFC2253 are escaped. If
\textbf{ASN1\_STRFLGS\_ESC\_CTRL} is set control characters are escaped.
If \textbf{ASN1\_STRFLGS\_ESC\_MSB} is set characters with the MSB set
are escaped: this option should \textbf{not} be used if the terminal
correctly interprets UTF8 sequences.

Escaping takes several forms.

If the character being escaped is a 16 bit character then the form
``\textbackslash{}UXXXX'' is used using exactly four characters for the
hex representation. If it is 32 bits then ``\textbackslash{}WXXXXXXXX''
is used using eight characters of its hex representation. These forms
will only be used if UTF8 conversion is not set (see below).

Printable characters are normally escaped using the backslash
`\textbackslash{}' character. If \textbf{ASN1\_STRFLGS\_ESC\_QUOTE} is
set then the whole string is instead surrounded by double quote
characters: this is arguably more readable than the backslash notation.
Other characters use the ``\textbackslash{}XX'' using exactly two
characters of the hex representation.

If \textbf{ASN1\_STRFLGS\_UTF8\_CONVERT} is set then characters are
converted to UTF8 format first. If the terminal supports the display of
UTF8 sequences then this option will correctly display multi byte
characters.

If \textbf{ASN1\_STRFLGS\_IGNORE\_TYPE} is set then the string type is
not interpreted at all: everything is assumed to be one byte per
character. This is primarily for debugging purposes and can result in
confusing output in multi character strings.

If \textbf{ASN1\_STRFLGS\_SHOW\_TYPE} is set then the string type itself
is printed out before its value (for example ``BMPSTRING''), this
actually uses \emph{ASN1\_tag2str()}.

The content of a string instead of being interpreted can be ``dumped'':
this just outputs the value of the string using the form \#XXXX using
hex format for each octet.

If \textbf{ASN1\_STRFLGS\_DUMP\_ALL} is set then any type is dumped.

Normally non character string types (such as OCTET STRING) are assumed
to be one byte per character, if \textbf{ASN1\_STRFLGS\_DUMP\_UNKNOWN}
is set then they will be dumped instead.

When a type is dumped normally just the content octets are printed, if
\textbf{ASN1\_STRFLGS\_DUMP\_DER} is set then the complete encoding is
dumped instead (including tag and length octets).

\textbf{ASN1\_STRFLGS\_RFC2253} includes all the flags required by
RFC2253. It is equivalent to: \\ ASN1\_STRFLGS\_ESC\_2253 \textbar{}
ASN1\_STRFLGS\_ESC\_CTRL \textbar{} ASN1\_STRFLGS\_ESC\_MSB \textbar{}
\\ ASN1\_STRFLGS\_UTF8\_CONVERT \textbar{} ASN1\_STRFLGS\_DUMP\_UNKNOWN
ASN1\_STRFLGS\_DUMP\_DER

\hyperdef{}{SEEux5fALSO}{\section{\hyperref[SEEux5fALSO]{SEE
ALSO}}\label{SEEux5fALSO}}

\emph{X509\_NAME\_print\_ex}(3), \emph{ASN1\_tag2str}(3)

\hyperdef{}{HISTORY}{\section{\hyperref[HISTORY]{HISTORY}}\label{HISTORY}}

TBA

\begin{longtable}[c]{@{}ll@{}}
\toprule\addlinespace
2014-01-06 & 1.0.1f
\\\addlinespace
\bottomrule
\end{longtable}

\end{document}
